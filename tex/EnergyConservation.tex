\section{Potential energy and conservation of energy}

%%%%%%%%%%%%%%%%%%%%%%%%%%%%%%%%%%%
%%
%% Multiple Choice
%%
%%%%%%%%%%%%%%%%%%%%%%%%%%%%%%%%%%%
\subsection{Multiple Choice}


%this question is slightly weird. Not sure how I would approach it. Should students know how much energy is held in nuclear binding energy?
\question The Darlington Nuclear Generating station in Ontario converts nuclear binding energy into electrical energy. What electrical power does the Darlington Nuclear Generating station produce?
\begin{choices} 
	\choice \SI{2.512e6}{kg\cdot m^3/s^2}
	\CorrectChoice \SI{3.512e6}{kg\cdot m^2/s^3}
	\choice \SI{4.512e6}{kg^2\cdot m^3/s^2}
	\choice \SI{5.512e6}{kg\cdot m^3/s^3}
\end{choices}

\question The Force from Star Wars is found to be conservative, so that a potential energy function, $U(\vec r)$ (aka ``The Potential''), can be defined. Captain Jean-Luc Picard uses The Force to create a Force Field around his ship in order to repel an incoming rebel ship full of jedis, klingons and other rebellious space people. As the incoming rebel ship powers through the repelling force field, the rebel ship's Potential Energy due to The Force:
\begin{choices} 
	\choice decreases.
	\choice remains the same.
	\CorrectChoice increases.
	\choice not enough information to tell.
\end{choices}

\question A bird with a mass of \SI{200}{g} is resting in a tree. It has a potential energy of \SI{10}{J} with zero potential energy defined to be where the ground is. What is the height of the bird relative to the ground?
\begin{checkboxes}
\choice \SI{19.6}{m}
\choice \SI{9.8}{m}
\choice \SI{7.2}{m}
\CorrectChoice \SI{5.1}{m} \correct
\end{checkboxes}

%Question submitted by David Cutler
\question If a lightning bolt were to discharge all of its (roughly) one billion Joules into a \SI{2.0}{g} bullet, and we assume that 100\% of that energy is converted into horizontal kinetic energy, how fast would the bullet travel immediately after being hit?
\begin{checkboxes}
\choice \SI{7.1e5}{m/s}
\CorrectChoice \SI{1.0e6}{m/s} \correct
\choice \SI{1.0e12}{m/s}
\choice \SI{3.0e8}{m/s}
\end{checkboxes}


\question A vicu\~na with a mass of \SI{35}{kg} is launched through the air in a parabolic arc. It has an initial kinetic energy of \SI{28}{kJ} and travels a total horizontal distance of \SI{100}{m}, reaching its max height of \SI{150}{m} at a distance of \SI{50}{m} from where it was launched. Ignoring air resistance, what is the vicu\~na's speed when it hits the ground?
\begin{checkboxes}
\choice \SI{1.3}{m/s}
\CorrectChoice \SI{40}{m/s} \correct
\choice \SI{36.6}{m/s}
\choice \SI{20}{m/s}
\end{checkboxes}


\question What is true about the kinetic energy of a one-dimensional, vertical mass-spring system?
\begin{checkboxes}
\choice It is at a maximum at the highest and lowest point and zero at equilibrium point
\CorrectChoice It is zero at the highest and lowest point and at a maximum at the equilibrium point \correct
\choice It is constant for the entire motion cycle
\choice There is no kinetic energy, only a balance between elastic potential energy and gravitational potential energy
\end{checkboxes}

\question Two children are on a playground competing to see who can throw a lawn dart the highest. Unfortunately, they fail to notice that they are standing under a tree. As soon as the game begins, they each throw their lawn dart straight upwards, but neither one comes down. One is stuck \SI{20}{m} high in the tree and the other one is stuck \SI{30}{m} high in the tree. Which of the following statements is true? Select all that apply:
\begin{checkboxes}
\CorrectChoice The lawn darts both currently have the same kinetic energy \correct
\choice The lawn darts both currently have the same gravitational potential energy
\choice The lawn darts both experienced the same amount of net work since they were thrown
\CorrectChoice The path of each lawn dart is needed to determine the net work done on each dart \correct
\choice The tree is currently doing work on the lawn darts to keep them from falling
\end{checkboxes}

%Submitted by Nicholas Everton
\question An object is released from a height and falls under the influence of gravity. Which of the following statements is true in this situation if you neglect air resistance? [Select all that apply]
\begin{checkboxes}
\choice The mechanical energy of the object is decreasing
\choice The kinetic energy of the object is increasing and the potential energy is increasing
\CorrectChoice The kinetic energy of the object is increasing and the potential energy is decreasing \correct
\choice The mechanical energy of the object is increasing
\CorrectChoice The mechanical energy of the object is constant \correct
\end{checkboxes}

%from Stephanie
\question The net force acting on a particle is conservative and increases the kinetic energy by \SI{500}{J}. What is the change in the potential energy, $\Delta U$, and the total energy, $\Delta E$, of the particle?   
\begin{checkboxes}
\choice $\Delta U = \SI{-500}{J}\,|\,\Delta E = \SI{500}{J}$
\CorrectChoice $\Delta U = \SI{-500}{J}\,|\,\Delta E = 0$ \correct
\choice $\Delta U = \SI{-250}{J}\,|\,\Delta E = 0$
\choice Not enough information
\end{checkboxes}

%from Stephanie
\question A \SI{500}{g} squirrel climbs up a tree \SI{2.3}{m} high. What is its change in gravitational potential energy?
\begin{checkboxes}
\choice \SI{11270}{J} 
\choice \SI{9.8}{J}
\CorrectChoice \SI{11.3}{J} \correct
\choice \SI{112.7}{J}
\end{checkboxes}

%Quentin Sanders
\question A child is sliding down a frictionless water slide. The slope is very small at the top of the slide, then the slide gets steeper and steeper towards the middle and finally begins to level out again near the bottom. At what point on the slide will the child be moving the fastest?
\begin{checkboxes}
\choice at the top of the slide
\choice at the steepest part of the waterslide, in the middle
\CorrectChoice  at the bottom of the slide \correct
\choice the child moves at a constant speed
\end{checkboxes}	

%Talia Castillo
\question  Can a particle's mechanical energy equal its potential energy?
\begin{checkboxes}
\choice No, because these are two different types of energy
\CorrectChoice Yes, as long as it has no kinetic energy \correct
\choice  Yes, as long as it has no gravitational energy
\choice No, because mechanical energy has different dimensions than potential energy
\end{checkboxes}	


\question A particle of mass $m$ is constrained to move along the $x$ axis. The only net force acting on the particle is conservative and can be modelled by a potential energy function $U(x)=\frac{1}{4}k_3x^4-\frac{1}{2}k_1x^2$, where $x$ is the particle's position expressed in metres, and $k_3$ and $k_1$ are positive numbers. Which statement is correct?
\begin{checkboxes} 
\choice The force is zero when $x=\frac{k_1}{k_3}$ and in the positive $x$ direction when $x>\frac{k_1}{k_3}$
\choice The force is zero when $x=\frac{k_1}{k_3}$ and in the positive $x$ direction when $x<\frac{k_1}{k_3}$
\choice The force is zero when $x=\sqrt{\frac{k_1}{k_3}}$ and in the positive $x$ direction when $x>\sqrt{\frac{k_1}{k_3}}$
\CorrectChoice The force is zero when $x=\sqrt{\frac{k_1}{k_3}}$ and in the positive $x$ direction when $x<\sqrt{\frac{k_1}{k_3}}$ \correct
\end{checkboxes}

\question Is the force of drag on an accelerating car conservative?
\begin{checkboxes}
\choice Yes
\CorrectChoice No
\end{checkboxes}

\question As you walk on a windy beach, you find that the force exerted on you by the wind is constant in magnitude and direction. Is this force conservative?
\begin{checkboxes}
\CorrectChoice Yes
\choice No
\end{checkboxes}

\question The work done by a non-conservative force on an object that moves from position A to position B
\begin{checkboxes}
\CorrectChoice depends on the path taken between A and B
\choice does not depend on the path taken between A and B
\end{checkboxes}



%%%%%%%%%%%%%%%%%%%%%%%%%%%%%%%%%%%
%
% long answer
%
%%%%%%%%%%%%%%%%%%%%%%%%%%%%%%%%%%%
\subsection{Long answers}
%from Stephanie
\question A \SI{140}{g} baseball is dropped from a tree \SI{15.0}{m} above the ground. 
\begin{parts}
\part With what speed would it hit the ground if air resistance could be ignored? 
\part If it actually hits the ground with a speed of \SI{7.5}{m/s}, what is the average value of the drag force that was exerted on the ball by the air?
\end{parts}

\begin{finalanswer}
\begin{enumerate}[(a)]
\item \SI{17.1}{m/s}
\item \SI{1.11}{N}
\end{enumerate}
\end{finalanswer}
\begin{solution}
\textbf{a)}
In the absence of friction/drag, no energy is dissipated from the system (consisting of the baseball). Since we can model the force of gravity using a potential energy, there are no (other) external forces doing work on the system. Thus, the change in mechanical energy will be zero. We will model this as a one dimension system along the $y$ axis, with positive $y$ upwards, and $y=0$ at the ground.

Initially, the ball is at rest and so has no kinetic energy. The initial mechanical energy of the system is the gravitational potential energy of the ball, a height $y_1$ above the ground:
\begin{align*}
E_1 = mgy_1
\end{align*}
where $y_1=\SI{15.0}{m}$. When the ball lands (if the ground is at $y_2=\SI{0}{m}$, it has both potential and kinetic energy terms in its mechanical energy:
\begin{align*}
E_2 = mgy_2+\frac{1}{2}mv_2^2
\end{align*}
Since there are no dissipative forces, $W=E_2-E_1=0\to E_1=E_2$:
\begin{align*}
E_1 &= E_2\\
mgy_1 &= mgy_2+\frac{1}{2}mv_2^2\\
\therefore v_2&=\sqrt{2g(y_1-y_2)}=\sqrt{2(\SI{9.8}{m/s^2})(\SI{15.0}{m})}=\SI{17.1}{m/s}
\end{align*}

\textbf{b)} In this case, we have a non-conservative force, $F_d$, (the force of drag) that performs (negative) work on the system,  $W_d=F_d(y_2-y_1)$. From conservation of energy, we have:
\begin{align*}
W_d &= E_2-E_1\\
F_d(y_2-y_1) &=mgy_2+\frac{1}{2}mv_2^2 - mgy_1\\
\therefore F_d &= \frac{1}{(y_2-y_1)}\left( mg(y_2-y_1)+\frac{1}{2}mv_2^2  \right)\\
&=\frac{1}{(\SI{-15.0}{m})}\left((\SI{0.14}{kg})(\SI{9.8}{m/s^2})(\SI{-15.0}{m})+\frac{1}{2}(\SI{0.14}{kg})(\SI{7.5}{m/s})^2\right)\\
&=\SI{1.11}{N}
\end{align*}
The force is positive, indicating that it is upwards, as expected.
\end{solution}

%from Stephanie
\question A \SI{4.0}{kg} block slides along a horizontal surface with a coefficient of kinetic friction $\mu_k\,=\,0.30$. The block has a speed $v=\SI{3.0}{m/s}$ when it strikes a massless spring head-on. 
\begin{parts}
\part If the spring has force constant $k\,=\,\SI{200}{N/m}$, what is the maximum compression of the spring? 
\part What minimum value of the coefficient of static friction, $\mu_s$, will assure that the spring remains compressed at the maximum compressed position? 
\part If $\mu_s$ is less than this, what is the speed of the block when it detaches from the decompressing spring?
 \end{parts}
\textit{Hint: Detachment occurs when the spring reaches its uncompressed length}

\begin{finalanswer}
\begin{enumerate}[(a)]
\item \SI{0.3695}{m}
\item 1.8852
\item \SI{2.158}{m/s}
\end{enumerate}
\end{finalanswer}
\begin{solution}
\textbf{a)} Only the spring restoring force and friction have components that are exerted in the direction of motion, so they are the only forces doing work. Since the spring is a conservative force, we can model its effect using potential energy. The force of friction is dissipative and does (negative) work on the system. 

We can define a coordinate system, $x$, with the origin at the uncompressed position of the spring, and the positive direction in the direction of the motion of the mass. At $t=0$, the mass is at $x=0$ with $v_1=\SI{3.0}{m/s}$ (in the positive direction). The initial mechanical energy of the system is:
\begin{align*}
E_1=\frac{1}{2}mv_1^2+\frac{1}{2}kx_1^2=\frac{1}{2}mv_1^2
\end{align*}
where $x_1=\SI{0}{m}$. When the spring is maximally compressed, at position $x_2$, the mass is at rest and the mechanical energy is given by the potential energy stored in the spring:
\begin{align*}
E_2=\frac{1}{2}mv_2^2+\frac{1}{2}kx_2^2=\frac{1}{2}kx_2^2
\end{align*}
where $v_2=\SI{0}{m/s}$. In the vertical direction, only gravity and the normal force act, so the normal force has the same magnitude as the weight of the mass. The force of friction while the mass is moving along positive $x$ is $\vec f_k=-\mu_kN\hat i=-\mu_kmg\hat i$, and the work done by the force of friction is:
\begin{align*}
W_f=-\mu_kmg(x_2-x_1)=-\mu_kmgx_2
\end{align*}
By conservation of energy, we have:
\begin{align*}
W_f &= E_2 - E_1\\
-\mu_kmgx_2 &= \frac{1}{2}kx_2^2-\frac{1}{2}mv_1^2\\
\frac{1}{2}kx_2^2+\mu_kmgx_2-\frac{1}{2}mv_1^2&=0
\end{align*}
which is a quadratic equation for $x_2$, with solutions:
\begin{align*}
x_2&=\frac{-mg\mu_k\pm\sqrt{(mg\mu_k)^2-4(\frac{1}{2}k)(-\frac{1}{2}mv^2_1)}}{2(\frac{1}{2}k)}=\frac{-mg\mu_k\pm\sqrt{(mg\mu_k)^2+kmv^{2}_1}}{k}\\
&=\frac{mg\mu_k}{k}\Bigg(-1\pm\sqrt{1+\frac{kmv^{2}_1}{(mg\mu_k)^2}}\Bigg)\\
&=\frac{(\SI{4.0}{kg})(\SI{9.80}{m/s^2})(0.3)}{\SI{200}{N/m}}\Bigg(-1\pm\sqrt{1+\frac{(\SI{200}{N/m})(\SI{4.0}{kg})(\SI{3.0}{m/s})^2}{(\SI{4.0}{kg})^2(\SI{9.8}{m/s^2})^2(0.30)^2}}\Bigg)\\
&=\SI{0.3695}{m}
\end{align*}
where we have chosen the positive solution, as the mass was moving in the positive direction.

\textbf{b)}When the spring is maximally compressed, the mass is momentarily at rest, and it will stay at rest if the magnitude of the restoring force of the spring ($F=kx_2$) is less than the magnitude of the force of static friction, $f_s=\mu_sN=\mu_smg$:
\begin{align*}
F &\leq f_s\\
kx_2 &\leq \mu_smg\\
\therefore \mu_s &\geq \frac{kx_2}{mg}=\frac{(\SI{200}{N/m})(\SI{0.3695}{m})}{(\SI{4.0}{kg})(\SI{9.80}{m/s^2})}=1.8852
\end{align*}

\textbf{c)}In this case, we evaluate the final mechanical energy at the point where the mass leaves the spring (at $x_3=0$, but with velocity, $v_3$, in the negative $x$ direction):
\begin{align*}
E_3=\frac{1}{2}mv_3^2
\end{align*}
During the phase where the mass compressed the spring and was then pushed back by the spring, the force of kinetic friction did negative work given by:
\begin{align*}
W_f = -\mu_kmg(x_2-x_1)+\mu_kmg(x_3-x_2)=-2\mu_kmgx_2
\end{align*}
since the force is always in the opposite direction of the displacement. By conservation of energy:
\begin{align*}
W_f &= E_3 - E_1\\
-2\mu_kmgx_2 &= \frac{1}{2}mv_3^2 - \frac{1}{2}mv_1^2\\
4\mu_kgx_2&=v_3^2-v_1^2\\
\therefore v_3 &= \sqrt{v_1^2-4\mu_kgx_2}\\
&=\sqrt{(\SI{3.0}{m/s})^2-4(0.3)(\SI{9.80}{m/s^2})(\SI{0.3695}{m})}\\
&=\SI{2.158}{m/s}
\end{align*}
\end{solution}

\question A small mass $m$ starts at rest the top of a sphere, and slides down the frictionless surface of the sphere. At what angle $\theta$ will the mass fall off of the sphere (see Figure \ref{fig:energyconservation:massBall})?
\capfig{0.2\textwidth}{figures/EnergyConservation/massBall.png}{\label{fig:energyconservation:massBall}A small mass sliding down a frictionless sphere.}

\begin{finalanswer}
\SI{48.19}{\degree}
\end{finalanswer}
\begin{solution}
The mass will fall off, when its normal force is zero. A free body diagram for the mass is shown in Figure \ref{fig:energyconservation:massBall_FBD}.
\capfig{0.1\textwidth}{figures/EnergyConservation/massBall_FBD.png}{\label{fig:energyconservation:massBall_FBD} Free body diagram for small mass sliding down a frictionless sphere.}
Until it falls off the sphere, the mass is moving around in a circle, so sum of the forces towards the centre of the sphere must equal $mv^2/R$:
\begin{align*}
mg\cos\theta-N=m\frac{v^2}{R}
\end{align*}
when the normal force is zero, this means that:
\begin{align}
\label{eq:massBallCondition}
g\cos\theta=\frac{v^2}{R}
\end{align}
We can use conservation of energy to find how $v$ depends on $\theta$. At some angle $\theta$, the mass will have changed height by:
\begin{align*}
\Delta h=R(1-\cos\theta)
\end{align*}
Using conservation of energy, the change in gravitational potential energy must equal the change in kinetic energy:
\begin{align*}
mg\Delta h &= \frac{1}{2}mv^2\\
\therefore v^2 &= 2g\Delta h=2gR(1-\cos\theta)
\end{align*}
Substituting this back into equation \ref{eq:massBallCondition}:
\begin{align*}
g\cos\theta&=\frac{v^2}{R} \\
g\cos\theta&=2g(1-\cos\theta)\\
\cos\theta&=\frac{2}{3}\\
\therefore \theta &= \cos^{-1}\left( \frac{2}{3} \right)=\SI{48.19}{\degree}
\end{align*}
independent of the mass, and of the radius of the sphere.
\end{solution}


\question We can model protons inside of a nucleus as being repelled by an electrostatic force, while being attracted by a nuclear force. If we place one proton at the origin of a coordinate system ($r=0$), we can model the potential energy of a second proton, a distance $r$ from the first proton. The potential energy from the electric force can be written as:
\begin{align*}
U^E(r) = \frac{k}{r}
\end{align*}
where $k=\SI{1.0e-28}{Jm}$. The potential energy from the nuclear force can be modelled as:
\begin{align*}
U^N(r) = -U_0\frac{r_0}{r}e^{-\frac{r}{r_0}}
\end{align*}
where $U_0=\SI{1e-13}{J}$, and $r_0=\SI{1.5e-15}{m}$
\begin{parts}
\part Write an expression for the total force on one proton as a function of the distance between protons
\part Show that the nuclear force is attractive, whereas the electric force is repulsive
\part Make a plot of the two potential energy functions, and their sum, between $r=\SI{0.5e-15}{m}$ and $r=\SI{5e-15}{m}$.
\part Is there an equilibrium point where the proton feels no net force from the proton at the origin? If yes, what is the corresponding distance between protons and is it a stable equilibrium? (Note that you may need to find this point numerically, rather than analytically).
\end{parts}
\textbf{Note that this is a very poor model of a nucleus, as you will see.}

\begin{finalanswer}
\begin{enumerate}[(a)]
\item $\vec F(r)=\left[\frac{k}{r^2}-U_0\frac{r_0}{r}e^{-\frac{r}{r_0}}\left(\frac{1}{r} + \frac{1}{r_0}\right) \right] \hat r$
\item From above, we can see that the electric force points away from the origin (positive $\hat r$) and is thus repulsive, whereas the nuclear force points towards the origin (negative $\hat r$) and is attractive.
\item \capfig{0.75\textwidth}{figures/EnergyConservation/NuclearPotential.png}{\label{fig:energyconservation:NuclearPotentialfinal} The electric, nuclear and total potential energy of a proton as a function of distance from another proton.}
\item The force is close to zero at $r = 1.78 \times 10^{-15} m$
\end{enumerate}
\end{finalanswer}
\begin{solution}
\textbf{a)} We find the force for each potential, and then add them together to get the total force:
\begin{align*}
\vec F^E(r)&=-\frac{d}{dr}U^E(r)\hat r=\frac{k}{r^2}\hat r
\end{align*}
\begin{align*}
\vec F^N(r)&=-\frac{d}{dr}U^N(r)\hat r=-\frac{d}{dr}\left(-U_0\frac{r_0}{r}e^{-\frac{r}{r_0}}\right)\hat r\\
&=U_0\left( -\frac{r_0}{r^2}e^{-\frac{r}{r_0}}- \frac{1}{r}e^{-\frac{r}{r_0}}\right)\hat r\\
&=-U_0e^{-\frac{r}{r_0}}\left(\frac{r_0}{r^2} + \frac{1}{r}\right)\hat r\\
&=-U_0\frac{r_0}{r}e^{-\frac{r}{r_0}}\left(\frac{1}{r} + \frac{1}{r_0}\right)\hat r\\
\end{align*}
The total force is thus:
\begin{align*}
\vec F(r)&=\vec F^E(r)+\vec F^N(r)\\
&=\left[\frac{k}{r^2}-U_0\frac{r_0}{r}e^{-\frac{r}{r_0}}\left(\frac{1}{r} + \frac{1}{r_0}\right) \right] \hat r
\end{align*}
\textbf{b)} From above, we can see that the electric force points away from the origin (positive $\hat r$) and is thus repulsive, whereas the nuclear force points towards the origin (negative $\hat r$) and is attractive.

\textbf{c)} To plot this in python, we can do:
\begin{verbatim}
import pylab as pl
import numpy as np

#some constants
U0=1e-13
r0=1.5e-15
k=1e-28

#The nuclear potential energy
def Y(r):
    return -U0*r0/r*np.exp(-r/r0)

#The electric potential energy
def C(r):
    return k/r
 
#Make a plot:
rvals = np.linspace(0.5e-15,5e-15,1000)
Yvals = Y(rvals)
Cvals = C(rvals)

pl.figure(figsize=(10,8))
pl.plot(rvals,Yvals,label='nuclear')
pl.plot(rvals,Cvals,label='electric')
pl.plot(rvals,Yvals+Cvals,label='combined')
pl.xlabel('distance between protons [m]', fontsize=14)
pl.ylabel('potential energy [J]', fontsize=14)
pl.legend()
pl.grid()
pl.show()
\end{verbatim}


\capfig{0.75\textwidth}{figures/EnergyConservation/NuclearPotential.png}{\label{fig:energyconservation:NuclearPotential} The electric, nuclear and total potential energy of a proton as a function of distance from another proton.}

\textbf{d)} The condition for the force to be zero is:
\begin{align*}
\frac{k}{r^2}-U_0\frac{r_0}{r}e^{-\frac{r}{r_0}}\left(\frac{1}{r} + \frac{1}{r_0}\right) &=0
\end{align*}
and we can see that this happens on the plot, corresponding to the point where the total potential energy has a slope of zero. We can find this point numerically in python:
\begin{verbatim}
#Define the total force:
def F(r):
    return k/r**2-U0*r0/r*np.exp(-r/r0)*(1/r+1/r0)

#Evaluate for all point in r:
Fvals = F(rvals)

#Find the smallest value:
minF=10e15
imin=0
for i in range(Fvals.size):
    if abs(Fvals[i])<minF:
        imin=i
        minF=abs(Fvals[i])
        
print("The force is close to zero at\
r = {:.2f} x 10^(-15) m".format(rvals[imin]*1e15))

The force is close to zero at r = 1.78 x 10^(-15) m
\end{verbatim}
This is an unstable equilibrium.
\end{solution}



\question The top floor of your neighbour's house is on fire, and you offer to help the firemen by using the water from your fancy swimming pool to put out the fire. You happen to have an old pump in your garage, but the label indicating its electrical power consumption is illegible, so you're not sure if the pump is powerful enough to pump the water to the required height of \SI{20}{m}, through the \SI{2.0}{cm} diameter of the fire hose. If the pump is 50\% efficient in converting electrical power to kinetic energy of the water being pumped, what electrical power must the pump be rated for in order to help your neighbour?

\textit{Assume that the temperature of the water is \SI{0}{\degree C} and that the density of water is $\rho=\SI{1}{g/cm^3}$. Also assume that there are no guanacos in the burning house.}

\begin{finalanswer}
\SI{2438.25}{W}
\end{finalanswer}
\begin{solution}
We need to determine the power required to pump a continuous cylinder of water with a radius of $r=\SI{0.01}{m}$ to a height of $h=\SI{20}{m}$. Let the speed of the water coming out of the nozzle be $v$.

Suppose that we look at the mass, $\Delta m$, of water that comes out of the nozzle over a period of time $\Delta t$. We can think of a cylinder of water coming out of the nozzle over that period of time. The water will travel a distance $v\Delta t$, so the length of the cylinder of water will be $L=v\Delta t$. The cross section of the cylinder is the cross section of the fire hose, $A=\pi r^2$. The mass of water will be:
\begin{align*}
\Delta m = \rho AL=\rho \pi r^2 v\Delta t
\end{align*}
By conservation of energy, a mass element, $\Delta m$, will need a kinetic energy of at least $\Delta m gh$, in order to reach a height $h$. 
\begin{align*}
\frac{1}{2}\Delta mv^2 &= \Delta m gh \\
\therefore v&=\sqrt{2gh}
\end{align*}
In a period of time $\Delta t$, the pump must provide enough kinetic energy, $\Delta K$, for the mass $\Delta m$ of water to reach the height, $h$ (assuming you aim the hose vertically):
\begin{align*}
\Delta K&=\frac{1}{2}\Delta m v^2 = \frac{1}{2} (\rho \pi r^2 v\Delta t) v^2\\
&=\frac{1}{2} \rho \pi r^2 \Delta t v^3\\
&=\frac{1}{2} \rho \pi r^2 \Delta t (2gh)^{\frac{3}{2}}
\end{align*}
Thus, the rate at which energy needs to be provided to the water by the pump (the power) is:
\begin{align*}
P^W&=\frac{\Delta K}{\Delta t}=\frac{1}{2} \rho \pi r^2 (2gh)^{\frac{3}{2}}
\end{align*}
Since the pump is only 50\% efficient at converting electrical energy into kinetic energy of the water, the electrical power requirement is:
\begin{align*}
P^E&=2P^W=\pi\rho  r^2 (2gh)^{\frac{3}{2}}\\
&=\pi(\SI{1000}{kg/m^3})(\SI{0.01}{m})^2[2(\SI{9.80}{m/s^2})(\SI{20}{m})]^{\frac{3}{2}}\\
&=\SI{2438.25}{W}
\end{align*}

\end{solution}


\question In the ``advanced'' approach to classical physics, we use only scalar quantities to determine the motion of a particle, instead of Newton's Second Law, which relies on vectors. For a particle of mass, $m$, moving in one dimension, we can define the Lagrangian, $L$, given by the difference between the kinetic energy of a particle, $K$, and its potential energy, $U$:
\begin{align*}
L(x,v)=K-U
\end{align*}
where, in general, the Lagrangian depends on the velocity of the particle ($v$, through the kinetic energy), and its position ($x$, through the potential energy). 
The equation of motion for a particle in one dimension ($x$) is given by the Euler-Lagrange equation:
\begin{align*}
\frac{d}{dt}\left(\die{L}{v}\right)-\die{L}{x}=0
\end{align*}
Show that the Euler-Lagrange equation is equivalent to Newton's Second Law in the case of a particle of mass $m$ in vertical free fall near the surface of the Earth. 

\textit{Hint: Write the kinetic and potential energies for the particle at some point in time, when it has a speed $v$ and is at a position $x$ above the ground. This will give you the Lagrangian, to which you can apply the Euler-Lagrange equation.}

\begin{solution}
The kinetic and potential energies, at some point in time are given by:
\begin{align*}
K &=\frac{1}{2}mv^2\\
U &=mgx
\end{align*}
where $v$ is the speed of the particle, and $x$ is its distance above the ground (where we define zero potential energy). The Lagrangian is thus:
\begin{align*}
L&=K-U=\frac{1}{2}mv^2-mgx
\end{align*}
Applying the Euler-Lagrange equation, we first take the two partial derivatives:
\begin{align*}
\die{L}{v} &= mv\\
\die{L}{x} &= mg
\end{align*}
and then the time derivative of the $\die{L}{v}$ term:
\begin{align*}
\frac{d}{dt}\left(\die{L}{v}\right) =\frac{d}{dt}\left(mv\right)=m\frac{dv}{dt}=ma
\end{align*}
Putting this into the Euler-Lagrange equation:
\begin{align*}
\frac{d}{dt}\left(\die{L}{v}\right)-\die{L}{x}&=0\\
ma - mg &=0\\
\therefore ma&=mg
\end{align*}
exactly as one would find from Newton's Second Law.
\end{solution}

%Zaremba 2006 104 Final
\question A child has a mass $M$ and a height $h$. They stand on a trampoline which, under their weight, compresses by a distance, $d$, equal to one tenth of their height (their feet are a distance $d$ below where the trampoline would rest if nobody stood on it), as shown in the left panel of Figure \ref{fig:energyconservation:Trampoline}. You may treat the trampoline as a spring that obeys Hooke's Law ($F(x)=-kx$).

By how much should the trampoline be compressed (with respect to its rest position, $x$ on the middle panel figure), so that it would launch the child a distance $h$ above the trampoline's rest position (your feet end up a distance $h$ above the rest position of the trampoline, as shown in the right panel)?

Give your answer in terms of $h$.
\capfig{0.7\textwidth}{figures/EnergyConservation/Trampoline.png}{\label{fig:energyconservation:Trampoline} Determine the compression $x$ to launch you at a height $h$.}

\begin{finalanswer}
$x = \frac{h}{10}(1 + \sqrt{21})$
\end{finalanswer}
\begin{solution}
Since the spring compresses by a distance $d=\frac{h}{10}$ under the weight $Mg$, we can determine the spring constant from Hooke's law:
\begin{align*}
Mg & = kd=k\frac{h}{10} \\
\therefore k &= \frac{10Mg}{h}
\end{align*}
We choose the uncompressed location of the trampoline as our zero for gravitational and spring potential energy. If the spring is compressed by an unknown distance $x$, then our total mechanical energy at the bottom is:
\begin{align*}
E_1 = -Mgx+\frac{1}{2}kx^2
\end{align*}
The energy at the height of the jump is only gravitational potential energy:
\begin{align*}
E_2 = Mgh
\end{align*}
These two are equal, so we can solve for x:
\begin{align*}
-Mgx+\frac{1}{2}kx^2 &= Mgh\\
\frac{1}{2}kx^2-Mgx-Mgh &=0
\end{align*}
This is a quadratic, with solution:
\begin{align*}
x &= \frac{Mg \pm \sqrt{(Mg)^2+4\frac{1}{2}kMgh}}{2\frac{1}{2}k}\\
&=\frac{Mg \pm \sqrt{(Mg)^2+2kMgh}}{k}\\
&=\frac{Mg \pm \sqrt{(Mg)^2+2\frac{10Mg}{h}Mgh}}{k}\\
&=\frac{Mg \pm \sqrt{(Mg)^2+20(Mg)^2}}{k}\\
&=\frac{Mg \pm \sqrt{21(Mg)^2}}{k}\\
&=\frac{Mg}{\frac{10Mg}{h}}(1 \pm \sqrt{21})\\
&=\frac{h}{10}(1 \pm \sqrt{21})\\
\end{align*}
where we choose the positive root:
\begin{align*}
x = \frac{h}{10}(1 + \sqrt{21})
\end{align*}
\end{solution} 

%Olivia W
\question A non-linear spring exerts a restoring force given by $\vec F(x)=(-k_1x-k_3x^3)\hat x$.
\begin{parts}
\part Show that the force is conservative.
\part Find the potential energy function for this spring.
\part Check your answer for part b) by recovering the force function from the potential energy function.
\end{parts}

\begin{finalanswer}
(b) $U(x) = \frac{1}{2}k_1x^2 + \frac{1}{4}k_3x^4 + C$
\end{finalanswer}
\begin{solution}
\begin{parts}
\part Since the force depends on the position, it could be conservative. We can check this with the following conditions:
\begin{align*}
\die{F_z}{y}-\die{F_y}{z}&=0\\
\die{F_x}{z}-\die{F_z}{x}&=0\\
\die{F_y}{x}-\die{F_x}{y}&=0
\end{align*}
The function has only an $x$ component, so we get:
\begin{align*}
\die{F_z}{y}-\die{F_y}{z}&=0-0&=0\\
\die{F_x}{z}-\die{F_z}{x}&=\die{}{z}(-k_1x-k_3x^3)-0&=0\\
\die{F_y}{x}-\die{F_x}{y}&=0-\die{}{y}(-k_1x-k_3x^3)&=0
\end{align*}
All the conditions are zero, so the force is indeed conservative. 
\part To determine the potential energy function, let's calculate the work done by the spring from position $x_A$ to position $x_B$.
\begin{align*}
W &=\int_A^B \vec F(\vec r) \cdot d\vec l\\
&=\int_{x_A}^{x_B} (-k_1x-k_3x^3)\hat x \cdot dx \hat x\\
&=\int_{x_A}^{x_B} (-k_1x-k_3x^3)dx=\left[-\frac{1}{2}k_1x^2-\frac{1}{4}k_3x^4 \right]_{x_A}^{x_B}\\
&=-\left( \frac{1}{2}k_1(x_B^2-x_A^2)+\frac{1}{4}k_3(x_B^4-x_A^4) \right)
\end{align*}
Comparing with:
\begin{align*}
-W &= U(\vec r_B) - U(\vec r_A) = U(x_B) - U(x_A)
\end{align*}
We can identify the potential energy for this non-linear spring:
\begin{align*}
U(x) = \frac{1}{2}k_1x^2 + \frac{1}{4}k_3x^4 + C
\end{align*}
\part We can recover the force function by taking the derivative:
\begin{align*}
F(x)&=\frac{d}{dx}U_x\\
&=-\frac{d}{dx}\left(\frac{1}{2}k_1x^2+\frac{1}{4}k_3x^4\right )+C\\
F(x)&=-(k_1x+k_3x^3)\hat x
\end{align*}
So we have checked that our potential energy function is correct. 
\end{parts}
\end{solution}

%Based on Giancolli 8-26
\question You need to design a horizontal spring to stop a car with a mass $m=\SI{1e3}{kg}$ moving with a speed of $v=\SI{100}{km/h}$. The spring is attached to a vertical wall on one end, and the car will collide with the other end of the spring. What should the spring constant, $k$, be if the occupants of the car are to experience an acceleration (deceleration) no greater than $4g$, where $g=\SI{9.8}{m/s^2}$? \textbf{Hint:} Think about the location at which the force from the spring will be the greatest, as that will correspond to the greatest acceleration.
\begin{solution}
We can use conservation of energy to model this. The kinetic energy of the car ($E_A$) must be converted into the potential energy of the spring ($E_B$):
\begin{align*}
E_A&=E_B\\
\frac{1}{2}mv^2&=\frac{1}{2}kx^2\\
mv^2&=kx^2
\end{align*}
where $x$ is the maximal compression of the spring. The maximal acceleration will be felt when the spring is maximally compressed, so we require that the force exerted by the spring when compressed by the distance $x$ results in an acceleration of the car that is $a=-4g$:
\begin{align*}
F&=-kx=ma=-4mg\\
\therefore x&=\frac{4mg}{k}
\end{align*}
We can substitute this into the conservation of energy equation:
\begin{align*}
mv^2&=k\left(\frac{4mg}{k}\right)^2\\
v^2&=\frac{16mg^2}{k}\\
\therefore k&=\frac{16mg^2}{v^2}=\frac{16(\SI{1e3}{kg})(\SI{9.8}{m/s^2})^2}{(\SI{27.78}{m/s})^2}=\SI{55314.6}{N/m}
\end{align*}
\end{solution}

\question The ``Les Attelas'' chairlift in the Swiss ski resort of Verbier can carry \num{2400} people per hour from an altitude of $\SI{2227}{m}$ to an altitude of $\SI{2722}{m}$. Estimate the total electrical power required by the chairlift if it is 50\% efficient at converting electrical power into mechanical work. \textbf{Hint:} You are not given enough data, you will have to estimate (and justify) values for any missing data.
\begin{solution}
The lift must carry $N=\num{2400}$ people per hour over a change in height, $h$:
\begin{align*}
h=(\SI{2722}{m})-(\SI{2227}{m})=\SI{495}{m}
\end{align*}
In one hour, this corresponds to a total work that must be done by the chairlift given by:
\begin{align*}
W=Nmgh
\end{align*}
where $m$ is the mass of a skier, which is not given. We can estimate that an average skier weighs $m=\SI{75}{kg}$. The power that must be exerted by the chairlift is thus:
\begin{align*}
P=\frac{W}{(\SI{1}{hour})}=\frac{(\num{2400})(\SI{75}{kg})(\SI{9.8}{m/s^2})(\SI{495}{m})}{(\SI{3600}{s})}=\SI{242.5}{kW}
\end{align*}
Since the chairlift is only 50\% efficient, it will consume twice as much electrical power, or $P=\SI{485.1}{kW}$
\end{solution}

\question In the ``advanced'' approach to classical physics, we use only scalar quantities to determine the motion of a particle, instead of Newton's Second Law, which relies on vectors. For a particle of mass, $m$, moving in one dimension, we can define the Lagrangian, $L$, given by the difference between the kinetic energy of a particle, $K$, and its potential energy, $U$:
\begin{align*}
L(x,v)=K-U
\end{align*}
where, in general, the Lagrangian depends on the velocity of the particle ($v$, through the kinetic energy), and its position ($x$, through the potential energy). 
The equation of motion for a particle in one dimension ($x$) is given by the Euler-Lagrange equation:
\begin{align*}
\frac{d}{dt}\left(\die{L}{v}\right)-\die{L}{x}=0
\end{align*}
Show that the Euler-Lagrange equation is equivalent to Newton's Second Law in the case of a particle of mass $m$ attached to a horizontal spring with spring constant $k$. The mass can move on a frictionless horizontal surface.

\textit{Hint: Write the kinetic and potential energies for the particle at some point in time, when it has a speed $v$ and is at a position $x$ relative to the rest position of the spring. This will give you the Lagrangian, to which you can apply the Euler-Lagrange equation.}
\begin{solution}
The kinetic and potential energies, at some point in time are given by:
\begin{align*}
K &=\frac{1}{2}mv^2\\
U &=\frac{1}{2}kx^2
\end{align*}
where $v$ is the speed of the particle, and $x$ is its distance from the rest position of the spring (where we define zero potential energy). The Lagrangian is thus:
\begin{align*}
L&=K-U=\frac{1}{2}mv^2-\frac{1}{2}kx^2
\end{align*}
Applying the Euler-Lagrange equation, we first take the two partial derivatives:
\begin{align*}
\die{L}{v} &= mv\\
\die{L}{x} &= -kx
\end{align*}
and then the time derivative of the $\die{L}{v}$ term:
\begin{align*}
\frac{d}{dt}\left(\die{L}{v}\right) =\frac{d}{dt}\left(mv\right)=m\frac{dv}{dt}=ma
\end{align*}
Putting this into the Euler-Lagrange equation:
\begin{align*}
\frac{d}{dt}\left(\die{L}{v}\right)-\die{L}{x}&=0\\
ma + kx &=0\\
\therefore ma&=-kx
\end{align*}
exactly as one would find from Newton's Second Law.
\end{solution}


\question A mass-less spring with a linear restoring force has a spring constant $k=\SI{100}{N/m}$ and a rest length of $L=\SI{1.0}{m}$. The spring is placed at the bottom of an incline that makes an angle $\theta=\SI{30}{\degree}$ with respect to the horizontal, as shown in Figure \ref{fig:energyconservation:SpringIncline}. A block of mass $m=\SI{0.5}{kg}$ is then placed on the spring, and the spring is compressed by a distance $d=\SI{0.5}{m}$ (so that the block is a distance $L-d$ from the bottom of the incline). The block is then released and slides up the incline to a distance $D$ from the bottom before sliding back down. The coefficient of kinetic friction between the block and the incline is $\mu_k=\num{0.3}$. 

\begin{parts}
	\part[2] What will be the speed of the block when it leaves the spring?
	\part[2] At what distance, $D$, measured from the bottom of the incline, will the block stop?
	\part[2] The amount of thermal energy, $\Delta E$, that is required to heat a block of mass $m$ so that it changes temperature by an amount $\Delta T$ (in degrees Kelvin) is given by:
	\begin{align*}
	\Delta E=mC\Delta T
	\end{align*}
	where $C$ is called the heat capacity of the block and depends on the material with which the block is made. Instead of modelling friction as a non-conservative force, we can model it as a change in the thermal energy of the block, if we account for this type of energy. We will assume that all of the thermal energy from friction will heat up the block (rather than the incline).
	
	Again, we launch the block by placing it on the spring so that the spring is compressed by a distance $d=\SI{0.5}{m}$. If the block has a heat capacity of $C=\SI{897}{J/kg/\degree K}$ (that for aluminium) and stops at a distance $D=\SI{5}{m}$ from the bottom of the incline, what will be its change in temperature? Comment on whether this would be a practical experiment to perform. 
	
\end{parts}

\capfig{0.45\textwidth}{figures/EnergyConservation/SpringIncline.png}{\label{fig:energyconservation:SpringIncline}A block is placed on spring of rest length $L$, which is then compressed and released so that the block moves up the incline.}

\begin{solution}
	\begin{parts}
		\part We use conservation of energy to model the motion of the block. We choose gravitational potential to be zero  at the position when the spring is compressed (when the block is a distance $L-d$ from the bottom of the incline).  We choose the spring potential energy to be zero when the block is a distance $L$ from the bottom of the incline (the rest position of the spring). Finally, we choose an $x$ axis to be parallel to the incline (increasing $x$ upwards), with the origin located a distance $L-d$ from the bottom of the incline. The block will leave the spring at position $x=d$.
		
		When the spring is compressed, the mechanical energy of the block is given by:
		\begin{align*}
		E_A=\frac{1}{2}kd^2
		\end{align*}
		When the block leaves the spring, the mechanical energy of the block is:
		\begin{align*}
		E_B=\frac{1}{2}mv^2+mgd\sin\theta
		\end{align*}
		The force of kinetic friction will do work given by:
		\begin{align*}
		W_f=-\mu_kNd=-\mu_kmg\cos\theta d
		\end{align*}
		where $N=mg\cos\theta$ is the normal force applied by the incline. Applying conservation of energy:
		\begin{align*}
		W^{NC}&=\Delta E=E_B-E_A\\
		-\mu_kmg\cos\theta d &= \frac{1}{2}mv^2+mgd\sin\theta-\frac{1}{2}kd^2\\
		\frac{1}{2}mv^2 &= \frac{1}{2}kd^2 -mgd\sin\theta-\mu_kmg\cos\theta d\\
		\therefore v&=\sqrt{\frac{kd^2 -2mgd\sin\theta-2\mu_kmg\cos\theta d}{m}}\\
		&=\sqrt{\frac{kd^2 -2mgd(\sin\theta+\mu_k\cos\theta)}{m}}\\
		&=\sqrt{\frac{(\SI{100}{N/m})(\SI{0.5}{m})^2 -2(\SI{0.5}{kg})(\SI{9.8}{N/kg})(\SI{0.5}{m})(\sin(\SI{30}{\degree})+(0.3)\cos(\SI{30}{\degree}))}{(\SI{0.5}{kg})}}\\
		&=\SI{6.52}{m/s}
		\end{align*}
		
		\part We use conservation of energy again. Since we already know the speed of the mass when it leaves the spring from part a, we can compare the energy of the mass when it leaves the spring to that at the top of the trajectory. We choose the gravitational potential energy to be zero at the bottom of the incline. When the mass leaves the spring, it has total mechanical energy (gravitational potential energy plus kinetic energy):
		\begin{align*}
		E_B=\frac{1}{2}mv^2+mgL\sin\theta
		\end{align*}
		At the top of the trajectory, the mechanical energy of the block is given by its gravitational potential energy:
		\begin{align*}
		E_C=mgD\sin\theta
		\end{align*}
		The force of kinetic friction will do negative work over a distance $D-L$ given by:
		\begin{align*}
		W_f=-\mu_kN(D-L)=-\mu_kmg\cos\theta (D-L)
		\end{align*}
		Again, applying conservation of energy, we can find $D$:
		\begin{align*}
		W^{NC}&=\Delta E=E_C-E_B\\
		-\mu_kmg\cos\theta (D-L) &= mgD\sin\theta - \frac{1}{2}mv^2-mgL\sin\theta\\
		\frac{1}{2}v^2+gL\sin\theta+\mu_kgL\cos\theta &=gD\sin\theta+\mu_kg\cos\theta D\\
		\therefore D&=\frac{v^2+2gL(\sin\theta+\mu_k\cos\theta)}{2g(\sin\theta+\mu_k\cos\theta)}\\
		&=\frac{v^2}{2g(\sin\theta+\mu_k\cos\theta)}+L\\
		&=\frac{(\SI{6.52}{m/s})^2}{2(\SI{9.8}{N/kg})(\sin(\SI{30}{\degree})+(0.3)\cos(\SI{30}{\degree}))}+(\SI{1.0}{m})\\
		&=\SI{3.855}{m}
		\end{align*}
		
		\part Again, we model the motion using conservation of energy, but instead of accounting for friction as doing non-conservative work, we count is a different type of energy. 
		
		When the spring is compressed, we assume that the block has no thermal energy (we choose thermal energy to be zero at whatever temperature the block has), so that the energy of the block is given by:
		\begin{align*}
		E_A=\frac{1}{2}kd^2
		\end{align*}
		where we chose gravitational potential energy to be zero where the block starts. At the top of the trajectory, (having travelled a distance $D-L+d$), the energy of the block is given by its gravitational potential energy and its thermal energy:
		\begin{align*}
		E_C=mg(D-L+d)\sin\theta+mC\Delta T
		\end{align*}
		By conservation of energy, these must be the same, and we can determine the change in temperature of the block:
		\begin{align*}
		E_A &= E_C\\
		\frac{1}{2}kd^2 &= mg(D-L+d)\sin\theta+mC\Delta T\\
		mC\Delta T &=\frac{1}{2}kd^2-mg(D-L+d)\sin\theta\\
		\Delta T &= \frac{kd^2-2mg(D-L+d)\sin\theta}{2mC}\\
		&= \frac{(\SI{100}{N/m})(\SI{0.5}{m})^2-2(\SI{0.5}{kg})(\SI{9.8}{m/s^2})((\SI{5}{m})-(\SI{1}{m})+(\SI{0.5}{m}))\sin(\SI{30}{\degree})}{2(\SI{0.5}{kg})(\SI{897}{J/kg/\degree K})}\\
		&=\SI{3.289e-3}{\degree K}
		\end{align*}
		It does not heat up by very much, so it would not be very practical to measure this small change in temperature. In reality, the block would heat up less than we calculated, since some of the thermal energy would heat up the incline.
	\end{parts}
\end{solution}

