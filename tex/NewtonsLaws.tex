
\chapter{Newton's Laws}
In this chapter, we introduce Newton's Laws, which is a succinct theory of physics that describes an incredibly large number of phenomena in the natural world. Newton's Laws are one possible formulation of what we call ``Classical Physics'' (as opposed to ``Modern Physics'' which include Quantum Mechanics and Special Relativity). Newton's Laws make the connection between dynamics (the causes of motion) and the kinematics of motion (the description of that motion). 
\label{chap:NewtonsLaws}
 \vspace{1cm}
\begin{learningObjectives}
\item Understand Newton's Three Laws
\item Understand the concept of force and how to identify a force
\item Understand the concepts of mass and inertia
\item Understand free body diagrams
\end{learningObjectives}

\section{Newton's Three Laws}
Newton's classical theory of physics is based on the three following laws:
\begin{itemize}
\item \textbf{Law 1}: An object will remain in its state of motion, be it at rest or moving with constant velocity, unless an external force is exerted on the object.
\item \textbf{Law 2}: An object's acceleration is proportional to the net force exerted \textbf{on the object}, inversely proportional to the mass of the object, and in the same direction as the net force exerted on the object.
\item \textbf{Law 3}: If one object exerts a force on another object, the second object exerts a force on the first object that is equal in magnitude and opposite in direction.
\end{itemize}
The three statements above are sufficient to describe almost all of the natural phenomena that we experience in our lives. Concepts such as energy, centre of mass, and torque, which you may have already encountered, are derived naturally from Newton's Laws. In order to build models to describe specific experiments or observations using Newton's Laws, one needs to understand the two main mathematical concepts that are introduced by the theory: force and mass. The next two sections introduce these concepts, before we describe how to use Newton's Laws.

\section{Force}
A force is a mathematical tool that is introduced in Newton's theory of physics. A force is not a real ``thing''; there are no forces in the real world, you cannot give someone a force, or buy a force at the supermarket. A force is purely a mathematical tool, so it is important to fight your intuition about what a force is and to stick to well-defined rules for using forces to build models.

Mathematically, a \textbf{force is represented by a vector}, and thus has a magnitude and a direction. The SI unit for the magnitude of a force is the Newton, abbreviated $\si{N}$. A force is used to describe how the motion of an object is affected by external agents. It is important to note that a force can be exerted by an inanimate being; that is, there is no intent - no conscious decision to push or pull - that is associated with a force.

When you push a block along a horizontal surface, we would model the motion of the block as being related to a force that you exert on the block in the direction that you are pushing and with a magnitude that is proportional to how hard you are pushing. Newton's third law states that the block will exert a force on you that is of equal magnitude but in the opposite direction; if we want to model \textit{your motion}, we will need to include that force exerted by the block \textit{on you}. 

If you are pulling on a cart, we would model the motion of the cart by including a force that is exerted on the cart by you. The force would be represented by a vector in the direction that you are pulling with a magnitude based on how hard you are pulling. Similarly, to model your motion, we would include a force vector that is equal in magnitude and opposite in direction to represent the force exerted by the cart on you.

One way to quantify a force is to use a spring scale. Springs have a natural ``rest length'' if not acted upon by external forces. If you try to stretch a spring, it will ``want'' to come back to its normal rest length; it exerts a force on your hand in the opposite direction that you pulled on the spring. You may have noticed that the more you stretch a spring, the harder you have to pull on it. We can quantify the magnitude of a force by the distance that the forces causes a spring to stretch, since that distance increases with what we conceptualize as a force. For example, one could designate a ``standard spring'' to be one that extends (or compresses) by $\SI{1}{cm}$ when a force of $\SI{1}{N}$ is exerted on the spring\footnote{In practice, $\SI{1}{N}$ is defined as the force that causes a mass of $\SI{1}{kg}$ to accelerate by $\SI{1}{m/s^2}$}. 

\subsection{Types of forces}



\subsection{The net force}


\section{Mass and inertia}


\section{Applying Newton's Laws}

\section{Free body diagrams}



\newpage
\section{Summary}
\vspace{2cm}
\begin{chapterSummary}
\item Something interesting
\end{chapterSummary}