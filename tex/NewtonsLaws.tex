
\chapter{Newton's Laws}
In this chapter, we introduce Newton's Laws, which is a succinct theory of physics that describes an incredibly large number of phenomena in the natural world. Newton's Laws are one possible formulation of what we call ``Classical Physics'' (as opposed to ``Modern Physics'' which include Quantum Mechanics and Special Relativity). Newton's Laws make the connection between dynamics (the causes of motion) and the kinematics of motion (the description of that motion). 
\label{chap:NewtonsLaws}
 \vspace{1cm}
\begin{learningObjectives}
\item Understand Newton's Three Laws
\item Understand the concept of force and how to identify a force
\item Understand the concepts of mass and inertia
\item Understand free body diagrams
\end{learningObjectives}

\section{Newton's Three Laws}
Newton's classical theory of physics is based on the three following laws:
\begin{itemize}
\item \textbf{Law 1}: An object will remain in its state of motion, be it at rest or moving with constant velocity, unless an external force is exerted on the object.
\item \textbf{Law 2}: An object's acceleration is proportional to the net force exerted \textbf{on the object}, inversely proportional to the mass of the object, and in the same direction as the net force exerted on the object.
\item \textbf{Law 3}: If one object exerts a force on another object, the second object exerts a force on the first object that is equal in magnitude and opposite in direction.
\end{itemize}
The three statements above are sufficient to describe almost all of the natural phenomena that we experience in our lives. Concepts such as energy, centre of mass, torque, etc, which you may have already encountered, are derived naturally from Newton's Laws. In order to build models to describe specific experiments or observations using Newton's Laws, one needs to understand the two main mathematical concepts that are introduced by the theory: force and mass. The next two sections introduce these concepts, before we describe how to use Newton's Laws. 

TODO: Briefly discuss each law


\section{Force}
A force is a mathematical tool that is introduced in Newton's theory of physics. A force is not a real ``thing''; there are no forces in the real world, you cannot give someone a force, or buy a force at the supermarket. A force is a purely mathematical tool, so it is important to fight your intuition about what a force is and to stick to well-defined rules for using forces to build models.

Mathematically, a \textbf{force is represented by a vector}, and thus has a magnitude and a direction. The SI unit for the magnitude of a force is the Newton, abbreviated $\si{N}$. A force is used to describe how the motion of an object is affected by external agents. It is important to note that a force can be exerted by an inanimate being; that is, there is no intent - no conscious decision to push or pull - that is associated with a force.

When you push a block along a horizontal surface, we would model the motion of the block as being related to a force that you exert on the block in the direction that you are pushing and with a magnitude that is proportional to how hard you are pushing. Newton's third law states that the block will exert a force on you that is of equal magnitude but in the opposite direction; if we want to model \textit{your motion}, we will need to include that force exerted by the block \textit{on you}. 

If you are pulling on a cart, we would model the motion of the cart by including a force that is exerted on the cart by you. The force would be represented by a vector in the direction that you are pulling with a magnitude based on how hard you are pulling. Similarly, to model your motion, we would include a force vector that is equal in magnitude and opposite in direction to represent the force exerted by the cart on you. When modelling the motion of an object, it is important to consider only the forces exerted on that object.

One way to quantify a force is to use a spring scale. Springs have a natural ``rest length'' if not acted upon by external forces. If you try to stretch a spring, it will ``want'' to come back to its normal rest length; it exerts a force on your hand in the opposite direction that you pulled on the spring. You may have noticed that the more you stretch a spring, the harder you have to pull on it. We can quantify the magnitude of a force by the distance that the forces causes a spring to stretch, since that distance increases with what we conceptualize as a force. For example, one could designate a ``standard spring'' to be one that extends (or compresses) by $\SI{1}{cm}$ when a force of $\SI{1}{N}$ is exerted on the spring\footnote{In practice, $\SI{1}{N}$ is defined as the force that causes a mass of $\SI{1}{kg}$ to accelerate by $\SI{1}{m/s^2}$}. 

\subsection{Types of forces}
There are several ways for a force to be exerted on an object. Some of the forces can be classified as ``contact forces'' as they arise from something making contact with the object (such as you pushing on the object). Other forces can be exerted ``at a distance''; for example, the force of gravity from the Earth can be exerted on a bird in flight, even if the bird is not in contact with the Earth. In this section, we present the most common types of force that arise when modelling the real world. When determining the forces that are acting on an object, it is usually a good idea to run down this list to see if any of these forces should be included.

\subsubsection{Weight}
Weight is the force exerted by gravity. While all objects with mass exert an attractive force of gravity on other objects, the force is usually negligible unless the mass of one of the objects is large. For an object near the surface of the Earth, we can, to a very good degree of approximation, assume that the only force of gravity on the object is from the Earth. We usually label the force of gravity on an object as $\vec F_g$. All objects near the surface of the Earth will experience a weight, as long as they have a mass. If an object has a mass, $m$, and is located near the surface of the Earth, it will experience a force that:
\begin{itemize}
\item Points towards the centre of the Earth (as illustrated in Figure \ref{fig:newtonslaws:weight}).
\item Has a magnitude of $F_g=mg$, where $g$ is the magnitude of the Earth's gravitational field, and in most locations on Earth\footnote{The value of $g$ decreases as you move further from the center of Earth} has a value of approximately $g=\SI{9.8}{N/kg}$.
\end{itemize}

\capfig{0.1\textwidth}{figures/NewtonsLaws/weight.png}{\label{fig:newtonslaws:weight}The weight force on an object near the surface of the Earth points towards the centre of the Earth (downwards).}
Although we have not yet introduced the concept of mass, it is worth noting that mass and weight are different (they had different dimensions). Mass is an intrinsic property of an object, whereas weight is a force of gravity that is exerted on that object because it has mass. On Earth, when we measure our weight, we are measuring $mg$, which also relates to our mass since, on Earth, weight and mass are related by a factor of $g=\SI{9.8}{N/kg}$; this is usually what leads to the confusion between mass and weight.


\subsubsection{Normal forces}
Normal forces are exerted when two surfaces are in contact and ``pushing'' against each other. For example, if a block is resting on a horizontal table, the table will exert a normal force on the block that is upwards. The force is called ``normal'' because it is normal (i.e. perpendicular) to the interface between the two objects. The normal force from a surface on an object points in the direction from the surface to the object in such as way that it is perpendicular to the interface between the surface and the object. Because of Newton's Third Law, whenever an object experiences a normal force from a surface, the object also exerts a force of the same magnitude (in the opposite direction) on the surface. The magnitude of the normal force exerted by a surface on an object in general depends on the other forces that are exerted on the object. For example, if a block is on a table, it will experience a stronger normal force if you exert a downwards force on the block.

Figure \ref{fig:newtonslaws:normal} shows two examples of the normal force on a block that is exerted by a surface (it is explicitly assumed that the block also experiences a downwards force from gravity that is not shown). In both cases, the normal force, $\vec N$, is perpendicular to the interface and in the direction that goes from the interface towards the object.

\capfig{0.5\textwidth}{figures/NewtonsLaws/normal.png}{\label{fig:newtonslaws:normal}The normal force, $\vec N$, exerted by a horizontal surface on a block (left side) and by an inclined surface (right side). In both cases, the normal force on the object is perpendicular to the interface between the object and the surface and points in the direction from the interface towards the object.}


\subsubsection{Friction forces}


\subsubsection{Tension forces}


\subsubsection{Applied forces}


\subsubsection{Spring forces}



\section{Mass and inertia}
Mass is a property of an object that quantifies how much matter the object contains. To be precise, we refer to ``inertial mass'', because mass in the context of Newton's Laws is a measure of the acceleration that an object will experience as the result of a net force. That is, if two objects have different masses and are acted upon by the same net force, the object with the higher mass will experience the smaller acceleration. The object with the higher mass as a stronger tendency to remain in the same state of motion (i.e. velocity) than the lighter object.

\section{Applying Newton's Laws}

\subsection{Free body diagrams}
\subsection{The net force}


\newpage
\section{Summary}
\vspace{2cm}
\begin{chapterSummary}
\item Something interesting
\end{chapterSummary}