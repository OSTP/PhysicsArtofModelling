\section{Newton's Laws}

%%%%%%%%%%%%%%%%%%%%%%%%%%%%%%%%%%%
%%
%% Multiple Choice
%%
%%%%%%%%%%%%%%%%%%%%%%%%%%%%%%%%%%%

\subsection{Multiple Choice}

\question[1] \label{question:newtonslaws:twoblocks} Two identical blocks of mass $m$ are at rest on a horizontal surface, as shown in Figure \ref{fig:newtonslaws:twoblocks}. The coefficients of static friction, $\mu$, between the two blocks and between the bottom block and the ground are the same. What is the minimum magnitude of a horizontal force $\vec F$ applied to the top block that will make the bottom block move?
\begin{choices} 
	\choice $F \geq \mu mg$
	\choice $F \geq 2\mu mg$
	\choice $F \geq 3\mu mg$
	\CorrectChoice It is not possible to make the bottom block move by pushing on the top block.
\end{choices}
\capfig{0.30\textwidth}{figures/NewtonsLaws/twoblocks.png}{\label{fig:newtonslaws:twoblocks} Two blocks (Question \ref{question:newtonslaws:twoblocks}).}


\question You push a heavy crate along a horizontal surface, moving it in the North direction at constant speed. Which statement is true?
\begin{checkboxes} 
	\choice You exert a force on the ground in the North direction.
	\CorrectChoice The ground exerts a force on you in the North direction.
	\choice The crate exerts a force on you in the North direction.
	\choice The crate exerts no force on you.
\end{checkboxes}

%Submitted by Genevieve Fawcett
\question If you punch a wall and break your hand but not the wall, what force is responsible for the damage to your hand?
\begin{checkboxes}
\choice The force you exert on the wall
\CorrectChoice The force the wall exerts on your hand, equal to the force you exert on the wall \correct
\choice The force the wall exerts on your hand, greater than the force you exert on the wall
\choice It depends on the material of the wall
\end{checkboxes}.

%Submitted by Emily Mendelson
\question Name all the forces acting  on an object that is sitting at rest on an inclined plane.
\begin{checkboxes}
\choice The object is at rest, therefore no forces are acting on the object
\choice The force of gravity
\choice The force of tension, the normal force, and the force of gravity
\CorrectChoice The force of friction, the normal force, and the force of gravity \correct
\end{checkboxes}

%Submittred by Emily Wener
\question Which of the following statements is true about weight?
\begin{checkboxes}
\choice Weight and mass describe the same intrinsic property of an object.
\choice Weight is an intrinsic property of an object.
\CorrectChoice The weight of an object can change depending on an object's location on Earth. \correct
\choice Two of the above statements are correct.
\choice None of the above.
\end{checkboxes}

\question Which of the following could be the mass of a tea cup in space?
\begin{checkboxes}
\choice \SI{0}{g}
\choice \SI{0}{N}
\CorrectChoice \SI{200}{g} \correct
\choice \SI{1.9}{N}
\end{checkboxes}


\question A dog with a mass of \SI{20}{kg}, as measured in the inertial reference frame of the Earth, is in an elevator standing on a standard bathroom scale. The elevator starts off stationary, then accelerates to a constant velocity, and then slows to a stop when it reaches the top floor. When will the scale display \SI{20}{kg}? Select all that apply:
\begin{checkboxes}
\CorrectChoice Before the elevator starts moving \correct
\choice While the elevator is accelerating upwards
\CorrectChoice While the elevator is moving upwards at its constant velocity \correct
\choice While the elevator is decelerating
\CorrectChoice After the elevator has stopped \correct
\end{checkboxes}

\question A car having a mass of \SI{900}{kg} collides head-on with a truck having a mass of \SI{1800}{kg}. In this collision, the magnitude of the force exerted on the car by the truck, $F_{CT}$ and the force exerted on the truck by the car, $F_{TC}$ are such that
\begin{checkboxes}
\choice $F_{CT} > F_{TC}$
\CorrectChoice $F_{CT} = F_{TC}$ \correct
\choice $F_{CT} < F_{TC}$
\choice Not enough information given to determine the answer
\end{checkboxes}


%From Midyear Exam F17
%easy
\question Excited at the idea of your first university physics exam, you decide to study for the exam by creating little demonstrations to showcase your knowledge to your friends. You derive an equation to convince your friends that a feather and a ballpoint pen will take the same amount of time to fall a distance $h$, even when the ballpoint pen is much heavier than the feather. You then perform your experiment, and the ballpoint pen hits the floor significantly earlier than the feather. You correctly explain to your friends that:
\begin{checkboxes} 
\choice The feather and pen both had the same acceleration, but the feather had more drag
\choice The feather and pen both had the same force from gravity, but the feather had more drag
\choice The feather and pen both had the same net force on them, but the feather is lighter, so it had less acceleration
\CorrectChoice The feather and pen had different accelerations \SI{1}{s} after being released. \correct
\end{checkboxes}


%Daniel Barake
\question You are at the supermarket pushing a cart full of groceries. To keep the cart moving at constant speed, you notice that you have to keep applying a force to the cart. You conclude that a continuous force must be needed to sustain continuous motion. This statement is:
\begin{checkboxes} 
\choice True, since the natural state of all objects is to be at rest. Eventually, all objects will be at rest, so to keep an object moving, a force needs to be applied.
\choice True, as this can be tested with other objects and vehicles such as cars and boats (when the motor/propeller on the boat is turned off, the boat slows down to a stop).
\CorrectChoice False. The force you apply to keep the cart moving at constant speed is only to counteract a frictional force. \correct
\end{checkboxes}
\newpage


%Yumian Chen
\question Which of the following options would not affect the drag force acting on an object moving through a fluid?
\begin{checkboxes} 
\choice The velocity of the object
\choice The density of the fluid
\choice The size of the object
\CorrectChoice The mass of the object \correct
\end{checkboxes}

\question You push a crate that is twice as heavy as you along a horizontal surface.
\begin{checkboxes}
\choice The crate exerts a force on you that is twice the magnitude of the force that you exert on the crate.
\choice The crate exerts a force on you that is half the magnitude of the force that you exert on the crate.
\CorrectChoice The crate exerts a force on you that is the same magnitude as the force that you exert on the crate.
\choice None of the above, as it will depend whether the crate is moving at constant speed or accelerating.
\end{checkboxes}

%Colin McNichols (modified)
\question A person is standing on a bathroom scale that indicates their mass in \si{kg}. The scale reading depends on the amount that a spring inside the scale is compressed as a result of the person standing on the scale. If that person stands on that scale in an elevator that is accelerating upwards in a building, what happens to the reading on the scale relative to the reading outside of the elevator?
\begin{checkboxes}
\CorrectChoice It increases.
\choice It stays the same.
\choice It decreases.
\end{checkboxes}

%Bennet Fahey
\question If a bowling ball and a chocolate chip cookie collide,
\begin{checkboxes}
\choice The bowling ball exerts a greater force on the cookie.
\choice The cookie exerts a greater force on the bowling ball.
\CorrectChoice They both exert equal and opposite forces on each other.
\end{checkboxes}


%%%%%%%%%%%%%%%%%%%%%%%%%%%%%%%%%%%
%
% Long Answer
%
%%%%%%%%%%%%%%%%%%%%%%%%%%%%%%%%%%%
\subsection{Long answers}
%Ryan U
\question Four boxes connected by ropes are lying on a frictionless surface, as in Figure \ref{fig:newtonslaws:4box}. The boxes have masses $m_1=\SI{6.0}{kg}$, $m_2=\SI{4.0}{kg}$, $m_3=\SI{3.0}{kg}$, $m_4=\SI{1.0}{kg}$.  Box 4 begins to accelerate to the right with an acceleration of \SI{2.0}{m/s^2}.
\begin{parts}
\part Find the tension in each rope and determine the rope with the highest tension
\part If the boxes all have the same mass, would the tensions be the same in each rope?
\part Box 4 stops accelerating when it hits \SI{3.0}{m/s}.  What is the tension in the rope between box 1 and 2 a moment later?
\end{parts}


\capfig{0.5\textwidth}{figures/NewtonsLaws/4box.png}{\label{fig:newtonslaws:4box}Four boxes on a frictionless surface.}

\begin{finalanswer}
\begin{enumerate}[(a)]
\item \begin{align*}
T_{12}&=\SI{12.0}{N}\\
T_{23}&=\SI{20.0}{N}\\
T_{34}&= \SI{26.0}{N}\\
\end{align*}
The maximum force is between boxes 3 and 4, with a value of $T_{34}=\SI{26.0}{N}$
\item No, the tension in the rope between boxes 3 and 4 will always be the highest, as that rope has to pull all of the other masses.
\item If all of the boxes are moving with a constant speed, then the tension in all of the ropes is zero (the boxes are moving on a frictionless surface).
\end{enumerate}
\end{finalanswer}
\begin{solution}
\textbf{a)}
Each box has the same acceleration, so we can work out the forces on each box, and thus the tension in the rope. Since the acceleration is horizontal, we only need to consider the tensions from the ropes (as these are the only forces with components in the horizontal direction). Starting with box 1 and moving towards the right:
\begin{align*}
\sum{F_{Box 1}}&=T_{12}=m_1a=(\SI{6.0}{kg})(\SI{2.0}{m/s^2})=\SI{12.0}{N}\\
\sum{F_{Box 2}}&=T_{23}-T_{12}=m_2a\\
\therefore T_{23} &= m_2a+(\SI{12.0}{N})=(\SI{4.0}{kg})(\SI{2.0}{m/s^2})+(\SI{12.0}{N})=\SI{20.0}{N}\\
\sum{F_{Box 3}}&=T_{34}-T_{23}=m_ra \\
\therefore T_{34} &= m_3a+(\SI{12.0}{N})= (\SI{3.0}{kg})(\SI{2.0}{m/s^2})+(\SI{20.0}{N})=\SI{26.0}{N}\\
\end{align*}
The maximum force is between boxes 3 and 4, with a value of $T_{34}=\SI{26.0}{N}$

\textbf{b)}
No, the tension in the rope between boxes 3 and 4 will always be the highest, as that rope has to pull all of the other masses.

\textbf{c)}
If all of the boxes are moving with a constant speed, then the tension in all of the ropes is zero (the boxes are moving on a frictionless surface).
\end{solution}

%Ryan U
\question Two boxes sit in a frictionless V-shaped pit with a \SI{90}{\degree} angle as shown in Figure \ref{fig:newtonslaws:Vboxes}. 
\begin{parts}
\part Draw a free-body diagram for box 1.
\part Find the magnitude of each force on box 1, assuming that the mass of each box is \SI{15}{kg}.
\end{parts}
\capfig{0.4\textwidth}{figures/NewtonsLaws/Vboxes.png}{\label{fig:newtonslaws:Vboxes} Two boxes sitting in a V-shaped pit.}

\begin{finalanswer}
\begin{enumerate}[(a)]
\item There are four forces acting on box 1:
\begin{enumerate}
\item Its weight ($\vec W_1$)
\item A force from box 2 ($\vec F_B$)
\item Two normal forces, one from each side of the pit($\vec N_1$ and $\vec N_2$)
\end{enumerate}
 \capfig{0.2\textwidth}{figures/NewtonsLaws/Vboxes_FBD.png}{\label{fig:newtonslaws:VboxesSolfinal} Free body diagram for box 1.}
\item \begin{align*}
F_B&=N_2=\SI{103.94}{N}\\
N_1&=N_2+F_B=\SI{207.89}{N}\\
W_1&=\SI{147.00}{N}\\
\end{align*}
\end{enumerate}
\end{finalanswer}
\begin{solution}
\textbf{a)} There are four forces acting on box 1:
\begin{enumerate}
\item Its weight ($\vec W_1$)
\item A force from box 2 ($\vec F_B$)
\item Two normal forces, one from each side of the pit($\vec N_1$ and $\vec N_2$)
\end{enumerate}
 \capfig{0.2\textwidth}{figures/NewtonsLaws/Vboxes_FBD.png}{\label{fig:newtonslaws:VboxesSol} Free body diagram for box 1.}
 
\textbf{b)}
We must sum the $x$ and $y$ components of the forces and set those sums equal to zero, since the box is not accelerating. We can define a coordinate system such that $x$ point to the right and $y$ points up. The weight of box 1 is:
\begin{align*}
\vec W_1=m\vec g=(\SI{15}{kg})(\SI{-9.8}{m/s^2}\hat j)=(-\SI{147}{N})\hat j
\end{align*} 
The sum of the forces on box 1 in the $x$ direction is:
\begin{align*}
\sum F_x^1 &= N_{1x}+N_{2x}+F_{Bx} = 0\\
0&= -N_1\cos(\SI{45}{\degree})+N_2\cos(\SI{45}{\degree})+F_B\cos(\SI{45}{\degree})
\end{align*}

The sum of the forces on box 1 in the $y$ direction is:
\begin{align*}
\sum F_y^1 &= N_{1y}+N_{2y}+F_{By}+W_{1y} = 0\\
0&= N_1\sin(\SI{45}{\degree})+N_2\sin(\SI{45}{\degree})-F_B\sin(\SI{45}{\degree})-W_1
\end{align*}

Note that we have 3 unknowns ($N_1$, $N_2$, $F_B$), but only 2 equations. We thus need to also consider the forces on box 2. Box 2 has 3 forces acting on it (the reaction force from $\vec F_B$, which we will call $\vec F'_B$, a normal force $N_3$, and its weight, $\vec W_2$), as shown in Figure \ref{fig:newtonslaws:VboxesSol2}. 
\capfig{0.2\textwidth}{figures/NewtonsLaws/Vboxes_FBD2.png}{\label{fig:newtonslaws:VboxesSol2} Free body diagram for box 2.}

The sum of the forces on box 2 in the $x$ direction is:
\begin{align*}
\sum F_x^2 &= N_{3x}+F'_{Bx} = 0\\
0&= N_3\cos(\SI{45}{\degree})-F'_B\cos(\SI{45}{\degree})
\end{align*}
This means that $N_3=F'_B$.

The sum of the forces on box 2 in the $y$ direction is:
\begin{align*}
\sum F_y^2 &= N_{3y}+F'_{By}+W_{2y} = 0\\
0&= N_3\sin(\SI{45}{\degree})+F'_B\sin(\SI{45}{\degree})+W_{2y}\\
0&= 2F'_B\sin(\SI{45}{\degree})-W_2\\
\therefore F'_B&=\frac{W_2}{2\sin(\SI{45}{\degree})}=\frac{\SI{147}{N}}{2\sin(\SI{45}{\degree})}=\SI{103.94}{N}
\end{align*}
where the magnitude of $F_B'$ is the same as $F_B$, since it is the corresponding reaction force, and the weight $W_2$ has the same magnitude as the first weight $W_1$. We can substitute this value back into the equations for box 1:
\begin{align*}
0&= -N_1\cos(\SI{45}{\degree})+N_2\cos(\SI{45}{\degree})+F_B\cos(\SI{45}{\degree}) \\
0&= N_1\sin(\SI{45}{\degree})+N_2\sin(\SI{45}{\degree})-F_B\sin(\SI{45}{\degree})-W_1
\end{align*}
Substituting and solving:
\begin{align*}
0&= -N_1+N_2+F_B \\
\therefore N_1&=N_2+F_B\\
0 &=(N_2+F_B)\sin(\SI{45}{\degree})+N_2\sin(\SI{45}{\degree})-F_B\sin(\SI{45}{\degree})-W_1\\
0 &= 2N_2\sin(\SI{45}{\degree})=W_1\\
\therefore N_2 &= \frac{W_1}{2\sin(\SI{45}{\degree})}=\frac{\SI{147}{N}}{2\sin(\SI{45}{\degree})}\\
\end{align*}
The magnitudes of the four forces on box 1 are thus:
\begin{align*}
F_B&=N_2=\SI{103.94}{N}\\
N_1&=N_2+F_B=\SI{207.89}{N}\\
W_1&=\SI{147.00}{N}\\
\end{align*}

\end{solution}

%Ryan U (highly modified)
\question Three blocks are connected by inextensible strings to a system of 2 massless and frictionless pulleys, as shown in Figure \ref{fig:newtonslaws:pulleys}. Block 3 is on a frictionless horizontal surface.
\begin{parts}
\part If block 1 has mass $m_1=\SI{3.0}{kg}$, block 2 has mass $m_2=\SI{1.0}{kg}$, and block 3 has mass $m_3=\SI{3.0}{kg}$, draw the free body diagram for each block and determine the acceleration of block 2 (magnitude and direction).
\part If block 1 has mass $m_1=\SI{1.0}{kg}$, block 2 has mass $m_2=\SI{3.0}{kg}$, and block 3 has mass $m_3=\SI{3.0}{kg}$, draw the free body diagram for each block and determine the acceleration of block 2 (magnitude and direction).
\end{parts}

\capfig{0.4\textwidth}{figures/NewtonsLaws/pulleys.png}{\label{fig:newtonslaws:pulleys} System of 3 blocks connected with two pulleys.}

\begin{finalanswer}
\begin{enumerate}[(a)]
\item \capfig{0.3\textwidth}{figures/NewtonsLaws/pulleys_FBD1.png}{\label{fig:newtonslaws:pulleys_FBD1} Free body diagram for each block, part a).} $\SI{2.8}{m/s^2}\;\;\text{(upwards)}$
\item  \capfig{0.3\textwidth}{figures/NewtonsLaws/pulleys_FBD2.png}{\label{fig:newtonslaws:pulleys_FBD2} Free body diagram for each block, part b).}
$-\SI{4.9}{m/s^2}\;\;\text{(downwards)}$
\end{enumerate}
\end{finalanswer}
\begin{solution}
\textbf{a)}In this case, block 2 will accelerate upwards. The free body diagram is shown in Figure \ref{fig:newtonslaws:pulleys_FBD1}.
\capfig{0.3\textwidth}{figures/NewtonsLaws/pulleys_FBD1.png}{\label{fig:newtonslaws:pulleys_FBD1} Free body diagram for each block, part a).}
We can write the sum of the forces on each block along the direction of motion (which we will define as positive for the case where block 2 moves upwards). Blocks 1 and 2 move in the vertical direction, so we write the forces in the vertical direction, and block 3 moves in the horizontal direction, so for it we write the sum of the forces in the horizontal direction. For each block, we can choose which direction is positive, and so we choose the positive direction for each block to correspond to a positive acceleration when block 2 moves upwards (thus, for block 1 the positive direction is down, for block 2 it is up, and for block 3 it is to the left).
\begin{align*}
W_1-T_1 &= m_1a \\
T_1-W_2-T_2 &= m_2a \\
T_2 &= m_3a \\
\end{align*}
We have 3 equations and 3 unknowns ($T_1$, $T_2$, and $a$). Adding the first two equations, and then substituting in the third equations:
\begin{align*}
W_1-W_2-T_2 &= (m_1+m_2)a\\
W_1-W_2-m_3a &= (m_1+m_2)a\\
W_1-W_2&= (m_1+m_2+m_3)a\\
\therefore a&=\frac{W_1-W_2}{(m_1+m_2+m_3)}\\
  &=\frac{(\SI{3.0}{kg})(\SI{9.8}{m/s^2})-(\SI{1.0}{kg})(\SI{9.8}{m/s^2})}{(\SI{3.0}{kg})+(\SI{1.0}{kg})+(\SI{3.0}{kg})}=\SI{2.8}{m/s^2}\;\;\text{(upwards)}
\end{align*}
\textbf{b)}In this case, block 2 will accelerate downwards, and block 3 will not affect the situation, since the rope between blocks 2 and block 3 will be slack. The free body diagram is shown in Figure \ref{fig:newtonslaws:pulleys_FBD2}.
\capfig{0.3\textwidth}{figures/NewtonsLaws/pulleys_FBD2.png}{\label{fig:newtonslaws:pulleys_FBD2} Free body diagram for each block, part b).}
We only need to consider blocks 1 and 2. Again, we write the sum of the forces on each block along the direction of motion (which we will define as positive for the case where block 2 moves upwards).
\begin{align*}
W_1-T_1 &= m_1a \\
T_1-W_2&= m_2a \\
\end{align*}
We have 2 equations and 2 unknowns ($T_1$ and $a$). Adding the two equations, we find $a$:
\begin{align*}
W_1-W_2 &= (m_1+m_2)a\\
\therefore a&=\frac{W_1-W_2}{(m_1+m_2)}\\
&=\frac{(\SI{1.0}{kg})(\SI{9.8}{m/s^2})-(\SI{3.0}{kg})(\SI{9.8}{m/s^2})}{(\SI{1.0}{kg})+(\SI{3.0}{kg})}=-\SI{4.9}{m/s^2}\;\;\text{(downwards)}
\end{align*}

\end{solution}


%OpenStax 5-48 (modified)
\question A fireman with mass $m=\SI{90}{kg}$ slides down the pole at the fire station with acceleration $a=\SI{5.0}{m/s^2}$. 
\begin{parts}
\part What force (magnitude and direction) does the pole exert on the fireman?
\part What force (magnitude and direction) does the fireman exert on the pole?
\part If the pole is \SI{10.0}{m} long, how much time does it take the fireman to reach the bottom?
\part What speed will the fireman have when she hits the ground?
\end{parts}

\begin{finalanswer}
\begin{enumerate}[(a)]
\item $\SI{432.0}{N}\;\;(upwards)$
\item $\SI{432.0}{N} downwards$
\item \SI{2.0}{s}
\item \SI{10}{m/s}
\end{enumerate}
\end{finalanswer}
\begin{solution}
\textbf{a)} The fireman is accelerating with less than $g$, so the pole is exerting an upwards force:
\begin{align*}
\sum F &= mg-F_p=ma\\
\therefore F_p &= m(g-a)=(\SI{90}{kg})( (\SI{9.8}{m/s^2}) - (\SI{5.0}{m/s^2}) )=\SI{432.0}{N}\;\;(upwards)
\end{align*}

\textbf{b)} The fireman exerts a force on the pole that is equal in magnitude but opposite in direction to the answer in part a), namely \SI{432.0}{N} downwards.

\textbf{c)} Given the acceleration, the initial speed (\SI{0}{m/s}) and total distance covered, we can find the time to go down the pole (assuming positive downwards):
\begin{align*}
x&=x_0+v_0t+\frac{1}{2}at^2\\
\therefore t&=\sqrt{\frac{2(x-x_0)}{a}}=\sqrt{\frac{2(\SI{10}{m})}{(\SI{5.0}{m/s^2})}}=\SI{2.0}{s}
\end{align*}

\textbf{d)} After \SI{2.0}{s} with an acceleration of $a=\SI{5.0}{m/s^2}$, and starting from rest, the speed will be:
\begin{align*}
v(t)=at=(\SI{5.0}{m/s^2})(\SI{2.0}{s})=\SI{10}{m/s}
\end{align*}
ouch!

\end{solution}

%Openstax 5-94
\question A \SI{10.0}{kg} object is initially moving east with a speed of \SI{15}{m/s}. A force then acts on it for \SI{2.0}{s}, after which it moves north-west, also with a speed of \SI{15}{m/s}. What are the magnitude and direction of the average force that acted on the object during that \SI{2.0}{s} interval?
\begin{finalanswer}
\SI{138.54}{N} and points in the direction \SI{67.5}{\degree} West of North.
\end{finalanswer}
\begin{solution}
If we define East as the positive $x$ axis and North as the positive $y$ axis, then the initial and final velocity vectors are:
\begin{align*}
\vec v_0&=(\SI{15}{m/s})\hat i \\
\vec v_f&=(\SI{15}{m/s})\cos(\SI{135}{\degree})\hat i +(\SI{15}{m/s})\sin(\SI{135}{\degree})\hat j\\
\end{align*}
The average acceleration vector is:
\begin{align*}
\vec a &=\frac{\vec v_f-\vec v_0}{\Delta t}=\frac{(\SI{15}{m/s})\cos(\SI{135}{\degree})\hat i +(\SI{15}{m/s})\sin(\SI{135}{\degree})\hat j-(\SI{15}{m/s})\hat i }{(\SI{2.0}{s})}\\
&=(\SI{-12.80}{m/s^2})\hat i+(\SI{5.30}{m/s^2})\hat j
\end{align*}
The average for vector is thus:
\begin{align*}
\vec F = m\vec a= (\SI{-128.0}{N})\hat i+(\SI{53.0}{N})\hat j
\end{align*}
which has a magnitude of \SI{138.54}{N} and points in the direction \SI{67.5}{\degree} West of North (or \SI{22.5}{\degree} North of West). 
\end{solution}


\question A little souvenir vicu\~na figurine with a mass of $m=\SI{50}{g}$ hangs by a thin string from the rear view mirror of your car. As you accelerate, you notice that the string makes an angle of \SI{10.0}{\degree} from the vertical. What is the acceleration of your car?

\begin{finalanswer}
$\SI{1.73}{m/s^2}$
\end{finalanswer}
\begin{solution}
The only two forces on the figurine are gravity and the tension in the string, as shown in Figure \ref{fig:newtonslaws:figurine_FBD}.
\capfig{0.05\textwidth}{figures/NewtonsLaws/figurine_FBD.png}{\label{fig:newtonslaws:figurine_FBD} Free body diagram for the vicu\~na figurine.}
The sum of the forces must be in the direction of motion of the car ($x$), and equal to $ma$. Letting $y$ be the vertical direction, we write the sum of forces in each direction:
\begin{align*}
\sum F_x &= T\sin\theta = ma\\
\sum F_y &= T\cos\theta-mg = 0
\end{align*}
Solving for $T$ in the second equation and substituting into the first equation:
\begin{align*}
 T&=\frac{mg}{\cos\theta}\\
 mg\frac{\sin\theta}{\cos\theta}&=ma\\
\therefore a&=g\tan\theta=(\SI{9.8}{m/s^2})\tan(\SI{10}{\degree})=\SI{1.73}{m/s^2}
\end{align*}
\end{solution}

\question You are called in to investigate a car accident, in which a car struck a llama on a dry road. The driver claims that they were going below the speed limit of $\SI{50}{km/h}$ and hit the breaks, locking the cars' wheels, as soon as they struck the animal. You notice that there are skid marks from the car, which you  measure to be $\SI{25}{m}$ in length. You also look up the coefficients of static friction and kinetic friction between the tires and the dry road, and find that they are $\mu_s=0.9$ and $\mu_k=0.6$, respectively.
\begin{parts}
\part At what speed can you say that the driver was going before the car started skidding?
\part If the road had been wet and the coefficients of friction correspondingly lower, would the skid marks be shorter or longer?
\end{parts}

\begin{solution}
\begin{parts}
\part The skid marks indicate that, for a distance of $\SI{25}{m}$, the car decelerated from an unknown initial speed, $v_0$, to a final speed $v=0$. 
The acceleration of the car must be found from Newton's Second Law. The forces exerted on the car are:
\begin{itemize}
\item $\vec F_g$, its weight, with magnitude $mg$ and pointing down.
\item $\vec N$, a normal force from the road, with magnitude $mg$ and pointing up.
\item $\vec f_k$, a force of kinetic friction with magnitude $\mu_kN$ and pointing backwards. This force is due to the tires skidding (kinetic friction) and is responsible for the deceleration.
\end{itemize}
The acceleration of the car is given by:
\begin{align*}
\sum F=f_k&=ma\\
\mu_kmg&=ma\\
\therefore a &=\mu_k g
\end{align*}
We can now use one of the kinematic equations for constant acceleration (the acceleration is in the negative direction if we choose $x_0=0$ and $x=\SI{18}{m}$:
\begin{align*}
v^2-v_0^2&=2a(x-x_0)\\
v_0^2&=-2a(x-x_0)=2\mu_kg(x-x_0)\\
\therefore v_0=\sqrt{2mu_kg(x-x_0)}&=\sqrt{2(0.6)(\SI{9.8}{m/s^2})(\SI{25}{m})}=\SI{9.4}{m/s}=\SI{43.6}{km/h}
\end{align*}
Based on the length of the skid marks, the car was not speeding when it started skidding. The driver could have been going over the speed limit, if they braked without skidding for a distance before locking the wheels.
\part The skid marks would be longer. The car would have a smaller deceleration (since $a$ is proportional to $\mu_k$), so the distance to decelerate and change the speed by a given amount is longer.
\end{parts}
\end{solution}

\question Two blocks with masses, $m_1$ (block 1), and $m_2$ (block 2), are placed on an incline, as shown in Figure \ref{fig:newtonslaws:blockrace}. The coefficients of kinetic friction between block 1 and the incline is $\mu_{k1}$ and the coefficient of kinetic friction between block 2 and the incline is $\mu_{k2}$. Block 1 is placed a distance $L$ from the bottom of the incline, while block $2$ is placed a distance $D$ from block 1. You can neglect the dimensions of each block.
\begin{parts}
\part Find an expression for the distance $D$ that will result in the blocks colliding exactly at the bottom of the incline, if they are both released at time $t=0$. Give your answer in terms of $L$, $\theta$, $\mu_{k1}$, $\mu_{k2}$.
\part How much time goes by between the blocks being released and the blocks reaching the bottom of the incline?
\end{parts}
\capfig{0.5\textwidth}{figures/NewtonsLaws/blockrace.png}{\label{fig:newtonslaws:blockrace}Two blocks on an incline at $t=0$.}
\begin{solution}
\begin{parts}
\part We first start by identifying the forces on each block, in order to determine their acceleration. Each block will have the following three forces exerted on it:
\begin{itemize}
\item $\vec F_g$, its weight.
\item $\vec N$, a normal force exerted by the plane.
\item $\vec f_k$, a force of kinetic friction exerted by the plane, with magnitude $\mu_kN$.
\end{itemize}
The forces are illustrated in the free-body diagram in Figure
\capfig{0.2\textwidth}{figures/NewtonsLaws/blockrace_FBD.png}{\label{fig:newtonslaws:blockrace_FBD}Free-body diagram for either block in Figure \ref{fig:newtonslaws:blockrace}.}
We can write Newton's Second Law for each block $i$, choosing the coordinate system that is illustrated in Figure \ref{fig:newtonslaws:blockrace_FBD} ($x$ parallel to the incline and positive down, $y$ perpendicular to the incline and positive up).:
\begin{align*}
\sum F_x &= F_{gi}\sin\theta-f_{ki}=m_ia_i
\sum F_y &= N-F_{gi}\cos\theta=m_ia_i
\end{align*}
where $i$ is 1 or 2, depending on the block. The $y$ equation gives the magnitude of the normal force:
\begin{align*}
N_1=m_ig\cos\theta
\end{align*}
which can be substituted into the $x$ equation to obtain the acceleration of block $i$:
\begin{align*}
m_ig\sin\theta-\mu_{ki}N_i&=m_ia_i\\
m_ig\sin\theta-\mu_{ki}m_ig\cos\theta&=m_ia_i\\
\therefore a_i&=g(\sin\theta-\mu_{ki}\cos\theta)
\end{align*}
By knowing the acceleration of each block, we can determine the time that it takes for each block to reach the bottom of the incline. For block 1, this is given by:
\begin{align*}
L&=\frac{1}{2}a_1t^2\\
\therefore t^2 &=2\frac{L}{a_1}
\end{align*}
For block 2:
\begin{align*}
L+D&=\frac{1}{2}a_2t^2\\
\therefore t^2 &=2\frac{L+D}{a_2}
\end{align*}
Equating the two times squared:
\begin{align*}
2\frac{L}{a_1}&=2\frac{L+D}{a_2}\\
\therefore D &= L\left(1-\frac{a_2}{a_1}\right)=L\left(1-\frac{\sin\theta-\mu_{k2}\cos\theta}{\sin\theta-\mu_{k1}\cos\theta}\right)
\end{align*}
This makes sense, since if the two coefficients of kinetic friction are the same, we find that $D=0$, as expected. It also makes sense because if $\mu_{k2}>\mu_{k1}$, then $D$ is negative, implying that block 2 should be placed in front of block 1 (as it will have more friction).
\part We can use block 1, since we know its acceleration and the distance that it travelled:
\begin{align*}
x(t)&=x_0+v_{0x}t+\frac{1}{2}at^2\\
L&=\frac{1}{2}a_1t^2\\
\therefore t&=\sqrt{\frac{2L}{a_1}}=\sqrt{\frac{2L}{\sin\theta-\mu_{k1}\cos\theta}}
\end{align*}
\end{parts}
\end{solution}

