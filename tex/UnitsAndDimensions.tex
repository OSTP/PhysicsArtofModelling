\section{Units, Dimensions, Estimates}

%%%%%%%%%%%%%%%%%%%%%%%%%%%%%%%%%%%
%%
%% Multiple Choice
%%
%%%%%%%%%%%%%%%%%%%%%%%%%%%%%%%%%%%
\subsection{Multiple Choice}
\question A motor is rotating at \SI{420}{rpm} (rotations per minute). What does this correspond to in \si{Hz} (rotations per second)?
\begin{checkboxes}
\CorrectChoice \SI{7}{Hz} \correct
\choice \SI{17.5}{Hz}
\choice \SI{25200}{Hz}
\end{checkboxes}

\question In solving a physics problem you end up with units of \si{m} in the numerator and units of \si{m/s} in the denominator. The units for your answer are:
\begin{checkboxes}
\choice \si{m^2/s}
\choice \si{m^2}
\choice \si{m}
\choice \si{1/s}
\CorrectChoice  \si{s} \correct
\end{checkboxes}

\question The density of a solid object is defined as the ratio of the mass of the object to its volume. The dimension of density is:
\begin{checkboxes}
\choice M/L
\CorrectChoice  M/L$^3$ \correct
\choice L$^3$/M
\choice ML
\end{checkboxes}

\question Power is defined as the rate of work per time, power = work/time.  If the dimensions of power are ML$^2$T$^{-3}$, what are the dimensions of work?  
\begin{checkboxes}
\choice MLT$^{-3}$
\choice ML$^2$T$^{-1}$
\choice ML$^{3}$T$^{-3}$
\CorrectChoice ML$^{2}$T$^{-2}$ \correct
\choice ML$^2$T$^{-4}$
\end{checkboxes}

\question The position, $x(t)$, of an object is given by the equation $x(t) = A + Bt +Ct^2$, where $t$ refers to time. Note that the dimension of position is length ($[x]=L$).  What are the dimensions of $A$, $B$, and $C$?  
\begin{checkboxes}
\choice [A]=L, [B]=L, [C]=L
\choice [A]=L, [B]=T, [C]=T
\choice [A]=L, [B]=T, [C]=T$^2$
\CorrectChoice [A]=L, [B]=L/T, [C]=LT$^{-2}$ \correct
\choice [A]=L/T, [B]=LT$^{-2}$, [C]=LT$^{-3}$
\end{checkboxes}

\question By repeating a measurement many times, you can reduce:
\begin{checkboxes}
\choice the systematic error in the value
\CorrectChoice the random (statistical) error in the value \correct
\end{checkboxes}





%%%%%%%%%%%%%%%%%%%%%%%%%%%%%%%%%%%
%
% long answer
%
%%%%%%%%%%%%%%%%%%%%%%%%%%%%%%%%%%%
\subsection{Long answers}

\question While observing a particularly fast guanaco in a herd, you measured that it covered a distance of $d=\SI{100.0\pm 5.0}{m}$ in a time $t=\SI{7.0 \pm 1.0}{s}$. What is the speed of the guanco, with uncertainty?

\textit{Note that speed is defined as distance over time. You are free to use whichever method that you like to calculate the speed and its uncertainty, but you must explain what you did in your answer.}

\begin{finalanswer}
$v = \SI{14.3\pm 2.2}{m/s}$
\end{finalanswer}

\begin{solution}
With Min-Max:
\begin{align*}
v^{max} &= \frac{d^{max}}{t^{min}}=\frac{(\SI{105.0}{m})}{(\SI{6.0}{s})}=\SI{17.5}{m/s}\\
v^{min} &= \frac{d^{min}}{t^{max}}=\frac{(\SI{95.0}{m})}{(\SI{8.0}{s})}=\SI{11.875}{m/s}\\
\therefore v &= \SI{14.6875 \pm 2.8125}{m/s}= \SI{14.7 \pm 2.8}{m/s}
\end{align*}

With relative errors (same as derivative), not added in quadrature:
\begin{align*}
v &= \frac{d}{t}=\frac{(\SI{100.0}{m})}{(\SI{7.0}{s})}=\SI{14.286}{m/s}\\
\frac{\sigma_v}{v} &= \frac{\sigma_d}{d}+\frac{\sigma_t}{t}=\frac{(\SI{5.0}{m})}{(\SI{100.0}{m})}+\frac{(\SI{1.0}{s})}{(\SI{7.0}{s})} = 0.1929\\
\therefore v &= \SI{14.286 \pm 2.755}{m/s}= \SI{14.3\pm 2.8}{m/s}
\end{align*}

With relative errors (same as derivative), added in quadrature (this is the best possible way to do it):
\begin{align*}
v &= \frac{d}{t}=\frac{(\SI{100.0}{m})}{(\SI{7.0}{s})}=\SI{14.286}{m/s}\\
\frac{\sigma_v}{v} &= \sqrt{ \left(\frac{\sigma_d}{d}\right)^2+\left(\frac{\sigma_t}{t} \right)^2}=\sqrt{ \left(\frac{(\SI{5.0}{m})}{(\SI{100.0}{m})}\right)^2+\left(\frac{(\SI{1.0}{s})}{(\SI{7.0}{s})}\right)^2} = 0.1513\\
\therefore v &= \SI{14.286 \pm 2.162}{m/s}= \SI{14.3\pm 2.2}{m/s}
\end{align*}
\end{solution}

%Giancolli 1-21 -fixed
\question A light year is defined as the distance that light travels in one year, at a speed of \SI{2.998e8}{m/s}.
\label{q:unitsanddimensions:lightyear}
\begin{parts}
\part How many metres are in a light year?
\part An astronomical unit (AU) is defined as the average distance between the Earth and the Sun, and is given by $\SI{1}{AU}=\SI{1.50e8}{km}$. Express the distance light travels in one year in AU.
\part An officer of the United Federation of Planets pulls your spaceship over. The officer claims that you were speeding, and that the speed limit in this section of the galaxy is \SI{5.0}{AU/h}. Unfortunately, your spaceship's speedometer measures speed in \si{m/s}, and it reads that you were travelling at \SI{3.2e8}{m/s}. Was the officer correct to pull you over?
\end{parts}

\begin{finalanswer}
\begin{enumerate}[(a)]
\item $\SI{1}{ly}=\SI{9.45e15}{m}$
\item $\SI{6.3e4}{AU/ly}$
\item Yes, the officer was correct to pull you over (Additionally, you violated the universal speed limit, so your spaceship will be impounded). 
\end{enumerate}
\end{finalanswer}
\begin{solution}
\textbf{a)} A light-year in meters:
\begin{align*}
\SI{1}{ly}=(\SI{2.998e8}{m/s})(60\cdot 60\cdot 24\cdot 365\si{s})=\SI{9.45e15}{m}
\end{align*}
\textbf{b)} From \si{ly} to \si{AU}:
\begin{align*}
\frac{(\SI{9.45e15}{m/ly})}{(\SI{1.50e11}{m/AU})}=\SI{6.3e4}{AU/ly}
\end{align*}
\textbf{c)} Conversion to \si{AU/h}:
\begin{align*}
(\SI{3.2e8}{m/s})\cdot\left(\frac{1}{\num{1.50e11}}\si{AU/m}\right)\cdot(\SI{3600}{s/h})=\SI{7.68}{AU/h}
\end{align*}
\end{solution}

%Giancolli 1-58 fixed
\question A watchmaker wants to test the accuracy of their watches. The watchmaker finds that their watch loses \SI{6}{s} in a year. How precise are the watches (expressed as a percentage)?

\begin{finalanswer}
$\num{1.9e-5}\%$
\end{finalanswer}

\begin{solution}
There are \num{3.15e7} seconds in a year (see question \ref{q:unitsanddimensions:lightyear}). The precision is thus:
\begin{align*}
100\frac{(\SI{6}{s})}{(\SI{3.15e7}{s})}=\num{1.9e-5}\%
\end{align*}
\end{solution}


%Giancolli 1-38 -fixed
\question In 1899, Max Plank suggested that there was a measurement of time that could be defined by only physical constants. Show that the following combination gives a quantity of time:
\begin{align*}
t_P=\sqrt{\frac{Gh}{c^5}}
\end{align*}
where, $G$, is Newton's gravitational constant, $h$ is Planck's constant, and $c$ is the speed of light. This combination of physical constants gives what is called Plank time, which is potentially the smallest possible measureable time interval.

\begin{solution}
First, we need to look up the dimensions of theses quantities, and then combine them:
\begin{align*}
[G]&=\frac{L^3}{MT^2}\\
[h]&=\frac{ML^2}{T}\\
[c^5]&=\frac{L^5}{T^5}\\
\therefore [t_P]&=\sqrt{\frac{\frac{L^3}{MT^2}\frac{ML^2}{T}}{\frac{L^5}{T^5}}}=T
\end{align*}
\end{solution}

\question Estimate the combined volume of all of the students in the PHYS 104/106 class. 

\textit{The numerical answer that you get, as long as it is reasonable, is not as important as the process. Describe in detail what assumptions you make in order to arrive at your answer, and comment on whether your answer is reasonable.}

\begin{finalanswer}
Should be about \SI{20}{m^3}
\end{finalanswer}

\begin{solution}
You need to:
\begin{itemize}
\item estimate the number of students in the class ($\sim$200)
\item estimate the average volume of a person (e.g. by assuming they are made of water, and have an average weight of $\sim$\SI{70}{kg}, \SI{70}{l}, so about \SI{0.1}{m^3} )
\end{itemize}
Should be about \SI{20}{m^3}
\end{solution}

\question Estimate the number of dentists in Toronto.  

\textit{The numerical answer that you get, as long as it is reasonable, is not as important as the process. Describe in detail what assumptions you make in order to arrive at your answer, and comment on whether your answer is reasonable.}

\begin{finalanswer}
$\sim 620$ dentists 
\end{finalanswer}

\begin{solution}
You need to estimate:
\begin{itemize}
\item population of Toronto (about 3 million)
\item hours everyone spends at the dentist (maybe a third of the population goes to the dentist for an hour per year), so about 1 million dentist hours
\item number of hours worked by 1 dentist per year (say, 6 weeks vacation, 35 hours for 46 weeks, 1610 hours)
\end{itemize}
So if all of the dentists are fully booked, you need about $\frac{\num{1e6}}{1610}=\sim 620$ dentists. Note that the estimate for the population that can see a dentist can be refined, making some arguments about income distribution, fraction of jobs with dental benefits, etc. 
\end{solution}

\question You decide to measure's Newton's universal constant of gravity, $G$, which you determine by measuring the force of attraction between two spheres of mass $m=\SI{10.000\pm 0.001}{kg}$ placed a distance $r=\SI{1.00\pm0.05}{cm}$ apart. You determine that the force of attraction is $F=\SI{6.5 \pm 1.0e-5}{N}$. $G$ is given by:
\begin{align*}
G = \frac{Fr^2}{m^2}
\end{align*}

Based on these measurements, what is the value of $G$ that you determine, with uncertainty?

\textit{You are encouraged to determine $G$ using QExpy and a Jupyter notebook. Save the notebook to pdf and upload it as your answer to show your work. You may need to copy the answer from the notebook to format it for significant figures and units.}

\begin{finalanswer}
$G=\SI{6.5\pm 1.2e-11}{m^3 kg^{-1} s^{-2}}$
\end{finalanswer}

\begin{solution}
Using QExpy
\begin{verbatim}
m = q.Measurement(10,0.001)
r = q.Measurement(0.01,0.0005)
F = q.Measurement(6.5e-5,1e-5)

G = F*r**2/m**2
q.set_sigfigs(2)
q.set_print_style("scientific")
print(G)
>>(65 +/- 12)*10^(-12)
\end{verbatim}
So you get $G=\SI{6.5\pm 1.2e-11}{m^3 kg^{-1} s^{-2}}$.
\end{solution}

\question The period of oscillation, $T$, of a pendulum (how long it takes to swing back and forth), depends on the length of the pendulum, $L$, and the acceleration due to gravity, $g$. Use dimensional analysis to:
\begin{parts}
\part determine a formula for the period of oscillation, $T$, in terms of $L$ and $g$.
\part show (or argue) that the period cannot depend on the mass of the pendulum
\end{parts}
\begin{solution}
\begin{parts}
\part We need to determine the exponents, $a$ and $b$, such that the following equation is dimensionally correct (has the same dimensions on both sides):
\begin{align*}
T = L^ag^b
\end{align*}
On the left, the dimensions must be time (also the letter $T$), so we need the dimension of $L^ag^b$ to also be equal to time. The dimension of $L^ag^b$ is given by:
\begin{align*}
[L] &= L\\
[g] &= L/T^2\\
\therefore [L^ag^b]&=L^aL^b/T^{2b} = L^{a+b}T^{-2b}
\end{align*}
Since this must be equal to dimensions of time, we must have:
\begin{align*}
a+b &=0 \\
-2b &= 1
\end{align*}
which we can easily solve for $a$ and $b$:
\begin{align*}
b &= -\frac{1}{2}\\
a &= -b = \frac{1}{2}
\end{align*}
The oscillation period is thus given by:
\begin{align*}
T &= L^ag^b = L^{\frac{1}{2}}g^{-\frac{1}{2}}=\sqrt{\frac{L}{g}}
\end{align*}
You can easily look up that the period of a simple pendulum is given by:
\begin{align*}
T=2\pi\sqrt{\frac{L}{g}}
\end{align*}
which is the same that we found, up to a dimensionless constant ($2\pi$).
\part There is no way that $T$ can depend on the mass of the pendulum, as there are no other quantities whose base dimensions include mass and would allow mass to be cancelled out. Note that if the formula were expected to depend on some physical constant that had mass in it, then the formula for $T$ could depend on mass. However, given the constraint of the formula for $T$ depending only on the length of the pendulum and the acceleration due to Earth's gravity, there is no way to incorporate the mass of the pendulum.
\end{parts}
\end{solution}