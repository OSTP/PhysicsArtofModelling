
\chapter{Electric charges and fields}
\label{chapter:chargesfields}
In this chapter, we start to look at the theories that describe electric phenomena. We will still use the framework for dynamics that was developed by Newton, but will now start to introduce the theories of electromagnetism which describe the electric, and later, the magnetic force. This analogous to how we introduce Newton's Universal Theory of Gravity to describe the source of the gravitational force. 


\begin{learningObjectives}{
 \item Understand the definition of an electric charge.
 \item Understand Coulomb's model for the electric force.
 \item Understand the definition of an electric field.
 \item Understand how to calculate the electric field from a continuous distribution of charge.
 }
\end{learningObjectives}

\begin{opening}
\begin{MCquestion}{If you rub a balloon against a carpet and bring it near your head, your hair will stand up and try to touch the baloon.}
\item The electric charge of the balloon is opposite of that on your hair.
\item Your hair has no net electric charge, this is an example of charge separation and induction. \correct
\end{MCquestion}
\end{opening}

\section{Electric charge}
You have likely experienced or heard about electric charge in your life. For example, on a dry Winter day, you might find that after rubbing your bare feet on a polyester carpet you feel a small electric shock upon touching a metallic surface such as a doorknob. This is a manifestation of the electric charge that has built up on you being released onto the doorknob. You probably also have a notion of the existence of positive and negative charges, and that equal charges repel each other whereas opposite charges attract. In this chapter, we develop the description of how these charges can accumulate and how they can exert attractive or repulsive forces. 

Ordinary matter is made of atoms, which are themselves made of a small positive nucleus (containing positive protons and neutral neutrons) surrounded by a ``cloud'' of negatively charged electrons. Within a solid object, the atoms in the object can be modelled as being effectively stationary due to inter-atomic forces that hold the atoms together. As a result, the protons (the positively charged part of atoms) can be considered to be fixed in space. The negative electrons, depending on the material, can often move from one atom to another. If an atom looses an electron to another atom, it becomes positive, whereas the atom that acquired the extra electron becomes negative. We define the net charge on an atom (or an object) based on whether there are more protons (positive), more electrons (negative) or an equal amount (neutral). By default, atoms are neutral and have an equal number of protons and electrons.

When you rub your feet on the carpet, electrons are being removed from one surface (your feet) and deposited on the other (the carpet). Your feet thus acquire a net positive charge (having lost electrons). When you touch a doorknob, the little spark comes from electrons jumping from the doorknob and onto your body (and down to your feet to fill the empty spaces left by the electrons lost to the carpet). The reason that the electrons leave your feet in the first place is that different materials have different ``affinities'' for electrons. When you rub two materials together (placing their atoms in close proximity), electrons will transfer to the material with the highest affinity for electrons. This way of creating a net charge is called ``charging by friction''.

The ``triboelectic series'' is a list of materials that tend to give up or acquire electrons when they are placed in close contact; some common materials from the series are shown in Figure \ref{fig:chargesfields:triboseries}.  The greatest charge is generated by rubbing together materials that are the furthest away in the diagram. Rubbing silk on a piece of wood will not create as much charge as rubbing silk on glass.

\capfig{\textwidth}{figures/ChargesFields/triboseries.png}{\label{fig:chargesfields:triboseries}A sample of a triboelectric series of materials. The materials on the right-hand side have the greatest affinity to acquire electrons.}

\begin{checkpoint}\label{cp:chargesfields:tribo}
\begin{MCquestion}{If you rub a glass rod with silk, which object ends up with an excess of electrons?}
\item glass rod.
\item silk. \correct
\item neither, they remain neutral.
\item both will acquire an excess of electrons.
\end{MCquestion}
\end{checkpoint}

\subsection{Conductors and insulators}
We can broadly classify materials into conductors (such as metals), and insulators (such as wood), depending on how easily the electrons can move around in the material. In a conductor, electrons (rather, the outer electron(s) of an atom) are only loosely bound to their nucleus, and they can thus move around freely. In an insulator, the electrons are tightly bound to the nuclei of their atoms and cannot easily move around. There is a third class of materials, semi-conductors, that fall somewhere between a conductor and an insulator. In a semi-conductor, electrons are typically bound to their atoms, but any additional electrons present in the material can move around as if they are in a conductor. 

Within a conductor, for example a solid metallic sphere, charges can move around freely. If that sphere has a net charge, for example an excess of electrons, all of those electrons will be located at the outer surface of the sphere. This is illustrated by showing the charges on the surface of the charged sphere in the left panel of Figure \ref{fig:chargesfields:conductioncharge}. This is because the electrons repel each other and will settle in a position where they are, on average, the furthest from all of the other electrons. If an initially neutral conducting sphere is connected to the charged sphere by a conducting wire (right panel of Figure \ref{fig:chargesfields:conductioncharge}), some of the electrons will ``conduct'' (transfer) onto the surface of the neutral sphere, so that, on average, they are further from all other electrons. This way of adding charge to the neutral sphere is called ``charging by conduction'', and the second sphere will remain charged if the connection between spheres is broken.

\capfig{0.7\textwidth}{figures/ChargesFields/conductioncharge.png}{\label{fig:chargesfields:conductioncharge}Charging by conduction: a neutral conducting sphere is connected to a negatively charged conduction sphere and the charges can ``spread out more'' if some of the charges move onto the neutral sphere.}

\subsection{Electrostatic induction}
Electrostatic induction allows one to ``induce'' a charge by using the fact that charges can move freely in a conductor. The left panel of Figure \ref{fig:chargesfields:induction} shows a (neutral) rod made of a conducting material, with electrons distributed uniformly throughout its volume. In the right panel, a negatively charged sphere is brought next to the rod. Since the rod is conducting, electrons in the rod can easily move and they will thus accumulate on the end of the rod that is furthest from the negative sphere (which repels the electrons). Those electrons will leave positive empty spaces, which can be modelled as positive charges, on the side closest to the sphere. The electrons in the rod will only accumulate on one end for as long as the force from the negative sphere is less than the repulsive force from the electrons that have already accumulated. In practice, such an equilibrium is reached almost instantly. In equilibrium, we say that the rod is ``polarized'', or that the ``charges in the rod have separated''.

Note that we can model this as if it where positive charges that move in rod instead of negative charges. We would model that the positive charges are attracted to the negative sphere, and thus accumulate on the end of the rod closest to the sphere, leaving a negative charge on the other end. The choice to call electrons ``negative'' is completely arbitrary. For convenience, we usually build models assuming that positive charges can easily move around, even if in reality it is almost always actually (negative) electrons that move.

\capfig{0.9\textwidth}{figures/ChargesFields/induction.png}{\label{fig:chargesfields:induction}Electrostatic induction: when a negatively charged sphere is brought close to a neutral conducting rod, the electrons in the rod, which can move freely, accumulate on the side of the rod furthest from the sphere, leaving an excess of positive charge near the sphere.}

We can create a net charge on the polarized rod if we provide a conducting path for charges to leave (or enter) the rod. The Earth can be modelled as a very large reservoir of both positive and negative charges. By connecting the rod to the Earth (we say that we connect the rod to ``ground''), we provide a path for the electrons in the rod to be even further from the negatively charged sphere, and they can thus leave the rod entirely in order to go into the ground. This is illustrated in the right-hand panel of Figure \ref{fig:chargesfields:inductioncharge}. If we then disconnect the rod from the ground, it has now acquired an overall positive charge, as in the right hand panel. We call this ``charging by induction''. We can also think of this in terms of positive charges moving into the rod from the Earth; when we connect the rod to the ground, the positive charges in the Earth can move into the rod and get close to the negatively charged sphere. If we disconnect the rod from the ground, the rod stays positive, just as we conclude when negative charges move.

\capfig{0.9\textwidth}{figures/ChargesFields/inductioncharge.png}{\label{fig:chargesfields:inductioncharge}Charging by induction: when we connect the polarized rod to the ground, electrons can leave the rod. If we now disconnect the rod from ground, the rod is left with an overall positive charge. }

\section{The Coulomb force}
Coulomb was the first to provide a detailed quantitative description of the force between charged objects. Nowadays, we use the (derived) SI unit of ``Coulomb'' (C) to represent charge. The ``charge'' of an object corresponds to the net excess (or lack) of electrons on the object. An electron has a charge of $-e=\SI{-1.6e-19}{C}$. Thus, an object with a charge of $\SI{-1}{C}$ has an excess of about $\num{1e19}$ electrons on it. If an object has an excess of electrons, it is negatively charged and we indicate this with a negative sign on the charge of the object. An object with a (positive) charge of $\SI{1}{C}$ thus has a deficit of electrons.

Through careful studies of the force between two charged spheres, Coulomb observed\footnote{Others had initially observed the inverse square Law, but Coulomb was the first to formalize the theory.} that:
\begin{itemize}
\item The force is attractive if the objects have opposite charges and repulsive if the objects have the same charge.
\item The force is inversely proportional to the squared distance between spheres.
\item The force is larger if the charges involved are larger. 
\end{itemize}
This leads to Coulomb's Law for the electric force (or simply ``Coulomb's Law''), $\vec F_{12}$, exerted on a point charge $Q_1$ by another point charge $Q_2$:
\begin{align*}
\Aboxed{\vec F_{12}=k\frac{Q_1Q_2}{r^2}\hat r_{21}}
\end{align*}
where $\hat r_{21}$ is the unit vector from $Q_2$ to $Q_1$ and $r$ is the distance between the two charges, as illustrated in Figure \ref{fig:chargesfields:coulombforce}. $k=\SI{8.99e9}{N\cdot m^2/C^{2}}$ is simply a proportionality constant (``Coulomb's constant'') to ensure that the quantity on the right will have units of Newtons when all other quantities are in S.I. units. 
\capfig{0.5\textwidth}{figures/ChargesFields/coulombforce.png}{\label{fig:chargesfields:coulombforce}Vectors involved in applying Coulomb's Law.}
If the two charges have positions $\vec r_1$ and $\vec r_2$, respectively, then the vector $\hat r_{21}$ is given by:
\begin{align*}
\hat r_{21} = \frac{1}{||\vec r_2 - \vec r_1||}\vec r_2 - \vec r_1
\end{align*}
Coulomb's Law is mathematically identical to the gravitational force in Newton's Universal Theory of Gravity. Rather than quantity of mass determining the strength of the gravitational force, it is the quantity of charge that determines the strength of the electric force. The only major difference is that gravity is always attractive, whereas the Coulomb force can be repulsive.

The product $Q_1Q_2$ in the numerator of Coulomb's force is positive if the two charges have the same sign (both positive or both negative) and negative if the charges have opposite signs. Again, referring to Figure \ref{fig:chargesfields:coulombforce}, if the two charges are positive, the force on $Q_1$ will point in the same direction as $\hat r_{21}$ and thus be repulsive. If, instead, the two charges have opposite signs, the product $Q_1Q_2$ will be negative and the force vector on $Q_1$ will point in the opposite direction from $\hat r_{21}$ and the force is attractive. 
\begin{example}{Calculate the magnitude of the electric force between the electron and the proton in a hydrogen atom and compare this to the gravitational force between them.}
We model this by assuming that the electron and proton are point charges a distance of $\SI{1}{\angstrom}=\SI{1e-10}{m}$ apart (1 \r Angstrom). The proton and electron have the same charge with magnitude $e=\SI{1.6e-19}{C}$, so the (attractive) electric force between them has a magnitude of:
\begin{align*}
F^e = k\frac{Q_1}{Q_2}{r^2}=(\SI{9e9}{N\cdot m^2/C^{2}})\frac{(\SI{1.6e-19}{C})(\SI{1.6e-19}{C})}{(\SI{1e-10}{m})^2}=\SI{2.3e-8}{N}
\end{align*}
which is a small number, but acting on a very small mass. In comparison, the force of gravity between an electron () and a proton () is given by:
\begin{align*}
F^g=G\frac{M_1M_2}{r^2}=(\SI{6.67e-11}{Nm^2/kg^2})\frac{(\SI{50}{kg})(\SI{50}{kg})}{(\SI{0.5}{m})^2}=TODO!
\end{align*}


\textbf{Discussion:} TODO. 
\end{example}


\section{The electric field}

\section{The electric dipole}


\newpage
\section{Summary}

\begin{chapterSummary}
 Something that was learned
\end{chapterSummary}

\newpage
\begin{importantEquations}
\medskip
\begin{multicols}{2}
\textbf{Momentum of a point particle:}
\begin{align*}
\vec p = m\vec v \\
\frac{d}{dt}\vec p = \sum \vec F = \vec F^{net}
\end{align*}
\columnbreak
\\
\textbf{Position of the Centre of Mass \\ of a system:}
\begin{align*}
\vec r_{CM} &=\frac{1}{M}\sum_i m_i\vec r_i 
\end{align*}
\medskip
\end{multicols}
\end{importantEquations}

\newpage
\section{Thinking about the material}

\begin{chapteractivity}{Reflect and research}
{
\item Explain
}
\end{chapteractivity}

\begin{chapteractivity}{To try at home}
{
\item Try
}
\end{chapteractivity}

\begin{chapteractivity}{To try in the lab}
{
\item Propose an experiment
}
\end{chapteractivity}

\newpage
\section{Sample problems and solutions}
\subsection{Problems}
\begin{problem}{soln:template:ballistic}{\label{prob:template:ballistic} 

}
\end{problem}
\newpage
\subsection{Solutions}
\begin{solution}{prob:template:ballistic}\label{soln:template:ballistic}

\end{solution}

