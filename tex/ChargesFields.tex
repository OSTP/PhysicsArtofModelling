
\chapter{Electric charges and fields}
\label{chapter:chargesfields}
In this chapter, we start to look at the theories that describe electric phenomena. We will still use the framework for dynamics that was developed by Newton, but will introduce Coulomb's model for the electric force (just as we have introduced Newton's model for the gravitational force). 

\begin{learningObjectives}{
 \item Understand Coulomb's model for the electric force.
 \item Understand the definition of an electric charge.
 \item Understand the definition of an electric field.
 \item Understand how to calculate the electric field from a continuous distribution of charge.
 }
\end{learningObjectives}

\begin{opening}
\begin{MCquestion}{If you rub a balloon against a carpet and bring it near your head, your hair will stand up and try to touch the baloon.}
\item The electric charge of the balloon is opposite of that on your hair.
\item Your hair has no net electric charge, this is an example of charge separation and induction. \correct
\end{MCquestion}
\end{opening}

\section{Electric charge}

\section{The Coulomb force}

\section{The electric field}

\section{The electric dipole}


\newpage
\section{Summary}

\begin{chapterSummary}
 Something that was learned
\end{chapterSummary}

\newpage
\begin{importantEquations}
\medskip
\begin{multicols}{2}
\textbf{Momentum of a point particle:}
\begin{align*}
\vec p = m\vec v \\
\frac{d}{dt}\vec p = \sum \vec F = \vec F^{net}
\end{align*}
\columnbreak
\\
\textbf{Position of the Centre of Mass \\ of a system:}
\begin{align*}
\vec r_{CM} &=\frac{1}{M}\sum_i m_i\vec r_i 
\end{align*}
\medskip
\end{multicols}
\end{importantEquations}

\newpage
\section{Thinking about the material}

\begin{chapteractivity}{Reflect and research}
{
\item Explain
}
\end{chapteractivity}

\begin{chapteractivity}{To try at home}
{
\item Try
}
\end{chapteractivity}

\begin{chapteractivity}{To try in the lab}
{
\item Propose an experiment
}
\end{chapteractivity}

\newpage
\section{Sample problems and solutions}
\subsection{Problems}
\begin{problem}{soln:template:ballistic}{\label{prob:template:ballistic} 

}
\end{problem}
\newpage
\subsection{Solutions}
\begin{solution}{prob:template:ballistic}\label{soln:template:ballistic}

\end{solution}

