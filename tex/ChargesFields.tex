\section{Charges and the electric field}

%%%%%%%%%%%%%%%%%%%%%%%%%%%%%%%%%%%
%%
%% Multiple Choice
%%
%%%%%%%%%%%%%%%%%%%%%%%%%%%%%%%%%%%
\subsection{Multiple Choice}

\question A net electric charge is placed on a solid conducting sphere. How does the charge distribute itself?
\begin{checkboxes}
	\choice It remains where it was placed.
	\choice It migrates to the centre of the sphere. 
	\CorrectChoice It distributes itself uniformly on the surface of the sphere. 
	\choice None of the above.  
\end{checkboxes}

\question An electric dipole is made of two charges, $+Q$ and $-Q$, separated by a distance $l=\SI{1}{cm}$. When placed in an electric field with magnitude $E=\SI{100}{V/m}$, the dipole experiences a maximal torque of magnitude $\tau=\SI{1.3e-3}{N\cdot m}$. What is the magnitude of $Q$?
\begin{checkboxes}
	\choice $Q=\SI{1.3e-6}{C}$
	\choice $Q=\SI{1.3e-5}{C}$
	\choice $Q=\SI{1.3e-4}{C}$
	\CorrectChoice $Q=\SI{1.3e-3}{C}$
\end{checkboxes}

\question  A glass rod is rubbed with a piece of fabric, and the glass rod gains a negative charge. What can be said?
\begin{checkboxes}
	\choice Protons were transferred from the glass to the piece of fabric.
	\choice Electrons were transferred from the glass to the piece of fabric.
	\choice Protons were transferred from the piece of fabric to the glass.
	\CorrectChoice Electrons were transferred from the piece of fabric to the glass.
\end{checkboxes}

\question A sphere of radius $R_1$ made of an insulating material has a total charge $+Q$. A spherical shell (inner radius $R_2>R_1$) made of an insulating material is concentric and completely envelopes the sphere. The shell has a net positive charge $+2Q$. An electron is placed at rest exactly between the sphere and the shell. In which direction does the electron move? 
\begin{checkboxes}
	\CorrectChoice Towards the sphere.
	\choice Towards the shell. 
	\choice The electrons stays at rest. 
	\choice Not enough information to tell. 
\end{checkboxes}

%Sarmud Mahmood
\question  Which of the following is true?
\begin{checkboxes}
	\choice Electric field vectors point in the opposite direction as the electric force that a positive test charge would undergo.
	\choice Electric fields are scalar quantities and thus do not have a vector direction.
	\CorrectChoice Electric field vectors point in the same direction as the force a positive test charge would experience.
	\choice None of the Above 
\end{checkboxes}

\question A sphere of radius $r$ has a net charge $+Q$ distributed uniformly through its volume. Which material could the sphere be made of?
\begin{checkboxes}
\choice Copper
\choice Gold
\choice Superconducting aluminium
\CorrectChoice Plastic \correct
\end{checkboxes}

\question Materials in which the electrons are bound very tightly to the nuclei are referred to as?
\begin{checkboxes}
\choice conductors
\CorrectChoice insulators \correct
\choice semi-conductors
\choice solid
\end{checkboxes}

\question In a region of space, the electric field is given by $\vec E(x)=-ax\hat x$, where $a$ is a constant. A charge $q$ is released near $x=0$, and the only force on the charge is that from the electric field. Which statement is true?
\begin{checkboxes}
\CorrectChoice If $q$ is positive, it will undergo simple harmonic motion \correct
\choice If $q$ is negative, it will undergo simple harmonic motion
\choice Regardless of the sign of $q$, it will undergo simple harmonic motion
\choice Regardless of the sign of $q$, it will not undergo simple harmonic motion
\end{checkboxes}

\question A positively charged rod, $A$, is brought close to (but not touching) one end of a neutral conducting rod, $B$. If a positive charge $q$ is brought close to the other end of rod $B$ (opposite the side with rod $A$), what force will $q$ feel?
\begin{checkboxes}
\choice A force attracting it to rod $B$
\CorrectChoice A force repelling it from rod $B$ \correct
\choice No force since rod $B$ is neutral
\end{checkboxes}


\question Two point charges of equal magnitude but unknown sign are placed close to each other. The force between them will be:
\begin{checkboxes}
\choice Repulsive
\choice Attractive
\choice Zero
\CorrectChoice Not enough information to answer \correct
\end{checkboxes}

%Question submitted by Sam Connolly
\question What happens when a molecule with a permanent electric dipole moment is placed in an electric field?
\begin{checkboxes}
\choice The molecule gains an additional induced dipole
\CorrectChoice Torque generated from the electric field aligns the molecule's dipole with the external field \correct
\choice Nothing happens
\end{checkboxes}

\question Which particles are responsible for excess charge in material?
\begin{checkboxes}
\choice protons
\CorrectChoice electrons \correct
\choice neutrons
\choice photons
\end{checkboxes}


\question Consider a 2D system in which you have a positive charge $+2Q$ located at (-3,0) and another positive charge $+4Q$ located at an unknown position (x,y). If a third charge B is placed at the origin (0,0), where does the $+4Q$ charge have to be located for B to remain stationary?
\begin{checkboxes}
\choice $(-3 \sqrt{2},0)$
\choice $ (0, 6) $
\choice $(\sqrt{2}, 6 )$
\CorrectChoice $(\ 3 \sqrt{2},0)$ \correct
\end{checkboxes}  

%Submitted by Quentin Sanders
\question A positive charge is free to move. Another positive charge is fixed in place one metre to the West of the charge. A negative charge is fixed in place one metre to the South of the first charge. In what direction will the first charge move?
\begin{checkboxes}
\choice North
\choice South
\choice East
\choice West
\choice SW
\CorrectChoice SE \correct
\choice NW
\choice NE
\end{checkboxes}  

\question A positively charged sphere is brought near a neutral conducting metallic rod, as shown in Figure \ref{fig:chargesfields:ChargeInduction}.
\capfig{0.3\textwidth}{figures/ChargesFields/ChargeInduction.png}{\label{fig:chargesfields:ChargeInduction}}
\begin{choices} 
\CorrectChoice There is no electric field in the rod \correct
\choice The total electric field in the rod is to the right
\choice The total electric field in the rod is to the left 
\end{choices}

\question A positively charged sphere is brought near a neutral conducting metallic rod, as shown in Figure \ref{fig:chargesfields:ChargeInduction}. What can you say about charges on the rod?
\capfig{0.3\textwidth}{figures/ChargesFields/ChargeInduction.png}{\label{fig:chargesfields:ChargeInduction}A charged sphere brought close to a neutral conducting rod.}
\begin{choices} 
\CorrectChoice The right side of the rod is positively charged, but the net charge on the rod is zero. \correct
\choice The left side of the rod is positively charged, but the net charge on the rod is zero.
\choice Both sides of the rod are positively charged.
\choice Neither side of the rod is charged.
\end{choices}

\question Three charges ($q$, $2q$, and $-q$) are held fixed at the corners of a right angle isosceles triangle as shown in Figure \ref{fig:chargesfields:3Charges}. Which statement is true?
\capfig{0.2\textwidth}{figures/ChargesFields/3Charges.png}{\label{fig:chargesfields:3Charges} Three charges at the corners of a right angle triangle.}
\begin{choices} 
\choice The net electric force vector is the same on each charge
\choice If released, $q$ would move in the positive $x$ and negative $y$ direction
\CorrectChoice The $x$ component of the net electric force on $q$ is larger in magnitude than the $y$ component of the net electric force on $q$ \correct
\choice If released, $-q$ would move in the positive $y$ direction
\end{choices}

\question A charge $q_1=1$\,nC (1\,nC=$10^{-9}$\,C) is placed at the origin of a coordinate system. A charge $q_2=0.5$\,nC is placed at a position $x=0.5$\,m and $y=0.5$\,m. What is the force vector on $q_1$?
\begin{checkboxes}
\choice $\vec F = (9.00, 9.00, 0)\times10^{-9}$\,N 
\choice $\vec F = (-9.00, -9.00, 0)\times10^{-9}$\,N 
\choice $\vec F = (6.36, 6.36, 0)\times10^{-9}$\,N 
\CorrectChoice $\vec F = (-6.36, -6.36, 0)\times10^{-9}$\,N \correct
\end{checkboxes}

\question An electron ($m=9.11\times 10^{-31}$\,kg, $q=-1.6\times 10^{-19}$\,C) is in a circular orbit with radius $r=1.0\times 10^{-10}$\,m around a proton ($m=1.67\times 10^{-27}$\,kg, $q=1.6\times 10^{-19}$\,C), much like in a hydrogen atom. Assuming that only the Coulomb force from the proton is acting on the electron and that the proton is held fixed in space, what is the frequency of the electron's orbit?
\begin{checkboxes}
\choice 2.53$\times 10^{10}$\,Hz 
\choice 5.91$\times 10^{13}$\,Hz\
\CorrectChoice 2.53$\times 10^{15}$\,Hz\ \correct
\choice 3.58$\times 10^{15}$\,Hz
\end{checkboxes}

\question An electric dipole is made from two (opposite) charges with magnitude, $Q$, separated by a distance, $l$, and placed such that the centre of the dipole is at the origin of a coordinate system. At a distance $R >> l$ from the origin, what can be said of the magnitude of the electric field from the dipole?
\begin{checkboxes}
\CorrectChoice It is weaker than that from a single positive charge $Q$ placed at the origin \correct
\choice It is stronger than that from a single positive charge $Q$ placed at the origin
\choice It is zero, because no net charge is enclosed by a gaussian sphere of radius $R$
\choice It is the same strength as the field from a single charge $2Q$ placed at the origin
\end{checkboxes}

\question An electric dipole is made from two (opposite) charges with magnitude $Q=42.0$\,C separated by a distance $l=2.0$\,m, and placed at the origin of a coordinate system. The origin is half-way between the two charges, the dipole is aligned along the x-axis such that the positive charge is at $x=+1.0$\,m. How much work is required to bring a charge $q=-1.6\times 10^{-19}$\,C from infinity to the origin?
\begin{checkboxes}
\choice 1.344 $\times 10^{-17}$\,J
\CorrectChoice 0\,J \correct
\choice -1.344 $\times 10^{-17}$\,J
\choice -1.344 $\times 10^{-17}$\,eV
\end{checkboxes}



%%%%%%%%%%%%%%%%%%%%%%%%%%%%%%%%%%%
%
% long answer
%
%%%%%%%%%%%%%%%%%%%%%%%%%%%%%%%%%%%
\subsection{Long answers}
%Giancolli - not sure which one...
\question Four charges of magnitude $+q$ are located at each corner of a square which has sides of length $d$. A charge $Q$ is placed at a distance $b$ orthogonal to the plane which the square lies on and at an equal distance from each charge at the corners of the square. What is the magnitude of the net force exerted by the four charges on the charge $Q$?

\begin{finalanswer}
\begin{align*}
F_{4q}&=4F_q=\frac{1}{\pi\epsilon_0}\frac{qQb}{\left(b^2+\left( \frac{d^2}{2}\right)\right)^{\frac{3}{2}}}
\end{align*}
\end{finalanswer}
\begin{solution}
Figure \ref{fig:chargesfields:OneQuarter} shows the distance between the charge $Q$ and any one of the four charges on the square. 
\capfig{0.35\textwidth}{figures/ChargesFields/OneQuarter.png}{\label{fig:chargesfields:OneQuarter}Distance between $Q$ and one of the charges.}
By symmetry, the components of the force that are not parallel to the segment of length $b$ will cancel. The component of the force from one charge that is parallel to the segment of length $b$ is thus given by:
\begin{align*}
F_q&=\frac{1}{4\pi\epsilon_0}\frac{qQ}{b^2+\left( \frac{d^2}{2}\right)}\cos\theta\\
&=\frac{1}{4\pi\epsilon_0}\frac{qQ}{b^2+\left( \frac{d^2}{2}\right)}\frac{b}{\sqrt{b^2+\left( \frac{d^2}{2}\right)}}\\
&=\frac{1}{4\pi\epsilon_0}\frac{qQb}{\left(b^2+\left( \frac{d^2}{2}\right)\right)^{\frac{3}{2}}}
\end{align*}
The total force is simply the result above multiplied by 4:
\begin{align*}
F_{4q}&=4F_q=\frac{1}{\pi\epsilon_0}\frac{qQb}{\left(b^2+\left( \frac{d^2}{2}\right)\right)^{\frac{3}{2}}}
\end{align*}
\end{solution}

% past 106 problem, also Haliday & Resnick
\question A thin glass rod is bent into a semicircle of radius $a$. A charge $+Q$ is uniformly distributed along the upper half, and a charge $-Q$ is uniformly distributed along the lower half, as shown in Figure \ref{fig:chargesfields:HalfCircle}. Find the electric field $\vec E$ (magnitude and direction) at point $P$ (the centre of the circle).
\capfig{0.25\textwidth}{figures/ChargesFields/HalfCircle.png}{\label{fig:chargesfields:HalfCircle}A glass rod bent into a half circle with charges distributed on it.}
\begin{finalanswer}
\begin{align*}
\vec E=-\frac{2k\lambda}{a}\hat y
\end{align*}
where we have also expressed the result in terms of $k=\frac{1}{4\pi\epsilon_0}$ or $\lambda$.
\end{finalanswer}
\begin{solution}
We can divide the glass rod into small segments of charge $dq$ and length $dl$, and sum (integrate) the electric field contributions from each charge element. The charge elements will be positive in the top half and negative in the bottom half. Figure \ref{fig:chargesfields:HalfCircle_Sol} shows that for each charge element on the top half, there is a charge element on the bottom half that creates an electric field with a horizontal component that cancels the one from the top half. 
\capfig{0.35\textwidth}{figures/ChargesFields/HalfCircle_Sol.png}{\label{fig:chargesfields:HalfCircle_Sol}Two symmetrically chosen charge elements whose horizontal components of electric field cancel.}
We choose to label each charge element by the angle that the vector from $P$ to the charge element makes with the horizontal ($\theta$ in the figure). Each charge element will have a charge equal to:
\begin{align*}
dq&=\lambda dl=\lambda ad\theta
\end{align*}
where $\lambda$ is the charge per unit length on the glass, and is given by:
\begin{align*}
\lambda=\frac{Q}{\frac{2\pi a}{4}}=\frac{2Q}{\pi a}
\end{align*}
Since the horizontal components of the electric field will cancel, we only need to consider the vertical components, which point downwards:
\begin{align*}
dE_y&=-\frac{1}{4\pi\epsilon_0}\frac{dq}{r^2}\sin\theta\\
&=-\frac{1}{4\pi\epsilon_0}\frac{\lambda ad\theta}{a^2}\sin\theta \\
&=-\frac{1}{4\pi\epsilon_0}\frac{2Q ad\theta}{\pi a a^2}\sin\theta \\
&=-\frac{1}{4\pi\epsilon_0}\frac{2Q}{\pi a^2}\sin\theta d\theta \\
\end{align*}
which will point downward for each segment of the half circle. The magnitude of the vertical component of the electric field will be the same for both segments, so we can calculate it for the positive segment and double the result. The vertical component of the electric field for the positive segment is given by:
\begin{align*}
E_y^{+Q} &= \int dE_y=-\int_{0}^{+\frac{\pi}{2}}\frac{1}{4\pi\epsilon_0}\frac{2Q}{\pi a^2}\sin\theta d\theta \\
&=-\frac{1}{4\pi\epsilon_0}\frac{2Q}{\pi a^2}\int_{0}^{+\frac{\pi}{2}}\sin\theta d\theta \\
&=-\frac{1}{4\pi\epsilon_0}\frac{2Q}{\pi a^2}[-\cos\theta ]_{0}^{+\frac{\pi}{2}}\\
&=-\frac{1}{4\pi\epsilon_0}\frac{2Q}{\pi a^2}
\end{align*}
The total electric field is thus given by:
\begin{align*}
\vec E=2E_y^{+Q}\hat y&=-\frac{1}{\pi^2\epsilon_0}\frac{Q}{a^2}\hat y\\
&=-\frac{4kQ}{\pi a^2}\hat y\\
&=-\frac{2k\lambda}{a}\hat y
\end{align*}
where we have also expressed the result in terms of $k=\frac{1}{4\pi\epsilon_0}$ or $\lambda$.
\end{solution}

%source? Giancolli?
\question An electron is constrained to move along the central axis of a ring which has a radius $R$ and a net charge $Q$ that is uniformly distrubuted.
\begin{parts}
\part Show that the electron will experience a force of the form $F_z = -kz$ when it is near the centre of the ring (where $z$ is the distance from the centre of ring and is small compared to $R$).
\part Derive a formula for the frequency of the oscillation that occurs in part (a)
\end{parts}
\textbf{Hint:} Note that for $z<<R$:
\begin{align*}
\frac{1}{z^2+R^2}=\frac{1}{R^2\left(\frac{z^2}{R^2}+1 \right)}\sim\frac{1}{R^2}
\end{align*}
since $\frac{z^2}{R^2}<<1$
\begin{finalanswer}
\begin{enumerate}[(a)]
\item N/A
\item \begin{align*}
f =\frac{1}{2\pi}\sqrt{\frac{Qe}{4\pi\epsilon_0R^3m_e}}
\end{align*}
where $m_e$ is the mass of the electron.
\end{enumerate}
\end{finalanswer}
\begin{solution}
\begin{parts}
\part Figure \ref{fig:chargesfields:ElectronRing} shows a the ring and a small element of charge, $dq$, with a length $dl$. If the charge per unit length on the ring is $\lambda$, then $dq$ is given by:
\begin{align*}
dq=\lambda dl=\frac{Q}{2\pi R}dl
\end{align*}
\capfig{0.40\textwidth}{figures/ChargesFields/ElectronRing.png}{\label{fig:chargesfields:ElectronRing} Ring of radius $R$ and charge $Q$.}
The total electric field will point in the positive $z$ direction, as for each section of the ring, there will be another section of the ring whose electric field cancels the $x$ and $y$ components. We thus only need to consider the $z$ components of the electric field. For a charge element $dq$, the electric field at a distance $z$ from the origin is given by:
\begin{align*}
dE_z&=\frac{1}{4\pi\epsilon_0}\frac{dq}{r^2}\cos\theta\\
&=\frac{1}{4\pi\epsilon_0}\frac{\lambda dl}{r^2}\frac{z}{r}\\
\end{align*}
If we sum (integrate) the electric field contributions from all of the elements of the ring we obtain the total electric field at $z$. The integral is trivial, since non of the variables above change along the ring. We label each charge $dq$ by their position $l$ along the circumference of the ring, and integrate $l$ from 0 to the circumference of the ring:
\begin{align*}
E_z&=\int dE_z=\int_0^{2\pi R}\frac{1}{4\pi\epsilon_0}\frac{\lambda z}{r^3}dl\\
&=\frac{1}{4\pi\epsilon_0}\frac{\lambda z}{r^3}\int_0^{2\pi R}dl\\
&=\frac{1}{4\pi\epsilon_0}\frac{\lambda z}{r^3}(2\pi R)\\
&=\frac{1}{4\pi\epsilon_0}\frac{Q}{2\pi R}\frac{z}{r^3}(2\pi R)\\
&=\frac{Q}{4\pi\epsilon_0}\frac{1}{r^3}z\\
&=\frac{Q}{4\pi\epsilon_0}\frac{1}{(R^2+z^2)^\frac{3}{2}}z\\
&=\frac{Q}{4\pi\epsilon_0}\frac{1}{R^3(1+\frac{z^2}{R^2})^\frac{3}{2}}z\\
\end{align*}
If $z<<R$, then $\frac{z}{R}<<1$ and $1+\frac{z^2}{R^2}\sim 1$, and we can write the force on the electron as:
\begin{align*}
F=-eE_Z&\sim -\frac{Qe}{4\pi\epsilon_0R^3}z\\
&=-k_sz
\end{align*}
where $e$ is the charge of the electron, and we have introduced an effective spring constant $k_s$:
\begin{align*}
k_s =\frac{Qe}{4\pi\epsilon_0R^3} 
\end{align*}
\part The frequency of the simple harmonic oscillator is given by:
\begin{align*}
f &= \frac{1}{2\pi} \omega= \frac{1}{2\pi}\sqrt{\frac{k_s}{m_e}}\\
&=\frac{1}{2\pi}\sqrt{\frac{Qe}{4\pi\epsilon_0R^3m_e}}
\end{align*}
where $m_e$ is the mass of the electron.
\end{parts}
\end{solution}

%original...
\question A charge $-q$ is held fixed in space. An electron is placed a distance, $d$, from the fixed charge, and released. The electron then accelerates away from the charge. What speed does the electron have:
\begin{parts}
\part when it reaches a distance $D$ ($D>d$)?
\part when it reaches infinity?
\end{parts}
\begin{finalanswer}
\begin{enumerate}[(a)]
\item \begin{align*}
v=\sqrt{\frac{qe}{2\pi\epsilon_0m_e}\left(\frac{1}{d}-\frac{1}{D} \right)}
\end{align*}
\item \begin{align*}
v=\sqrt{\frac{qe}{2\pi\epsilon_0m_e}\frac{1}{d}}
\end{align*}
\end{enumerate}
\end{finalanswer}
\begin{solution}
\begin{parts}
\part In order to find the electron's speed, we calculate the work done by the electric force. We choose a coordinate system such that the charge $-q$ lies at the origin, and the electron is initially at a position $x=d$ on the $x$ axis. We can then calculate the work done as the electron moves to $x=D$. The force on the electron is directed in the positive $x$ direction:
\begin{align*}
W &= \int_d^D\vec F \cdot d\vec x=\int_d^D \frac{qe}{4\pi\epsilon_0}\frac{1}{x^2}\hat x \cdot d\vec x\\
&=\frac{qe}{4\pi\epsilon_0}\int_d^D \frac{1}{x^2}d x=\frac{qe}{4\pi\epsilon_0}\left[-\frac{1}{x} \right]_d^D\\
&=\frac{qe}{4\pi\epsilon_0}\left(\frac{1}{d}-\frac{1}{D} \right)
\end{align*}
which is positive as expected (since $D>d$). The speed can then be found by work-energy theorem:
\begin{align*}
W&=\Delta K=\frac{1}{2}m_ev^2\\
\therefore v&=\sqrt{\frac{2W}{m_e}}\\
&=\sqrt{\frac{2}{m_e}\frac{qe}{4\pi\epsilon_0}\left(\frac{1}{d}-\frac{1}{D} \right)}\\
&=\sqrt{\frac{qe}{2\pi\epsilon_0m_e}\left(\frac{1}{d}-\frac{1}{D} \right)}\\
\end{align*}
\part In the limit ${D\to\infty}$, $\frac{1}{D}\to 0$, and the speed of the electron is found to be:
\begin{align*}
v=\sqrt{\frac{qe}{2\pi\epsilon_0m_e}\frac{1}{d}}
\end{align*}
\end{parts}
\end{solution}


%Giancolli - not sure which one...
\question Consider the square charge distribution shown in Figure \ref{fig:chargesfields:SquareCharge}, with $q=\SI{2.6e-8}{C}$ and $a=\SI{1.5}{m}$.
\begin{parts}
\part What is the force on the charge $q$ due to the other three charges (magnitude and direction)?
\part What is the electric field at the centre of the square?
\end{parts}
\capfig{0.25\textwidth}{figures/ChargesFields/SquareCharge.png}{\label{fig:chargesfields:SquareCharge}Charges arranged on a square.}
\begin{finalanswer}
\begin{enumerate}[(a)]
\item $F=\SI{1.60e-5}{N}$, $\phi=\SI{121.16}{\degree}$
\item \SI{587.66}{N/C}
\end{enumerate}
\end{finalanswer}
\begin{solution}
\begin{parts}
\part We define a coordinate system whose origin is at $q$, with positive $y$ upwards and postive $x$ to the right. The force vectors from each other charge on $q$ are given by:
\begin{align*}
\vec F_{2q} &= \frac{k2q^2}{a^2}(-\hat x)\\
\vec F_{2q} &= \frac{k3q^2}{2a^2}\frac{1}{\sqrt 2}(-\hat x+\hat y)\\
\vec F_{4q} &= \frac{k4q^2}{a^2}(+\hat y)\\
\end{align*}
The total force is found by adding these vectors together:
\begin{align*}
\vec F &= \left(\frac{kq^2}{a^2}\right)\left[ -\left(2+ \frac{3}{2\sqrt 2}\right)\hat x +\left(4+ \frac{3}{2\sqrt 2}\right)\hat y  \right] \\
&=	(\SI{-8.27e-6}{N})\hat x +(\SI{1.37e-5}{N})\hat y
\end{align*}
The magnitude, and direction with respect to the positive $x$ axis are:
\begin{align*}
F &= \SI{1.60e-5}{N} \\
\phi &= \SI{90}{\degree}+\tan^{-1}\left(\frac{8.27}{13.7}\right)=\SI{121.16}{\degree}
\end{align*}
\part By symmetry, the $x$ component of the electric field will cancel (there are $5q$ on either side of the centre). The $y$ component is given by:
\begin{align*}
E_y &=\frac{2kq}{\sqrt 2 a^2}(-1-2+3+4)\\
&=\frac{8kq}{\sqrt 2 a^2}\\
&=\SI{587.66}{N/C}
\end{align*}
\end{parts}
\end{solution}

%classic...
\question Determine the electric field at a point which is a distance, $z$, above an infinite plane with surface charge density $\sigma$ (in \si{C/m^2}). 

\textbf{Hint:} You may think of the plane as a sum of line charges. You can use the result that the magnitude of the electric field a distance $r$ away from an infinite line of charge with linear charge density $\lambda$ is given by:
\begin{align*}
E(r) = \frac{\lambda}{2\pi\epsilon_0 r}
\end{align*}
You may also need the following anti-derivative:
\begin{align*}
\int \frac{1}{x^2+a^2}dx=\frac{1}{a}\tan^{-1}\left( \frac{x}{a}\right) + C
\end{align*}
\begin{finalanswer}
$E=\sigma/(2\epsilon_0)$
\end{finalanswer}
\begin{solution}
We break up the plane into an infinite number of infinitely long line elements of width $dy$, as shown in Figure \ref{fig:chargesfields:InfinitePlane_Sol}. If we choose a particular line element as shown in the figure, it will create an electric field in the $yz$ plane. However, for each position $y$ of a line element, a line element at $-y$ will create an electric field whose $y$ component cancels the first line element. Thus, the total field will be in the $z$ direction, and we only need to consider the $z$ component of the field from each line element.
\capfig{0.4\textwidth}{figures/ChargesFields/InfinitePlane_Sol.png}{\label{fig:chargesfields:InfinitePlane_Sol}Electric field from a line element of the surface.}
The $z$ component of the electric field at point $P$ on the $z$ axis that comes from a line element at position $y$ is given by:
\begin{align*}
dE_z = \frac{\lambda}{2\pi\epsilon_0 r}\cos\theta
\end{align*}
where $r$ is the distance to the line element, $\theta$ is the angle between the vector from $P$ to the line element and the $z$ axis, and $\lambda$ is the charge per unit length of the line element. Since we are only given the surface charge density of the plane, $\sigma$, and not the linear density, we need to convert one to the other. Suppose that we have a finite line element of length $L$ and width $dy$; the charge, $dq$, on that element will be given by its area ($Ldy$) times the surface charge density:
\begin{align*}
dq = \sigma Ldy
\end{align*}
The charge per unit length is then that charge divided by the length of the element:
\begin{align*}
\lambda = \frac{dq}{L}=\sigma dy
\end{align*}
The $z$ component of the field from the line element is thus:
\begin{align*}
dE_z &= \frac{\sigma dy}{2\pi\epsilon_0 r}\cos\theta \\
&=\frac{\sigma}{2\pi\epsilon_0} \frac{z}{r^2} dy\\
&=\frac{\sigma}{2\pi\epsilon_0} \frac{z}{y^2+z^2} dy
\end{align*}
We can now sum all the contributions from $y=-\infty$ to $y=+\infty$ to get the total electric field:
\begin{align*}
E_z &= \int dE_z=\frac{\sigma z}{2\pi\epsilon_0}\int_{-\infty}^{+\infty} \frac{1}{y^2+z^2} dy\\
&=\frac{\sigma z}{2\pi\epsilon_0}\left[ \frac{1}{z}\tan^{-1}\left(\frac{y}{z}\right) \right]_{-\infty}^{+\infty}\\
&=\frac{\sigma}{2\pi\epsilon_0}\left[ \tan^{-1}(+\infty) - \tan^{-1}(-\infty) \right]\\
&=\frac{\sigma}{2\pi\epsilon_0}\left[ \frac{\pi}{2}- \frac{-\pi}{2} \right]\\
&=\frac{\sigma}{2\epsilon_0}
\end{align*}
which does not depend on how far one is from the plane.
\end{solution}

%Giancolli 21-62
\question A dipole consisting of charges $+e$ and $-e$ separated by \SI{0.85}{nm} is in an electric field with strength $E =\SI{4.2e4}{N/C}$.
\begin{parts}
\part What is the magnitude of the dipole moment?
\part What is the torque on the dipole when its dipole moment is perpendicular to the electric field?
\part What is the torque on the dipole when its dipole moment makes an angle of 45 degrees with respect to the electric field?
\part What is the work required to rotate the dipole from being oriented parallel to the field to being antiparallel to the field?
\end{parts}
\begin{finalanswer}
\begin{enumerate}[(a)]
\item $p=\SI{1.360e-28}{Cm}$
\item $\tau=\SI{5.712e-23}{Nm}$
\item $\tau=\SI{4.039e-23}{Nm}$
\item $W=\SI{1.142e-22}{J}$
\end{enumerate}
\end{finalanswer}
\begin{solution}
\begin{parts}
\part We simply multiply the magnitude of the charge by the distance between them:
\begin{align*}
p=ql=(\SI{1.6e-19}{C})(\SI{0.85e-9}{m})=\SI{1.360e-28}{Cm}
\end{align*}
\part The torque on the dipole is given by:
\begin{align*}
\tau = \vec p \times \vec E=pE\sin\theta=Ep=(\SI{4.2e5}{N/C})(\SI{1.36e-28}{Cm})=\SI{5.712e-23}{Nm}
\end{align*}
where $\sin(\SI{90}{\degree})=1$.
\part The torque on the dipole is given by:
\begin{align*}
\tau = \vec p \times \vec E=pE\sin\theta=Ep=(\SI{5.712e-23}{Nm})\frac{1}{\sqrt{2}}=\SI{4.039e-23}{Nm}
\end{align*}
where $\sin(\SI{45}{\degree})=\frac{1}{\sqrt{2}}$.
\part We can find this easily by calculating the change in potential energy:
\begin{align*}
U=-\vec p\cdot\vec E=-pE\cos\theta
\end{align*}
When the dipole is parallel, the angle is $\theta=\SI{0}{\degree}$, and when anti-parallel, it is $\theta=\SI{180}{\degree}$. The work done, is the final minus the initial potential energy:
\begin{align*}
W&=\Delta U=-pE\cos(\SI{180}{\degree}) + pE\cos(\SI{0}{\degree})=2pE\\
&=2(\SI{5.712e-23}{Nm})=\SI{1.142e-22}{J}
\end{align*}
which is positive, as expected (we need to provide energy to displace the dipole from its equilibrium). 
\end{parts}
\end{solution}


% Giancolli 21-67. Fixed, but I may add a diagram to this question. It took a few read-throughs to really understand what it was asking.
\question A dipole has two charges seperated by a distance $l$. Show that at a point along the axis of the dipole (the line which passes through $+Q$ and $-Q$) the electric field has the magnitude:
\begin{align*}
E=\frac{2p}{4\pi\epsilon_0d^3}
\end{align*}
Where $p$ is the dipole moment, $d$ is the distance from a point along the axis of the dipole to the centre of the dipole, and $d >> l$
\begin{solution}
We can introduce an $x$ axis such that the origin is at the centre of the dipole, the positive charge is at $x=+\frac{l}{2}$ and the negative charge is located at $x=+\frac{l}{2}$. The electric field at some point $x=d$ ($d>0$) is given by summing the electric fields from the negative and positive charges. The electric field will be parallel to the $x$-axis, so we only calculate its $x$ component:
\begin{align*}
E_x&=E_{+Q}+E_{-Q}=\frac{1}{4\pi\epsilon_0}\left( \frac{Q}{\left(d-\frac{l}{2} \right)^2}+\frac{-Q}{\left(d+\frac{l}{2} \right)^2}\right)\\
&=\frac{Q}{4\pi\epsilon_0}\left( \frac{\left(d+\frac{l}{2} \right)^2-\left(d-\frac{l}{2} \right)^2}{\left(d+\frac{l}{2} \right)^2\left(d-\frac{l}{2} \right)^2}\right)\\
&=\frac{Q}{4\pi\epsilon_0}\left( \frac{d^2+dl+\frac{1}{4}l^2-d^2+dl-\frac{1}{4}l^2}{\left(d^2+dl+\frac{1}{4}l^2\right)\left(d^2-dl+\frac{1}{4}l^2\right)}\right)\\
&=\frac{Q}{4\pi\epsilon_0}\left( \frac{2dl}{\left(d^2+dl+\frac{1}{4}l^2\right)\left(d^2-dl+\frac{1}{4}l^2\right)}\right)\\
&=\frac{Q}{4\pi\epsilon_0}\left( \frac{2dl}{d^2\left(1+\frac{l}{d}+\frac{1}{4}\frac{l^2}{d^2}\right)d^2\left(1-\frac{l}{d}+\frac{1}{4}\frac{l^2}{d^2}\right)}\right)\\
&\sim \frac{Q}{4\pi\epsilon_0}\left( \frac{2dl}{d^4}\right)\\
&=\frac{1}{4\pi\epsilon_0}\frac{2p}{d^3}
\end{align*}
where we took the limit that $\frac{l}{d}<<1$ and recognized the $Ql=p$. The electric field points in the same direction as the dipole, which is true on either side of the $x$ axis (when $d<0$).
\end{solution}

%Giancolli 21- 64
\question A dipole is made up of two positive charges $Q$.
\begin{parts}
\part Show that the electric field on the \textbf{perpendicular bisector} of the dipole is given by:
\begin{align*}
E=\frac{2Q}{4\pi\epsilon_0r^2}
\end{align*}
Where $r$ is the distance from the centre of the dipole, $l$ is the distance between the two charges, and $r>>l$.
\part Why does the electric field in this dipole decrease by a factor of $r^2$? (The electric field decreases by a factor of $r^3$ in a typical dipole)
\end{parts}
\begin{finalanswer}
\begin{enumerate}[(a)]
\item N/A
\item Since both charges have the same sign, a long distance away, they will look like a single charge with charge $2Q$.
\end{enumerate}
\end{finalanswer}
\begin{solution}
\begin{parts}
\part Figure \ref{fig:chargesfields:DipoleSameCharge_Sol} shows that only the vertical components of the electric field will contribute, and that the electric field will thus be vertical. 
\capfig{0.2\textwidth}{figures/ChargesFields/DipoleSameCharge_Sol.png}{\label{fig:chargesfields:DipoleSameCharge_Sol}Electric field from a dipole with both charges positive and equal.}
The total electric field along the bisector is given by summing the vertical (and equal) contribution from each charge:
\begin{align*}
E&=2E_1\cos\theta=\frac{2Q}{4\pi\epsilon_0\left(r^2+\frac{l^2}{4} \right)}\cos\theta\\
&=\frac{2Q}{4\pi\epsilon_0\left(r^2+\frac{l^2}{4} \right)}\frac{r}{\left(r^2+\frac{l^2}{4} \right)^{\frac{1}{2}}}=\frac{2Q}{4\pi\epsilon_0}\frac{r}{\left(r^2+\frac{l^2}{4} \right)^{\frac{3}{2}}}\\
&=\frac{2Q}{4\pi\epsilon_0}\frac{r}{r^3\left(1+\frac{l^2}{4r^2} \right)^{\frac{3}{2}}}\\
&\sim \frac{2Q}{4\pi\epsilon_0r^2}
\end{align*}
where in the last line, we took the limit that $r>>l$ (so that $\frac{l}{r}<<1$). 
\part Since both charges have the same sign, a long distance away, they will look like a single charge with charge $2Q$.
\end{parts}
\end{solution}

%original...
\question A dipole is made from two small oppositely charged spheres each carrying a charge with magnitude $Q=\SI{2e-2}{C}$, and separated by a distance of $l=\SI{1}{cm}$. The horizontal dipole is held at its centre of mass and is free to rotate in the horizontal plane about a vertical axis through its centre of mass. The dipole has a moment of inertia $I=\SI{0.001}{kgm^2}$. The dipole is placed in a uniform horizontal electric field of magnitude $E=\SI{1000}{N/C}$.
\begin{parts}
\part Show that for small angles between the electric field vector and the dipole vector, the dipole will experience simple harmonic motion.
\part What is the frequency of the simple harmonic oscillations?
\end{parts}
\begin{finalanswer}
\begin{enumerate}[(a)]
\item
\end{enumerate}
\end{finalanswer}
\begin{solution}
\begin{parts}
\part The only net torque on the dipole is from the force from the electric field:
\begin{align*}
\tau = -pE\sin\theta 
\end{align*}
where we have inserted a minus sign to indicate that this is a restoring torque, in the opposite direction of increasing angle $\theta$. The net torque is equal to the moment of inertia times the angular acceleration:
\begin{align*}
-pE\sin\theta &= I\alpha\\
\therefore \alpha &= -\frac{pE}{I}\sin\theta\sim-\frac{pE}{I}\theta
\end{align*}
where in the last equality, we made the small angle approximation ($\sin\theta\sim\theta$). This has the form for simple harmonic motion:
\begin{align*}
\frac{d^2\theta}{dt^2}&=-\omega^2 \theta\\
\omega &=\sqrt{\frac{pE}{I}}
\end{align*}
\part The frequency of the oscillations is given by:
\begin{align*}
f&=\frac{\omega}{2\pi}=\frac{1}{2\pi}\sqrt{\frac{pE}{I}}\\
&=\frac{1}{2\pi}\sqrt{\frac{QlE}{I}}=\frac{1}{2\pi}\sqrt{\frac{(\SI{2e-2}{C})(\SI{0.01}{m})(\SI{1000}{N/C})}{(\SI{0.001}{kgm^2})}}\\
&=\SI{2.25}{Hz}
\end{align*}
\end{parts}
\end{solution}

%original, can also ask for potential
\question \label{q:chargesfields:E_field_from_rod} A rod made of an insulating material has a length, $L$, and carries a total charge $+Q$. The rod is placed along the x-axis, such that one end is at the origin and the other end is at a position $x=-L$ (Figure \ref{fig:chargesfields:ChargedRod}). Give an expression for the electric field vector at a point $x>0$ along the x-axis, as shown.
\capfig{0.3\textwidth}{figures/ChargesFields/ChargedRod.png}{\label{fig:chargesfields:ChargedRod} Charged rod of length $L$ carrying charge $Q$.}

\textbf{Hint:} If you need to integrate $\int_a^b (u+c)^n du$, where $c$ is a constant, you can use substitution: for example, let $v = u+c$ (so that $dv=du$), change the limits of the integral, and integrate $\int_{a+c}^{b+c}v^ndv$.
\begin{finalanswer}
\begin{align*}
E = k\frac{Q}{L}\left[\frac{1}{x}-\frac{1}{L+x}\right]
\end{align*}
\end{finalanswer}
\begin{solution}
The field is in the positive x-direction, so we calculate its magnitude. Let $a=0$ at $x=0$ and $a=+L$ when $x=-L$ (positive $a$ in the negative $x$ direction), and use $a$ as our variable of integration:
\begin{align*}
E = \int_{0}^{L}k\frac{\lambda da}{(a+x)^2}
\end{align*}
Integrate by letting $b=a+x$, $da=db$, change limits of integration:
\begin{align*}
E = \int_{x}^{L+x}k\frac{\lambda db}{b^2}=k\lambda \left[-\frac{1}{b}\right]_{x}^{L+x}=k\frac{Q}{L}\left[\frac{1}{x}-\frac{1}{L+x}\right]
\end{align*}

\end{solution}




\question Four uniformly electrically charged rods of length $2a$ are arranged in a square as shown in Figure \ref{fig:ChargesFields:FourRods}. Two of the rods carry a total positive charge $Q$, while two of the rods carry a negative charge $-Q$. What is the electric field vector at the centre of the square (use the given coordinate system)?
\capfig{0.3\textwidth}{figures/ChargesFields/FourRods.png}{\label{fig:ChargesFields:FourRods} Four uniformly charged rods arranged in a square.}
\begin{finalanswer}
	$E = \frac{2\sqrt{2}kQ}{aL}\hat x - \frac{2\sqrt{2}kQ}{aL}\hat y=\frac{\sqrt{2}kQ}{L^2}(\hat x - \hat y)$
\end{finalanswer}
\begin{solution}
	In order to determine the total electric field at the centre, we can determine the magnitude of the field from any one of the rods and then vectorially add the fields together. Figure \ref{fig:ChargesFields:Rod_dq} shows a diagram for calculating the electric field from a rod of length $2a$ carrying positive charge.
	\capfig{0.3\textwidth}{figures/ChargesFields/Rod_dq.png}{\label{fig:ChargesFields:Rod_dq} Calculating the field a distance $a$ from the centre of a charged rod of length $2a$.}
	The net electric field will be in the $x$ direction:
	\begin{align*}
	\vec E=E_x\hat x=\int_0^Q\frac{kdq}{r^2}\cos\theta \hat x
	\end{align*}
	We can write the charge element, $dq$, as:
	\begin{align*}
	dq=\frac{Q}{L}dy=\frac{Q}{L}\frac{dy}{d\theta}{d\theta}
	\end{align*}
	where $dy$ is a small length element along the rod and $d\theta$ is the corresponding angle that is subtended. We can write $y$ (the vertical position of $dq$) in terms of $\theta$:
	\begin{align*}
	y&=a\tan\theta\\
	\therefore\frac{dy}{d\theta}=\frac{a}{\cos^2\theta}
	\end{align*}
	So that the charge element becomes:
	\begin{align*}
	dq=\frac{Q}{L}\frac{a}{\cos^2\theta}d\theta
	\end{align*}
	Note that the distance $r$ from the charge to the point where we calculate the electric field is given by:
	\begin{align*}
	r&=\frac{a}{\cos\theta}\\
	\therefore\frac{1}{r^2}&=\frac{\cos^2\theta}{a^2}
	\end{align*}
	Putting this altogether (and changing $L=2a$):
	\begin{align*}
	E_x &= \int_0^Q\frac{kdq}{r^2}\cos\theta = \frac{kQ}{aL}\int_{-\frac{\pi}{4}}^{\frac{\pi}{4}}\cos\theta d\theta\\
	&=\frac{kQ}{aL}(\sin(\pi/4)-\sin(-\pi/4))=\frac{2kQ}{\sqrt{2}aL}=\frac{\sqrt{2}kQ}{2L^2}
	\end{align*}
	We can now add the electric field from the four charged rods together. Both the positive and negative vertical rods will create a field in the positive $x$ direction (each with the magnitude above), whereas the two horizontal rods will create fields in the negative $y$ direction. The total electric field is thus:
	\begin{align*}
	\vec E = \frac{2\sqrt{2}kQ}{aL}\hat x - \frac{2\sqrt{2}kQ}{aL}\hat y=\frac{\sqrt{2}kQ}{L^2}(\hat x - \hat y)
	\end{align*}
\end{solution}

\question Two uniformly electrically charged rods of length $2a$ are arranged perpendicular to each other with their ends touching, as shown in Figure \ref{fig:ChargesFields:TwoRods}. One rod carries a positive charge $Q$, while the other rod carries a negative charge $-Q$. What is the electric field vector at point $P$ (shown) which is a distance $a$ from the midpoint of each of the two rods?
\capfig{0.3\textwidth}{figures/ChargesFields/TwoRods.png}{\label{fig:ChargesFields:TwoRods} Two uniformly charged rods that are perpendicular to each other.}
\begin{finalanswer}
	$\vec E = \frac{\sqrt{2}kQ}{aL}\hat x + \frac{\sqrt{2}kQ}{aL}\hat y=\frac{\sqrt{2}kQ}{2L^2}(\hat x + \hat y)$
\end{finalanswer}
\begin{solution}
	In order to determine the total electric field at point P, we can determine the magnitude of the field from any one of the rods and then vectorially add the fields together. Figure \ref{fig:ChargesFields:Rod_dq} shows a diagram for calculating the electric field from a rod of length $2a$ carrying positive charge.
	\capfig{0.3\textwidth}{figures/ChargesFields/Rod_dq.png}{\label{fig:ChargesFields:Rod_dq} Calculating the field a distance $a$ from the centre of a charged rod of length $2a$.}
	The net electric field will be in the $x$ direction:
	\begin{align*}
	\vec E=E_x\hat x=\int_0^Q\frac{kdq}{r^2}\cos\theta \hat x
	\end{align*}
	We can write the charge element, $dq$, as:
	\begin{align*}
	dq=\frac{Q}{L}dy=\frac{Q}{L}\frac{dy}{d\theta}{d\theta}
	\end{align*}
	where $dy$ is a small length element along the rod and $d\theta$ is the corresponding angle that is subtended. We can write $y$ (the vertical position of $dq$) in terms of $\theta$:
	\begin{align*}
	y&=a\tan\theta\\
	\therefore\frac{dy}{d\theta}=\frac{a}{\cos^2\theta}
	\end{align*}
	So that the charge element becomes:
	\begin{align*}
	dq=\frac{Q}{L}\frac{a}{\cos^2\theta}d\theta
	\end{align*}
	Note that the distance $r$ from the charge to the point where we calculate the electric field is given by:
	\begin{align*}
	r&=\frac{a}{\cos\theta}\\
	\therefore\frac{1}{r^2}&=\frac{\cos^2\theta}{a^2}
	\end{align*}
	Putting this altogether (and changing $L=2a$):
	\begin{align*}
	E_x &= \int_0^Q\frac{kdq}{r^2}\cos\theta = \frac{kQ}{aL}\int_{-\frac{\pi}{4}}^{\frac{\pi}{4}}\cos\theta d\theta\\
	&=\frac{kQ}{aL}(\sin(\pi/4)-\sin(-\pi/4))=\frac{2kQ}{\sqrt{2}aL}=\frac{\sqrt{2}kQ}{2L^2}
	\end{align*}
	We can now add the electric field from the two charged rods together. The positive rod will create a field in the positive $x$ direction, whereas the negative rod will create a field in the positive $y$ direction. The total electric field is thus:
	\begin{align*}
	\vec E = \frac{\sqrt{2}kQ}{aL}\hat x + \frac{\sqrt{2}kQ}{aL}\hat y=\frac{\sqrt{2}kQ}{2L^2}(\hat x + \hat y)
	\end{align*}
\end{solution}

\question Four uniformly electrically charged rods of length $2a$ are arranged in a square as shown in Figure \ref{fig:ChargesFields:FourRods2}. Three of the rods carry a total positive charge $Q$, while one of the rods carries a negative charge $-Q$. What is the electric field vector at the centre of the square (use the given coordinate system)?
\capfig{0.3\textwidth}{figures/ChargesFields/FourRods2.png}{\label{fig:ChargesFields:FourRods2} Four uniformly charged rods arranged in a square.}
\begin{finalanswer}
	$\vec E = \frac{\sqrt{2}kQ}{L^2}\hat x $
\end{finalanswer}
\begin{solution}
	In order to determine the total electric field at the centre, we can determine the magnitude of the field from any one of the rods and then vectorially add the fields together. Figure \ref{fig:ChargesFields:Rod_dq} shows a diagram for calculating the electric field from a rod of length $2a$ carrying positive charge.
	\capfig{0.3\textwidth}{figures/ChargesFields/Rod_dq.png}{\label{fig:ChargesFields:Rod_dq} Calculating the field a distance $a$ from the centre of a charged rod of length $2a$.}
	The net electric field will be in the $x$ direction:
	\begin{align*}
	\vec E=E_x\hat x=\int_0^Q\frac{kdq}{r^2}\cos\theta \hat x
	\end{align*}
	We can write the charge element, $dq$, as:
	\begin{align*}
	dq=\frac{Q}{L}dy=\frac{Q}{L}\frac{dy}{d\theta}{d\theta}
	\end{align*}
	where $dy$ is a small length element along the rod and $d\theta$ is the corresponding angle that is subtended. We can write $y$ (the vertical position of $dq$) in terms of $\theta$:
	\begin{align*}
	y&=a\tan\theta\\
	\therefore\frac{dy}{d\theta}=\frac{a}{\cos^2\theta}
	\end{align*}
	So that the charge element becomes:
	\begin{align*}
	dq=\frac{Q}{L}\frac{a}{\cos^2\theta}d\theta
	\end{align*}
	Note that the distance $r$ from the charge to the point where we calculate the electric field is given by:
	\begin{align*}
	r&=\frac{a}{\cos\theta}\\
	\therefore\frac{1}{r^2}&=\frac{\cos^2\theta}{a^2}
	\end{align*}
	Putting this altogether (and changing $L=2a$):
	\begin{align*}
	E_x &= \int_0^Q\frac{kdq}{r^2}\cos\theta = \frac{kQ}{aL}\int_{-\frac{\pi}{4}}^{\frac{\pi}{4}}\cos\theta d\theta\\
	&=\frac{kQ}{aL}(\sin(\pi/4)-\sin(-\pi/4))=\frac{2kQ}{\sqrt{2}aL}=\frac{\sqrt{2}kQ}{2L^2}
	\end{align*}
	We can now add the electric field from the four charged rods together. The electric fields from the two horizontal positive rods just cancel out. Both the positive and negative vertical rods will create a field in the positive $x$ direction (each with the magnitude above). The total electric field is thus:
	\begin{align*}
	\vec E = \frac{\sqrt{2}kQ}{L^2}\hat x 
	\end{align*}
\end{solution}








