
\chapter{Electric charges and fields}
\label{chapter:chargesfields}
In this chapter, we start to look at the theories that describe electric phenomena. We will still use the framework for dynamics that was developed by Newton, but will now start to introduce the theories of electromagnetism which describe the electric, and later, the magnetic force. This analogous to how we introduce Newton's Universal Theory of Gravity to describe the source of the gravitational force. 


\begin{learningObjectives}{
 \item Understand the definition of an electric charge.
 \item Understand Coulomb's model for the electric force.
 \item Understand the definition of an electric field.
 \item Understand how to calculate the electric field from a continuous distribution of charge.
 }
\end{learningObjectives}

\begin{opening}
\begin{MCquestion}{If you rub a balloon against a carpet and bring it near your head, your hair will stand up and try to touch the baloon.}
\item The electric charge of the balloon is opposite of that on your hair.
\item Your hair has no net electric charge, this is an example of charge separation and induction. \correct
\end{MCquestion}
\end{opening}

\section{Electric charge}
You have likely experienced or heard about electric charge in your life. For example, on a dry Winter day, you might find that after rubbing your bare feet on a polyester carpet you, that you feel a small electric shock upon touching a metallic surface such as a doorknob. This is a manifestation of the electric charge that has built up on you being released onto the doorknob. You probably also have a notion of the existence of positive and negative charges, and that equal charges repel each other whereas opposite charges attract. In this chapter, we develop the description of how these charges can accumulate and how they can exert attractive or repulsive forces. 

Ordinary matter is made of atoms, which are themselves made of a small positive nucleus (containing positive protons and neutral neutrons) surrounded by a ``cloud'' of negatively charged electrons. Within a solid object, the atoms in the object can be modelled as being effectively stationary due to inter-atomic forces that hold the atoms together. As a result, the protons (the positively charged part of atoms) can essentially be considered to be fixed. The negative electrons, depending on the material, can often move from one atom to another. If an atom looses an electron to another atom, it becomes positive, whereas the atom that acquired the extra electron becomes negative. We define the net charge on an atom based on whether there are more protons (positive), more electrons (negative) or an equal amount (neutral). By default, atoms are neutral and have an equal number of protons and electrons.

When you rub your feet on the carpet, electrons are being removed from one surface (your feet) and deposited on the other (the carpet). Your feet thus acquire a net positive charge (having lost electrons. When you touch a doorknob, the little spark comes from electrons jumping from the doorknob and onto your body (and down to your feet to fill the empty spaces left by the electrons lost to the carpet). The reason that the electrons leave your feet in the first place is that different materials have different ``affinities'' for electrons. When you rub two materials together (placing their atoms in close proximity), electrons will transfer to the material with the highest affinity for electrons. This way of creating a net charge is called ``charging by friction''. 

The ``triboelectic series'' is a list of materials that tend to give up or take up electrons and some common materials from the series are shown in Figure \ref{fig:chargesfields:triboseries}.

\capfig{0.9\textwidth}{figures/ChargesFields/triboseries.png}{\label{fig:chargesfields:triboseries}A sample of a triboelectric series of materials.}

TODO: Checkpoint question: You rub glass with silk, which one lost electrons?


\subsection{Charging by conduction and induction}
\subsection{Conductors and insulators}

\section{The Coulomb force}

\section{The electric field}

\section{The electric dipole}


\newpage
\section{Summary}

\begin{chapterSummary}
 Something that was learned
\end{chapterSummary}

\newpage
\begin{importantEquations}
\medskip
\begin{multicols}{2}
\textbf{Momentum of a point particle:}
\begin{align*}
\vec p = m\vec v \\
\frac{d}{dt}\vec p = \sum \vec F = \vec F^{net}
\end{align*}
\columnbreak
\\
\textbf{Position of the Centre of Mass \\ of a system:}
\begin{align*}
\vec r_{CM} &=\frac{1}{M}\sum_i m_i\vec r_i 
\end{align*}
\medskip
\end{multicols}
\end{importantEquations}

\newpage
\section{Thinking about the material}

\begin{chapteractivity}{Reflect and research}
{
\item Explain
}
\end{chapteractivity}

\begin{chapteractivity}{To try at home}
{
\item Try
}
\end{chapteractivity}

\begin{chapteractivity}{To try in the lab}
{
\item Propose an experiment
}
\end{chapteractivity}

\newpage
\section{Sample problems and solutions}
\subsection{Problems}
\begin{problem}{soln:template:ballistic}{\label{prob:template:ballistic} 

}
\end{problem}
\newpage
\subsection{Solutions}
\begin{solution}{prob:template:ballistic}\label{soln:template:ballistic}

\end{solution}

