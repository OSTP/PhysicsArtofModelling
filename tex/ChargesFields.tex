
\chapter{Electric charges and fields}
\label{chapter:chargesfields}
In this and subsequent chapters, we start to look at the theories that describe electric and magnetic phenomena. Within the framework for dynamics that was developed by Newton, we will introduce the theories of electromagnetism which describe the electric force, the magnetic force, and how these two interact. This first chapter introduces the description of the electric force, analogously to how we introduced Newton's Universal Theory of Gravity to describe the gravitational force. 


\begin{learningObjectives}{
 \item Understand the definition of an electric charge.
 \item Understand the difference between an insulator and a conductor.
 \item Understand different mechanisms for charging objects.
 \item Understand Coulomb's model for the electric force.
 \item Understand the definition of an electric field.
 \item Understand how to calculate the electric field from a continuous distribution of charge.
 \item Understand how to model an electric dipole.
 }
\end{learningObjectives}

\begin{opening}
\begin{MCquestion}{If you rub a balloon against a carpet and bring it near your head, your hair will stand up and try to touch the balloon.}
\item The electric charge of the balloon is opposite of that on your hair.
\item Your hair has no net electric charge, this is an example of charge separation and induction. \correct
\end{MCquestion}
\end{opening}

\section{Electric charge}
You have likely experienced or heard about electric charge in your life. For example, on a dry Winter day, you might find that after rubbing your bare feet on a polyester carpet you feel a small electric shock upon touching a metallic surface such as a doorknob. This is a manifestation of the electric charge that has built up on you being released onto the doorknob. You probably also have a notion of the existence of positive and negative charges, and that equal charges repel each other whereas opposite charges attract. In this chapter, we develop the description of how these charges can accumulate and how they exert attractive or repulsive forces on each other. 

Ordinary matter is made of atoms, which are themselves made of a small positive nucleus (containing positive protons and neutral neutrons) surrounded by a ``cloud'' of negatively charged electrons. Within a solid object, the atoms in the object can be modelled as being effectively stationary due to inter-atomic forces that hold the atoms together. As a result, the protons (the positively charged part of atoms) can be considered to be fixed in space. The negative electrons, depending on the material, can often move from one atom to another. If an atom looses an electron to another atom, it becomes positive, whereas the atom that acquired the extra electron becomes negative. We define the net charge on an atom (or an object) based on whether there are more protons (positive), more electrons (negative) or an equal amount (neutral). By default, atoms are neutral and have an equal number of protons and electrons. The reason that anything acquires a net electric charge is because it acquired an excess (or deficit) of electrons from another object. We say that ``charge is conserved'', since the number of electrons does not change and if one object became positively charged, a different object must have become negatively charged by the same amount, so that the total charge (in the Universe) is zero.

When you rub your feet on the carpet, electrons are being removed from one surface (your feet) and deposited on the other (the carpet). Your feet thus acquire a net positive charge (having lost electrons). When you touch a doorknob, the little spark comes from electrons jumping from the doorknob and onto your body. The reason that the electrons leave your feet in the first place is that different materials have different ``affinities'' for electrons. When you rub two materials together (placing their atoms in close proximity), electrons will transfer to the material with the highest affinity for electrons. This way of creating a net charge is called ``charging by friction''.

The ``triboelectic series'' is a list of materials that tend to give up or acquire electrons when they are placed in close contact with each other; some common materials from the series are shown in Figure \ref{fig:chargesfields:triboseries}.  The greatest charge is generated by rubbing together materials that are the furthest away in the diagram. Rubbing silk on a piece of glass result in more charge than rubbing wool on the same piece of glass.

\capfig{\textwidth}{figures/ChargesFields/triboseries.png}{\label{fig:chargesfields:triboseries}A sample of a triboelectric series of materials. The materials on the right-hand side have the greatest affinity to acquire electrons.}

\begin{checkpoint}\label{cp:chargesfields:tribo}
\begin{MCquestion}{If you rub a glass rod with silk, which object ends up with an excess of electrons?}
\item glass rod.
\item silk. \correct
\item neither, they remain neutral.
\item both will acquire an excess of electrons.
\end{MCquestion}
\end{checkpoint}

\subsection{Conductors and insulators}
We can broadly classify materials into conductors (such as metals), and insulators (such as wood), depending on how easily the electrons can move around in the material. In a conductor, electrons (rather, the outer electron(s) of an atom) are only loosely bound to their nucleus, and they can thus move around the material freely. In an insulator, the electrons are tightly bound to the nuclei of their atoms and cannot easily move around. There is a third class of materials, semi-conductors, that falls somewhere between a conductor and an insulator. In a semi-conductor, electrons are typically bound to their atoms, but any additional electrons present in the material can move around as if they are in a conductor. 

Within a conductor, such as a solid metallic sphere, charges can move around freely. If that sphere has a net charge, for example an excess of electrons, those excess electrons will migrate to the outer surface of the sphere. Electrons in the sphere repel each other and will quickly settle into a position where they are, on average, the furthest from all of the other electrons, which occurs if all of the electrons migrate to the surface. This is illustrated by showing the charges on the surface of the charged sphere in the left panel of Figure \ref{fig:chargesfields:conductioncharge}. If an initially neutral conducting sphere is connected to the charged sphere by a conducting wire (right panel of Figure \ref{fig:chargesfields:conductioncharge}), some of the electrons will ``conduct'' (transfer) onto the surface of the neutral sphere, so that, on average, they are further from all other electrons. This way of adding charge to the neutral sphere is called ``charging by conduction'', and the second sphere will remain charged if the connection between spheres is broken.

\capfig{0.7\textwidth}{figures/ChargesFields/conductioncharge.png}{\label{fig:chargesfields:conductioncharge}Charging by conduction: a neutral conducting sphere is connected to a negatively charged conducting sphere. The charges can ``spread out more'' if some of the charges move (``conduct'') from the charge sphere onto the neutral sphere.}

\subsection{Electrostatic induction}
Electrostatic induction allows one to ``induce'' a charge by using the fact that charges can move freely in a conductor. The left panel of Figure \ref{fig:chargesfields:induction} shows a (neutral) rod made of a conducting material, with electrons distributed uniformly throughout its volume. In the right panel, a negatively charged sphere is brought next to the rod. Since the rod is conducting, electrons in the rod can easily move and they will thus accumulate on the end of the rod that is furthest from the negative sphere (which repels the electrons). Those electrons will leave positive empty spaces, which can be modelled as positive charges, on the side closest to the sphere. The electrons in the rod will only accumulate for as long as the force from the negative sphere is less than the repulsive force from the electrons that have already accumulated. In practice, such an equilibrium is reached almost instantly. In equilibrium, we say that the rod is ``polarized'', or that the ``charges in the rod have separated'', although the rod is overall still neutral.

Note that we can model this as if it where positive charges that move inside of the rod instead of negative charges. The positive charges are attracted to the negative sphere, and thus accumulate on the end of the rod closest to the sphere, leaving a negative charge on the other end. The choice to call electrons ``negative'' is completely arbitrary. For convenience, we usually build models assuming that positive charges can easily move around, even if, in reality, it is almost always actually (negative) electrons that move.

\capfig{0.9\textwidth}{figures/ChargesFields/induction.png}{\label{fig:chargesfields:induction}Electrostatic induction: when a negatively charged sphere is brought close to a neutral conducting rod, the electrons in the rod, which can move freely, accumulate on the side of the rod furthest from the sphere, leaving an excess of positive charge near the sphere.}

We can create a net charge on the polarized rod if we provide a conducting path for charges to leave (or enter) the rod. The Earth can be modelled as a very large reservoir of both positive and negative charges. By connecting the rod to the Earth (we say that we connect the rod to ``ground''), we provide a path for the electrons in the rod to be even further from the negatively charged sphere, and they can thus leave the rod entirely in order to go into the ground. This is illustrated in the right-hand panel of Figure \ref{fig:chargesfields:inductioncharge}.

If we then disconnect the rod from the ground, it has now acquired an overall positive charge, as in the right hand panel. We call this ``charging by induction''. We can also think of this in terms of positive charges moving into the rod from the Earth; when we connect the rod to the ground, the positive charges in the Earth can move into the rod and get close to the negatively charged sphere. If we disconnect the rod from the ground, the rod stays positive, just as we conclude when using a model where it is the negative charges that move\footnote{Unless magnetism is involved, it is not possible to distinguish between a flow of positive charges moving in one direction or negative charges moving in the opposite direction.}.

\capfig{0.9\textwidth}{figures/ChargesFields/inductioncharge.png}{\label{fig:chargesfields:inductioncharge}Charging by induction: when we connect the polarized rod to the ground, electrons can leave the rod. If we now disconnect the rod from ground, the rod is left with an overall positive charge. }

\section{The Coulomb force}
Coulomb was the first to provide a detailed quantitative description of the force between charged objects. Nowadays, we use the (derived) SI unit of ``Coulomb'' (C) to represent charge. The ``charge'' of an object corresponds to the net excess (or lack) of electrons on the object. An electron has a charge of $-e=\SI{-1.6e-19}{C}$, which is a very large charge. Thus, an object with a charge of $\SI{-1}{C}$ has an excess of about $\num{1e19}$ electrons on it. If an object has an excess of electrons, it is negatively charged and we indicate this with a negative sign on the charge of the object. An object with a (positive) charge of $\SI{1}{C}$ thus has a deficit of $\num{1e19}$ electrons.

Through careful studies of the force between two charged spheres, Coulomb observed\footnote{Others had initially observed the inverse square law for the electric force, but Coulomb was the first to formalize the theory.} that:
\begin{itemize}
\item The force is attractive if the objects have opposite charges and repulsive if the objects have the same charge.
\item The force is inversely proportional to the squared distance between spheres.
\item The force is larger if the charges involved are larger. 
\end{itemize}
This leads to Coulomb's Law for the electric force (or simply ``Coulomb's Law''), $\vec F_{12}$, exerted on a point charge $Q_1$ by another point charge $Q_2$:
\begin{align*}
\Aboxed{\vec F_{12}=k\frac{Q_1Q_2}{r^2}\hat r_{21}}
\end{align*}
where $\hat r_{21}$ is the unit vector from $Q_2$ to $Q_1$ and $r$ is the distance between the two charges, as illustrated in Figure \ref{fig:chargesfields:coulombforce}. $k=\SI{8.99e9}{N\cdot m^2/C^{2}}$ is simply a proportionality constant (``Coulomb's constant'') to ensure that the quantity on the right will have units of Newtons when all other quantities are in S.I. units. In some instances, it is more convenient to use the ``permittivity of free space'', $\epsilon_0$, rather than Coulomb's constant, in which case Coulomb's Law has the form:
\begin{align*}
\vec F_{12}=\frac{1}{4\pi\epsilon_0}\frac{Q_1Q_2}{r^2}\hat r_{21}
\end{align*}
where $\epsilon_0=\frac{1}{4\pi k}=\SI{8.85e-12}{C^2\cdot N^{-1}\cdot m^{-2}}$ is a more fundamental constant, as we will see in later chapters.

\capfig{0.5\textwidth}{figures/ChargesFields/coulombforce.png}{\label{fig:chargesfields:coulombforce}Vectors involved in applying Coulomb's Law.}
If the two charges have positions $\vec r_1$ and $\vec r_2$, respectively, then the vector $\hat r_{21}$ is given by:
\begin{align*}
\hat r_{21} = \frac{\vec r_2 - \vec r_1}{||\vec r_2 - \vec r_1||}
\end{align*}
Coulomb's Law is mathematically identical to the gravitational force in Newton's Universal Theory of Gravity. Rather than quantity of mass determining the strength of the gravitational force, it is the quantity of charge that determines the strength of the electric force. The only major difference is that gravity is always attractive, whereas the Coulomb force can be repulsive.
\begin{checkpoint}\label{cp:chargesfields:conservative}
\begin{MCquestion}{The Coulomb force is conservative.}
\item True. \correct
\item False.
\end{MCquestion}
\end{checkpoint}
The product $Q_1Q_2$ in the numerator of Coulomb's force is positive if the two charges have the same sign (both positive or both negative) and negative if the charges have opposite signs. Again, referring to Figure \ref{fig:chargesfields:coulombforce}, if the two charges are positive, the force on $Q_1$ will point in the same direction as $\hat r_{21}$ (since all of the scalars are positive in Coulomb's Law) and thus be repulsive. If, instead, the two charges have opposite signs, the product $Q_1Q_2$ will be negative and the force vector on $Q_1$ will point in the opposite direction from $\hat r_{21}$ and the force is attractive. 

\begin{example}{Calculate the magnitude of the electric force between the electron and the proton in a hydrogen atom and compare this to the gravitational force between them.}
We model this by assuming that the electron and proton are point charges a distance of $\SI{1}{\angstrom}=\SI{1e-10}{m}$ apart (1 \r Angstrom is about the size of the hydrogen atom). The proton and electron have the same charge with magnitude $e=\SI{1.6e-19}{C}$, so the (attractive) electric force between them has a magnitude of:
\begin{align*}
F^e = k\frac{Q_1Q_2}{r^2}=(\SI{9e9}{N\cdot m^2/C^{2}})\frac{(\SI{1.6e-19}{C})(\SI{1.6e-19}{C})}{(\SI{1e-10}{m})^2}=\SI{2.3e-8}{N}
\end{align*}
which is a small number, but acting on a very small mass. In comparison, the force of gravity between an electron ($m_e=\SI{9.1e-31}{kg}$) and a proton ($m_p=\SI{1.7e-27}{kg}$) is given by:
\begin{align*}
F^g=G\frac{m_em_p}{r^2}=(\SI{6.7e-11}{Nm^2/kg^2})\frac{(\SI{9.1e-31}{kg})(\SI{1.7e-27}{kg})}{(\SI{1e-10}{m})^2}=\SI{1.04e-47}{N}
\end{align*}
\textbf{Discussion:} As we can see, the electric force between an electron and a proton is 39 orders of magnitude larger than the gravitational force! This shows that the gravitational force is extremely weak on the scale of particles and has essentially no effect in particle physics. Indeed, the best current theory of particle physics, and the most precisely tested theory in physics, the ``Standard Model'', does not need to include gravity in order to provide a spectacularly precise description of particles. One of the big challenges in theoretical physics is nonetheless to develop a theory that integrates the gravitational force with the other forces.
\end{example}
\begin{example}{\label{ex:chargesfields:chargetriangle}Three charges, $Q_1=\SI{1}{nC}$, $Q_2=\SI{-2}{nC}$, and $q=\SI{-1}{nC}$, are held fixed at the three corners of an equilateral triangle with sides of length $a=\SI{1}{cm}$, with a coordinate system as shown in Figure \ref{fig:chargesfields:chargetriangle}. What is the electric force vector on charge $q$? (Note that $\SI{1}{nC}=\SI{1e-9}{C}$).
\capfig{0.4\textwidth}{figures/ChargesFields/chargetriangle.png}{\label{fig:chargesfields:chargetriangle}Three charges arranged in an equilateral triangle of side $a$.}}
The net electric force on charge $q$ will be the vector sum of the forces from charges $Q_1$ and $Q_2$. We thus need to determine the force vectors on $q$ from each charge using Coulomb's Law, and then add those two vectors to obtain the net force on $q$. The force vectors exerted on $q$ from each charge are illustrated in Figure \ref{fig:chargesfields:chargetriangle_sol}.
\capfig{0.4\textwidth}{figures/ChargesFields/chargetriangle_sol.png}{\label{fig:chargesfields:chargetriangle_sol}Force vectors on charge $q$.}
The force from charge $Q_1$ has magnitude:
\begin{align*}
F_{q1}=\left |k\frac{Q_1q}{a^2}\right |=(\SI{9e9}{N\cdot m^2/C^{2}})\frac{(\SI{1e-9}{C})(\SI{1e-9}{C})}{(\SI{0.01}{m})^2}=\SI{9e-5}{N}
\end{align*}
and components:
\begin{align*}
\vec F_{q1}&=-F_{q1}\cos(\SI{60}{\degree})\hat x-F_{q1}\sin(\SI{60}{\degree})\hat y\\
&=-(\SI{4.5e-5}{N})\hat x-(\SI{7.8e-5}{N})\hat y
\end{align*}
Similarly, the force on $q$ from $Q_2$ has magnitude:
\begin{align*}
F_{q2}=\left |k\frac{Q_2q}{a^2}\right |=(\SI{9e9}{N\cdot m^2/C^{2}})\frac{(\SI{2e-9}{C})(\SI{1e-9}{C})}{(\SI{0.01}{m})^2}=\SI{1.8e-4}{N}
\end{align*}
and components:
\begin{align*}
\vec F_{q2}&=-F_{q2}\cos(\SI{60}{\degree})\hat x+F_{q2}\sin(\SI{60}{\degree})\hat y\\
&=-(\SI{9.0e-5}{N})\hat x+(\SI{1.6e-4}{N})\hat y
\end{align*}
Finally, we can add the two force vectors together to obtain the net force on $q$:
\begin{align*}
\vec F^{net}&=\vec F_{q1}+\vec F_{q2}\\
&=-(\SI{4.5e-5}{N})\hat x-(\SI{7.8e-5}{N})\hat y-(\SI{9.0e-5}{N})\hat x+(\SI{1.6e-4}{N})\hat y\\
&=-(\SI{13.5e-5}{N})\hat x+(\SI{8.2e-5}{N})\hat y
\end{align*}
which has a magnitude of $\SI{15.8e-5}{N}$.

\textbf{Discussion:} In this example, we determined the net force on a charge by making use of the superposition principle; namely, that we can treat the forces exerted on $q$ by $Q_1$ and $Q_2$ independently, without needing to consider the fact that $Q_1$ and $Q_2$ exert forces on each other.
\end{example}

\section{The electric field}
We define the electric field vector, $\vec E$, in an analogous way as we defined the gravitational field vector, $\vec g$. By defining the gravitational field vector, say, at the surface of the Earth, we can easily calculate the gravitational force exerted by the Earth on any mass, $m$, without having to use Newton's Universal Theory of Gravity. As you recall, we can define the gravitational field at some position $\vec r$, $\vec g(\vec r)$, from a point mass $M$ as the gravitational force per unit mass:
\begin{align*}
\vec g(\vec r) = -G \frac{M}{r^2}\hat r
\end{align*}
where $\vec r$ is a vector from the position of $M$ to where we want to know the gravitational field. As a result, the force exerted on a ``test mass'', $m$, located at position $\vec r$ relative to mass $M$ is given by:
\begin{align*}
\vec F^g=m\vec g= -G\frac{Mm}{r^2}\hat r
\end{align*}
which, of course, is the result from Newton's Theory of Gravity. As you recall, we can define the gravitational field for  any object that is not a point mass (e.g. the Earth), and use that field to find the force exerted by the Earth on any mass $m$, without having to re-calculate the gravitational field each time (which requires an integral or Gauss' Law). 

We proceed in an analogous was to define the ``Electric field'', $\vec E(\vec r)$, as the \textit{electric force per unit charge}. If we have a point charge, $Q$, located at the origin of a coordinate system, then the electric field from that point charge at some position $\vec r$ relative to the origin is given by:
\begin{align*}
\Aboxed{\vec E(\vec r) = k\frac{Q}{r^2}\hat r}
\end{align*}
If we place a ``test charge'', $q$, at position $\vec r$ in space, it will experience a force given by:
\begin{align*}
\vec F^e=q\vec E=k\frac{Qq}{r^2}\hat r
\end{align*}
just as we find from Coulomb's Law. 
\begin{checkpoint}\label{cp:chargesfields:edirection}
\begin{MCquestion}{A negative charge is placed at the origin of a coordinate system. At some point in space, the electric field from that charge}
\item points towards the origin. \correct
\item points away from the origin.
\end{MCquestion}
\end{checkpoint}
In Example \ref{ex:chargesfields:chargetriangle}, we determined the electric force on charge $q$, exerted by two other charges $Q_1$ and $Q_2$. If we now changed the value of $q$ and wanted to determine the force, we can use the electric field to simplify the process considerably. That is, we can determine the value of the electric field, $\vec E$, from $Q_1$ and $Q_2$ at the position of $q$, and then simply multiply that field vector by a charge $q$ to obtain the force on that charge, without having to add force vectors.
\begin{example}{\label{ex:chargesfields:epointcharges}Two charges, $Q_1=\SI{1}{nC}$, and $Q_2=\SI{-2}{nC}$ are held fixed at two corners of an equilateral triangle with sides of length $a=\SI{1}{cm}$, with a coordinate system as shown in Figure \ref{fig:chargesfields:chargetriangle}. What is the electric field vector at the third corner of the triangle? 
\capfig{0.4\textwidth}{figures/ChargesFields/fieldtriangle.png}{\label{fig:chargesfields:fieldtriangle}Two charges at the corners of an equilateral triangle of side $a$.}}
The net electric field at the third corner of the triangle will be the vector sum of the electric fields from charges $Q_1$ and $Q_2$. We thus need to determine the electric field vectors from each charge, and then add those two vectors to obtain the net electric field. The vectors are illustrated in Figure \ref{fig:chargesfields:fieldtriangle_sol}.
\capfig{0.4\textwidth}{figures/ChargesFields/fieldtriangle_sol.png}{\label{fig:chargesfields:fieldtriangle_sol}Electric field vectors from two charges at the corners of an equilateral triangle of side $a$.}
The electric field from charge $Q_1$ has magnitude:
\begin{align*}
E_1=\left |k\frac{Q_1}{a^2}\right |=(\SI{9e9}{N\cdot m^2/C^{2}})\frac{(\SI{1e-9}{C})}{(\SI{0.01}{m})^2}=\SI{9e4}{N/C}
\end{align*}
and components:
\begin{align*}
\vec E_1&=E_1\cos(\SI{60}{\degree})\hat x+E_1\sin(\SI{60}{\degree})\hat y\\
&=(\SI{4.5e4}{N/C})\hat x+(\SI{7.8e4}{N/C})\hat y
\end{align*}
Similarly, the electric field from $Q_2$ has magnitude:
\begin{align*}
E_2=\left |k\frac{Q_2}{a^2}\right |=(\SI{9e9}{N\cdot m^2/C^{2}})\frac{(\SI{2e-9}{C})}{(\SI{0.01}{m})^2}=\SI{1.8e5}{N/C}
\end{align*}
and components:
\begin{align*}
\vec E_2&=E_2\cos(\SI{60}{\degree})\hat x-E_2\sin(\SI{60}{\degree})\hat y\\
&=(\SI{9.0e4}{N/C})\hat x-(\SI{1.6e5}{N/C})\hat y
\end{align*}
Finally, we can add the two force vectors together to obtain the net force on $q$:
\begin{align*}
\vec E^{net}&=\vec E_1+\vec E_2\\
&=(\SI{4.5e4}{N/C})\hat x+(\SI{7.8e4}{N/C})\hat y+(\SI{9.0e4}{N/C})\hat x-(\SI{1.6e5}{N/C})\hat y\\
&=(\SI{13.5e4}{N/C})\hat x-(\SI{8.2e4}{N/C})\hat y
\end{align*}
which has a magnitude of $\SI{15.8e4}{N/C}$. By knowing the electric field at the empty corner of the triangle, we can now calculate the net electric force that would act on any charge placed in that location. For example, if we place a charge $q=\SI{-1}{nC}$ (as in Example \ref{ex:chargesfields:chargetriangle}), we can easily find the corresponding electric force:
\begin{align*}
\vec F_q &= q\vec E=(\SI{-1}{nC})\left[ (\SI{13.5e4}{N/C})\hat x-(\SI{8.2e4}{N/C})\hat y \right]\\
&=-(\SI{13.5e-5}{N})\hat x+(\SI{8.2e-5}{N})\hat y
\end{align*}
as we found previously. Note that the force on $q$ is in the opposite direction of the electric field vector. This is because $q$ is negative. The \textbf{electric field at some point in space thus points in the same direction as the force that a positive test charge would experience}.

\textbf{Discussion:} In this example, we determined the net electric field by making use of the superposition principle; namely, that we can treat the electric fields from $Q_1$ and $Q_2$ independently, without needing to consider the fact that $Q_1$ and $Q_2$ exert forces on each other. By knowing the electric field at some position in space, we can easily calculate the force vector on any test charge, $q$, placed at that position. Furthermore, the sign of the charge $q$ will determine in which direction the force will point (parallel to $\vec E$ for a positive charge and anti-parallel to $\vec E$ for a negative charge).
\end{example}

\subsection{Visualizing the electric field}
Generally, a ``field'' is something that has a different value at different positions in space. The pressure in a fluid under the presences of gravity is a field: the pressure is different at different heights in the fluid. Since pressure is a scalar quantity (a number), we call it a ``scalar field''. The electric field is called a ``vector field'', because it is a vector that is different at each position in space. One way to visualize the electric field is to draw arrows at different positions in space; the length of the arrow is then proportional to the strength of the electric field at that position, and the direction of the arrow then represents the direction of the electric field. The electric field for a point charge is shown using this method in Figure TODO.
%TODO A figure of a bunch of arrows point away from a point charge, with amplitude dropping off as 1/r^2.

One disadvantage of visualizing a vector field with arrows is that the arrows take up space, and it can be challenging to visualize how the field changes magnitude and direction continuously through space. For this reason, one usually uses ``field lines'' to visualize a vector field. Field lines are continuous lines with the following properties:
\begin{itemize}
\item The direction of the vector field at some point in space is tangent to the field line at that point.
\item Field lines have a direction to indicate the direction of the field vector along the tangent.
\item The magnitude of the field is proportional to the density of field lines at that point. The more field lines near a location in space, the larger the magnitude of the field vector at that point.
\end{itemize}
An example of using field lines to represent a vector field in space is shown in Figure \ref{fig:chargesfields:fieldlines}. The corresponding field vector is shown at two different positions in space. At both positions, the vector is tangent to the field line at that position in space and points in the direction of the little arrow drawn at the end of the field lines. The field vector at point $A$ has a larger magnitude than the one at point $B$, since the field lines are more concentrated at point $A$ than at point $B$ (there are more field lines per unit area at that position in space, the field lines are closer together).
\capfig{0.4\textwidth}{figures/ChargesFields/fieldlines.png}{\label{fig:chargesfields:fieldlines}An example of determining a field vector from the continuous field lines.}
\begin{checkpoint}\label{cp:chargesfields:efield}
\begin{MCquestion}{It is possible for field lines to cross?}
\item Yes.
\item No. \correct
\end{MCquestion}
\end{checkpoint}
Because the electric field vector always points in the direction of the force that would be exerted on a positive charge, electric field lines will point out from a positive charge and into a negative charge. The electric field lines for a combination of positive and negative charges is illustrated in Figure TODO.
%TODO A figure with 2 positive and 1 negative charge, with correct field lines.

\subsection{Electric field from a charge distribution}
So far, we have only considered Coulomb's Law for point charges (charges that are infinitely small and can be considered to exist at a single point in space). We can use the principle of superposition to determine the electric field from a charged extended/continuous object by modelling that object as being made of many point charges. The electric field from that object is then the sum of the electric field from the point charges that make up that object. 

Consider a charged wire that is bent into a semi-circle of radius $R$, as in Figure \ref{fig:chargesfields:semicircle}. The wire carries a net electric charge, $Q$, that is uniformly distributed along the length of the wire. We wish to determine the electric field vector at the centre of the circle. 
\capfig{0.2\textwidth}{figures/ChargesFields/semicircle.png}{\label{fig:chargesfields:semicircle}A charged wire bent into a semi-circle of radius $R$.}
We start by choosing a very small section of wire and model that section of wire as a point charge with infinitesimal charge $dq$ (as in Figure \ref{fig:chargesfields:semicircle_sol}). A distance $R$ from that point charge, the electric field from that point charge will have magnitude, $dE$, given by:
\begin{align*}
dE=k\frac{dq}{R^2}
\end{align*}
The electric field vector, $d\vec E$, from the point charge $dq$ is illustrated in Figure \ref{fig:chargesfields:semicircle_sol}.
\capfig{0.3\textwidth}{figures/ChargesFields/semicircle_sol.png}{\label{fig:chargesfields:semicircle_sol}Infinitesimal electric fields from point charges along the bent wire.}
Using the coordinate system that is show, we define $\theta$ as the angle made by the vector from the origin to the point charge $dq$ and the $x$-axis. The electric field vector from $dq$ is then given by:
\begin{align*}
d\vec E = dE\cos\theta \hat x - dE\sin\theta \hat y
\end{align*}
The total electric field at the origin will be obtained by summing the electric fields from the different $dq$ over the entire semi-circle:
\begin{align*}
\vec E &= \int d\vec E = \int \left(dE\cos\theta \hat x - dE\sin\theta \hat y\right)\\
&=\left( \int dE\cos\theta \right)\hat x -\left( \int dE\sin\theta \right)\hat y\\
\therefore E_x &= \int dE\cos\theta\\
\therefore E_y &= -\int dE\sin\theta\\
\end{align*}
We are thus left with two integrals to solve for the $x$ and $y$ components of the electric field, respectively. Before jumping into solving the integrals, it is useful to think about the symmetry of the problem. Specifically, consider a second point charge, $dq'$, located symmetrically about the $x$-axis from charge $dq$, as illustrated in Figure \ref{fig:chargesfields:semicircle_sol}. The charge $dq'$ will create a small electric field $d\vec E'$ as illustrated. When we add together $d\vec E$ and $d\vec E'$, the two $y$ components will cancel, and only the $x$ components will sum together. Similarly, for any $dq$ that we choose, there will always be another $dq'$ such that when we sum together their respective electric fields, the $y$ component will cancel. Thus, by symmetry, we can argue that the net $y$ component of the electric field, $E_y$, must be identically zero. We thus only need to evaluate the $x$ component of $\vec E$:
\begin{align*}
E_x = \int dE\cos\theta = \int k\frac{dq}{R^2} \cos\theta
\end{align*}
In order to solve this integral, we need to consider which variables change for different choices of the point charge $dq$. In this case, the distance $R$ is the same anywhere along the semi-circle, so only $\theta$ changes with different choices of $dq$, as $k$ is a constant. We thus need to express $dq$ in terms of $d\theta$ so that we can solve the integral. $d\theta$ corresponds to a small change in the angle $\theta$, and is the angle that is subtended by the charge $dq$. That is, the charge $dq$ covers a small arc length, $ds$, of the semi-circle, which is related to $d\theta$ by:
\begin{align*}
ds = Rd\theta
\end{align*}
The total charge on the wire is given by $Q$, and the wire has a length $\pi R$ (half the circumference of a circle). Since the charge is distributed uniformly on the wire, the charge per unit length of any piece of wire must be constant. In particular, $dq$ divided by $ds$ must be equal to $Q$ divided by $\pi R$:
\begin{align*}
\frac{dq}{ds}&=\frac{Q}{\pi R}\\
\therefore dq &=\frac{Q}{\pi R}ds=\frac{Q}{\pi}d\theta
\end{align*}
where in the last equality we used the relation $ds=Rd\theta$. We now have all the ingredients to solve the integral:
\begin{align*}
E_x &= \int k\frac{dq}{R^2} \cos\theta = \int_{-\pi/2}^{+\pi/2} k\frac{Q}{\pi R^2}\cos\theta d\theta\\
&= k\frac{Q}{\pi R^2}\int_{-\pi/2}^{+\pi/2}\cos\theta d\theta=k\frac{Q}{\pi R^2}\left[ \sin\theta \right]_{-\pi/2}^{+\pi/2}\\
&= k\frac{2Q}{\pi R^2}
\end{align*}
The total electric field vector at the centre of the circle is thus given by:
\begin{align*}
\vec E = k\frac{2Q}{\pi R^2} \hat x
\end{align*}
Note that if we had not realized that we did not need to solve the integral for the $y$ component, we would still find that it is zero:
\begin{align*}
E_y= -k\frac{Q}{\pi R^2}\int_{-\pi/2}^{+\pi/2}\cos\theta d\theta=-k\frac{Q}{\pi R^2}\left[ -\cos\theta \right]_{-\pi/2}^{+\pi/2}=0
\end{align*}
In order to determine the electric field at some point from any continuous charge distribution, the procedure is generally the same:
\begin{enumerate}
\item Make a \textit{good} diagram.
\item Choose a charge element $dq$.
\item Draw the electric field element, $d\vec E$, at the point of interest.
\item Write out the electric field element vector, $d\vec E$, in terms of $dq$ and any other relevant variables.
\item Think of symmetry: will any of the component of $d\vec E$ sum to zero over all of the $dq$?
\item Write the total electric field as the sum (integral) of the electric field elements.
\item Identify which variables change as one varies the $dq$ and choose an integration variable to express $dq$ and everything else in terms of that variable and other constants.
\item Do the sum (integral).
\end{enumerate}

\begin{example}{\label{ex:chargesfields:ring}A ring of radius $R$ carries a total charge $+Q$. Determine the electric field a distance $a$ from the centre of the ring, along the axis of symmetry of the ring.}
In order to determine the electric field, we carry out the procedure outlined above, and start by drawing a good diagram, as in Figure \ref{fig:chargesfields:ring}, showing: our coordinate system, our choice of $dq$, the electric field element vector $d\vec E$ that corresponds to $dq$, and variables ($r$, $\theta$) to specify the position of $dq$.
\capfig{0.3\textwidth}{figures/ChargesFields/ring.png}{\label{fig:chargesfields:ring}Determining the electric field on the axis of a ring of radius $R$ carrying charge $Q$.}
In this case, the figure is challenging to draw and visualize because of the three-dimensional nature of the problem. With the specific $dq$ that we chose, the electric field element vector is given by:
\begin{align*}
d\vec E = -dE\sin\theta \hat x + 0\hat y + dE\cos\theta \hat z 
\end{align*}
where $d\vec E$ has magnitude:
\begin{align*}
dE = k\frac{dq}{r^2}
\end{align*}
The $x$ and $z$ components of the total electric field will then be given by:
\begin{align*}
E_x &= -\int dE\sin\theta=-\int k\frac{dq}{r^2}\sin\theta\\
E_z &= \int dE\cos\theta=\int k\frac{dq}{r^2}\cos\theta \\
\end{align*}
In general, if we had chosen a $dq$ that is not along one of the axes of the coordinate system, the electric field element vector would have components in all three directions. However, if we consider the symmetry of the ring, we can note that once we sum together all of the electric field elements, only the $z$ components will survive. Indeed, we have shown in Figure \ref{fig:chargesfields:ring} that for each $dq$, there will be a $dq'$ located on the opposite side of the ring that will create an electric field element that will cancel all by the $z$ component of the field element from $dq$. We thus only need to consider the $z$ components of the electric field elements when determining the total electric field:
\begin{align*}
\vec E = E_z\hat z
\end{align*}
We now have to evaluate the integral for the $z$ component of the electric field:
\begin{align*}
E_z &= \int k\frac{dq}{r^2}\cos\theta \\
\end{align*}
and determine which quantities change as we move $dq$ around the ring. In this case, both $r^2$ and $\cos\theta$ are the same for all elements on the ring, and the integral is trivial:
\begin{align*}
E_z &= k\frac{1}{r^2}\cos\theta\int dq=k\frac{Q}{r^2}\cos\theta=kQ\frac{a}{(R^2+a^2)^\frac{3}{2}}  \\
\end{align*}
where the integral $\int dq$ simply means ``sum all of the charges $dq$ together'', which is equal to $Q$, the total charge on the ring. 
In the last equality, we replaced $\cos\theta$ with the variables $a$ and $R$ that are provided in the question.
\end{example}

\begin{example}{\label{ex:chargesfields:finiteline}You have rubbed a glass rod with a silk cloth such that the glass rod has acquired a positive charge. The rod has a length, $L$, a negligible cross-section, and has acquired a total charge, $Q$, that is uniformly distributed along the length of the rod. What is the electric field a distance $R$ from the centre of the rod?}
In order to determine the electric field, we carry out the procedure outlined above, and start by drawing a good diagram, as in Figure \ref{fig:chargesfields:finiteline}, showing: our coordinate system, our choice of $dq$ at a distance $y$ above the centre of the rod, the electric field element vector $d\vec E$ that corresponds to $dq$, and variables ($y$, $r$, $\theta$) to specify the position of $dq$.
\capfig{0.3\textwidth}{figures/ChargesFields/finiteline.png}{\label{fig:chargesfields:finiteline}Determining the electric field a distance $R$ from the centre of a rod of length $L$ carrying charge $Q$.}
We define the origin to be located at the point where we want to determine the electric field, and the angle $\theta$ to be the angle between the horizontal and the position vector of $dq$. We can write the electric field element vector as:
\begin{align*}
d\vec E = dE\cos\theta \hat x - dE\sin\theta \hat y
\end{align*}
where $d\vec E$ has magnitude:
\begin{align*}
dE = k\frac{dq}{r^2}
\end{align*}
The $x$ and $y$ components of the total electric field will then be given by:
\begin{align*}
E_x &= \int dE\cos\theta=\int k\frac{dq}{r^2}\cos\theta \\
E_y &= -\int dE\sin\theta=-\int k\frac{dq}{r^2}\sin\theta\\
\end{align*}
Again, before proceeding with the integrals, we consider symmetry. Specifically, if we consider a charge $dq'$ located symmetrically about the $x$ axis from $dq$ (as illustrated in Figure \ref{fig:chargesfields:finiteline}), we see that the $y$ component of the electric field element $d\vec E'$ that it creates will cancel the $y$ component of $d\vec E$. For each choice of $dq$, there will exist a corresponding choice $dq'$ which will result in the $y$ component of the net electric field being zero. We thus only need to evaluate the $x$ component of the total electric field:
\begin{align*}
\vec E = E_x \hat x = \left(\int k\frac{dq}{r^2}\cos\theta\right) \hat x
\end{align*}
Within the integrand, both $r$ and $\theta$ will change as we sum over the different charges $dq$ along the rod. A straightforward option to write the integral is to use $y$ as the integration constant, and to write $dq$, $r$, and $\cos\theta$ in terms of $y$. The charge $dq$ covers an infinitesimal length of the rod, $dy$. Since the rod is uniformly charged, the charge per unit length must be the same over a small length $dy$ as it is over the whole length of the rod:
\begin{align*}
\frac{dq}{dy}&=\frac{Q}{L}\\
\therefore dq &= \frac{Q}{L} dy
\end{align*}
It is often useful to introduce a constant charge per unit length, $\lambda=\frac{Q}{L}$, so that we can write the charge $dq$ as:
\begin{align*}
dq = \lambda dy
\end{align*}
We can also express $r^2$ and $\cos\theta$ in terms of $y$ (and $R$, which is constant):
\begin{align*}
r^2 &= y^2+R^2\\
\cos\theta&=\frac{R}{r}=\frac{R}{\sqrt{y^2+R^2}}
\end{align*}
Finally, we can combine this all into an integral that we can evaluate:
\begin{align*}
 E_x &=\int k\frac{dq}{r^2}\cos\theta\\
 &=k\int_{-L/2}^{L/2} \lambda \frac{1}{y^2+R^2}\frac{R}{\sqrt{y^2+R^2}} dy\\
 &=kR\lambda\int_{-L/2}^{L/2} \frac{1}{(y^2+R^2)^{\frac{3}{2}}} dy\\
 &=kR\lambda \left[  \frac{y}{R^2\sqrt{y^2+R^2}}\right]_{-L/2}^{L/2}\\
 \therefore E_x &= \frac{k\lambda}{R}\frac{L}{\sqrt{\left(\frac{L}{2}\right)^2+R^2}}  
\end{align*}
If the rod were infinitely long (or very long compared to the distance $R$), the electric field becomes:
\begin{align*}
\lim_{L\to\infty}E_x=\frac{2k\lambda}{R}
\end{align*}
By using the charge per unit length, $\lambda$, we were able to easily generalize our result to that expected for an infinitely long rod with uniform charge density.

Solving the integral above in terms of the integration variable $y$ is difficult without some knowledge of integrals. For this specific integral, the easiest method to use from calculus is ``trig substitution''. We show below how we can arrive at a much easier integral if we had instead chosen the angle $\theta$ as the integration variable instead of $y$, and we will see that this is a physical illustration of the ``trig substitution method'' from calculus!

We go back to step 7 in our procedure and choose $\theta$ as the integration variable for the integral:
\begin{align*}
E_x &=\int k\frac{dq}{r^2}\cos\theta\\
\end{align*}
That is, we need to express $1/r^2$ and $dq$ in terms of $\theta$. Referring to Figure \ref{fig:chargesfields:finiteline}, we have:
\begin{align*}
r &= \frac{R}{\cos\theta}\\
\therefore \frac{1}{r^2}&=\frac{\cos^2\theta}{R^2}\\
y &= R\tan\theta\\
\therefore dy &= \frac{dy}{d\theta}d\theta=\frac{R}{\cos^2\theta}d\theta\\
\therefore dq &= \lambda dy =\lambda\frac{R}{\cos^2\theta}d\theta
\end{align*}
The only difficulty is in determining the angle $d\theta$ subtended by $dq$, which was determined above by first relating $dy$ and $d\theta$. With these substitutions, the integral becomes trivial:
\begin{align*}
E_x &=\int k\frac{dq}{r^2}\cos\theta\\
&=k\int_{-\theta_0}^{\theta_0} \lambda\frac{R}{\cos^2\theta} \frac{\cos^2\theta}{R^2} \cos\theta d\theta=\frac{k\lambda}{R}\int_{-\theta_0}^{\theta_0}\cos\theta d\theta=\frac{k\lambda}{R}\left[\sin\theta \right]_{-\theta_0}^{\theta_0}\\
&=\frac{2k\lambda}{R}\sin\theta_0
\end{align*}
where $\theta_0$ is the angle subtended by half of the rod. Referring to Figure \ref{fig:chargesfields:finiteline}, we can easily see that:
\begin{align*}
\sin\theta_0=\frac{L/2}{\sqrt{\left(\frac{L}{2}\right)^2+R^2}}
\end{align*}
So that the total electric field is given by:
\begin{align*}
E_x &=\frac{2k\lambda}{R}\sin\theta_0=\frac{k\lambda}{R}\frac{L}{\sqrt{\left(\frac{L}{2}\right)^2+R^2}}
\end{align*}
as found before. Furthermore, in the limit of an infinitely long rod, the angle $\theta_0$ tends to $\frac{\pi}{2}$, so that the electric field becomes:
\begin{align*}
E_x=\lim_{\theta_0\to\frac{\pi}{2}}\frac{2k\lambda}{R}\sin\theta_0=\frac{2k\lambda}{R}
\end{align*}
\textbf{Discussion:} In this example, we saw how to apply the principle of superposition to determine the electric field near a finite and a infinite line of charge with constant charge per unit length. We showed that it was relatively straightforward to set up the integral in terms of $dy$, but not so easy to solve the integral. We then showed that by using $\theta$ as the integration variable, we could arrive at a much easier integral. This change of variable corresponds to a physical variable in our problem, but is also the basis for the more abstract ``trig substitution'' method used to solve integrals in calculus.
\end{example}

\begin{example}{Calculate the electric field a distance, $a$, above a infinite plane that carries uniform charge per unit area, $\sigma$.}
In this case, we need to determine the field above an object that is two dimensional (a plane). In the previous examples (a ring, a line of charge), we modelled a one dimensional object (e.g. the line), as being made of many point charges (0-dimensional objects). We treated those point charges has having an infinitesimal length along the object so that we could sum them together to obtain the object (e.g. $dy$ was the length of the charge for the rod/line of charge).
 
In order to model the two-dimensional object (the plane), we model it has being the sum of many one dimensional objects. We can model a plane either as a rectangle of width, $W$, and length, $L$, as shown in the left panel of Figure \ref{fig:chargesfields:planecharge} or as a disk of radius, $R$, as shown in the right panel. To model an infinite plane, we can then take the limit of either $L$ and $W$ going to infinity (rectangle), or of $R$ going to infinity (disk). We can model the rectangle as being the sum of many lines of \textbf{finite} length, $L$, and infinitesimal width, $dx$. Similarly, we can model the disk as the sum of infinitesimally thin rings of \textbf{finite} radius, $r$, and thickness, $dr$. In both cases, we know how to model the field from a line of charge (Example \ref{ex:chargesfields:finiteline}) or from a ring (Example \ref{ex:chargesfields:ring}). 
\capfig{0.8\textwidth}{figures/ChargesFields/planecharge.png}{\label{fig:chargesfields:planecharge}A two-dimensional object such as a plane modelled as a the sum of infinitely thin lines (left panel) or as the sum of infinitely thin rings (right panel).}
We proceed by modelling the plane as a disk made up of infinitesimal rings. Our infinitesimal charge, $dq$, is thus that of a ring of radius $r$ and thickness $dr$, as illustrated in Figure \ref{fig:chargesfields:disk}.
\capfig{0.3\textwidth}{figures/ChargesFields/disk.png}{\label{fig:chargesfields:disk}Modelling the field from a disk as the sum of fields from concentric thin rings.}
We know from Example \ref{ex:chargesfields:ring} that the magnitude of the electric field a distance $a$ from the centre of the ring, along its axis of symmetry (the $z$ axis in Figure \ref{fig:chargesfields:disk}), is given by:
\begin{align*}
dE = kdq\frac{a}{(r^2+a^2)^\frac{3}{2}} 
\end{align*}
By symmetry, for all of the different infinitesimal rings that make up the disk, the field will always point along the $z$ axis. In order to determine the total field, we sum (integrate) the values of $dE$, over all of the rings, from a radius of $r=0$ to a radius $r=R$. For each ring, the value of $r$ will be different, so we need to express $dq$ in terms of $dr$ in order to perform the integral. We know that the plane has a uniform charge per unit area given by $\sigma$. The charge $dq$ of an infinitesimal ring is given by:
\begin{align*}
dq = \sigma dA=\sigma 2\pi r dr
\end{align*}
where $dA=2\pi r dr$ is the area of the infinitesimal ring of radius $r$ and thickness $dr$ (think of unfolding the ring into a rectangle of height $dr$ and length $2\pi r$, the circumference of the circle, in order to determine the area). We now have all of the ingredients in order to determine the total electric field:
\begin{align*}
E &= \int dE = \int_0^R kdq\frac{a}{(r^2+a^2)^\frac{3}{2}}  = 2\pi k a \sigma \int_0^R \frac{r}{(r^2+a^2)^\frac{3}{2}}dr\\
&=2\pi k a \sigma \left[  \frac{-1}{\sqrt{r^2+a^2}}\right]_0^R=2\pi k  \sigma\left(1-\frac{a}{R^2+a^2} \right)
\end{align*}
Finally, we can take the limit of $R\to\infty$ in order to get the electric field above an infinite plane:
\begin{align*}
E=\lim_{R\to\infty}2\pi k  \sigma\left(1-\frac{a}{R^2+a^2} \right)=2\pi k\sigma=\frac{\sigma}{2\epsilon_0}
\end{align*}
where we used $\epsilon_0$ in the last equality as the result is a little cleaner without the factors of $\pi$. Note that for an infinite plane of charge, the electric field does not depend on the distance (our variable $a$) from the plane!

\textbf{Discussion:} In this example, we showed how we can model a two-dimensional charge distribution as the sum of one-dimensional charge distributions. In particular, we showed that an infinite plane of charge can be modelled as the sum of many lines charges or of many rings of charge (we chose the latter in the above). We also found that the electric field above an infinite plane of charge does not depend on the distance from the plane; that is, the electric field is constant above an infinite plane of charge. 
\end{example}
\section{The electric dipole}
Electric dipoles are a specific combination of a positive charge $+Q$ held at a fixed distance, $l$, from an equal and opposite charge, $-Q$, as illustrated in Figure \ref{fig:chargesfields:dipole}. Dipoles can be represented by their ``electric dipole vector'' (or ``electric dipole moment''), $\vec p$, defined to point in the direction \textbf{from the negative charge to the positive charge}, with magnitude:
\begin{align*}
p=Ql
\end{align*}
\capfig{0.3\textwidth}{figures/ChargesFields/dipole.png}{\label{fig:chargesfields:dipole}An electric dipole.}
Dipoles arise often in nature, for example, a water molecule can be modelled as a dipole, because the two hydrogen atoms are not symmetrically arranged around the oxygen atom. The electrons in a water molecule tend to stay closer to the oxygen atom, which acquires an excess of 2 electrons, while each proton has a deficit of 1 electron, resulting in a separation of charge (polarization), which can be modelled as a an electric dipole, as in Figure \ref{fig:chargesfields:h20}.
\capfig{0.1\textwidth}{figures/ChargesFields/h20.png}{\label{fig:chargesfields:h20}A water molecule can be modelled as an electric dipole.}
When a dipole is immersed in a uniform electric field, as illustrated in Figure \ref{fig:chargesfields:dipoleinfield}, the net force on the dipole is zero because the force on the positive charge will always be equal and in the opposite direction from the force on the negative charge. 
\capfig{0.6\textwidth}{figures/ChargesFields/dipoleinfield.png}{\label{fig:chargesfields:dipoleinfield}An electric dipole in a uniform electric field.}
Although the net force on the dipole is zero, there is still a net torque about its centre that will cause the dipole to rotate (unless the dipole vector is already parallel to the electric field vector). If the dipole vector makes an angle of $\theta$ with the electric field vector (as in Figure \ref{fig:chargesfields:dipoleinfield}), the magnitude of the net torque on the dipole about its centre is given by:
\begin{align*}
\tau=\frac{l}{2}F^+\sin\theta+\frac{l}{2}F^-\sin\theta=\frac{l}{2}QE\sin\theta+\frac{l}{2}QE\sin\theta=QlE\sin\theta=pE\sin\theta
\end{align*}
In Figure \ref{fig:chargesfields:dipoleinfield}, the torque vector is into the page (the forces will make it rotate clockwise), which is the same direction as the cross product, $\vec p \times \vec E$. Note that the magnitude of the torque is also equal to the magnitude of the cross product. Thus, in general, the torque vector on a dipole, $\vec p$, from an electric field, $\vec E$, is given by:
\begin{align*}
\Aboxed{\vec \tau =\vec p \times \vec E}
\end{align*}
In particular, note that the torque is zero when the dipole and electric field vectors are parallel. Thus, a dipole will always experience a torque that tends to align it with the electric field vector. The dipole is thus in a stable equilibrium when it is parallel to the electric field.
\begin{checkpoint}\label{cp:chargesfields:efield}
\begin{MCquestion}{When an electric dipole is such that its dipole vector is anti-parallel to the electric field vector, the dipole is}
\item not in equilibrium.
\item in a stable equilibrium.
\item in an unstable equilibrium. \correct
\end{MCquestion}
\end{checkpoint}

We can also model the behaviour of the dipole using energy. If a dipole is rotated away from its equilibrium orientation and released, it will gain (rotational) kinetic energy as it tries to return to equilibrium, and will oscillate about the equilibrium position. When the dipole is held out of equilibrium, we can think of it has having potential energy. To determine the functional form of that potential energy function, we consider the work done in rotating the dipole from an angle $\theta_1$ to an angle $\theta_2$ (where the angle is between the dipole and the electric field vectors):
\begin{align*}
W&=\int_{\theta_1}^{\theta_2} \tau d\theta=\int_{\theta_1}^{\theta_2} -pE\sin\theta d\theta=-pE\int_{\theta_1}^{\theta_2} \sin\theta d\theta\\
&=pE[\cos\theta]_{\theta_1}^{\theta_2}=pE\cos\theta_2-pE\cos\theta_1
\end{align*}
where the negative sign in the torque is to indicate that the torque is in the opposite direction from increasing $\theta$ (in Figure \ref{fig:chargesfields:dipoleinfield}, the torque is clockwise whereas the angle $\theta$ increases counter-clockwise). The net work done in going from position $\theta_1$ to $\theta_2$ is the negative of the change in potential energy in going from $\theta_1$ to $\theta_2$. Thus, we define the potential energy of an electric dipole, $\vec p$, in an electric field, $\vec E$, as:
\begin{align*}
\Aboxed{U=-pE\cos\theta=-\vec p\cdot \vec E}
\end{align*}
which has a negative sign, and we also recognize that this is equivalent to the scalar product between $\vec p$ and $\vec E$. Note that the negative sign makes sense because systems experience a force/torque that will decrease their potential energy. When the angle is zero, $\cos\theta$, is maximal. Since we need the position with $\theta=0$ to have the lowest potential energy, the minus sign guarantees that all values of $\theta$ other than zero will give a potential energy that is higher. Remember that only changes in potential energy are relevant, so the minus sign should not bother you, although you should think about whether it makes sense.
\newpage
\section{Summary}

\begin{chapterSummary}
Objects can acquire a net charge if they acquire a net excess or deficit of electrons. Charges are never created, they are only transferred from one object to another. One can charge an object by friction, conduction, or induction. Materials can be classified broadly as conductors, where electrons can move freely in a material, or conductors, in which electrons remain tightly bound to the atoms in the material. If a conducting object acquires a net charge, those charges will migrate to the surface of the conductor.

Coulomb was the first to quantitatively describe the electric force exerted on a point charge, $Q_1$, by a second point charge, $Q_2$, located a distance, $r$, away:
\begin{align*}
\vec F_{12}=k\frac{Q_1Q_2}{r^2}\hat r_{21}=\frac{1}{4\pi\epsilon_0}\frac{Q_1Q_2}{r^2}\hat r_{21}
\end{align*}
where $\hat r_{21}$ is the unit vector from  $Q_2$ to $Q_1$. One can write the force using either Coulomb's constant, $k$, or the permittivity of free space, $\epsilon_0$. Coulomb's force is attractive if the product $Q_1Q_2$ is negative, and repulsive if the product is positive. Thus, charges of the same sign exert a repulsive force on each other, whereas opposite charges exert an attractive for on each other. 

Mathematically, Coulomb's Law is identical to the gravitational force in Newton's Universal Theory of Gravity, which implies that it is conservative. The electric field vector at some position in space is defined to be the electric force per unit charge at that position in space. That is, at some position in space where the electric field vector is $\vec E$, a charge, $q$, will experience an electric force:
\begin{align*}
\vec F=q\vec E
\end{align*} 
much like a mass, $m$, will experience a gravitational force, $m\vec g$, in a position in space where the gravitational field is $\vec g$. A positive charge will experience a force in the same direction as the electric field, whereas a negative charge will experience a force in the direction opposite of the electric field. The electric field at position, $\vec r$, from a point charge, $Q$, located at the origin, is given by:
\begin{align*}
\vec E = k\frac{Q}{r^2}\hat r
\end{align*}
One can visualize an electric field by using ``field lines''. The field vector at any point in space has a magnitude that is proportional to the number of field lines at that point, and a direction that is tangent to the field lines at that point.

We can model the electric field from a continuous charged object (i.e. not a point charge) by modelling the object as being made up of many point charges. Often, it is easiest to model an $N$-dimensional object as being the sum of objects of dimension $N-1$ and an infinitesimal length in the remaining dimension. For example, we modelled a line of charge as the sum of point charges that have an infinitesimal length, and we modelled a disk of charge as the the sum of rings that have an infinitesimal thickness. In general, the strategy to model the electric field from a continuous distribution of charge is the same:

\begin{enumerate}
\item Make a \textit{good} diagram.
\item Choose a charge element $dq$.
\item Draw the electric field element, $d\vec E$, at the point of interest.
\item Write out the electric field element vector, $d\vec E$, in terms of $dq$ and any other relevant variables.
\item Think of symmetry: will any of the component of $d\vec E$ sum to zero over all of the $dq$?
\item Write the total electric field as the sum (integral) of the electric field elements.
\item Identify which variables change as one varies the $dq$ and choose an integration variable to express $dq$ and everything else in terms of that variable and other constants.
\item Do the sum (integral).
\end{enumerate}

Finally, we introduced the electric dipole, which is an object comprised of two equal and opposite charges, $+Q$ and $-Q$, held at fixed distance, $l$, from each other. One can model an electric dipole using its dipole vector, $\vec p$, defined to point in the direction from $-Q$ to $+Q$, with magnitude:
\begin{align*}
p=Ql
\end{align*} 
When a dipole is immersed in a uniform electric field, $\vec E$, it will experience a torque given by:
\begin{align*}
\vec\tau=\vec p\times \vec E
\end{align*}
The torque will act such as to align the vector $\vec p$ with the electric field vector. We can define a potential energy, $U$, to model the energy that is stored in a dipole when it is not aligned with the electric field:
\begin{align*}
U=-\vec p \cdot \vec E
\end{align*}
The point of lowest potential energy corresponds to the case when $\vec p$ and $\vec E$ are parallel, whereas the point of highest potential energy is when the two vectors are anti-parallel.
\end{chapterSummary}

\newpage
\begin{importantEquations}
\medskip
\begin{multicols}{2}
\textbf{Momentum of a point particle:}
\begin{align*}
\vec p = m\vec v \\
\frac{d}{dt}\vec p = \sum \vec F = \vec F^{net}
\end{align*}
\columnbreak
\\
\textbf{Position of the Centre of Mass \\ of a system:}
\begin{align*}
\vec r_{CM} &=\frac{1}{M}\sum_i m_i\vec r_i 
\end{align*}
\medskip
\end{multicols}
\end{importantEquations}

\newpage
\section{Thinking about the material}

\begin{chapteractivity}{Reflect and research}
{
\item Explain
}
\end{chapteractivity}

\begin{chapteractivity}{To try at home}
{
\item Try
}
\end{chapteractivity}

\begin{chapteractivity}{To try in the lab}
{
\item Propose an experiment
}
\end{chapteractivity}

\newpage
\section{Sample problems and solutions}
%TODO: Sample problem: show that for small angles, dipoles oscillate with SHM
%TODO: Sample problem: Electric field at the centre of a square wire of charge (some segments positive, some negative). No need to have to do the integral, just show that you can use the result in the example to get the field from something more complicated.
\subsection{Problems}
\begin{problem}{soln:template:ballistic}{\label{prob:template:ballistic} 

}
\end{problem}
\newpage
\subsection{Solutions}
\begin{solution}{prob:template:ballistic}\label{soln:template:ballistic}

\end{solution}

