\section{Simple Harmonic Motion}

%%%%%%%%%%%%%%%%%%%%%%%%%%%%%%%%%%%
%%
%% Multiple Choice
%%
%%%%%%%%%%%%%%%%%%%%%%%%%%%%%%%%%%%
\subsection{Multiple Choice}

\question You wish to double the period of a simple pendulum (a mass at the end of a string), which should you do?
\begin{checkboxes}
	\choice Double the mass of the bob
	\CorrectChoice Quadruple the length of the string
	\choice Double the length of the string
	\choice Quadruple the mass
\end{checkboxes}

\question You have constructed a clock that is based on the oscillations of a simple pendulum of length $L$. You bring your clock to a planet where gravity at the surface is double the strength of gravity at the surface of the Earth (where you designed the clock). What should the length of the pendulum be in order for your clock to still tick at the same rate on the new planet?
\begin{checkboxes}
	\CorrectChoice $2L$
	\choice $4L$
	\choice $\frac{1}{2}L$
	\choice $\frac{1}{4}L$
\end{checkboxes}

\question Which of the following simple harmonic oscillators has the lowest frequency?
\begin{checkboxes}
	\choice a mass $M$ on a horizontal frictionless surface attached to a horizontal spring with spring constant $2k$.
	\choice a mass $M$ attached to a vertical spring with spring constant $2k$.
	\CorrectChoice a mass $M$ on a horizontal frictionless surface attached to two horizontal springs in series, each with spring constant $k$.
	\choice a mass $M$ on a horizontal frictionless surface attached to two horizontal springs in parallel, each with spring constant $k$.
	\choice None of the above, they all oscillate with the same frequency.
\end{checkboxes}

\question The position of a frictionless horizontal spring mass-system is described by $x(t)=A\cos(\omega t + \phi)$, where $x=0$ corresponds to the rest length of the spring. At time $t=0$ the mass has its maximal speed and is moving in the positive $x$ direction. What is the value of $\phi$?
\begin{checkboxes}
	\choice $0$
	\choice $\frac{\pi}{4}$
	\choice $\frac{\pi}{2}$
	\CorrectChoice $\frac{3\pi}{2}$
\end{checkboxes}

\question The position of a frictionless horizontal spring mass-system is described by $x(t)=A\cos(\omega t + \phi)$, where $x=0$ corresponds to the rest length of the spring. At time $t=0$ the mass has its maximal speed and is moving in the negative $x$ direction. What is the value of $\phi$?
\begin{checkboxes}
	\choice $\frac{\pi}{4}$
	\CorrectChoice $\frac{\pi}{2}$
	\choice $\frac{3\pi}{2}$
	\choice $\pi$
\end{checkboxes}

\question The position of a frictionless horizontal spring mass-system is described by $x(t)=A\cos(\omega t + \phi)$, where $x=0$ corresponds to the rest length of the spring. At time $t=0$ the mass has a speed of zero and its position is along the negative $x$ axis. What is the value of $\phi$?
\begin{checkboxes}
	\choice $\frac{\pi}{4}$
	\choice $\frac{\pi}{2}$
	\choice $\frac{3\pi}{2}$
	\CorrectChoice $\pi$
\end{checkboxes}


%Victor Li
\question  Which one of the following increases the period of motion of a horizontal frictionless mass-spring system?
\begin{checkboxes}
	\CorrectChoice Increasing the mass
	\choice Increasing the spring constant
	\choice Increasing the amplitude of the motion
	\choice Increasing the maximum speed of the mass
\end{checkboxes}


%Tashifa Imtiaz
\question  A basketball is bouncing up and down in a straight line. Assume that the collision with the ground is elastic and that there is no air resistance. Is this an example of simple harmonic motion?
\begin{checkboxes}
	\choice Yes 
	\CorrectChoice No
\end{checkboxes}

\question A mass attached to a horizontal spring is undergoing simple harmonic motion. If the maximum amplitude of the displacement from the equilibrium position of the spring is increased by a factor of 3, by what factor does the maximal acceleration of the mass increase?
\begin{checkboxes}
\choice $\sqrt{3}$
\CorrectChoice 3 \correct
\choice 9
\choice it does not change
\end{checkboxes}

\question A mass attached to a horizontal spring is undergoing simple harmonic motion. If the maximum amplitude of the displacement from the equilibrium position of the spring is increased by a factor of 3, by what factor does the period of the oscillations change?
\begin{checkboxes}
\choice $\sqrt{3}$
\choice 3
\choice 9
\CorrectChoice it does not change \correct
\end{checkboxes}

\question A mass attached to a horizontal spring is undergoing simple harmonic motion. If the maximum amplitude of the displacement from the equilibrium position of the spring is increased by a factor of 3, by what factor does the mechanical energy of the mass change (assuming zero potential energy when the spring is at its equilibrium length)?
\begin{checkboxes}
\choice $\sqrt{3}$
\choice 3
\CorrectChoice 9 \correct
\choice it does not change
\end{checkboxes}

%Question submitted by Matt Routliffe
\question A one-dimensional system is at rest. If the potential energy graph on both the left and right side has a negative slope, what type of equilibrium is it currently in?
\begin{checkboxes}
\choice Stable Equilibrium
\CorrectChoice Unstable Equilibrium \correct
\choice Cannot be answered with information given
\end{checkboxes}


%based on question by Nick Brown
\question A mass attached to a horizontal spring is undergoing simple harmonic motion. If the maximum amplitude of the displacement from the equilibrium position of the spring is increased by a factor of 3, by what factor does the maximal acceleration of the mass increase?
\begin{checkboxes}
\choice $\sqrt{3}$
\CorrectChoice 3 \correct
\choice 9
\choice it does not change
\end{checkboxes}

\question A mass attached to a horizontal spring is undergoing simple harmonic motion. If the maximum amplitude of the displacement from the equilibrium position of the spring is increased by a factor of 3, by what factor does the period of the oscillations change?
\begin{checkboxes}
\choice $\sqrt{3}$
\choice 3
\choice 9
\CorrectChoice it does not change \correct
\end{checkboxes}

\question A mass attached to a horizontal spring is undergoing simple harmonic motion. If the maximum amplitude of the displacement from the equilibrium position of the spring is increased by a factor of 3, by what factor does the mechanical energy of the mass change (assuming zero potential energy when the spring is at its equilibrium length)?
\begin{checkboxes}
\choice $\sqrt{3}$
\choice 3
\CorrectChoice 9 \correct
\choice it does not change
\end{checkboxes}


\question A pendulum of length \SI{1}{m} in Mars' gravitational potential (g = \SI{3.7}{m/s^2}) swings from its maximum amplitude of $\theta_0 = \SI{4}{\degree}$.  The bob of the pendulum has mass of \SI{2}{kg}, and the rest of the rod is massless.  At the bottom of the arc, the pendulum collides with a horizontal spring at equilibrium with $k=\SI{100}{kg/s^2}$.  The bob detaches from the pendulum and attaches to the spring and starts oscillating in the horizontal plane.  How much time does it take from the maximum amplitude of the pendulum, until the maximum compression of the spring?  
\begin{checkboxes}
\choice 1.04 s 
\CorrectChoice 1.26 s\correct
\choice 2.08 s
\choice 0.76 s
\end{checkboxes}


\question A torsional spring is brought from Earth to a another planet with a smaller gravitational acceleration. How will its period on the new planet compare to its period on Earth?
\begin{checkboxes}
\choice It will be larger
\choice It will be smaller
\CorrectChoice It will be the same \correct
\end{checkboxes}

\question A mass attached to a horizontal spring is undergoing simple harmonic motion. If the maximum amplitude of the displacement from the equilibrium position of the spring is increased by a factor of 3, by what factor does the maximal acceleration of the mass increase?
\begin{checkboxes}
\choice $\sqrt{3}$
\CorrectChoice 3 \correct
\choice 9
\choice it does not change
\end{checkboxes}

\question A mass attached to a horizontal spring is undergoing simple harmonic motion. If the maximum amplitude of the displacement from the equilibrium position of the spring is increased by a factor of 3, by what factor does the period of the oscillations change?
\begin{checkboxes}
\choice $\sqrt{3}$
\choice 3
\choice 9
\CorrectChoice it does not change \correct
\end{checkboxes}

\question A mass attached to a horizontal spring is undergoing simple harmonic motion. If the maximum amplitude of the displacement from the equilibrium position of the spring is increased by a factor of 3, by what factor does the mechanical energy of the mass change (assuming zero potential energy when the spring is at its equilibrium length)?
\begin{checkboxes}
\choice $\sqrt{3}$
\choice 3
\CorrectChoice 9 \correct
\choice it does not change
\end{checkboxes}

\question A simple pendulum is observed to have a period of \SI{1}{s} when undergoing simple harmonic motion in a vertical plane. How far from the axis of rotation is the mass at the end of the pendulum? 
\begin{choices} 
\choice \SI{12}{cm}
\CorrectChoice \SI{25}{cm} \correct
\choice \SI{50}{cm}
\choice \SI{78}{cm}
\end{choices}

\question A simple pendulum is made by suspending a mass $m$ at the end of a string of length $L$. One measures the angle, $\theta$, between the vertical and the position of the mass ($\theta=0$ when the mass is at the lowest point of the trajectory). What can you say about the motion of the mass, if it released in such a way that it swings in a plane?
\begin{checkboxes} 
\CorrectChoice It is simple harmonic motion only if the amplitude of the oscillations is small \correct
\choice It is simple harmonic motion regardless of the amplitude of the oscillations
\choice It is never simple harmonic motion 
\choice It is uniform circular motion
\end{checkboxes}

\question In a simple harmonic oscillator, if we double the mass and halve the spring constant, what will happen to the period of the oscillations?
\begin{checkboxes}
\choice It will stay the same
\CorrectChoice It will double \correct
\choice It will be halved
\choice It will be multiplied by $\sqrt{2}$
\end{checkboxes}

%%%%%%%%%%%%%%%%%%%%%%%%%%%%%%%%%%%
%
% long answer
%
%%%%%%%%%%%%%%%%%%%%%%%%%%%%%%%%%%%
\subsection{Long answers}
%From Marie Vidal
\question A simple pendulum is constructed by suspending a mass $m$ from a massless inextensible string of length $l$. The angle between the the vertical and the string is $\theta$ (where $\theta=0$ is the equilibrium position of the pendulum). Use the Lagrangian method to show that the equation of motion of the system is given by:
\begin{align*}
\alpha = - \frac{g}{l} \sin \theta
\end{align*}
where $\alpha$ is the angular acceleration. The Lagrangian is given by the difference between the kinetic energy, $K$, and potential energy, $U$, of the pendulum:
\begin{equation}
L(\theta, \omega) = K(\omega) - U(\theta)
\end{equation}
where the kinetic energy of the pendulum depends on its angular velocity, $\omega$, and its potential energy depends on its angular position, $\theta$. From the Lagrangian, the equation of motion for the pendulum is given by the Euler-Lagrange equation:
\begin{equation}
\die{L}{\theta} - \frac{d}{dt} \left(\die{L}{\omega} \right)= 0
\end{equation}

\begin{solution}
We can write the Lagrangian starting with the linear speed $v=\omega l$ and the height above the ground, $h=l (1 -\cos \theta)$, and then convert to angular quantities:
\begin{align*}
L &= K - U\\
  &= \frac{1}{2} m v^2 - mgh \\
  &= \frac{1}{2} m l^2 \omega^2 - mg l (1 -\cos \theta)\\
  &= \frac{1}{2} m l^2 \omega^2 - mgl + mg\cos\theta
\end{align*}
To apply the Euler-Lagrange equation, we need the following quantities:
\begin{align*}
\die{L}{\theta} &= - mg l \sin\theta\\
\die{L}{\omega} &= m l^2 \omega \\
\frac{d}{dt} \left(\die{L}{\omega} \right) &=\frac{d}{dt}( m l^2) \omega=ml^2\alpha
\end{align*}
Applying the Euler-Lagrange equation:
\begin{align*}
\die{L}{\theta} - \frac{d}{dt} \left(\die{L}{\omega} \right) &= 0\\ 
- mg l \sin\theta - m l^2 \alpha &= 0\\
\therefore \alpha &= -\frac{g}{l} \sin \theta
\end{align*}
as required.
\end{solution}

%From Marie Vidal
\question Consider two simple pendula, A and B, made from a mass at the end of a massless inextensible string. The string of pendulum A has a length of $l_A=\SI{1}{m}$, while the length of the string of pendulum B is unknown. Both pendula are released at the same time. After pendulum A has completed 12 complete oscillations, pendulum B has completed 10. What is the length of the string on pendulum B?
\begin{finalanswer}
\SI{1.44}{m}
\end{finalanswer}
\begin{solution}
The period of a simple pendulum with a string of length $l$ is given by:
\begin{align*}
 T = 2\pi\sqrt{\frac{l}{g}}
\end{align*}
For each pendulum, we thus have:
\begin{align*}
T_A &= 2\pi\sqrt{\frac{l_A}{g}}\\
T_B &= 2\pi\sqrt{\frac{l_B}{g}}
\end{align*}
We can take the ratio of the two periods:
\begin{align*}
\frac{T_A}{T_B} = \sqrt{\frac{l_A}{l_B}} = \frac{10}{12}
\end{align*}
Solving for $l_B$:
\begin{align*}
l _B = l_A \left( \frac{T_B}{T_A} \right)^2 = \SI{1}{m}\left(\frac{12}{10} \right)^2=\SI{1.44}{m}
\end{align*}
\end{solution}

%From Giancolli, 14-11 -fixed
\question A uniform metre-stick of mass $M$ is attached to a wall by a pivot. The free end of the metre stick is then attched to a spring with a spring constant $k$, as shown in Figure \ref{fig:SHO:RodSpring}. If the metre stick makes small oscillations, what is its frequency? [Hint: Use torques!] 
\capfig{0.2\textwidth}{figures/SHO/RodSpring.png}{\label{fig:SHO:RodSpring}A horizontal uniform metre stick of mass $M$ attached to a hinge (on the left) and a vertical spring on the right.}
\begin{finalanswer}
$f=\frac{1}{2\pi}\sqrt{\frac{3k}{M}}$
\end{finalanswer}
\begin{solution}
As suggested by the hint, we write the torques on the stick. If we write the torques about the hinge, then we can ignore the unknown forces exerted at the hinge. Thus, the only two forces acting on the metre stick are gravity (exerted at the centre) and the restoring force from the spring (exerted at the end). In equilibrium, the spring is stretched by some amount $y_0$, so that the sum of the torques about the hinge is zero:
\begin{align*}
\sum \tau &= 0\\
Mg\frac{1}{2}L-Lky_0=0
\end{align*} 

We introduce the angle $\theta$ to be the angle between the horizontal and the beam, when it is displaced from the horizontal. We choose $\theta$ to be positive when the beam is below the horizontal, corresponding to stretching the spring.

If the angle $\theta$ is small, then the extension of the spring will be given by $y_0+\Delta y\sim y_0+L\sin\theta\sim y_0+L\theta$, where $\Delta y$ corresponds to the vertical displacement of the tip of the metre stick (and we ignore the fact that, as the stick rotates, the tip will also have a horizontal displacement). The force from the spring, for a positive angle $\theta$ (downwards) is thus given by:
\begin{align*}
F_k = -k(y_0+\Delta y) = -ky_0-kL\theta
\end{align*}
For a positive displacement $\theta$, gravity will exert a positive torque, while the spring will exert a negative torque. If we assume that the angle $\theta$ is small enough, then we can assume that both gravity and the spring force are perpendicular to the beam, and write the sum of torques as:
\begin{align*}
\sum \tau &= I\alpha\\
Mg\frac{1}{2}L-Lky_0-kL^2\theta &=I\alpha
\end{align*}
Using the equilibrium condition, the contribution from gravity cancels, and we get the equation for simple harmonic motion:
\begin{align*}
I\alpha &=-kL^2\theta \\
\frac{d^\theta}{dt^2}&=-\frac{kL^2}{I}\theta\\
&=-\frac{kL^2}{\frac{1}{3}ML^2}\theta\\
&=-\frac{3k}{M}\theta=-\omega^2\theta\\
\therefore \omega = \sqrt{\frac{3k}{M}}
\end{align*}
where we used the moment of inertia for a rod about its end, $I=\frac{1}{3}ML^2$. The frequency is then given by:
\begin{align*}
f = \frac{\omega}{2\pi}=\frac{1}{2\pi}\sqrt{\frac{3k}{M}}
\end{align*}
\end{solution}

\question You have been hired to design an efficient long-distance transport system to compete with the HyperLoop. You propose to dig a straight tunnel to connect two points across the Earth, as illustrated in Figure \ref{fig:SHO:SHMTransport}. The tunnel is under vacuum (so that there is no friction), and a train on a frictionless track is used to move in the tunnel. The train is only powered by gravity, thus making the system very energy efficient. You assume that the Earth has a uniform density.
\capfig{0.2\textwidth}{figures/SHO/SHMTransport.png}{\label{fig:SHO:SHMTransport}A tunnel connecting two points on Earth.}
\begin{parts}
\part Show that the motion of the train in the tunnel is simple harmonic motion.
\part If you had such a tunnel constructed between Kingston, ON, Canada, and Coinsins, Switzerland, how long would the one-way journey take? How does this compare to a commercial airline flight?
\end{parts}
\textbf{Hint}: Remember to use Gauss' Law to take into account the fact that the force of gravity is less nearer the centre of the Earth.
\begin{finalanswer}
\begin{enumerate}[(a)]
\item You should get the expression $\frac{d^2x}{dt^2}=-G\frac{M}{R^3}x$, which shows that it is simple harmonic motion.
\item \SI{42.18}{min}, which is approximately 10 times faster than a flight.
\end{enumerate}
\end{finalanswer}
\begin{solution}
\begin{parts}
\part Figure \ref{fig:SHO:SHMTransport_FBD} shows the forces on the train when it is in the tunnel, a distance $x$ from the centre of the tunnel. The force of gravity is towards the centre of the Earth, and thus has a component that is not cancelled by the normal force which ``powers'' the train. 
\capfig{0.4\textwidth}{figures/SHO/SHMTransport_FBD.png}{\label{fig:SHO:SHMTransport_FBD}Free body diagram for forces on the train when in the tunnel.}
The normal force is always perpendicular to the tunnel. The net force on the train comes from the component of gravity that is not cancelled by the the normal force. This component if responsible for the acceleration of the train. If we take the $x$-axis to be parallel to the tunnel, the origin to be at the centre of the tunnel, and the positive $x$ direction to be to the right, Newton's Second Law gives:
\begin{align*}
\sum F_x &= m_Ta\\
-F_g\sin\theta &= m_Ta\\
\therefore \frac{d^2x}{dt^2}&=-\frac{1}{m_T}F_g\sin\theta
\end{align*}
where $m_T$ is the mass of the train. We need to know how both $\theta$ and $F_g$ depend on $x$ to determine if this is simple harmonic motion. Figure \ref{fig:SHO:SHMTransport_GEOM} shows the relation between the angle $\theta$, the distance between the train and the centre of the Earth, $r$, and the position of the train in the tunnel, $x$. 
\capfig{0.3\textwidth}{figures/SHO/SHMTransport_GEOM.png}{\label{fig:SHO:SHMTransport_GEOM} Geometry for the tunnel.}
Gauss' Law tells us that only the mass of the Earth at radii smaller than the position of the train will contribute to the force of gravity. The mass of the Earth below radius $r$ is given by:
\begin{align*}
m(r)&=\rho \frac{4}{3}\pi r^3\\
&=\frac{M}{\frac{4}{3}\pi R^3} \frac{4}{3}\pi r^3\\
&=M\frac{r^3}{R^3}
\end{align*} 
where $\rho$ is the density of the Earth, $M$ is the mass of the Earth, and $R$ is the radius of the Earth. The force of gravity at position $r$ is thus given by:
\begin{align*}
F_g &= G\frac{m_Tm(r)}{r^2}\\
&=G\frac{m_T}{r^2}M\frac{r^3}{R^3}\\
&=G\frac{m_TM}{R^3}r
\end{align*}
Putting this into the equation of motion:
\begin{align*}
\frac{d^2x}{dt^2}&=-\frac{1}{m_T}F_g\sin\theta \\
&=-\frac{1}{m_T}G\frac{m_TM}{R^3}r\sin\theta\\
&=-G\frac{M}{R^3}r\sin\theta\\
&=-G\frac{M}{R^3}x\\
\therefore \frac{d^2x}{dt^2}&=-G\frac{M}{R^3}x
\end{align*}
where we have used the fact that $r\sin\theta=x$. This does indeed correspond to simple harmonic motion!
\part The frequency of the simple harmonic motion does not depend on the location of the tunnel. The angular frequency is given by:
\begin{align*}
\frac{d^2x}{dt^2}&=-\omega^2x\\
\therefore \omega &= \sqrt{G\frac{M}{R^3}}
\end{align*}
To go from one end of the tunnel to the other, it will take half of the period:
\begin{align*}
\frac{1}{2}T&=\frac{1}{2}\frac{2\pi}{\omega}\\
&=\pi\sqrt{\frac{R^3}{GM}}=\pi\sqrt{\frac{(\SI{6.37e6}{m})^3}{(\SI{6.67e-11}{Nm^2/kg^2})(\SI{5.97e24}{kg})}}\\
&=\SI{2531.1}{s}=\SI{42.18}{min}
\end{align*}
so just under an hour. A quick internet search reveals that flights between Montreal (close to Kingston) and Geneva (close to Coinsins) take about \SI{7}{hr}, so the tunnel is approximately 10 times faster! Interestingly, it will take \SI{42.18}{min} to travel between any two points on Earth, so the tunnel is slower between locations that you can cover in less than that amount of time!
\end{parts}
\end{solution}

\question A mass $m=\SI{300}{g}$ lies on a frictionless horizontal surface and is attached to a horizontal spring with a spring constant $k=\SI{3}{N/m}$. A coordinate system is given such that the $x$ axis is parallel to the motion of the mass under the action of the spring, and the origin is located at the un-stretched position of the spring. The position of the mass is given by:
\begin{align*}
x(t) = A\cos(\omega t+\phi)
\end{align*}
At time $t=\SI{2}{s}$, the position of the mass $m$ is $x=\SI{0.2}{m}$, and its velocity is $v=\SI{-0.5}{m/s}$.
\begin{parts}
\part Determine $A$, $\omega$, and $\phi$, and make a plot of $x(t)$ and $v(t)$ between $t=\SI{0}{s}$ and $t=\SI{10}{s}$.
\part What is the mechanical energy of the mass?
\part What is the speed of the mass at position $x=\frac{A}{2}$?
\end{parts}
\begin{finalanswer}
\begin{enumerate}[(a)]
\item $A=\SI{0.26}{m}$,$\omega=\SI{3.16}{rad/s}$, $\phi=\SI{-323.76}{\degree}$
\capfig{0.5\textwidth}{figures/SHO/SHM_plot.png}{\label{fig:SHO:SHM_plotanswer}Position and velocity for the SHO.}
\item \SI{0.0975}{J}
\item \SI{0.69}{m/s}
\end{enumerate}
\end{finalanswer}
\begin{solution}
\begin{parts}
\part The angular frequency for a simple harmonic oscillator is given by:
\begin{align*}
\omega=\sqrt{\frac{k}{m}}=\sqrt{\frac{(\SI{3}{N/m})}{(\SI{0.3}{kg})}}=\SI{3.16}{rad/s}
\end{align*}
The velocity of the mass is given by:
\begin{align*}
v(t) &= \frac{d}{dt}x(t)\\
&=-\omega A\sin(\omega t+\phi)
\end{align*}
Using the given values for $x$ and $v$ at $t=\SI{2}{s}$, we have:
\begin{align*}
x(\SI{2}{s}) &= A\cos((\SI{3.16}{rad/s})(\SI{2}{s})+\phi) = \SI{0.2}{m}\\
v(\SI{2}{s}) &= -(\SI{3.16}{rad/s})A\sin((\SI{3.16}{rad/s})(\SI{2}{s})+\phi) = \SI{-0.5}{m/s}\\
\end{align*}
Dividing the $v$ equation by the $x$ equation gives:
\begin{align*}
(\SI{3.16}{rad/s})\tan((\SI{3.16}{rad/s})(\SI{2}{s})+\phi)&=\frac{(\SI{0.5}{m/s})}{(\SI{0.2}{m})}=\SI{2.5}{s^{-1}}\\
\therefore (\SI{3.16}{rad/s})(\SI{2}{s})+\phi &=\tan^{-1}\left(\frac{(\SI{2.5}{s^{-1}})}{(\SI{3.16}{rad/s})} \right)\\
\phi &= \tan^{-1}\left(\frac{(\SI{2.5}{s^{-1}})}{(\SI{3.16}{rad/s})} \right)-(\SI{3.16}{rad/s})(\SI{2}{s})\\
&=\SI{-5.65}{rad}=\SI{-323.76}{\degree}
\end{align*}
where it is important to use radians, since $\omega =\SI{3.16}{rad/s}$ is defined in radians per second. Given the value of $\phi$, we can easily find the value of $A$ using the $x$ equation:
\begin{align*}
A &= \frac{\SI{0.2}{m}}{\cos((\SI{3.16}{rad/s})(\SI{2}{s})+(\SI{-5.65}{rad}))}\\
&=\SI{0.26}{m}
\end{align*}
We can plot this (Figure \ref{fig:SHO:SHM_plot}) with the simple python code:
\begin{verbatim}
import numpy as np
import pylab as pl

#Constants:
A = 0.26
w = 3.16
phi = -5.65

#1000 values of t between 0 and 10
tvals = np.linspace(0,10,1000)

#Define the values of x and v
xvals = A*np.cos(w*tvals+phi)
vvals = -w*A*np.sin(w*tvals+phi)

#Plot them:
pl.figure(figsize=(10,8))
pl.plot(tvals, xvals, label='x(t)')
pl.plot(tvals, vvals, label='v(t)')
pl.legend()
pl.grid()
pl.xlabel('time [s]')
pl.show()
\end{verbatim}
\capfig{0.5\textwidth}{figures/SHO/SHM_plot.png}{\label{fig:SHO:SHM_plot}Position and velocity for the SHO.}
\part The mechanical energy can be evaluated at any point of the trajectory. Since we already know the position and velocity at time $t=\SI{2}{s}$, we can find:
\begin{align*}
E&=\frac{1}{2}mv^2+\frac{1}{2}kx^2\\
&=\frac{1}{2}(\SI{0.3}{kg})(\SI{-0.5}{m/s})^2+\frac{1}{2}(\SI{3}{N/m})(\SI{0.2}{m})^2\\
&=\SI{0.0975}{J}
\end{align*}
\part Since energy is conserved, and we know the position of the mass ($x=\frac{1}{2}A=\SI{0.13}{m}$), we can easily find the speed:
\begin{align*}
\frac{1}{2}mv^2&= E-\frac{1}{2}kx^2 \\
\therefore v &=\sqrt{\frac{2E-kx^2}{m}}=\sqrt{\frac{2(\SI{0.0975}{J})-(\SI{3}{N/m})(\SI{0.13}{m})^2}{(\SI{0.3}{kg})}}\\
&=\SI{0.69}{m/s}
\end{align*}
\end{parts}
\end{solution}

%Giancolli 14-80 -fixed
\question A young man accidentally left his phone on the end of a diving board. He watches in horror as his friend jumps off the diving board, causing it to oscillate with simple harmonic motion at a frequency of \SI{3.1}{Hz}. If the phone loses contact with the diving board, it will fall into the water. What is the maximum possible amplitude of the diving board given that the the phone does not lose contant?
\begin{finalanswer}
\SI{2.58}{cm}
\end{finalanswer}
\begin{solution}
The condition of the pebble to remain on the board is that the acceleration never exceed $g$. When the board is at the top of its trajectory and start to go back down, the pebble will remain in contact as long as the acceleration of the board does not exceed $g$. The acceleration is maximal when the oscillator is at its maximum displacement from equilibrium. If the position of the end of the board is given by:
\begin{align*}
x(t)=A\cos(\omega t)
\end{align*}
such that the displacement is maximal at $t=\SI{0}{s}$. The acceleration is given by taking the time derivative twice:
\begin{align*}
a(t) = -\omega^2A\cos(\omega t)
\end{align*}
If we set the amplitude of the acceleration equal to $g$ at $t=\SI{0}{s}$ (when the board is maximally deflected - the acceleration is downwards, as is that due to gravity), then we find:
\begin{align*}
g &= \omega^2A\\
\therefore A&=\frac{g}{\omega^2}=\frac{g}{4\pi^2 f^2}=\frac{(\SI{9.8}{m/s^2})}{4\pi^2 (\SI{3.1}{Hz})^2}\\
&=\SI{2.58}{cm}
\end{align*} 
\end{solution}

\question A mass $m=\SI{2.0}{kg}$ is placed at the end of a massless inextensible string of length $L=\SI{1.0}{m}$ to form a simple pendulum. The mass is released from an angle $\theta=\SI{30}{\degree}$ from the vertical (assume that the small angle approximation is valid to describe the pendulum). When the mass arrives at the bottom of its trajectory, it detaches from the string and attaches to a horizontal spring with spring constant $k=\SI{3}{N/m}$. The spring is positioned such that it is at rest exactly at the position that the mass makes contact. 
\capfig{0.4\textwidth}{figures/SHO/PendSpring.png}{\label{fig:SHO:PendSpring}A mass swinging from a pendulum and attaching to a spring.}
\begin{parts}
\part How much time elapses between when the mass is released and when the spring is maximally compressed?
\part What is the maximum compression of the spring? 
\end{parts}
\begin{finalanswer}
\begin{enumerate}[(a)]
\item \SI{1.78}{s}
\item \SI{1.32}{m}
\end{enumerate}
\end{finalanswer}
\begin{solution}
\begin{parts}
\part We add the times for the mass to go from the initial height to the bottom of the pendulum, and then the time that it takes to compress the spring. Since both the pendulum and the spring are simple harmonic oscillators, we can just use their periods to find the times. In both cases, the times correspond to \textbf{one quarter} of the period of the oscillator. The time is thus:
\begin{align*}
t &= \frac{1}{4} (T_{pend}+T_{spring})\\
&= \frac{1}{4} \left( \frac{2\pi}{\omega_{pend}}+\frac{2\pi}{\omega_{spring}} \right)= \frac{2\pi}{4} \left(  \sqrt{\frac{L}{g}} + \sqrt{\frac{m}{k}}\right)\\
&= \frac{\pi}{2} \left(  \sqrt{\frac{(\SI{1.0}{m})}{(\SI{9.8}{m/s^2})}} + \sqrt{\frac{(\SI{2.0}{kg})}{(\SI{3}{N/m})}}\right)=\SI{1.78}{s}
\end{align*}
\part We use conservation of energy, with the initial potential gravitational energy converting to spring potential energy when the spring is maximally compressed a distance $x$. Setting zero gravitational potential energy at the bottom of the pendulum, the change in height is $h=L(1-\cos\theta)$. We thus find:
\begin{align*}
mgL(1-\cos\theta)&=\frac{1}{2}kx^2\\
\therefore x&=\sqrt{\frac{2mgL(1-\cos\theta)}{k}}\\
&=\sqrt{\frac{2(\SI{2.0}{kg})(\SI{9.8}{m/s^2})(\SI{1.0}{m})(1-\cos(\SI{30}{\degree}))}{(\SI{3}{N/m})}}
&= \SI{1.32}{m}
\end{align*}
\end{parts}
\end{solution} 

%Based on Giancolli 14-90
\question You are designing a ride for crias (baby llamas) consisting of a platform that is attached to a spring with spring constant $k=\SI{420}{N/m}$ and that slides along a frictionless surface. The platform has a mass of $M=\SI{45}{kg}$. The coefficient of static friction between a cria and the platform is $\mu_s=0.4$. You wish for the amplitude of the oscillations to be $A=\SI{1.2}{m}$. You realize that for the ride to be safe, the crias must have a minimum mass or they will slide off. What is that minimum mass?
\capfig{0.4\textwidth}{figures/SHO/LllamaRide.png}{\label{fig:SHO:LllamaRide}A cria on a platform that oscillates with SHM.}
\begin{finalanswer}
\SI{83.57}{kg}
\end{finalanswer}
\begin{solution}
The crias will slide off if the force of static friction between them and the platform is not large enough to provide the maximum acceleration of the platform. Given the position as a function of time for SHM, we can find the acceleration:
\begin{align*}
x(t)&=A\cos(\omega t)\\
\therefore v(t) &=-A\omega\sin(\omega t) \\
\therefore a(t) &=-A\omega^2\cos(\omega t)
\end{align*}
The maximum acceleration is thus:
\begin{align*}
a = A\omega^2
\end{align*}
The angular frequency of the SHM is give by:
\begin{align*}
\omega = \sqrt{\frac{k}{m+M}}
\end{align*}
We thus require that the force of static friction be at least as large as $ma$:
\begin{align*}
f=\mu_sN=\mu_smg &> ma\\
\therefore \mu_s g &> A\omega^2\\
 mu_s g &> A \frac{k}{m+M}\\
 \mu_smg+\mu_sMg &> Ak\\
 m &> \frac{Ak-\mu_sMg}{\mu_sg}\\
 &> \frac{(\SI{1.2}{m})(\SI{420}{N/m})-(0.4)(\SI{9.8}{m/s^2})(\SI{45}{kg})}{(0.4)(\SI{9.8}{m/s^2})}\\
 &>\SI{83.57}{kg}
\end{align*}
\end{solution}

<<<<<<< HEAD

\question A cylindrical cork of radius $R$ and length $L$ is floating in a liquid such that its axis of symmetry is vertical and half of the length of the cork is below the fluid (the density of the fluid is $\rho_f$), as in Figure \ref{fig:Cork}. For this problem, ignore any effects from friction, drag and the viscosity of the fluid.
\begin{parts}
\part Show that the density of the cork, $\rho_c$, is half that of the fluid ($\rho_c= \frac{1}{2}\rho_f$)
\part If the cork is displaced vertically by a distance $x$ ($x<\frac{L}{2})$ from its equilibrium (either pushed down or pulled up slightly compared to the depiction in the figure), show that the net force on the cork is given by:
\begin{align*}
F=-\pi R^2 \rho_fg x
\end{align*}
where the net force is in the opposite direction of the displacement, $x$.
\part If you press down on the cork slightly (so that part of the cork is still above the fluid) and release it, the cork will oscillate up and down. What should the length of the cork be for it to oscillate with a frequency of \SI{5}{Hz}? 
\end{parts}
\capfig{0.2\textwidth}{figures/SHO/Cork.png}{\label{fig:Cork}A cork floating in a liquid.}
\begin{solution}
\begin{parts}
\part The cork is in equilibrium when half of its volume is submerged. The buoyant force in this case is:
\begin{align*}
F_B=\pi R^2\frac{1}{2}L\rho_fg
\end{align*}
The weight of the cork, in terms of its density, is given by:
\begin{align*}
mg = (\pi R^2L\rho_c)g
\end{align*}
Since $F_B=mg$ in equilibrium, we have:
\begin{align*}
\pi R^2\frac{1}{2}L\rho_fg &=(\pi R^2L\rho_c)g\\
\therefore \rho_c&=\frac{1}{2}\rho_f
\end{align*}
\part Let us choose that positive $x$ is downwards. The net force on the cork is given by the sum of the weight (positive) and buoyancy (negative) forces:
\begin{align*}
\sum F &= mg - F_B=(\pi R^2L\rho_c)g-\pi R^2\left(\frac{1}{2}L+x\right)\rho_fg\\
&=(\pi R^2L\rho_c)g-\pi R^2\frac{1}{2}L\rho_fg -\pi R^2 \rho_fg x\\
&=-\pi R^2 \rho_fg x
\end{align*}
where the first two terms in the second line cancelled since $\rho_c= \frac{1}{2}\rho_f$ (part (a)).
\part The result from part (b) shows that the cork will undergo simple harmonic motion with frequency given by:
\begin{align*}
f&= \frac{\omega}{2\pi}=\frac{1}{2\pi} \sqrt{\frac{\pi R^2 \rho_fg }{m}}=\frac{1}{2\pi}\sqrt{\frac{\pi R^2 \rho_fg }{\pi R^2L\frac{1}{2}\rho_f}}=\frac{1}{2\pi}\sqrt{\frac{2g}{L}}\\
\therefore L&=\frac{2g}{4\pi^2 f^2}=\frac{g}{2\pi^2 f^2}=\frac{(\SI{9.8}{m/s^2})}{2\pi^2 (\SI{5}{Hz})^2}=\SI{1.99}{cm}
\end{align*}
\end{parts}
\end{solution}


%Giancolli 14-10 -fixed
\question A spring oscillating in simple harmonic motion has a frequency of $\omega_1 = \SI{1.2}{Hz}$ when a mass $m$ is at the end of the spring. The frequency of the spring becomes $\omega_2 = \SI{0.8}{Hz}$ when a mass $\delta m = \SI{0.72}{kg}$ is added to the mass $m$ at the end of the spring. What is the mass of $m$?
\begin{finalanswer}
	$m = \SI{0.576}{kg}$
\end{finalanswer}
\begin{solution}
	The frequencies given are directly related to the angular frequencies $\omega$ of the simple harmonic motion (through a factor of $2\pi$). In the first instance, the angular frequency is given by:
	\begin{align*}
	\omega_1 &= \sqrt{\frac{k}{m}}\\
	\therefore k&=\omega_1^2m
	\end{align*}
	allowing us to express the (unknown) spring constant in terms of the given frequency and the unknown mass $m$. When the additional mass is attached, the angular frequency is given by:
	\begin{align*}
	\omega_2 &= \sqrt{\frac{k}{m+\Delta m}}\\
	\end{align*}
	Since the spring constant remains the same, we can use our first expression to remove it, thus allowing us to solve for $m$:
	\begin{align*}
	\omega_2 &= \sqrt{\frac{\omega_1^2m}{m+\Delta m}}\\
	\omega_2^2(m+\Delta m)&=\omega_1^2m\\
	\omega_2^2\Delta m&=(\omega_1^2-\omega_2^2)m\\
	\therefore m&=\frac{\omega_2^2}{\omega_1^2-\omega_2^2}\Delta m=\frac{(\SI{0.80}{Hz})^2}{(\SI{1.2}{Hz})^2-(\SI{0.80}{Hz})^2}(\SI{0.72}{kg})=\SI{0.576}{kg}
	\end{align*}
	where we just used the frequencies since the factors of $2\pi$ all cancel. 
\end{solution}

%Giancolli 14-18 -fixed
\question  A tuning fork is built such that it produces a the musical note $A_6$ ($\SI{880}{Hz}$). When the tuning fork is producing an $A_6$, the tip of the prongs oscillate with an amplitude of $A=\SI{0.6}{mm}$. What is the maximum speed and acceleration of the tip of the tuning fork's prongs?
\begin{finalanswer}
	$v = \SI{3.318}{m/s}$,
	$a = \SI{1.834e4}{m/s^2}$
\end{finalanswer}
\begin{solution}
	The position of the tip of the prong is given by:
	\begin{align*}
	x(t) = A\cos(\omega t)
	\end{align*}
	where $A=\SI{0.6}{mm}$ and $\omega = (2\pi)(\SI{880}{Hz})$. The speed of the tip is given by:
	\begin{align*}
	v(t) =\frac{d}{dt}x(t)= -A\omega\sin(\omega t)
	\end{align*}
	and the acceleration is given by:
	\begin{align*}
	a(t) =\frac{d}{dt}v(t)= -A\omega^2\sin(\omega t)
	\end{align*}
	Thus, the maximal speed is given by:
	\begin{align*}
	v_{max}=A\omega=(\SI{0.6}{mm})((2\pi)(\SI{880}{Hz}))=\SI{3.318}{m/s}
	\end{align*}
	and the maximal acceleration is given by:
	\begin{align*}
	a_{max}=A\omega^2=(\SI{0.6}{mm})((2\pi)^2(\SI{880}{Hz}))^2=\SI{1.834e4}{m/s^2}
	\end{align*}
\end{solution}

%Giancolli 14-20 -fixed
\question A vertical spring has a mass $m = \SI{0.95}{kg}$ attached. When the mass is attached, the spring compresses by $x = \SI{0.15}{m}$ before reaching equilibrium. If the mass is pressed down upon, then released, how much time will pass before it reaches the point of equilibrium?
\begin{finalanswer}
	\SI{0.194}{s}
\end{finalanswer}
\begin{solution}
	Based on the amount that spring stretches, we can determine the spring constant:
	\begin{align*}
	mg &= kx\\
	\therefore k = \frac{mg}{x}
	\end{align*}
	It will take one quarter of an oscillation period to return to the equilibrium position from when the mass is released:
	\begin{align*}
	t=\frac{1}{4}T&=\frac{1}{4}\frac{2\pi}{\omega}=\frac{\pi}{2}\sqrt{\frac{m}{k}}\\
	&=\frac{\pi}{2}\sqrt{\frac{m x}{mg}}=\frac{\pi}{2}\sqrt{\frac{x}{g}}\\
	&=\frac{\pi}{2}\sqrt{\frac{(\SI{0.15}{m})}{(\SI{9.8}{m/s^2})}}=\SI{0.194}{s}
	\end{align*}
\end{solution}


\question Two blocks of mass $m$ are glued together and attached to a linear spring with rest length $L$ and spring constant $k$, as shown in Figure \ref{fig:SHO:twoblocksvertical}. Initially, the two blocks are at rest and in equilibrium. Suddenly, the glue between the blocks fails and the bottom block falls off. What is the amplitude of the oscillations of the block that remains attached to the spring?
\capfig{0.15\textwidth}{figures/SHO/twoblocksvertical.png}{\label{fig:SHO:twoblocksvertical}Two blocks attached to a vertical spring.}



\begin{finalanswer}
	$A = \frac{mg}{k}$
\end{finalanswer}

\begin{solution}
	
	First, we need to find the equilibrium position of the spring, when two blocks are suspended. In this case, the spring will extend such that the spring force is equal to the combined weight of the two blocks:
	\begin{align*}
	kx &= 2mg\\
	\therefore x &=\frac{2mg}{k}
	\end{align*}
	When one block falls off, the equilibrium position will be higher, and the single block will oscillate about that new equilibrium with an amplitude given by the difference between the old and new equilibria. The new equilibrium is given by:
	\begin{align*}
	kx'&=mg\\
	\therefore x'=\frac{mg}{k}
	\end{align*}
	The amplitude of the oscillation is then given by:
	\begin{align*}
	A = x-x'= \frac{mg}{k}
	\end{align*}
	
\end{solution}

=======
%Something along these lines:
%%. A pendulum of length 1 m in Mars' gravitational potential (g = 3:7 m/s2) swings from its maximum amplitude of
%Theta = 4.  The bob of the pendulum has mass of 2 kg, and the rest of the rod is massless. At the bottom of the arc, the pendulum collides with a horizontal spring at equilibrium with k= 100 kg / s2.  The bob detaches from the pendulum and attaches to the spring and starts oscillating in the horizontal plane.  How much time does it take from the maximum amplitude of the pendulum, until the maximum compression of the spring
>>>>>>> master
