\section{Rotational dynamics}

%%%%%%%%%%%%%%%%%%%%%%%%%%%%%%%%%%%
%%
%% Multiple Choice
%%
%%%%%%%%%%%%%%%%%%%%%%%%%%%%%%%%%%%
\subsection{Multiple Choice}


%Based on Ashley Braun
\question[1] A bug sits on a horizontal vinyl record at a distance $r$ from the centre of the record. The record starts playing (accelerating), and the bug is accelerated by a torque $\tau$. A second bug with the same mass as the first is located at a distance $\frac{r}{2}$ from the centre of the record. What is the net torque exerted on the second bug? Assume that both bugs do not move relative to the record as it accelerates.
\begin{checkboxes} 
	\CorrectChoice $\frac{1}{4}\tau$
	\choice  $\frac{1}{2}\tau$
	\choice $2\tau$
	\choice $4\tau$
\end{checkboxes}

\question You approach a closed door with a door handle to your left. You need to push the door handle to open the door, and thus you exert a torque on the door (relative to an axis through the hinges on the right of the door) that is directed
\begin{checkboxes} 
	\choice Upwards.
	\CorrectChoice  Downwards.
	\choice Away from you. 
	\choice Towards you.
\end{checkboxes}

\question \label{question:rotationaldynamics:personladder} A mass-less ladder has a total length $L$ and is placed against a wall such that it makes an angle $\theta$ with the horizontal, as shown in Figure \ref{fig:rotationaldynamics:personladder}. The coefficient of static friction between the ladder and the ground is $\mu_s$, while there is no friction between the ladder and the vertical wall. What is the maximum distance, $D$, along the ladder (as measured from the ground) that a person can climb up the ladder before the ladder starts to slide?
\begin{checkboxes} 
	\choice $D\leq L(1-\mu_s\sin\theta)$
	\choice $D\leq L(1-\mu_s\tan\theta)$
	\choice $D\leq L\mu_s\cos\theta$
	\CorrectChoice $D\leq L\mu_s\tan\theta$
\end{checkboxes}
\capfig{0.20\textwidth}{figures/RotationalDynamics/personladder.png}{\label{fig:rotationaldynamics:personladder} A person climbing a ladder (Question \ref{question:rotationaldynamics:personladder}).}


\question A travelling amusement park came to town, and your little brother decides that you both need to test the new spinning saucer ride. Your brother wants to go ``as fast as possible'', in attempt to take-off into outer-space. There are seats near the centre of the saucer, and seats on the outer-edge of the saucer. Where should you sit, in order to go the fastest?
\begin{checkboxes}
\choice Near the centre, because you will increase linear speed and angular velocity.
\choice Near the centre, because you will increase linear speed.
\CorrectChoice Near the outer-edge, because you will increase linear speed. %correct
\choice Near the outer-edge, because you will increase linear speed and angular velocity.
\end{checkboxes}

%Need to double-check this one
\question A construction worker has to turn a lever at an angle of 55$^{\circ}$. The length of the lever is $\SI{0.61}{m}$ long and is a force of $\SI{25}{N}$ is exerted to turn it. What is the net torque on the lever?
\begin{checkboxes}
\choice $\SI{9.0}{Nm}$
\choice $\SI{34}{Nm}$
\CorrectChoice $\SI{12}{Nm}$ \correct
\choice $\SI{22}{Nm}$
\end{checkboxes}

\question You are fed up with your non-physics course and decide to throw the book in the air. If you want the book to spin as fast as possible, which of the following axes should you spin it about? All of the axes pass through the centre of mass of the book. Ignore the thickness of the book and assume it is longer than it is wide.
\begin{checkboxes}
\CorrectChoice The axis along the length of the book, parallel to its surface \correct
\choice The axis along the width of the book, parallel to its surface
\choice The axis through the center of the book, perpendicular to its surface
\choice Both the length and width axes will work
\end{checkboxes}


\question A vicu\~na with a mass of \SI{50}{kg} stands on the corner of a giant spinning chocolate bar with dimensions \SI{50}{m} by \SI{100}{m} (assume uniform density and no thickness). The axis of rotation is in the center of the chocolate bar and perpendicular to the plane it forms. If the chocolate bar has a mass of \SI{1000}{kg}, what is the moment of inertia of the whole system? (The vicu\~na can be treated as a point mass)
\begin{checkboxes}
\CorrectChoice \SI{1.20e6}{kg m} \correct
\choice \SI{1.67e6}{kg m}
\choice \SI{6.41e6}{kg m}
\choice \SI{1.20e6}{kg^2 m}
\end{checkboxes}


\question A sphere with a radius of \SI{2}{m} and a mass of \SI{1}{kg} and uniform density begins spinning from rest about its centre.  With uniform angular acceleration it reaches a rotational speed of 2 revolutions per second over the course of 10 seconds.  If the applied force was tangential to the surface of the sphere, what was the magnitude of the applied force?
\begin{checkboxes}
\CorrectChoice \SI{1.01}{N} \correct
\choice \SI{1.21}{N}
\choice \SI{0.96}{N}
\choice \SI{2.01}{N}
\end{checkboxes}


\question While sitting in class, you find yourself watching the clock very closely. By watching the ticking of the second hand, you determine that it spends \SI{0.9}{s} stationary and then spends \SI{0.1}{s} to move to the next position. It does this 60 times in one minute for a full rotation. What is the average and maximum angular velocity of the second hand during one full rotation?
\begin{checkboxes}
\CorrectChoice $\omega_{AVG} = \SI{0.11}{rad\per s}$, $\omega_{MAX} = \SI{1.1}{rad\per s}$ \correct
\choice $\omega_{AVG} = \SI{0.002}{rad\per s}$, $\omega_{MAX} = \SI{0.02}{rad\per s}$
\choice $\omega_{AVG} = 0$, $\omega_{MAX} = \SI{1.1}{rad\per s}$
\choice $\omega_{AVG} = \omega_{MAX} = \SI{1.1}{rad\per s}$
\end{checkboxes}

% Question submitted by Haoyuan Wang
\question All parts of a rigid object rotating about a fixed axis have the same angular velocity $\omega$ and the same angular acceleration $\alpha$ at any instant.
\begin{checkboxes}
\CorrectChoice True \correct
\choice False
\end{checkboxes}

%Question submitted by Nicole Gaul
\question A moment of inertia : [select all that apply]
\begin{checkboxes}
\choice Has units of $m^2/kg$
\CorrectChoice Is the counterpart of mass in the formula for rotational kinetic energy \correct
\CorrectChoice Is the quantitative measure of rotational inertia \correct
\end{checkboxes}

\question You have two circular frisbees of the same mass and radius. One of them is a ring, and the other is the more traditional disk shape. You throw them exactly the same way. Which one will spin the fastest?
\begin{checkboxes}
\choice The ring
\CorrectChoice The disk \correct
\choice They spin at the same rate
\end{checkboxes}

\question Two forces of different magnitude are exerted at the same point on an object. Is it possible for them to produce the same torque about an axis that does not go through the point where the forces are applied?
\begin{checkboxes}
\CorrectChoice Yes \correct
\choice No
\end{checkboxes}

\question You ride a bicycle forward in a straight line. In which direction does the angular velocity of the wheels point
\begin{checkboxes}
\choice Forward
\choice Backward
\CorrectChoice To your left \correct
\choice To your right
\end{checkboxes}

%Question submitted by Adam McCaw
\question If a point mass A is double the distance from a rotation axis than another point mass B that is half of the mass of A, how do their moments or inertia about that axis relate?
\begin{checkboxes}
\choice $I_A=I_B$
\CorrectChoice $I_A=2I_B$ \correct
\choice $I_B=2I_A$
\choice $I_A=4I_B$
\end{checkboxes}


\question You have designed an engine that provides a constant torque regardless of its rotational speed. You place different shapes on the axle and measure their angular velocity after a fixed period of time. Which one will have the highest angular velocity?
\begin{checkboxes} 
\choice A solid disk of mass $M$ and radius $R$ (the axle is perpendicular to the plane of the disk and goes through its centre)
\CorrectChoice A solid disk of mass $M$ and radius $R/2$ (the axle is perpendicular to the plane of the disk and goes through its centre) \correct
\choice A ring of mass $M$ and radius $R/2$ (the axle is perpendicular to the plane of the ring and goes through its centre)
\choice A ring of mass $M$ and radius $R$ (the axle is perpendicular to the plane of the ring and goes through its centre)
\end{checkboxes}

\question You wish to accelerate a uniform solid disk of mass $M=\SI{20}{kg}$ and radius $R=\SI{0.5}{m}$ to an angular velocity of \SI{100}{rad/s} in \SI{2}{s}. The disk rotates about a fixed axis that goes through its centre and is perpendicular to the plane of the disk. How much power is required?
 \begin{checkboxes} 
\choice \SI{3125}{W}
\CorrectChoice \SI{6250}{W} \correct
\choice \SI{12500}{W}
\choice \SI{25000}{W}
\end{checkboxes}

\question Suppose that you spin a ball that is resting on a desk. As time passes by, the angular velocity of the ball relative to some axis decreases. In what direction is the angular acceleration of the ball, relative to that same axis?
\begin{checkboxes}
\choice The angular velocity is zero.
\CorrectChoice Anti-parallel to the angular velocity.
\choice Parallel to the angular velocity.
\choice Not enough information to tell.
\end{checkboxes}

\question You push on a door located in front of you to open it and exert a downwards torque on the door relative to its hinges. Relative to you, on which side are the door's hinges? 
\begin{checkboxes}
\CorrectChoice On your right.
\choice On your left. 
\choice Above you.
\choice At your feet.
\end{checkboxes}

\question Which is easier (i.e. requires less effort on your part)?
\begin{checkboxes}
\CorrectChoice Rotating a bar about its centre.
\choice Rotating a bar about its end.
\end{checkboxes}

\question You are driving in your car and go around a left turn. You notice that the shocks (springs) in your car compress more on one side than the other. 
\begin{checkboxes}
\CorrectChoice The right side of the car is lower in the turn \correct
\choice The left side of the car is lower in the turn
\end{checkboxes}

\question Why are the breaks on a motorcycle (and cars) bigger on the front wheel(s)?
\begin{checkboxes}
\choice Because there is more room on those wheels, as the back wheels are near the engine.
\CorrectChoice Because the normal force from the ground is bigger on the front wheel(s) than on the back wheel(s) when breaking \correct
\choice Because the normal force from the ground is bigger on the back wheel(s) than on the front wheel(s) when breaking
\end{checkboxes}

\question A speed skater leans towards the centre of the track as they go around it. If you consider the torques on the skater about the point where their skates contact the ice, which forces create a torque?
\begin{checkboxes}
\choice Gravity
\choice Gravity and the normal force from the ice
\choice Gravity, an inertial (centrifugal) force,  and the normal force from the ice
\CorrectChoice Gravity and an inertial (centrifugal) force \correct
\end{checkboxes}

\question A guanaco with a mass of $m$ is standing on a horizontal plank that is suspended on the side of Huayna Picchu (a mountain near Machu Picchu). The plank has a length $L$ and a mass $M$. The plank is anchored to the side of the mountain with a hinge and would swing freely if there were no rope attached to the plank. A rope is attached to the end of the plank that is away from the mountain and the rope is anchored into the mountain, as shown in Figure \ref{fig:rotationaldynamics:GuanacoPlank}. The rope makes an angle $\theta$ with the plank. The guanaco is standing a distance $\frac{3}{4}L$ from the mountain. What is the tension in the rope?
\capfig{0.2\textwidth}{figures/Statics/GuanacoPlank.png}{\label{fig:rotationaldynamics:GuanacoPlank} A guanaco on a plank.}
\begin{checkboxes} 
\choice $T=\frac{\left( \frac{1}{2}m+ \frac{3}{4}M\right)g}{\cos\theta}$
\choice $T=\frac{mg}{\sin\theta}$
\choice $T=\frac{\left( \frac{3}{4}m+ \frac{1}{2}M\right)g}{\cos\theta}$
\CorrectChoice $T=\frac{\left( \frac{3}{4}m+ \frac{1}{2}M\right)g}{\sin\theta}$ \correct
\end{checkboxes}


%%%%%%%%%%%%%%%%%%%%%%%%%%%%%%%%%%%
%
% long answer
%
%%%%%%%%%%%%%%%%%%%%%%%%%%%%%%%%%%%
\subsection{Long answers}

\question Four equal point masses $m$ are located at the corners of a square of length $a$. The masses are connected by rigid mass-less rods. The square is in the vertical plane, as shown in Figure \ref{fig:rotationaldynamics:square}.
\begin{parts}
\part What is the moment of inertia of the system about an axis that is perpendicular to the page and goes through the bottom left mass?
\part If the system is free to rotate under the influence of gravity about the same axis in part (a), what is its initial angular acceleration about that axis, just as it is released?
\end{parts}
\capfig{0.2\textwidth}{figures/RotationalDynamics/square.png}{\label{fig:rotationaldynamics:square} Four equal masses on the side of a square.}
\begin{finalanswer}
\begin{enumerate}[(a)]
\item $4ma^2$
\item $\frac{g}{2a}$
\end{enumerate}
\end{finalanswer}
\begin{solution}
\begin{parts}
\part The mass at the axis of rotation does not contribute to the moment of inertia. The masses above and to the right are a distance $a$ away from the axis of rotation, whereas the one that is diagonally opposite of the axis is at a distance $\sqrt a$ from the axis of rotation. The moment of inertia is thus:
\begin{align*}
I=ma^2+ma^2+m(\sqrt 2a)^2=4ma^2
\end{align*}
\part We calculate the torques from the weight of the masses about the axis. The mass directly above the axis creates no torque, since its weight is parallel to the vector between it and the rotation axis. The sum of the torques is:
\begin{align*}
\sum \tau &= mg\sqrt 2 a\sin(\SI{45}{degree})+mga\\
&=mg\sqrt 2 a\frac{1}{\sqrt 2}+mga=2mga\\
\end{align*}
The sum of the torques is the moment of inertia times angular acceleration:
\begin{align*}
\sum \tau &= I \alpha\\
2mga &= 4ma^2 \alpha\\
\therefore \alpha = \frac{g}{2a}
\end{align*}
\end{parts}
\end{solution}


% Modified question from phone app called "Brilliant", from Troy
\question A ceiling fan is powered on and rotating with constant angular velocity. You switch it off and measure that it takes 36 revolutions to reduce its angular velocity to half its original value. How many more revolutions will it take before it comes to rest? Assume constant angular acceleration.
\begin{finalanswer}
The fan will complete 12 more revolutions.
\end{finalanswer}
\begin{solution}
Let $\theta(t)$ be the angular position of one of the fan fins and let us take $\theta(t=0)=0$. Since the angular acceleration ($\alpha$) is constant (and negative), the angular position $\theta(t)$ and angular speed $\omega(t)$ are given by:
\begin{align*}
\theta(t)&=\omega_0t+\frac{1}{2}\alpha t^2\\
\omega(t)&=\omega_0+\alpha t
\end{align*}
where $\omega_0$ is the initial angular speed of the fan before it starts to decelerate, and we have used the fact that $\theta(t=0)=0$. We know that at some (unknown) time $t_1$, the angular displacement is 36 revolutions ($\theta(t_1)=36\times 2\pi$), and that the angular speed is $\omega(t_1)=\frac{1}{2}\omega_0 t$:
\begin{align*}
\theta(t_1)&=72\pi=\omega_0t_1+\frac{1}{2}\alpha t_1^2\\
\omega(t_1)&=\frac{1}{2}\omega_0=\omega_0+\alpha t_1
\end{align*}
We can use the second equation to express $t_1$ in terms of $alpha$ and $\omega_0$ and use it in the equation for $\theta(t_1)$:
\begin{align*}
t_1 &=-\frac{1}{2}\frac{w_0}{\alpha}\\
72\pi &= \omega_0\left(- \frac{1}{2}\frac{w_0}{\alpha}\right)+\frac{1}{2}\alpha \left( \frac{1}{2}\frac{w_0}{\alpha}\right)^2\\
&=-\frac{1}{2}\frac{w_0^2}{\alpha}+\frac{1}{8}\frac{w_0^2}{\alpha}\\
&=-\frac{3}{8}\frac{w_0^2}{\alpha}\\
\therefore \frac{w_0^2}{\alpha} &= -192\pi
\end{align*}

Later, at some time $t_2$, the fan comes to rest:
\begin{align*}
\theta(t_2)&=\omega_0t_2+\frac{1}{2}\alpha t_2^2\\
\omega(t_2)&=0=\omega_0+\alpha t_2
\end{align*}
Again, we can use the second equation to express $t_2$ in terms of $\alpha$ and $\omega_0$ and use it in the equation for $\theta(t_2)$:
\begin{align*}
t_2&=-\frac{\omega_0}{\alpha}\\
\theta(t_2)&=-\frac{\omega_0^2}{\alpha}+\frac{1}{2}\alpha \left(\frac{\omega_0}{\alpha}\right)^2\\
&=-\frac{\omega_0^2}{\alpha}+\frac{1}{2}\frac{\omega_0^2}{\alpha}\\
&=-\frac{1}{2}\frac{\omega_0^2}{\alpha}\\
&=-\frac{1}{2}(-192\pi)\\
&=96\pi
\end{align*}
where we used our expression for $\frac{\omega_0^2}{\alpha}$ determined above. We are asked to find $\theta(t_2)-\theta(t_1)$, the angular displacement between $t_1$ and $t_2$, which is:
\begin{align*}
\theta(t_2)-\theta(t_1)=96\pi-72\pi=12\times(2\pi)
\end{align*}
thus, the fan will complete 12 more revolutions.
\end{solution}

%Troy
\question You are working on a physics assignment the night before it is due, but you realize that you do not have enough time to complete it. You start to wonder if you could stop the Earth from rotating and stop the sun from rising the next day. Hopefully the ensuing confusion would give you more time to work on the assignment. Instead of working on your physics assignment, you decide to explore the feasibility of this scenario. Model the Earth as a perfect solid sphere rotating constantly without any precession.

\begin{parts}
\part What is the angular velocity of the Earth?
\part If your assignment is due in 12 hours, what is the minimum constant angular acceleration that you would need to stop the Earth's rotation by then?
\part Calculate the torque required to produce the angular acceleration you calculated in part (b).
\part To achieve the torque from part (c), you decide to install rocket engines onto the Earth. The Rocketdyne F-1 is the most powerful single-nozzle liquid-fuelled rocket engine ever flown. It has a thrust of approximately \SI{6800}{\kilo\newton}. What is the minimum number of F-1 engines that you need to stop the Earth?
\end{parts}

\textit{Note: The mass of the Earth is \SI{5.972e24}{\kilo\gram}, the radius of the Earth is \SI{6378}{km}, the moment of inertia of a solid sphere of radius $r$ is $I = \frac{2}{5}mr^2$.}
\begin{finalanswer}
\begin{enumerate}[(a)]
\item $\SI{7.27e-5}{rad/s}$
\item $\SI{-1.68e-9}{rad/s^2}$
\item $\SI{1.63e29}{Nm}$
\item $\num{3.76e15}$ rockets
\end{enumerate}
\end{finalanswer}
\begin{solution}
\begin{parts}
\part The angular velocity of the Earth is given by $2\pi$ divided by the length of one day:
\begin{align*}
\omega = \frac{\Delta\theta}{\Delta t} = \frac{(\SI{2\pi}{rad})}{(\SI{86400}{s})} = \SI{7.27e-5}{rad \per s}
\end{align*}

\part The acceleration is the change in angular speed over the period of SI{12}{hr}:
\begin{align*}
\alpha = \frac{\Delta\omega}{\Delta t} = \frac{(\SI{-7.27e-5}{rad/s})}{(\SI{43200}{s})} = \SI{-1.68e-9}{rad /s^2}
\end{align*}

\part The required torque is given by $I\alpha$, where $I$ is the moment of inertia of the Earth about its rotation axis:
\begin{align*}
\tau &= \alpha I  = (\SI{-1.68e-9}{rad/s^2})\left(\frac{2(\SI{5.972e24}{kg})(\SI{6378}{km})^2}{5}\right)\\
 &= \SI{1.63e29}{Nm}
\end{align*}

\part The most efficient way to generate the torque is to attach the rockets at the equator of the Earth so that their thrust is tangent to the Earth and in the direction opposite of the rotation. The total torque can be evaluated from the total force from all the rockets, $F$, assuming that the lever arm is the radius of the Earth, and that the force is tangent to the surface of the Earth:
\begin{align*}
\tau &= |r||F|\sin\theta = |r||F|\sin(\SI{90}{\degree}) = |r||F|\\
\therefore |F| &= \frac{\tau}{|r|} = \frac{(\SI{1.63e29}{Nm})}{(\SI{6378}{km})}\\
|F| &= \SI{2.56e22}{N}\\
\end{align*}
In terms of the thrust of an F-1 rocket engine, this gives
\begin{align*}
\frac{\SI{2.56e22}{\newton}}{\SI{6800}{\kilo\newton\per rocket}} = \SI{3.76e15}{rockets}\\
\end{align*}
That seems like a lot of rockets. Luckily, by solving this question, you have made progress on your assignment.
\end{parts}

\end{solution}


\question Two different blocks of mass $m_1$ and $m_2$, respectively, are on different sides of an incline, where the slopes of the two sides make angles of $\alpha$ and $\beta$ with the horizontal, respectively, as depicted in Figure \ref{fig:rotationaldynamics:InclinePulley}. The blocks are connected by a mass-less inextensible rope that goes around a frictionless pulley of mass $M$ and radius $R$ without slipping. The pulley can be modelled as a solid disk with moment of inertia $I=\frac{1}{2}MR^2$. The coefficient of kinetic friction between the blocks and the incline is $\mu_k$. Show that when the system is accelerating such that $m_1$ is moving downwards, the acceleration is given by:
\begin{align*}
a &= \frac{m_1(\sin\alpha-\mu_k\cos\alpha)-m_2(\sin\beta+\mu_k\cos\beta)}{m_1+m_2+\frac{1}{2}M}g
\end{align*}
\textit{Hint: The tension in the rope is not the same on either side of the pulley if the pulley has a moment of inertia. If the tensions were the same, the pulley would not rotate, as the net torque on it would be zero.}
\capfig{0.4\textwidth}{figures/RotationalDynamics/InclinePulley.png}{\label{fig:rotationaldynamics:InclinePulley}Two masses on an inclined plane with a pulley.}
\begin{solution}
The two blocks will have the same acceleration magnitude, equal to the tangential acceleration of the pulley. We will assume that the system accelerates to the left so that we can draw the friction forces, as in the free body diagram shown in Figure \ref{fig:rotationaldynamics:InclinePulley_FBD}.
\capfig{0.5\textwidth}{figures/RotationalDynamics/InclinePulley_FBD.png}{\label{fig:rotationaldynamics:InclinePulley_FBD}}
For each block, we choose the positive $x$ axis to be parallel to the acceleration (and positive in the direction of the acceleration), and the $y$ axis to be perpendicular to the plane and positive away from the plane. For block 1, Newton's second law gives for the $x$ and $y$ components respectively:
\begin{align*}
\sum F_x&=m_1g\sin\alpha-f_1-T_1=m_1a\\
\sum F_y&=N_1-m_1g\cos\alpha=0\\
\end{align*}
Using the $y$ equation to eliminate $N_1$, and using the fact that $f_1=\mu_kN_1$, we have:
\begin{align*}
m_1g\sin\alpha-\mu_km_1g\cos\alpha-T_1&=m_1a\\
\therefore m_1g(\sin\alpha-\mu_k\cos\alpha)-T_1&=m_1a
\end{align*}
For block 2, Newton's second law gives for the $x$ and $y$ components respectively:
\begin{align*}
\sum F_x&=T_2-f_2-m_2g\sin\beta=m_2a\\
\sum F_y&=N_2-m_2g\cos\beta=0\\
\end{align*}
Again, using the $y$ equation to eliminate $N_2$ and using $f_2=\mu_kN_2$ into the $x$ equation gives:
\begin{align*}
T_2-\mu_km_2g\cos\beta-m_2g\sin\beta&=m_2a\\
\therefore T_2-m_2g(\sin\beta+\mu_k\cos\beta)&=m_2a\\
\end{align*}
On the pulley, we define the angular acceleration to be positive in the direction counter-clockwise direction. It is related to the acceleration of the blocks by:
\begin{align*}
\alpha R = a
\end{align*}
The net torque on the pulley comes from the two tensions ($T_1$ exerting a positive torque and $T_2$ exerting a negative torque):
\begin{align*}
\sum \tau = T_1R-T_2R&=I\alpha\\
T_1R-T_2R&=\frac{1}{2}MR^2 \frac{a}{R}\\
T_1-T_2&=\frac{1}{2}Ma
\end{align*}
With the equations from the two blocks, we now have 3 equations and 3 unknowns ($a$, $T_1$, and $T_2$):
\begin{align*}
 m_1g(\sin\alpha-\mu_k\cos\alpha)-T_1&=m_1a\\
 T_2-m_2g(\sin\beta+\mu_k\cos\beta)&=m_2a\\
 T_1-T_2&=\frac{1}{2}Ma\\
\end{align*}
Let us use the third equation to express $T_2$ in terms of $T_1$, and substitute that into the second equation to get an expression for $T_1$:
\begin{align*}
T_2&=T_1-\frac{1}{2}Ma\\
T_1-\frac{1}{2}Ma-mg(\sin\beta+\mu_k\cos\beta)&=m_2a\\
\therefore T_1&=m_2a+\frac{1}{2}Ma+m_2g(\sin\beta+\mu_k\cos\beta)
\end{align*}
We can isolate $T_1$ from the equation that came from block 1, and set it equal to the above:
\begin{align*}
T_1&= m_1g(\sin\alpha-\mu_k\cos\alpha)-m_1a\\
m_2a+\frac{1}{2}Ma+m_2g(\sin\beta+\mu_k\cos\beta) &= m_1g(\sin\alpha-\mu_k\cos\alpha)-m_1a\\
(m_1+m_2+\frac{1}{2}M)a &=m_1g(\sin\alpha-\mu_k\cos\alpha)-m_2g(\sin\beta+\mu_k\cos\beta)\\
\therefore a &= \frac{m_1(\sin\alpha-\mu_k\cos\alpha)-m_2(\sin\beta+\mu_k\cos\beta)}{m_1+m_2+\frac{1}{2}M}g
\end{align*}
 
\end{solution}



\question One can model the blades of a helicopter as 3 identical uniform thin rods of length $L=\SI{4.0}{m}$ and mass $M=\SI{120}{kg}$, as shown in Figure \ref{fig:rotationaldynamics:HelicopterBlades}. The helicopter's blades accelerate from rest to \SI{450}{rpm} in \SI{10}{s}, and the helicopter's engine is 85\% efficient at delivering power to the rotor blades. 
\begin{parts}
\part What is the minimum required power for the helicopter's engine? Give your answer in horsepower (\SI{1}{hp}=\SI{745.7}{W})
\part If a small bird of mass $m=\SI{500}{g}$ managed to hold on to the end of one of the blades as they accelerated from rest, what speed would the blades have after \SI{10}{s} if you assume that power was delivered by the engine at the same rate as in part a)? Express your answer in rpms.
\part What is the centripetal acceleration felt by the bird in part b after $\SI{10}{s}$? Express your answer in $g$s.
\end{parts}

\capfig{0.3\textwidth}{figures/RotationalDynamics/HelicopterBlades.png}{\label{fig:rotationaldynamics:HelicopterBlades}A model for helicopter blades.}
\begin{finalanswer}
\begin{enumerate}[(a)]
\item $\SI{336.20}{hp}$
\item $\SI{448.98}{rpm}$
\item 902 $g$s
\end{enumerate}
\end{finalanswer}
\begin{solution}
\begin{parts}
\part
We model the helicopter blades as three thin rods being rotated about their end. The moment of inertia of the system of three blades is thus:
\begin{align*}
I=3I_{rod}=3\frac{1}{3}ML^2=ML^2=(\SI{120}{kg})(\SI{4.0}{m})^2=\SI{1920}{kgm^2}
\end{align*}
The rotational kinetic energy of the blades when rotating at \SI{450}{rpm} is:
\begin{align*}
K&=\frac{1}{2}I\omega^2\\
&=\frac{1}{2}(\SI{1920}{kgm^2})((\SI{450}{rot/min})(2\pi\si{rad/rot})(\frac{1}{60}\si{min/s}))^2\\
&=\frac{1}{2}(\SI{1920}{kgm^2})(\SI{47.12}{rad/s})^2\\
&=\SI{2.131e6}{J}
\end{align*}
The power required to bring them up to speed in \SI{10}{s} is:
\begin{align*}
P=\frac{K}{\Delta t}=\frac{(\SI{2.131e6}{J})}{(\SI{10}{s})}&=\SI{2.131e5}{W}
\end{align*}
Since the engine is only 85\% efficient, the engine power must be:
\begin{align*}
P^E=\frac{P}{0.85}=\SI{2.507e5}{W}=\SI{336.20}{hp}
\end{align*}
\part The bird changes the moment of inertia of the blades:
\begin{align*}
I=ML^2+mL^2=\SI{1920}{kgm^2}+(\SI{500}{g})(\SI{4.0}{m})=\SI{1928}{kgm^2}
\end{align*}
In \SI{10}{s}, the helicopter blades will have the same rotational kinetic energy as before, which we can use to solve for the speed:
\begin{align*}
K&=\frac{1}{2}I\omega^2\\
\therefore \omega &= \sqrt{\frac{2K}{I}}\\
&=\sqrt{\frac{2(\SI{2.131e6}{J})}{(\SI{1928}{kgm^2})}}\\
&=\SI{47.01}{rad/s}=\SI{448.98}{rpm}
\end{align*}
Thus the rotational speed of the blades is hardly affected, although the blade with the bird on it will have a tension force to keep the bird on it (see next part).
\part The centripetal acceleration in $g$s is given by:
\begin{align*}
a_c\frac{1}{g}=\omega^2L\frac{1}{g}=(\SI{47.01}{rad/s})^2(\SI{4.0}{m})\frac{1}{(\SI{9.8}{m/s^2})}=902
\end{align*}
That's a lot of $g$s! The bird is thus pulling with a force of about $ma_c\sim \SI{450}{N}$ on that blade.
\end{parts}
\end{solution}

\question A solid cone of mass $M$ and height $H$ has a circular base of radius $R$. The cone is rotated about its axis of symmetry as shown in Figure \ref{fig:rotationaldynamics:Cone}. Derive a formula for its moment of inertia about its axis of symmetry.
\capfig{0.15\textwidth}{figures/RotationalDynamics/Cone.png}{\label{fig:rotationaldynamics:Cone}A cone.}
\begin{finalanswer}
$\frac{3}{10}MR^2$
\end{finalanswer}
\begin{solution}
We will break the cone up into many small cylinders of radius $r(h)$ and infinitesimal height $dh$, as in Figure \ref{fig:rotationaldynamics:Cone_dh}. Each small cylinder will have a mass $dm=\rho\pi r(h)^2 dh$, where $\rho$ is the density of the cone. The moment of inertia of a small disk of mass $dm$ and radius $r$ will be:
\begin{align*}
dI=\frac{1}{2}dmr(h)^2
\end{align*} 
\capfig{0.15\textwidth}{figures/RotationalDynamics/Cone_dh.png}{\label{fig:rotationaldynamics:Cone_dh}A small disk in the cone.}

We can then add the moments of inertia of each little cylinder to get the total moment of inertia of the cone:
\begin{align*}
I=\int dI=\int_0^H \frac{1}{2} dm r(h)^2=\int_0^H \frac{1}{2}  r(h)^2\rho\pi r(h)^2 dh=\frac{1}{2}  \rho\pi\int_0^H  r(h)^4 dh
\end{align*}
We define the integration variable $h$ to be 0 at the vertex of the cone. We need to determine the function $r(h)$, knowing that $r(0)=0$ and $r(H)=R$, and that the function is linear, we have:
\begin{align*}
r(h) = \frac{R}{H}h
\end{align*}
The integral for the moment of inertia is thus:
\begin{align*}
I=\frac{1}{2} \rho\pi\left(\frac{R}{H}\right)^4\int_0^H  h^4 dh=\frac{1}{2} \rho\pi\left(\frac{R}{H}\right)^4\left[\frac{1}{5}h^5\right]_0^H=\frac{1}{10} \rho\pi R^4H
\end{align*}
Finally, we need to express the density in terms of the mass and dimensions of the cone. Just for fun and instead of looking up the volume of a cone, we can find the volume by summing the volume of the infinitesimal cylinders of radius $r(h)$ and height $dh$:
\begin{align*}
V&=\int dV=\int_0^H  \pi r(h)^2 dh=\pi\left(\frac{R}{H}\right)^2\int_0^H  h^2 dh=\pi\left(\frac{R}{H}\right)^2\left[\frac{1}{3}h^3\right]_0^H\\
&=\frac{1}{3}\pi R^2H
\end{align*}
The density is thus:
\begin{align*}
\rho=\frac{3M}{\pi R^2H}
\end{align*}
and the moment of inertia is:
\begin{align*}
I&=\frac{1}{10} \rho\pi R^4H=\frac{1}{10} \frac{3M}{\pi R^2H}\pi R^4H\\
&=\frac{3}{10}M R^2\\
\end{align*}
which is independent of the height!
\end{solution}


\question Two solid spheres of masses $M=\SI{4}{kg}$ and $m=\SI{2}{kg}$, with radii $R=\SI{20}{cm}$ and $r=\SI{10}{cm}$, respectively, are attached at either end of a rod of mass $m_r=\SI{5}{kg}$ and length $L=\SI{50}{cm}$ to form a dumbbell (see Figure \ref{fig:rotationaldynamics:Dumbbell}). Note that the spheres are attached to the ends of the rod (the rod does not penetrate the spheres). 
\capfig{0.4\textwidth}{figures/RotationalDynamics/Dumbbell.png}{\label{fig:rotationaldynamics:Dumbbell}A dumbbell.}
\begin{parts}
\part What is the moment of inertia of the system about its centre of mass?
\part A coordinate system is set up so that the origin is at the centre of mass of the dumbbell, and the $x$ axis is parallel to the rod, with the positive direction from the $M$ to $m$. A force:
\begin{align*}
\vec F=(\SI{0.8}{N})\hat i+(\SI{1.0}{N}) \hat j + (\SI{1.2}{N}) \hat k
\end{align*}is applied to the outer edge of the small mass $m$ (the edge of the sphere opposite of where it meets the rod). If $\vec F$ is the only force applied to the dumbbell, what is the angular acceleration vector of the dumbbell with respect to the origin of the coordinate system that is specified? What is its magnitude? 
\end{parts}
\begin{finalanswer}
\begin{enumerate}[(a)]
\item $\SI{1.121}{kgm^2}$
\item $\vec\alpha =(\SI{-0.589}{rad/s^2})\hat j+(\SI{0.491}{rad/s^2})\hat k$,
which has magnitude $\alpha=\SI{0.767}{rad/s^2}$.
\end{enumerate}
\end{finalanswer}
\begin{solution}
\begin{parts}
\part First, we need to find the centre of mass of the system. After that, we can use the parallel axis theorem to find the moments of inertia of the rod and two masses about the CM and them.

We choose the centre of the rod as the origin of a one dimension coordinate system, and choose the positive $x$ direction as going from $M$ to $m$. The sphere of mass $M$ has its centre of mass at $x_M=-\frac{L}{2}-R=\SI{-45}{cm}$, and the sphere of mass $m$ has its centre of mass at $x_m=\frac{L}{2}+r=\SI{35}{cm}$. The centre of mass of the rod is at $x_r=0$, by choice.

The centre of mass of the system is thus given by:
\begin{align*}
x_{CM}&=\frac{1}{M+m+m_r}(Mx_M+mx_m+m_rx_r)\\
&=\frac{1}{(\SI{4}{kg})+(\SI{2}{kg})+(\SI{5}{kg})}((\SI{4}{kg})(\SI{-0.45}{m})+(\SI{2}{kg})(\SI{0.35}{m})+(\SI{5}{kg})(0))\\
&=\SI{-0.1}{m}
\end{align*}

The moment of inertia of the rod is its moment of inertia about its centre plus $m_r(x_r-x_{CM})^2$ (by the parallel axis theorem):
\begin{align*}
I_{rod}&=\frac{1}{12}m_rL^2+m_r(x_r-x_{CM})^2\\
&=\frac{1}{12}(\SI{5}{kg})(\SI{0.5}{m})^2+(\SI{5}{kg})(\SI{-0.1}{m})^2\\
&=\SI{0.154}{kg m^2}
\end{align*}
The moments of inertia of the two rods are found in a similar manner:
\begin{align*}
I_M&=\frac{2}{5}MR^2+M(x_M-x_{CM})^2\\
&=\frac{2}{5}(\SI{4}{kg})(\SI{0.2}{m})^2+(\SI{4}{kg})((\SI{-0.45}{m})-(\SI{-0.1}{m}))^2\\
&=\SI{0.554}{kgm^2}\\
I_m&=\frac{2}{5}mr^2+m(x_m-x_{CM})^2\\
&=\frac{2}{5}(\SI{2}{kg})(\SI{0.1}{m})^2+(\SI{2}{kg})((\SI{0.35}{m})-(\SI{-0.1}{m}))^2\\
&=\SI{0.413}{kgm^2}
\end{align*}
The total moment of inertia about the centre of mass is thus:
\begin{align*}
I &= I_{rod}+I_M+I_m\\
&=(\SI{0.154}{kg m^2})+(\SI{0.554}{kgm^2})+(\SI{0.413}{kgm^2})=\SI{1.121}{kgm^2}
\end{align*}
\part Given the force, we first calculate the torque, from which we can find the angular acceleration vector. The force is applied at the outer edge of the smaller sphere, which corresponds to a point on the $x$ axis given by:
\begin{align*}
\vec r &= (\frac{L}{2}+2r-x_{CM})\hat i=((\SI{0.25}{m})+2(\SI{0.1}{m})-(\SI{-0.1}{m}))\hat i\\
&=(\SI{0.55}{m})\hat i
\end{align*}
The torque from the force is given by the vector product:
\begin{align*}
\vec\tau &=\vec r\times \vec F=[(\SI{0.55}{m})\hat i]\times[(\SI{0.8}{N})\hat i+(\SI{1.0}{N}) \hat j + (\SI{1.2}{N}) \hat k]\\
&=(\SI{0.55}{m})(\SI{0.8}{N})(\hat i\times \hat i)+(\SI{0.55}{m})(\SI{1.0}{N})(\hat i\times \hat j)+(\SI{0.55}{m})(\SI{1.2}{N})(\hat i\times \hat k)\\
&=(0)+(\SI{0.55}{Nm})(\hat k)+(\SI{0.66}{Nm})(-\hat j)\\
&=(\SI{-0.66}{Nm})\hat j+(\SI{0.55}{Nm})\hat k
\end{align*}

The angular acceleration is given by the angular version of Newton's Second Law. The torque that we calculated above is the only torque on the dumbbell, giving the angular acceleration vector
\begin{align*}
\sum \vec\tau &= I\vec\alpha = (\SI{-0.66}{Nm})\hat j+(\SI{0.55}{Nm})\hat k\\
\therefore \vec\alpha &=\frac{1}{(\SI{1.121}{kgm^2})}(\SI{-0.66}{Nm})\hat j+(\SI{0.55}{Nm})\hat k\\
&=(\SI{-0.589}{rad/s^2})\hat j+(\SI{0.491}{rad/s^2})\hat k
\end{align*}
which has magnitude $\alpha=\SI{0.767}{rad/s^2}$.
\end{parts}
\end{solution}



%% From Olivia (or Emma?)
\question Derive a formula for the moment of inertia of a solid Double Chocolate Donut of radius $r$ rotated about its axis of symmetry, as illustrated in Figure \ref{fig:rotationaldynamics:donut}.

\capfig{0.5\textwidth}{figures/RotationalDynamics/donut.png}{\label{fig:rotationaldynamics:donut} This is a Double Chocolate Donut.}
\begin{solution}

In order to solve this problem, we should imagine that the Double Chocolate Donut\footnote{Please take special note of the asymmetry of the icing of this Double Chocolate Donut. In physics, we must often perform approximations to model physical situations. We can appropriately approximate this Double Chocolate Donut by ascribing its shape to that of a regular torus.} is composed of small donut cylinders of radius r.

So, each cylinder must have a volume of: $V=\Delta xy\Delta z$

Using the formula for density, D= M/V, we can solve for the mass of a small cylinder.

\begin{align*}
\rho=m/v
m=\rho v
m=\Delta xy\Delta z\rho
\end{align*}

Now, we must put variables x,y and z in terms of the variables theta, r and X (the height);

Let us first look at the y variable, which is the height of the Double Chocolate Donut;

This is the equation for a circle with radius ``2r'', which is:

\begin{align*}
r^{2}=y^{2}+(x-r)^{2}
\end{align*}

Solving in terms of y, we get:

\begin{align*}
y=\sqrt{r^{2}-(x-r)^{2}}
\end{align*}

Now, plug in r=2r

\begin{align*}
y=\sqrt{2r-(x-2r)^{2}}
y=\sqrt{2r^{2}+2rx-x^{2}}
\end{align*}

For the actual height, we must multiply by 2 for the top and bottom semicircles. Also, we have to multiply by the distance from the rotational axis, which is: $(R+x)$ equation. So we get:

\begin{align*}
y=2\sqrt{2r^{2}+2rx-x^{2}}(R+x)
\end{align*}

To solve further from here, we need to put dz in terms of R, r, X and theta. Z here is the width, which depends on the distance from the centre multiplied by the angle:

\begin{align*}
dz=(R+x)dt\theta
\end{align*}

Plugging it all in, we'll need a double integral:

\begin{align*}
I=\int_{0}^{2\pi}\int_{0}^{2r} 2\rho\sqrt{2r^{2}+2rx-x^{2}}(R+x)^{2}(R+x)dxd\theta
\end{align*}

Simplifying this we get:

\begin{align*}
I=\int_{0}^{2\pi}\int_{0}^{2r} 2\rho\sqrt{2r^{2}+2rx-x^{2}}(R+x)^{3}dxd\theta
\end{align*}

This is the correct expression for the moment of inertia of a solid Double Chocolate Donut. Note that solving complicated double integrals is beyond the scope of the course and is an exercise in mathematics rather than physics. For those who would like to solve it for fun, solving the above expression gives the moment of inertia for a solid donut to be:

\begin{align*}
I=2\rho \sqrt{\frac{8r^{3}(10R^{3}+30rR^{2}+39r^{2}R+19r^{3}}{15}}
\end{align*}
\end{solution}

%%%%%%%%%%%%%%%%%%%%%
% Statics
%%%%%%%%%%%%%%%%%%%%%

\question The system in Figure \ref{fig:rotationaldynamics:hinge} is in static equilibrium. A mass of $m=\SI{225}{kg}$ hangs from the end of the uniform strut whose mass is $M=\SI{45.0}{kg}$. 
\begin{parts}
\part Find the tension, $T$, in the cable.
\part Find the horizontal and vertical force components exerted on the sturt by the hinge.
\capfig{0.5\textwidth}{figures/Rotationaldynamics/hinge.png}{\label{fig:rotationaldynamics:hinge} The geometry of the system.}
\end{parts}
\begin{finalanswer}
\begin{enumerate}[(a)]
\item \SI{6626.59}{N}
\item $F_{hx}=\SI{5738.80}{N}$, $F_{hy}=\SI{5959.30}{N}$
\end{enumerate}
\end{finalanswer}
\begin{solution}
\begin{parts}
\part Since the system is at rest, the sum of the forces and torques on the strut must be zero. The force of tension from the cable, $\vec T$, is directed at $\SI{30}{\degree}$ below the horizontal. The force from the hinge on the strut will have both a vertical and a horizontal component. Figure \ref{fig:rotationaldynamics:hinge_a} shows the forces on the strut (including where they are exerted, since this matters for torques). If we define the $x$ axis to be horizontal (positive to the right), and the $y$ axis to be vertical (positive upwards), we can write the sum of the forces as:
\begin{align*}
\sum F_x &=F_{hx}-T\cos(\SI{30}{\degree}) = 0\\
\sum F_y &=F_{hy}-T\sin(\SI{30}{\degree})-(m+M)g = 0
\end{align*}

\capfig{0.5\textwidth}{figures/Rotationaldynamics/hinge_annotated.png}{\label{fig:rotationaldynamics:hinge_a} Forces acting on the strut.}

Let $L$ be the length of the strut. We choose to write the torques about the hinge (so that the unknown forces at the hinge are not needed). The torques are all in the same plane, so that we can deal with only their magnitudes. Let us choose clockwise as corresponding to positive torque. The sum of the torques must be zero:
\begin{align*}
\sum \tau&=Mg\frac{L}{2}\cos(\SI{45}{\degree})+mgL\cos(\SI{45}{\degree})-TL\sin(\SI{15}{\degree})=0\\
T\sin(\SI{15}{\degree}) &= g\cos(\SI{45}{\degree})\left(\frac{M}{2}+m \right)\\
\therefore T &= (\SI{9.8}{m/s^2})\frac{\cos(\SI{45}{\degree})}{\sin(\SI{15}{\degree})}\left(\frac{(\SI{45.0}{kg})}{2}+(\SI{225}{kg}) \right)=\SI{6626.59}{N}
\end{align*}

\part Knowing the tension from part (a), we can now use the sum of the forces equations in $x$ and $y$ to get the forces on the hinge:
\begin{align*}
F_{hx}&=T\cos(\SI{30}{\degree})=(\SI{6626.59}{N})\cos(\SI{30}{\degree})=\SI{5738.80}{N}\\
F_{hy}&=T\sin(\SI{30}{\degree})+(m+M)g=(\SI{6626.59}{N})\sin(\SI{30}{\degree})+(\SI{9.8}{m/s^2})(\SI{270}{kg})=\SI{5959.30}{N}
\end{align*}
\end{parts}

\end{solution}


%%%%%%%%%%%%%%%%%%%%%%%%%%%%%%%%%%%
%
% This question taken from Physics for Scientists and Engineers, a Strategic Approach
%
%%%%%%%%%%%%%%%%%%%%%%%%%%%%%%%%%%%
\question A $L=\SI{3.0}{m}$ long rigid uniform beam with a mass of $M=\SI{100}{kg}$ is supported at each end, as shown in Figure \ref{fig:rotationaldynamics:student}. A $m=\SI{80}{kg}$ student stands a distance $d=\SI{2.0}{m}$ away from support 1. How much upward force does each support exert on the beam if the system is at rest?

\capfig{0.6\textwidth}{figures/Rotationaldynamics/student.png}{\label{fig:rotationaldynamics:student} }
\begin{finalanswer}
\SI{751.33}{N}
\end{finalanswer}
\begin{solution} 
Since the system is at rest, the sum of the forces and torques on the beam must be zero. We can choose to write the torques about support point 1, so that the unknown force at support point 1 does not contribute. Let $F_1$ and $F_2$ be the upward components of the forces on support 1 and 2, respectively. Let us choose clockwise for positive torques:
\begin{align*}
\sum \tau &= (\SI{1.5}{m})Mg+(\SI{2.0}{m})mg-(\SI{3.0}{m})F_2=0\\
\therefore F_2& = \frac{(\SI{9.8}{m/s^2})}{(\SI{3.0}{m})}((\SI{1.5}{m})(\SI{100}{kg})+(\SI{2.0}{m})(\SI{80}{kg}))\\
&=\SI{1012.67}{N}
\end{align*}

To find $F_1$, we can either write the torgues about support point 2, or use the sum of the forces in the vertical direction. Choosing the latter (with positive upwards), we have:

\begin{align*}
\sum F &= F_1+F_2-Mg-mg=0\\
\therefore F_1&=(M+m)g-F_2=(\SI{180}{kg})(\SI{9.8}{m/s^2})-(\SI{1012.67}{N})\\
&=\SI{751.33}{N}
\end{align*}
\end{solution}

%Based on Giancolli 12-59
\question You wish to pull a sphere of radius $R$ and mass $M$ over a step of height $h$, by exerting a tangential force $\vec F$ at the top of the sphere as shown in Figure \ref{fig:rotationaldynamics:SphereStep}. What is the minimum magnitude of the force that is required?
\capfig{0.4\textwidth}{figures/Rotationaldynamics/SphereStep.png}{\label{fig:rotationaldynamics:SphereStep} }
\begin{finalanswer}
$F=Mg\sqrt{\frac{h}{2R-h}}$
\end{finalanswer}
\begin{solution}
Figure \ref{fig:rotationaldynamics:SphereStep_FBD} shows the forces acting on the sphere and the relevant dimensions. The sphere will move as soon as $F$ is large enough to cause it to have a net torque. 
\capfig{0.4\textwidth}{figures/Rotationaldynamics/SphereStep_FBD.png}{\label{fig:rotationaldynamics:SphereStep_FBD} } 
Since we do not know the normal force, we evaluate the torques about the corner of the step. The lever arm of $F$ is $2R-h$, and the lever arm of the weight, $x$ on the diagram, is given by:
\begin{align*}
(R-h)^2+x^2&=R^2\\
\therefore x&=\sqrt{R^2-(R-h)^2}=\sqrt{2Rh-h^2}=\sqrt{h(2R-h)}
\end{align*}
The sum of the torques, taking clockwise as positive, is given by:
\begin{align*}
\sum \tau &= F(2R-h)-Mg\sqrt{h(2R-h)}=0\\
\therefore F&=Mg\frac{\sqrt{h(2R-h)}}{2R-h}=Mg\sqrt{\frac{h}{2R-h}}
\end{align*}

\end{solution}

%Based on Giancolli 12-80
\question A uniform ladder with a length of $L=\SI{6.0}{m}$ and a mass of $M=\SI{20}{kg}$ leans against a frictionless wall (so that the wall can only exert a horizontal force on the ladder). The ladder makes an angle $\theta=\SI{20}{\degree}$ with the wall, as shown in Figure \ref{fig:rotationaldynamics:Ladder}. Determine the minimum value of the coefficient of static friction between the ladder and the ground for the ladder not to slip when a person weighing $m=\SI{75}{kg}$ stands two-thirds of the way up the ladder.
\capfig{0.2\textwidth}{figures/Rotationaldynamics/Ladder.png}{\label{fig:rotationaldynamics:Ladder}A ladder leaning against a frictionless wall.}
\begin{finalanswer}
$\mu_s =\tan\theta\left(\frac{\frac{1}{2}M-\frac{2}{3}m}{(m+M)}  \right)$
\end{finalanswer}
\begin{solution}
The forces one the ladder when the person is on it are shown in Figure \ref{fig:rotationaldynamics:Ladder_FBD}.
\capfig{0.2\textwidth}{figures/Rotationaldynamics/Ladder_FBD.png}{\label{fig:rotationaldynamics:Ladder_FBD}Forces on the ladder.}

The sum of the torques and forces must be zero, since the ladder is not moving. The sum of the forces in the vertical direction allow us to determine the normal force:
\begin{align*}
\sum F_y&=N-(m+M)g=0\\
\therefore N&=(m+M)g
\end{align*}
The force of static friction at the bottom of the ladder is thus give by:
\begin{align*}
f_s=\mu_sN=\mu_s(m+M)g
\end{align*}
where $\mu_s$ is the coefficient of static friction between the bottom of the ladder and the ground. We take the torques above the top of the ladder, so that we do not need to solve for $\vec R$, the force from the wall. We will take clockwise as positive torques:
\begin{align*}
\sum\tau&=N\sin\theta L-f_s\cos\theta L-Mg\sin\theta\frac{L}{2}-mg\sin\theta\frac{L}{3}=0\\
f_s\cos\theta&=N\sin\theta-\frac{1}{2}Mg\sin\theta-\frac{1}{3}mg\sin\theta\\
f_s&=N\tan\theta-\frac{1}{2}Mg\tan\theta-\frac{1}{3}mg\tan\theta\\
\mu_s(m+M)g &=(m+M)g\tan\theta-\frac{1}{2}Mg\tan\theta-\frac{1}{3}mg\tan\theta\\
\mu_s(m+M)g &=g\tan\theta\left(\frac{1}{2}M-\frac{2}{3}m  \right) \\
\therefore \mu_s &=\tan\theta\left(\frac{\frac{1}{2}M-\frac{2}{3}m}{(m+M)}  \right) \\
\end{align*}
\end{solution}


%Harrison/Zaremba 2000 Final exam
\question A thin rod of length $L=\SI{0.5}{m}$ and mass $M=\SI{1.0}{kg}$ makes an angle $\theta=\SI{60}{\degree}$ with the horizontal. It is held in position by a light horizontal string attached at the top end of the rod, as shown in Figure \ref{fig:rotationaldynamics:RodMass}. A mass $m$ is suspended from the top of the rod by a second string. If the coefficient of static friction between the rod and the ground at the point of contact is $\mu=0.50$, what is the largest mass $m$ that can be suspended such that the rod does not slip?
\capfig{0.4\textwidth}{figures/Rotationaldynamics/RodMass.png}{\label{fig:rotationaldynamics:RodMass}A mass hanging from a rod.}
\begin{finalanswer}
\SI{2.73}{kg}
\end{finalanswer}
\begin{solution}
The forces on the rod are shown in Figure \ref{fig:rotationaldynamics:RodMass_FBD}. Just before the rod slips, the force of friction will be equal to $\mu N$.
\capfig{0.4\textwidth}{figures/Rotationaldynamics/RodMass_FBD.png}{\label{fig:rotationaldynamics:RodMass_FBD}Forces on the rod.}
Since the rod is in static equilibrium, the sum of the forces and torques are both zero. The sum of the forces in the horizontal and vertical directions are:
\begin{align*}
T-\mu N&=0\\
mg+Mg-N&=0
\end{align*}
respectively.  The second equation gives the normal force, which we can substitute in to the first equation:
\begin{align*}
N&=(m+m)g\\
\therefore T&=\mu(m+M)g
\end{align*}
We can take the torques about the bottom (positive torques clockwise):
\begin{align*}
T\sin\theta L-mg\cos\theta L-Mg\cos\theta \frac{L}{2}&=0\\
T\sin\theta-(m+\frac{1}{2}M)g\cos\theta&=0\\
\therefore T&=(m+\frac{1}{2}M)g\frac{\cos\theta}{\sin\theta}=(m+\frac{1}{2}M)g\frac{1}{\tan\theta}\\
\end{align*}
Equating the two equations for $T$ and re-arranging for $m$:
\begin{align*}
\mu(m+M)g &= (m+\frac{1}{2}M)g\frac{1}{\tan\theta}\\
\mu\tan\theta m+\mu\tan\theta M&=m+\frac{1}{2}M\\
m(\mu\tan\theta-1) &= M(\frac{1}{2}-\mu\tan\theta)\\
\therefore m&= \frac{\frac{1}{2}-\mu\tan\theta}{\mu\tan\theta-1}M = \frac{\frac{1}{2}-(0.5)\tan(\SI{60}{\degree})}{0.5\tan(\SI{60}{\degree})-1}\SI{1.0}{kg}\\
&=\SI{2.73}{kg}
\end{align*}
\end{solution}

\question Chlo\"e loves her rocket shoes! I investigated whether it would be safe to buy her some \textit{actual} rocket shoes. To do this, I modelled Chlo\"e as a vertical uniform bar of length $L=\SI{80}{cm}$ and mass $M=\SI{13}{kg}$ with rocket shoes at the bottom, as in Figure \ref{fig:rotationaldynamics:RocketShoe}. I modelled the rocket shoes as being massless and having negligible dimensions (assumed to be a point at the end of the rod). The shoes provide a combined constant thrust of $F=\SI{20}{N}$ parallel to the ground for a duration of \SI{10}{s}. Although the shoes have wheels on them, I assumed that it would be equivalent to sliding on the ground with a kinetic friction coefficient of $\mu_k=0.1$.

\capfig{0.1\textwidth}{figures/RotationalDynamics/RocketShoe.png}{\label{fig:rotationaldynamics:RocketShoe} A model for Chlo\"e and her rocket shoes.}
\begin{parts}
\part If starting from rest, how fast will Chlo\"e be travelling after \SI{10}{s}?
\part Since I do not want to get too much exercise chasing after her, what total distance will she have travelled when she comes to rest? Remember, the rockets only burn for the first \SI{10}{s}; after that, she will slow back down and come to rest.
\part In order to not tip over while the rockets are on, Chlo\"e will need to lean forwards or backwards. What angle will she need to make with the vertical to remain stable as the rocket shoes push her forward? Make a diagram and indicate whether she needs to lean forwards or backwards.
\end{parts}

\begin{solution}
\begin{parts}
\part We need to determine the horizontal acceleration, which we can do from Newton's Second Law and a free body diagram. The forces acting on Chlo\"e are her weight, the normal force from the ground, the thrust from the rocket shoes, and the force of kinetic friction. If we choose positive $x$ in the direction of motion and positive $y$ upwards, the sum of the forces in each direction is:
\begin{align*}
\sum F_x &= F-f_k=ma\\
\sum F_y &= N-mg=0
\end{align*}
Since $f_k=\mu_kN$, we easily find the acceleration:
\begin{align*}
a&=\frac{1}{m}(F-f_k)=\frac{1}{m}(F-f_k)=\frac{1}{m}(F-\mu_kmg)\\
&=\frac{1}{(\SI{13}{kg})}((\SI{20}{N})-(0.1)(\SI{13}{kg})(\SI{9.8}{m/s^2}))=\SI{0.558}{m/s^2}
\end{align*}
After \SI{10}{s}, she will be travelling at a speed of:
\begin{align*}
v(t)=v_0+at=(\SI{0.558}{m/s^2})(\SI{10}{s})=\SI{5.558}{m/s}
\end{align*}
which corresponds to about \SI{20}{km/h}, a little fast for a small child. I would not consider this safe!
\part We can divide the motion in two phases. First, she has a constant acceleration for \SI{10}{s}, as found in part (a). After that, she will decelerate because of friction. During the acceleration phase, she travels a distance of:
\begin{align*}
x_1 = \frac{1}{2}at^2=\frac{1}{2}(\SI{0.558}{m/s^2})(\SI{10}{s})^2=\SI{27.9}{m}
\end{align*}
The deceleration due to friction is:
\begin{align*}
a=-\frac{f_k}{m}=-\frac{\mu_kmg}{m}=-\mu_kg=(0.1)(\SI{9.8}{m/s^2})=\SI{-0.98}{m/s^2}
\end{align*}
The time that it will take for her to come to rest is given by:
\begin{align*}
t = \frac{v_f-v_i}{a}=\frac{-(\SI{5.558}{m/s})}{(\SI{-0.98}{m/s^2})}=\SI{5.67}{s}
\end{align*}
The final position after decelerating is thus:
\begin{align*}
x_f &= x_1 +v_0t+\frac{1}{2}at^2\\
&=(\SI{27.9}{m}) +(\SI{5.558}{m/s})(\SI{5.67}{s})+\frac{1}{2}(\SI{-0.98}{m/s^2})(\SI{5.67}{s})^2\\
&=\SI{43.66}{m}
\end{align*}

\part The thrust from the rocket shoes will cause a torque about Chlo\"e's centre of mass that would cause her to fall backwards. She will thus need to lean forward so that she does not rotate. We have to be a little careful, because Chlo\"e is accelerating, so the sum of the torques is only zero about an axis that is co-moving with Chlo\"e, or instantaneously about the centre of mass.

The easiest way to solve this is to be in the frame of reference of Chlo\"e since we already know the acceleration. We take the sum of torques about the shoe so that we do not need to worry about the three forces at the shoe end of Chlo\"e. Since we are in an non-inertial frame of reference, we have to include an inertial force of $-M\vec a$ at the centre of mass.
\capfig{0.15\textwidth}{figures/RotationalDynamics/RocketShoe_forces.png}{\label{fig:rotationaldynamics:RocketShoe_forces}Force on Chlo\"e in the \textit{accelerating} frame of reference.}
Taking the torques about the shoe (positive clockwise), only gravity and the inertial force contribute. The torques sum to zero:
\begin{align*}
Mg\frac{L}{2}\sin\theta-Ma\cos\theta\frac{L}{2}&=0\\
g\sin\theta&=a\cos\theta\\
\therefore\tan\theta&=\frac{a}{g}=\frac{(\SI{0.558}{m/s^2})}{(\SI{9.8}{m/s^2})}=0.0570\\
\therefore \theta&=\tan^{-1}(0.0579)=\SI{3.26}{\degree}
\end{align*}
which is a small enough angle that it is reasonable to expect that she could maintain.

Note that if we take the torques about the centre of mass, then we don't need to worry about the inertial force, and the torques must be zero in all reference frames. Again, taking positive torque clockwise and about the centre of mass:
\begin{align*}
N\sin\theta\frac{L}{2}+f_k\cos\theta\frac{L}{2}-F\cos\theta\frac{L}{2}&=0\\
N\sin\theta+f_k\cos\theta-F\cos\theta&=0\\
mg\sin\theta+\mu_kmg\cos\theta-F\cos\theta&=0\\
mg\sin\theta &= (F-\mu_kmg)\cos\theta\\
\tan\theta &=\frac{F-\mu_kmg}{mg}=\frac{a}{g}
\end{align*}
where in the last line, we substituted in the expression for acceleration from part (a).
\end{parts}
\end{solution}

\question A majestic llama of mass $m=\SI{500}{kg}$ is walking on a horizontal plank of mass $M=\SI{200}{kg}$ and length $L=\SI{5}{m}$ that is attached to the vertical face of Huyana Picchu (a mountain in Peru). The plank is attached to the mountain with a hinge that allows the plank to rotate in the vertical plane. The other end of the plank is secured by a mass-less rope that makes an angle of $\theta =\SI{30}{\degree}$ with respect to the plank. The rope can sustain a maximum tension of $\SI{5000}{N}$. As the llama walks away from the mountain, the tension in the rope increases until the rope suddenly breaks, and the plank with the llama on it starts to rotate about the hinge. What is the (linear) acceleration of the llama the moment just after the rope breaks? 

Note that the moment of inertial of plank of mass $M$ and length $L$ rotated about one of its ends is $I=\frac{1}{3}ML^2$.
\capfig{0.3\textwidth}{figures/RotationalDynamics/llamaplank.png}{\label{fig:rotationaldynamics:llamaplank}An adventurous llama walking out on a plank.}
\begin{solution}
First we need to find how far the llama walks before the force of tension exceeds its maximal value. We can consider the torques on the plank about the hinge, so that the only forces that contribute a torque are the weight of the plank, the weight of the llama and the force of tension. The plank is in equilibrium (net torque of zero) until the llama is far enough (say a distance $x$ from the hinge). The sum of the torques is given by:
\begin{align*}
\sum \tau = Mg\frac{L}{2}+mgx-T\sin\theta L &= 0\\
\therefore x &= \frac{2T\sin\theta -Mg}{2mg}L
\end{align*}
We can solve for the position $x$ of the llama when $T$ is at its maximal value:
\begin{align*}
x &= \frac{2T\sin\theta -Mg}{2mg}L= \frac{2(\SI{5000}{N})\sin(\SI{30}{\degree})-(\SI{200}{kg})(\SI{9.8}{m/s^2})}{2(\SI{500}{kg})(\SI{9.8}{m/s^2})} (\SI{5}{m})= \SI{1.551}{m}
\end{align*}
When the llama reaches this position, the rope breaks, and the system comprised of the plank and llama will rotate about the hinge. The moment of inertia of the plank and llama is given by:
\begin{align*}
I&=I_{plank}+mx^2=\frac{1}{3}ML^2+mx^2\\
&=\frac{1}{3}(\SI{200}{kg})(\SI{5}{m})^2+(\SI{500}{kg})(\SI{1.551}{m})^2=\SI{2869.50}{kg\cdot m^2}
\end{align*}
The net torque on the plank and llama system about the hinge is that from the weights of the plank and llama:
\begin{align*}
\sum\tau &=  Mg\frac{L}{2}+mgx
\\&= (\SI{200}{kg})(\SI{9.8}{m/s^2})\frac{(\SI{5}{m})}{2}+(\SI{500}{kg})(\SI{9.8}{m/s^2})(\SI{1.551}{m})=\SI{12500}{N\cdot m}
\end{align*}
The angular acceleration of the plank and llama is then given by:
\begin{align*}
\sum\tau &= I\alpha\\
\therefore \alpha &= \frac{\sum\tau}{I}=\frac{(\SI{12500}{N\cdot m})}{(\SI{2869.50}{kg\cdot m^2})}=\SI{4.36}{rad/s}
\end{align*}
The linear acceleration of the llama is given by:
\begin{align*}
a &= \alpha r\\
a &= \alpha x = (\SI{4.36}{rad/s})(\SI{1.551}{m}) =\SI{6.77}{m/s^2}
\end{align*}
\end{solution}



\question A uniform wire of mass $M$ is bent into a closed semi-circle with radius $R$ (the wire has a curved section and a straight section), as shown in Figure \ref{fig:rotationaldynamics:semicircle}.
\capfig{0.30\textwidth}{figures/semicircle.png}{\label{fig:rotationaldynamics:semicircle} A wire bent into a closed semi-circle of radius $R$.}
\begin{parts}
	\part Show that the centre of mass of the wire is located at a distance:
	\begin{align*}
	h=\frac{2R}{2+\pi}
	\end{align*}
	from the centre of the semi-circle.
	\part The semi-circular wire from above is held with a pin that is placed at the centre of the straight section of wire (point $P$ in Figure \ref{fig:rotationaldynamics:semicircle_b}), so that the wire can rotate in the vertical plane about a horizontal axis that goes through the pin and that is perpendicular to the plane of the wire. What is the moment of inertia of the wire about this axis of rotation?
	
	\part A small bug of mass $m=\SI{5}{g}$ lands on the wire, directly below the point where the wire is suspended (left panel of Figure \ref{fig:rotationaldynamics:semicircle_b}), and starts to walk up the wire. When the bug is at a position such that a line from $P$ to the bug makes an angle $\theta=\SI{30}{\degree}$ with respect to the vertical (right panel), what angle, $\phi$, does the straight part of the wire make with the horizontal? Assume that the wire has a mass $M=\SI{100}{g}$.
\end{parts}
\capfig{0.60\textwidth}{figures/RotationalDynamics/semicircle_b.png}{\label{fig:rotationaldynamics:semicircle_b} The wire is suspended at point $P$ (midpoint of the straight section) so that it may rotate in the vertical plane. A bug of mass $m$ starts below point $P$ (left panel) and starts to walk up the wire, causing the wire to rotate (right panel).}

\begin{solution}
	\begin{parts}
		\part We introduce a coordinate system such that the origin is located at the centre of the semi-circle and such that the wire lies in the negative $y$ side, as shown in Figure \ref{fig:rotationaldynamics:semicircle_sol}.
		\capfig{0.30\textwidth}{figures/RotationalDynamics/semicircle_sol.png}{\label{fig:rotationaldynamics:semicircle_sol}A small mass element of the curved section of wire.}
		By symmetry, the $x$ coordinate of the centre of mass will be zero. We consider the wire to be made of two parts, a straight part with its centre of mass at the origin (a rod of length $2R$), and a curved part, whose $y$ coordinate of the centre of mass is given by:
		\begin{align*}
		y_{CM}=\frac{1}{M_{circ}}\int y dm
		\end{align*}
		where $M_{circ}$ is the mass of the circular part of the wire. We can choose $dm$ to be small piece of the wire of length $dl$ that subtends and arc $d\theta$, so that the mass element is given by:
		\begin{align*}
		dm =\lambda dl = \lambda Rd\theta=\frac{M}{L}Rd\theta
		\end{align*}
		where $\lambda$ is the mass per unit length of the wire. The $y$ position of the mass element is given by:
		\begin{align*}
		y=R\sin\theta
		\end{align*}
		where $\theta$ is taken as zero at the $x$ axis and increases counter-clockwise (such that $\sin\theta$ is negative). The $y$ position of the centre of mass of the curved part of the wire is then given by:
		\begin{align*}
		y_{CM}&=\frac{1}{M_{circ}}\int y dm=\frac{1}{M_{circ}}\int_{\pi}^{2\pi} y \frac{M}{L}Rd\theta\\
		&=\frac{1}{M_{circ}}\int_{\pi}^{2\pi} (R\sin\theta) \frac{M}{L}Rd\theta\\
		&=\frac{R^2M}{LM_{circ}}\int_{\pi}^{2\pi} \sin\theta d\theta=-\frac{R^2M}{LM_{circ}}[\cos\theta]_{\pi}^{2\pi}\\
		&=-\frac{2R^2M}{LM_{circ}}
		\end{align*}
		We also need to determine the fraction of the mass that is located in the circular part:
		\begin{align*}
		M_{circ}=\pi R\frac{M}{L}
		\end{align*}
		The $y$ position of the centre of mass of the bent part of the wire becomes:
		\begin{align*}
		y_{CM}&=-\frac{2R^2M}{LM_{circ}}=-\frac{2R}{\pi}
		\end{align*}
		Finally, combining this with the straight section of the bar ($M_s$), the centre of mass is given by:
		\begin{align*}
		y_{CM}&=\frac{M_s(0)-M_{circ}\left(\frac{2R}{\pi}\right)}{M}=-\frac{2R^2}{L}
		\end{align*}
		Noting that $L=2R+\pi R$, we can write this as:
		\begin{align*}
		y_{CM}&=-\frac{2R}{2+\pi}
		\end{align*} 
		\part We consider the wire as being comprised of a bar rotated about its centre of mass plus a semi-circle rotated about its centre. We first need to determine the mass of each section. The total length of the wire is:
		\begin{align*}
		L=2R+\pi R=(2+\pi)R
		\end{align*}
		The straight part thus has a mass:
		\begin{align*}
		M_{bar}=\frac{M}{L}2R=\frac{2M}{2+\pi}
		\end{align*}
		and the curved part:
		\begin{align*}
		M_{circ}=\frac{M}{L}\pi R=\frac{M \pi}{2+\pi}
		\end{align*}
		The total moment of inertia is thus:
		\begin{align*}
		I&=I_{bar}+I_{circ}=\frac{1}{12}M_{bar}(2R)^2+M_{circ}R^2\\
		&=\frac{1}{12}\frac{2M}{2+\pi}(2R)^2+\frac{M \pi}{2+\pi}R^2\\
		&=\left(\frac{1}{3}\right)\frac{2+3\pi}{6+3\pi}MR^2
		\end{align*}
		\part We can consider the torques about the point $P$ exerted on the wire by both the bug and the weight of the wire (exerted at the centre of mass found in part a). The torques have to be equal and opposite in direction since the system will be in equilibrium and there are no other torques exerted on the wire.
		\begin{align*}
		\tau_{bug} &= \tau_{g}\\
		mgR\sin\theta &= Mg\left(\frac{2R}{2+\pi}\right)\sin\phi\\
		\therefore \sin\phi&=\frac{m(2+\pi)}{2M}\sin\theta\\
		\end{align*}
		Using the numbers provided:
		\begin{align*}
		\phi=\sin^{-1}\left(\frac{(\SI{5}{g})(2+\pi)}{2(\SI{100}{g})}\sin(\SI{30}{\degree})  \right)=\SI{3.685}{\degree}
		\end{align*}
		Note that we would find the same answer if we considered that the centre of mass of the bug and wire system must remain at $x=0$.
	\end{parts}
\end{solution}





%TODO: Question library question: Given a coefficient of static friction, what is the maximum angle of the incline that would allow rolling without slipping.
%TODO: Question library question: A yo-yo that is falling, what is its linear acceleration?

