\section{Work and Energy}

%%%%%%%%%%%%%%%%%%%%%%%%%%%%%%%%%%%
%%
%% Multiple Choice
%%
%%%%%%%%%%%%%%%%%%%%%%%%%%%%%%%%%%%
\subsection{Multiple Choice}
%Submitted by Wei Zhuolin
\question A constant force applied by a person is pushing a \SI{200}{kg} box to the right. Before the person starts to push, the initial velocity of the box is \SI{2}{m/s} to the right. After \SI{200}{s} the velocity of the box is \SI{20}{m/s} to the right. What is the average power exerted by the person, assuming that the box moves on a frictionless surface?
\begin{checkboxes}
\choice 100 W
\CorrectChoice 198 W \correct
\choice 186 W
\choice 200 W
\end{checkboxes}


%Submitted by Sara Stephens
\question In which of the following scenarios is work being done?
\begin{checkboxes}
\choice Carrying a briefcase horizontally above the ground with a constant velocity
\choice A normal force from the floor on the second level of a building stops you from falling into the floor below
\choice The tension in a string pulling on a rubber stopper when you spin it around at a constant speed
\choice Both 1 and 3
\CorrectChoice None of the above \correct
\end{checkboxes}


%submitted by Shona Birkett
\question On Halloween, a clown enters a pumpkin carving contest. Part of the contest involves the participants lifting up their pumpkin after it has been carved to declare they are finished with their design. This also allows the audience of the contest to see the funky designs carved into the pumpkins. The clown completes his design and lifts up his pumpkin. What is the work done by the clown on the pumpkin if the clown lifts his pumpkin a total of \SI{2}{m} and the pumpkin's mass is \SI{6}{kg}?
\begin{checkboxes}
\CorrectChoice \SI{117.6}{J} \correct
\choice \SI{102.3}{J}
\choice \SI{120.0}{J}
\end{checkboxes}


%submitted by Jesse Fu
\question What are the SI units for power? 
\begin{checkboxes}
\choice \si{kg m s^{-2}} 
\choice \si{kg m^2 s^{-2}} 
\CorrectChoice \si{kg m^2 s^{-3}} \correct
\choice \si{kg m^3 s^{-3}} 
\end{checkboxes}


%submitted by Emma Lanciault
\question A worker horizontally pushes a sled with a force of \SI{40}{N} over a distance of \SI{6.0}{m}, while a frictional force of \SI{24}{N} acts on the wheelbarrow in a direction opposite to that of the worker. What net work is done on the wheelbarrow?
\begin{checkboxes}
\CorrectChoice \SI{96}{J} \correct
\choice \SI{144}{J}
\choice \SI{-144}{J}
\choice \SI{-96}{J}
\end{checkboxes}


%submitted by Daniel Tazbaz
\question A horizontal conveyor belt is moving a \SI{1}{m} tall guanaco at constant speed. The guanaco has a mass of \SI{90}{kg} and moves \SI{15}{m} along the conveyor belt, in \SI{10}{s}. What is the work done on the guanaco, assuming the conveyor belt has a 100\% efficiency?
\begin{checkboxes}
\choice \SI{13000}{J}
\CorrectChoice \SI{0}{J} \correct
\choice \SI{90}{J}
\choice Not enough information.
\choice Gaunacos are immune to work.
\end{checkboxes}

%From Mideyar Exam F17
%Emma Neary (modified to add efficiency and realistic power)
\question You are using an electric crane to lift a \SI{400}{kg} box onto the roof of a house. The crane has a power rating of \SI{1000}{W}, and is 50\% efficient at converting electrical power into mechanical work. If the house is \SI{10}{m} tall, what is the smallest time in which the crane can lift the box from the ground up to the roof?
\begin{checkboxes} 
\choice \SI{0.013}{s}
\choice \SI{7.8}{s}
\CorrectChoice \SI{78.4}{s} \correct
\choice \SI{156.8}{s}
\end{checkboxes}

%Tashifa Imtiaz
\question If \SI{500}{J} of net work is done to a \SI{20}{kg} mass with an initial speed of \SI{5}{m/s}, then what is the final speed of the mass?
\begin{checkboxes}
\choice \SI{7.64}{m/s}
\CorrectChoice \SI{8.67}{m/s}
\choice \SI{25.00}{m/s}
\choice \SI{30.00}{m/s}
\end{checkboxes}

\question The work done by a person that climbed up a staircase of height $h$:
\begin{checkboxes}
\choice is negative.
\choice is 0.
\CorrectChoice is positive.
\choice depends on our choice of whether positive is up or down.
\end{checkboxes}

\question You push a crate a distance $d$ up an incline (measured parallel to the incline). The crate started at rest and ended at rest, and there is friction between the incline and the crate. The work that you did:
\begin{checkboxes}
\choice is negative.
\choice is 0.
\CorrectChoice is positive.
\choice depends on how fast you pushed the crate up the incline.
\end{checkboxes}


%%%%%%%%%%%%%%%%%%%%%%%%%%%%%%%%%%%
%
% long answer
%
%%%%%%%%%%%%%%%%%%%%%%%%%%%%%%%%%%%
\subsection{Long answers}
%From James G
\question A sled and passenger with a combined mass of \SI{50}{kg} are pulled \SI{20}{m} across snowy flat ground ($\mu_k = 0.20$) at constant velocity, by a force, $\vec F$, directed \SI{25}{\degree} above the horizontal.
\begin{parts}
\part Draw a free body diagram of the sled and passenger. Be sure to show your coordinate system, show the velocity and acceleration of the sled, and have accurate lengths of force vectors.
\part What is the work done by the applied force? 
\part What is the work done by the frictional force?
\part What is the total work done? Calculate the total work and explain your result.
\end{parts} 

\begin{finalanswer}
\begin{parts}
\part The free body diagram:
\capfig{0.3\textwidth}{figures/WorkEnergy/Sled_FBD}{ A free body diagram of the sled and passenger system.}
\part $\SI{1792.80}{J}$
\part $\SI{-1792.80}{J}$
\part $\SI{0}{J}$
\end{parts}
\end{finalanswer}

\begin{solution}
\textbf{a)} The velocity should be pointing in the horizontal direction and in the same direction as the horizontal component of the applied force. The acceleration should be zero. As the acceleration is zero, the sum of the forces in each direction must be zero; relative magnitudes of vectors on the FBD should roughly reflect this (the vertical components of normal force, $\vec N$, and applied force cancel the weight, $M\vec g$, and the horizontal component of the applied force cancels the frictional force, $\mu_kN$).
\capfig{0.3\textwidth}{figures/WorkEnergy/Sled_FBD}{\label{fig:workenergy:Sled_FBD} A free body diagram of the sled and passenger system.}
\textbf{b)} To get the work done by the applied force, we have to first determine its magnitude. We can use Newton's laws to do so. The sum of the forces must be zero, because the sled+passenger are travelling at constant velocity, and we consider the combination as a single object of mass $M=\SI{50}{kg}$. Using the $x$ and $y$ coordinates illustrated in Figure \ref{fig:workenergy:Sled_FBD}, we have, for each direction:
\begin{align*}
\sum F_x &= F\cos\theta-\mu_kN = 0 \\
\sum F_y &= F\sin\theta+N-Mg = 0
\end{align*}
From the $x$ component, we can determine the normal force, which we then substitute into the $y$ equation:
\begin{align*}
N&=\frac{F\cos\theta}{\mu_k}\\
0&=F\sin\theta+\frac{F\cos\theta}{\mu_k}-Mg\\
\therefore F&=\frac{Mg}{\sin\theta+\frac{1}{\mu_k}\cos\theta}\\
&=\frac{(\SI{50}{kg})(\SI{9.8}{m/s^2})}{\sin(\SI{25}{\degree})+\frac{1}{(0.2)}\cos(\SI{25}{\degree})}=\SI{98.91}{N}
\end{align*}
The work done by the applied force is:
\begin{align*}
W_F&=\vec F\cdot\vec d=Fd\cos\theta\\
&=(\SI{98.91}{N})(\SI{20}{m})\cos(\SI{25}{\degree})=\SI{1792.80}{J}
\end{align*}
The work is positive, as the force has a component acting in the direction of motion.

\textbf{c)} The force of friction is equal in magnitude to $F\cos\theta$ (from the $x$ component of Newton's Second Law above). The angle that it makes with the displacement vector is $\SI{180}{\degree}$. The work done by the frictional force is thus:
\begin{align*}
W_f&=\vec f\cdot\vec d=F\cos(\SI{25}{\degree})d\cos(\SI{180}{\degree})=\SI{-1792.80}{J}\\
\end{align*}

\textbf{d)} The only forces doing work are the x-component of the applied force, and friction. Then $W_{total}=W_f+W_{F}=\SI{0}{J}$. We could have predicted this from the work-energy theorem; since the sled+passenger are moving at the same velocity the whole time, there is no change in kinetic energy, and the net work done by external forces must be zero.
\end{solution}



%From James G
\question A $\SI{7.0}{kg}$ body has three times the kinetic energy of a $\SI{20.0}{kg}$ body. Calculate the ratio of the speeds of these bodies.

\begin{finalanswer}
The $\SI{7.0}{kg}$ body is moving about 2.9 times faster than the $\SI{20.0}{kg}$ body.
\end{finalanswer}


\begin{solution}
The \SI{7.0}{kg} body has three times the kinetic energy of the \SI{20.0}{kg} body:

\begin{align*}
K_7 = 3 K_{20}
\end{align*}
Using the formula for kinetic energy ($K_i=\frac{1}{2} m_i v_i^2$) and cancelling factors of $\frac{1}{2}$:
\begin{align*}
m_7 v_7^2 = 3 m_{20} v_{20}^2
\end{align*}
Isolating the ration of speeds,
\begin{align*}
\frac{v_7}{v_{20}} = \sqrt[]{\frac{3 m_{20}}{m_7}} = \sqrt[]{\frac{3 (\SI{20.0}{kg})}{(\SI{7.0}{kg})}}   \simeq \mathbf{2.9}
\end{align*}
Therefore the $\SI{7.0}{kg}$ body is moving about \bf{2.9 times faster} than the $\SI{20.0}{kg}$ body.
\end{solution}

%From James G
\question Not all springs exhibit a linear restoring force ($F(x)=-kx$). If you stretch any spring far enough, it will exhibit \textit{non-linearity}, where terms in the force go with higher powers of x (e.g. $F(x)= - kx - a x^3 - b x^5 - ...$).

Consider a bungee cord that exerts a non-linear elastic force of magnitude:
\begin{align*}
F(x)=-k_1 x-k_3 x^3
\end{align*}
where $x$ is the distance the cord is stretched, and $k_1=\SI{200}{N/m}$ is the linear spring coefficient. Suppose that the cord is stretched a distance of 19 m, and it is found that this requires $\SI{24}{kJ}$.  What is the non-linear spring coefficient $k_3$ of the bungee cord (and its units)? 

\begin{finalanswer}
$\SI{-0.37}{N m^{-3}}$
\end{finalanswer}
\begin{solution}
We are given the functional form of the spring force, and know that it is stretched over a distance of \SI{19}{m}. We calculate the work done against the restoring force to stretch the bungee cord (we must apply a force that is in the opposite direction of the restoring force, and in the same direction as the displacement, thus producing positive work):
\begin{align*}
 W = \int_a^b -F(x) dx
\end{align*}
Setting $a=\SI{0}{m}$, $b=\SI{19}{m}$, using the given form of $F(x)$, and integrating:
\begin{align*}
W &= \int_{\SI{0}{m}}^{\SI{19}{m}} \left(k_1 x + k_3 x^3 \right) dx \\
&= \left. \left(\frac{k_1}{2} x^2 + \frac{k_3}{4} x^4 \right) \right|_{\SI{0}{m}}^{\SI{19}{m}} \\
&=  \frac{k_1}{2} (\SI{19}{m})^2 + \frac{k_3}{4} (\SI{19}{m})^4  
\end{align*}

Isolating $k_3$ and substituting in known values for $W$ and $k_1$:
\begin{align*}
k_3 &= \frac{4}{(\SI{19}{m})^4} \left[ W - \frac{k_1}{2} (\SI{19}{m})^2  \right] \\
&= \frac{4}{(\SI{19}{m})^4} \left[ (\SI{24000}{J}) - \frac{(\SI{200.}{N/m})}{2} (\SI{19}{m})^2 \right] \\
\therefore k_3 &= \SI{-0.37}{N m^{-3}}
\end{align*}
\end{solution}

%From James G
\question A person with a mass of \SI{80}{kg} runs up a flight of stairs \SI{20}{m} high in \SI{10}{s} at constant speed. 
\begin{parts}
\part Assuming the person's body is only 25$\%$ efficient in converting energy to work, how much power do they expend lifting themselves up the stairs?
\part The person's daily energy intake from food is \SI{10.5}{MJ} (2500 food calories) while maintaining a constant weight. What is the average power they produce over a day? How does this compare with their power production when running up the stairs?
\end{parts}

\begin{finalanswer}
\begin{parts}
\part $\SI{6272}{W}$
\part $\SI{121.5}{W}$.  This is about 50 times less than the power he produces while running up the stairs.
\end{parts}
\end{finalanswer}

\begin{solution}
\textbf{a)} Since the person ran up the stairs at constant speed, there is no change in kinetic energy. From the work-energy theorem, this means $W=\Delta K=0$.  The person exerts a constant force upwards (on average), $F_{person}$, that does positive work against gravity. The total work done in going up a distance $h=\SI{20}{m}$ can be written as the sum of the work done by the person, and the work done by gravity:
\begin{align*}
0&=W_{person}+W_g=F_{person}h+F_gh \\
\therefore   W_{person} &= mgh = (\SI{80}{kg})(\SI{9.8}{m/s^2})(\SI{20}{m})=\SI{15680}{J} 
\end{align*}
If the person were 100\% efficient in their conversion from energy to work, the power required to run up the stairs in t=\SI{20}{s} is:
\begin{align*}
P^W_{person}=\frac{W_{person}}{t}=\frac{mgh}{t}=\frac{(\SI{15680}{J})}{(\SI{10}{s})}=\SI{1568}{W}
\end{align*}
Given that they are only 25\% efficient in converting energy to work, they need to generate energy at a rate that is four times ($\frac{1}{0.25}$) higher:
\begin{align*}
P_{person}=\frac{1}{0.25}P^W_{person}=\frac{1}{0.25}(\SI{1568}{W})=\SI{6272}{W}
\end{align*}

\textbf{b)} Since the person is maintaining their weight, they must be expending \SI{1.05e7}{J} over the course of \SI{24}{hr}. Then, the average power that they expend in a day is:
\begin{align*}
P_{avg} &= \frac{E_{daily}}{t_{day}} \\
&= \frac{(\SI{1.05e7}{J})}{(\SI{86400}{s})}\\
&=\SI{121.5}{W}
\end{align*}
This is about 50 times less than the power they produce while running up the stairs.
\end{solution}

%Giancolli 7-72 -Fixed. Can make a more original scenario is preferred.
\question A \SI{60}{kg} meteorite strikes the earth, colliding with a a patch of mud. The meteorite moves \SI{4.0}{m} into the mud. The force between the meteorite and the mud can be modelled as:
\begin{align*}
F(x)=(\SI{590}{N/m^3})x^3
\end{align*}
where $x$ is the depth in the mud. Assuming that the meteorite is at rest once it has travelled \SI{4.0}{m}, what was the initial speed of the meteorite?
\begin{finalanswer}
$\SI{34.36}{m/s}$
\end{finalanswer}
\begin{solution}
Two forces act on the meteorite as it comes to rest: gravity which does positive work and the mud, which does negative work. Both forces are co-linear with the displacement. Since the meteorite ends at rest, the net (negative) work done by the forces  will be equal to the (negative) change in kinetic energy of the meteorite.

The work done on the meteorite is given by the net force exerted by the mud and gravity integrated over the \SI{5.0}{m} distance over which the meteorite stopped. The net force is anti-parallel to the displacement, so we expect that the work will be negative:
\begin{align*}
W&=\int_{\SI{0}{m}}^{\SI{4.0}{m}} [mg -F(x)]dx=\int_{\SI{0}{m}}^{\SI{4.0}{m}} [mg-(\SI{590}{N/m^3})x^3]dx\\
&=\left[(\SI{60}{kg})(\SI{9.8}{m/s^2})x - (\SI{590}{N/m^3})\frac{1}{4} x^4\right]_{\SI{0}{m}}^{\SI{4.0}{m}}\\
&=(\SI{60}{kg})(\SI{9.8}{m/s^2})(\SI{4.0}{m})-(\SI{590}{N/m^3})\frac{1}{4} (\SI{4.0}{m})^4=\SI{-35408}{J}
\end{align*}
This is equal to the change in kinetic energy:
\begin{align*}
W&=\Delta K=K_{final}-K_{initial}=0 - \frac{1}{2}mv^2\\
\therefore v&=\sqrt{\frac{-2W}{m}}=\sqrt{\frac{-2(\SI{-35408}{J})}{(\SI{60}{kg})}}=\SI{34.36}{m/s}
\end{align*}
\end{solution}

\question A skateboarder of mass $m$ is riding along a frictionless ramp, as show in Figure \ref{fig:workenergy:skateramp}. Using the coordinate system shown in the Figure, the ramp can be modelled as being horizontal for negative values of $x$, and then a parabola, $y(x)=a+bx+cx^2$, for positive values of $x$. The ramp ends at $x=x_0$. Show that the work done by gravity on the skateboarder over the path of the ramp (between $x=0$ and $x=x_0$) is given by:
\begin{align*}
W = -mg(bx_0+cx_0^2)
\end{align*}

\capfig{0.5\textwidth}{figures/WorkEnergy/skateramp.png}{\label{fig:workenergy:skateramp} A skateboarder on a ramp.}
\begin{solution}
Only gravity does work. The force of gravity is given by:
\begin{align*}
F_g = -mg\hat j
\end{align*}
We can write a small displacement vector, $d\vec r$, as:
\begin{align*}
d\vec r=dx\hat i+dy\hat j
\end{align*}
The work done by gravity is given by:
\begin{align*}
W&=\int \vec F_g \cdot d\vec r \\
&=\int (-mg\hat j) \cdot (dx\hat i+dy\hat j) \\
&=\int -mg dy=-mg\int dy
\end{align*}
We can integrate this with respect to $y$ if we use the correct limits for $y$, namely from $y(x=0)=a$ to $y(x=x_0)=a+bx_0+cx_0^2$:
\begin{align*}
W&=-mg\int_{a}^{a+bx_0+cx_0^2} dy=-mg \left[y\right]_{a}^{a+bx_0+cx_0^2}\\
&=-mg(bx_0+cx_0^2)
\end{align*}
as required. We could also have integrated this with respect to $x$, using substitution of variables:
\begin{align*}
-mg\int dy &=-mg\int_0^{x_0} \left(\frac{dy}{dx}\right)dx\\
&=-mg\int_0^{x_0} \left( b+2cx \right)dx\\
&=-mg \left[bx+cx^2  \right]_0^{x_0}\\
&=-mg(bx_0+cx_0^2)
\end{align*}
as before.
\end{solution}

\question A toy rocket of mass $m$ is tied to a string and forced to go around in a horizontal circle of radius $R$, as shown in Figure \ref{fig:workenergy:RocketCircle}. The thrust of the rocket has a constant magnitude $F$ and is always directed at an angle $\theta$ with respect to the tangent of circle.
\begin{parts}
\part If the rocket starts at rest, what will its speed be after 1 revolution (in terms of the given variables)?
\part How long will it take to complete the first revolution?
\part What is the tension in the string after 1 revolution?
\end{parts}
\capfig{0.3\textwidth}{figures/WorkEnergy/RocketCircle.png}{\label{fig:workenergy:RocketCircle} A rocket going around in a circle.}
\begin{finalanswer}
\begin{parts}
\part $\sqrt{\frac{4\pi R F\cos\theta}{m}}$
\part $\sqrt{\frac{4\pi Rm}{F\cos\theta}}$
\part $4\pi F\cos\theta+F\sin\theta$
\end{parts}
\end{finalanswer}

\begin{solution}
\textbf{a)} The component of the thrust tangent to the circle does work on the rocket. Since the force always makes an angle $\theta$ with the displacement, in one revolution, the total work done by the thrust is:
\begin{align*}
W_F=\int \vec F\cdot d\vec r=\int_0^{2\pi R} F\cos\theta dr=F(2\pi R)\cos\theta
\end{align*}
Since the rocket starts at rest, the change in kinetic energy is:
\begin{align*}
\Delta K &= \frac{1}{2}mv^2=W_F\\
\therefore v &= \sqrt{\frac{2W_F}{m}} = \sqrt{\frac{4\pi R F\cos\theta}{m}}
\end{align*}
Note that one can also easily solve this by using the tangential acceleration ($ma_T=F\cos\theta$), or torque ($\tau=F\cos\theta R=I\alpha$).

\textbf{b)} Again, there are many ways to solve this. If we use Newton's Second Law and 1D kinematics in the tangential direction (with constant acceleration):
\begin{align*}
ma_T=F\cos\theta\\
\therefore a_T=\frac{F\cos\theta}{m}
\end{align*}
If the initial position is $x=0$ and the final position is $x=2\pi R$:
\begin{align*}
x &= x_0 + v_0t+\frac{1}{2}a_Tt^2\\
\therefore t&=\sqrt{\frac{2(x-x_0)}{a_T}}\\
&= \sqrt{\frac{4\pi R}{a_T}}=\sqrt{\frac{4\pi Rm}{F\cos\theta}}
\end{align*}

\textbf{c)} In the radial direction, we only have the tension in the string and a component of the thrust. These sum to provide the centripetal force:
\begin{align*}
T-F\sin\theta&=m\frac{v^2}{R}\\
\therefore T&=m\frac{v^2}{R}+F\sin\theta\\
&=m\frac{4\pi R F\cos\theta}{mR}+F\sin\theta\\
&=4\pi F\cos\theta+F\sin\theta\\
\end{align*}
\end{solution}


\question A mass $m$ starts from rest and slides down an inclined plane that makes an angle $\theta$ with the horizontal (see Figure \ref{fig:workenergy:MassSlopeSpring}). The mass starts at a height $h$ above the ground. At the bottom of the ramp, the mass slides for a distance $d$ along the ground, before running into a spring and compressing it. The spring is non-linear and the restoring force is given by $F(x)=-k_1x-k_3x^3$, where $x$ is the amount that the spring is compressed relative to its rest position. The coefficient of kinetic friction between the mass and the inclined plane, and between the mass and the ground is $\mu_k$. 
\begin{parts}
\part Write an expression for the speed of the mass just before it makes contact with the spring.
\part What is the maximum value of the coefficient of kinetic friction that will allow the mass to make contact with the spring?
\part If instead, the coefficient of kinetic friction is zero everywhere, and the maximum compression of the spring is found to be $X$, what was the initial height of the block?
\end{parts}
\capfig{0.4\textwidth}{figures/WorkEnergy/MassSlopeSpring.png}{\label{fig:workenergy:MassSlopeSpring} A mass sliding down a slope.}
\begin{finalanswer}
\begin{parts}
\part $\sqrt{2g \left(h-h\frac{\mu_k}{\tan\theta}-d\mu_k\right)}$
\part $\frac{h}{\frac{h}{\tan\theta}+d}$
\part $\frac{\frac{1}{2}k_1X^2+\frac{1}{4}k_3X^4}{mg}$
\end{parts}
\end{finalanswer}
\begin{solution}
\textbf{a)} The work done by gravity ($W_G$, positive), friction along the ramp ($W_{f1}$, negative), and friction along the flat part ($W_{f2}$, negative) will equal the change in kinetic energy. We take the positive $x$ direction to be to the right, and the positive $y$ direction to be upwards.
\begin{align*}
W_G &= (-mg\hat j)\cdot(-h\hat j+\frac{h}{\sin\theta}\hat i)=mgh
\end{align*}
Along the incline, the magnitude of the normal force is given by $N=mg\cos\theta$. The force of friction is anti-parallel to the displacement down the ramp, and the work is thus:
\begin{align*}
W_{f1}= -f_1\frac{h}{\sin\theta}=-\mu_k mg\cos\theta\frac{h}{\sin\theta}=-mgh\frac{\mu_k}{\tan\theta}
\end{align*}
Along the straight part, the force of friction is $\mu_kmg$, and it is anti-parallel to the displacement, doing work
\begin{align*}
W_{f2}= -f_2d=\mu_kmgd
\end{align*}
The net work is thus:
\begin{align*}
W^{net}=W_G+W_{f1}+W_{f2}=mg \left(h-h\frac{\mu_k}{\tan\theta}-\mu_kd\right)
\end{align*}
This corresponds to the change in kinetic energy:
\begin{align*}
W^{net}&=\Delta K = \frac{1}{2}mv^2-0 \\
\therefore v&=\sqrt{\frac{2W^{net}}{m}} = \sqrt{2g \left(h-h\frac{\mu_k}{\tan\theta}-d\mu_k\right)} 
\end{align*}

\textbf{b)} The mass will not compress the spring if the speed of the mass is zero right when it touches the spring. We use the answer from above and set it to zero, giving:
\begin{align*}
\left(h-h\frac{\mu_k}{\tan\theta}-d\mu_k\right) &= 0\\
\therefore \mu_k = \frac{h}{\frac{h}{\tan\theta}+d}
\end{align*}

\textbf{c)} In this case, the only forces doing work are gravity and the spring. The work done by gravity is unchanged:
\begin{align*}
W_G &= (-mg\hat j)\cdot(-h\hat j+\frac{h}{\sin\theta}\hat i)=mgh
\end{align*}
The spring does negative work. If the spring compresses a distance $X$, it will perform work given by:
\begin{align}
W_S=\int_0^X (-k_1x-k_3x^3)dx=-\frac{1}{2}k_1X^2-\frac{1}{4}k_3X^4
\end{align}
The net work done by the spring and gravity is
\begin{align*}
W^{net}=W_G+W_S=mgh-\frac{1}{2}k_1X^2-\frac{1}{4}k_3X^4
\end{align*}
Since the mass starts at rest at the top of the slope and ends at rest at the bottom of the slope, the change in kinetic energy is zero. The net work must thus be zero:
\begin{align*}
W^{net} &= \Delta K=0\\
mgh-\frac{1}{2}k_1X^2-\frac{1}{4}k_3X^4&=0\\
\therefore h&=\frac{\frac{1}{2}k_1X^2+\frac{1}{4}k_3X^4}{mg}\\
\end{align*}


\end{solution}

%Olivia W
\question A constant force $\vec F$ is applied to an object. The object moves a total displacement $\vec d$ in the same direction as the force vector (Figure \ref{fig:workenergy:work1Dconstantsum}). 
\capfig{0.5\textwidth}{figures/WorkEnergy/work1Dconstantsum}{ \label{fig:workenergy:work1Dconstantsum} A force is applied to an object. The object's displacement $\vec d$ is divided into $N$ segments.}
Show that if you divided the displacement into $N$ segments of length $d_i$ (where $i=1,2,3,...,N$), the sum of the work done over each segment would be equal to the work done over the whole displacement $\vec d$. i.e. Show that
\begin{align*}
W=\sum_{i=1}^{N}\vec F \cdot \vec d_i
\end{align*}

\begin{solution}
We take the sum of the work done over each segment $\vec d_i$: 
\begin{align*}
\sum_{i=1}^{i=N} \vec F \cdot \vec d_i
\end{align*}
It sometimes helps to write out a few terms of the sum:
\begin{align*}
\sum_{i=1}^{i=N}\vec F \cdot \vec d_i
=\vec F \cdot \vec d_1+\vec F \cdot \vec d_2+\vec F \cdot \vec d_3+...+\vec F \cdot \vec d_N
\end{align*}
The scalar product (dot product) is distributive. This means that, since $\vec F$ is the same over each segment $d_i$, we can factor out $\vec F$:
\begin{align*}
\sum_{i=1}^{i=N}\vec F \cdot \vec d_i = \vec F \cdot (\vec d_1 + \vec d_2 + \vec d_3 +...+ \vec d_N)
\end{align*}
By design, the sum of $\vec d_1 + \vec d_2 + \vec d_3 +...+ \vec d_N$ is equal to $\vec d$, so we are left with:
\begin{align*}
\vec F \cdot (\vec d_1 + \vec d_2 + \vec d_3 +...+ \vec d_N) = \vec F \cdot \vec d
\end{align*}
which by definition is equal to the work done over the entire displacement $\vec d$:
\begin{align*}
\therefore W=\vec F \cdot \vec d=\sum_{i=1}^{N}\vec F \cdot \vec d_i
\end{align*}
\end{solution}

\question The vertical displacement of the centre of a trampoline can be modelled as a spring with spring constant $k=\SI{1e4}{N/m}$ that follows Hooke's Law ($F(x)=-kx$). If you ($m=\SI{50}{kg}$) leap (i.e. initial velocity of zero) onto the trampoline from a height $H=\SI{1.5}{m}$ above the surface of the trampoline, what is your total vertical displacement when you (momentarily) come to rest, when the trampoline is maximally compressed?
\begin{solution}
The only forces doing work on you are your weight (throughout the whole displacement), and the upwards force of the trampoline, once you make contact. Let $X$ be the distance by which the trampoline contracts, so that your total displacement is $X+H$.
The work done by gravity is (choosing positive $x$ downwards):
\begin{align*}
W_g=\vec F_g \cdot (H\hat x)+\vec F_g \cdot (X\hat x)=mg(H+X)
\end{align*}
since gravity is given by $\vec F_g=mg\hat x$. The work done by the spring is:
\begin{align*}
W_s=\int_0^X\vec F_s\cdot dx=\int_0^X -kx dx=-\frac{1}{2}kX^2
\end{align*}
Since you start at rest and end at rest, the net change in kinetic energy, and thus the net work done on you, is zero:
\begin{align*}
W^{net}&=W_g+W_s=0\\
mg(H+X) -\frac{1}{2}kX^2 &= 0\\
kX^2 - 2mgX - 2mgH &= 0\\
\end{align*}
We can solve the above quadratic equation for $X$:
\begin{align*}
\therefore X &= \frac{2mg \pm \sqrt{4m^2g^2+4k(2mgH)}}{2k}\\
&=\frac{mg \pm \sqrt{m^2g^2+2kmgH}}{k}\\
&=\frac{(\SI{50}{kg})(\SI{9.8}{m/s^2}) + \sqrt{(\SI{50}{kg})^2(\SI{9.8}{m/s^2})^2+2(\SI{1e4}{N/m})(\SI{50}{kg})(\SI{9.8}{m/s^2})(\SI{1.5}{m})}}{(\SI{1e4}{N/m})}\\
&=\SI{0.44}{m}
\end{align*}
Only the solution with the positive sign gives a positive value for $X$. Your total displacement is thus:
\begin{align*}
d &= H+X=(\SI{1.5}{m})+(\SI{0.44}{m})=\SI{1.94}{m}
\end{align*}

\end{solution}


%Joshua Rinaldo (clean up the solution for this)
\question Oh no, you've created a positive feedback loop! You've synthesized a material with a coefficient of friction $\mu = -0.2$ and constructed a system with two springs of $k = (100+60x)$N/m which are a distance $d = \SI{0.5}{m}$ from one another when uncompressed. If the springs are compressed a distance greater than $x_{break} = \SI{1.5}{m}$, they will snap and the system will break. The space between the two springs is covered in your $\mu = -0.2$ substance, but the space underneath the springs is frictionless, as shown in Figure \ref{fig:workenergy:feedbackloop}. Motion begins in the system when a block of mass $m = \SI{12}{kg}$ is fixed at a compressed distance of $x = \SI{0.5}{m}$ on one of the springs, then released. How many times will the block move from one spring to the other before the system breaks?
\capfig{0.5\textwidth}{figures/WorkEnergy/feedbackloop}{ \label{fig:workenergy:feedbackloop} A positive feedback loop.}
\begin{finalanswer}
	The block will move between springs 15 times before snapping a spring.
\end{finalanswer}
\begin{solution}
First, we must determine the initial energy stored in the compressed spring before the block is released:
\begin{align*}
W &= \int_{0}^{x}Fdx\\
&= \int_{0}^{0.5}(100x+60x^2)dx\\
&= (50x^2+15x^4)|_0^{0.5}\\
&= 12.5+0.9375\\
W&=\SI{13.4375}{J}
\end{align*}
Next, we will calculate the amount of energy required for the block to compress a spring to such an extend that it snaps.
\begin{align*}
W &= \int_{0}^{x}Fdx\\
&= \int_{0}^{1.5}(100x+60x^2)dx\\
&= (50x^2+15x^4)|_0^{1.5}\\
&= 112.5+75.9375\\
W&=\SI{188.4375}{J}
\end{align*}	
We now know that the block begins with \SI{13.4375}{J} and will break the spring it comes into contact with when it has \SI{188.43750}{J} or more. Now, we must calculate the distance that the block must travel over the $\mu = -0.2$ substance to gain the amount of energy needed to snap a spring.
\begin{align*}
W_{substance}& = -\mu mgd\\
\SI{188.4375}{J}-\SI{13.4375}{J}&= -(-0.2)(\SI{12}{kg})(\SI{9.81}{m/s^2})(d)\\
\SI{175}{J} &= (\SI{23.544}{N})d\\
d&=\frac{\SI{175}{J}}{\SI{23.544}{N}}\\
d&=7.4323 
\end{align*}
Now that we have found the total distance the block will travel before it is capable of snapping a spring, we can simply find how many movements from one spring to another this equates to.
\begin{align*}
n&=\frac{7.4323}{0.5}\\
n&=14.8646
\end{align*}
The block will become capable of snapping a spring during its 15th movement, which means that it will cross the $\mu = -0.2$ substance 15 times before snapping a spring and breaking the system.
\end{solution}



%Olivia W
%\question A \SI{200000}{kg} plane falls straight down from a height of 35000 feet. Superman flies to the plane as fast as he can and catches it when it is 33000 feet in the air. He applies a constant upwards force and manages to stop the plane when it is a mere 10 feet from the ground. Quantify Superman's power in watts.  
