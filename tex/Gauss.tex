
\chapter{Gauss' Law}
\label{chapter:gauss}
In this chapter, we take a detailed look at Gauss' Law applied in the context of the electric field. We already covered Gauss' Law briefly in Section \ref{sec:gravity:gauss} when we examined the gravitational field. Since the electric force is mathematically identical to the gravitational force, we can apply the same tools to model the electric field as we do the gravitational field. This chapter provided an in depth description of Gauss' Law in the context of the electric field. 

\begin{learningObjectives}{
 \item Understand the concept of flux for a vector field.
 \item Understand how to apply Gauss' Law quantitatively to determine an electric field.
 \item Understand how to apply Gauss' Law qualitatively to discuss charges on a conductor.
 }
\end{learningObjectives}

\begin{opening}
\begin{MCquestion}{A neutral spherical conducting shell encloses a point charge, $Q$, located at the centre of the shell. Due to separation of charge, the outer surface of the shell will acquire a net positive charge. What is the magnitude of that charge? }
\item less than $Q$.
\item exactly $Q$. \correct
\item more than $Q$.
\end{MCquestion}
\end{opening}

\section{Flux of the electric field.}
Gauss' Law can be expressed as a relation between the ``flux'' of the electric field through a closed surface and the charge that is enclosed by that surface. We start by examining the concept of flux. 

Flux is always defined based on:
\begin{itemize}
\item A surface.
\item A vector field (e.g. the electric field). 
\end{itemize}
and can be thought of as a measure of many field lines cross the surface. For that reason, one usually refers to the ``flux of the electric field through a surface''. This is illustrated in Figure \ref{fig:gauss:fluxangle} for a uniform horizontal electric field, and a flat surface, whose normal vector, $\vec A$, is shown. If the surface is parallel to the field (right panel, with the normal vector of the surface parallel to the field vector), then no field lines cross that surface and the flux through that surface is zero. If the surface is perpendicular to the field (left panel), then the flux through that surface is maximal. If the surface is rotated with respect to the electric field, as in the middle panel, then the flux through the surface is between zero and the maximal value.
\capfig{0.9\textwidth}{figures/Gauss/fluxangle.png}{\label{fig:gauss:fluxangle} Flux of an electric field through a surface that makes different angles with respect to the electric field. In the rightmost panel, there are no field lines crossing the surface, so the flux through the surface is zero. In the leftmost panel, the surface is oriented such that the flux through it is maximal.}
%TODO: I think we can improve this figure, making the area element more clearly a plane in 3D. 
We can define a vector, $\vec A$, associated with the surface such that the magnitude of $\vec A$ is equal to the area of the surface, and the direction of $\vec A$ is such that it is perpendicular to the surface. We define the flux, $\Phi_E$, of the electric field, $\vec E$, through the surface represented by vector, $\vec A$, as:
\begin{align*}
\Phi_E=\vec E\cdot \vec A=EA\cos\theta
\end{align*}
since this will have the same properties that we described above (e.g. no flux when $\vec E$ and $\vec A$ are perpendicular, flux proportional to number of field lines crossing the surface). Note that the flux is only defined up to an overall sign, as there are two possible choices for the direction of the vector $\vec A$, since it is only required to be perpendicular to the surface. By convention, we usually choose $\vec A$ so that the flux is positive, when the surface is ``open''. If the surface is ``closed'' (see Section \ref{sec:gauss:closedsurfaces} below), then the vector $\vec A$ is always defined to be normal to the surface and point outwards from the surface.

%TODO: Checkpoint: MC question about the units of electric flux.
\begin{example}{An electric field is uniform and given by, $\vec E=E\cos\theta\hat x+E\sin\theta\hat y $ throughout space. A rectangle is defined by the four points $(0,0,0)$, $(0,0,H)$, $(L,0,0)$, $(L,0,H)$. What is the flux of the electric field through the surface?}
The surface that is defined corresponds to a rectangle in the $xz$ plane with area $A=LH$. Since the rectangle lies in the $xz$ plane, a vector perpendicular to the surface will be along the $y$ direction. We choose the positive $y$ direction, since this will give a positive number for the flux (as the electric field has a positive component in the $y$ direction). The vector $\vec A$ is given by:
\begin{align*}
\vec A =A\hat y=LH\hat y
\end{align*}
The flux through the surface is thus given by:
\begin{align*}
\Phi_E&=\vec E\cdot \vec A=(E\cos\theta\hat x+E\sin\theta\hat y)\cdot(LH\hat y)\\
&=ELH\sin\theta
\end{align*}
where one should note that the angle $\theta$, in this case, is not the angle between $\vec E$ and $\vec A$, but rather the complement of the angle. 

\textbf{Discussion:} In this example, we calculated the flux of a uniform electric field through a rectangle of area $LH$. Since we knew the components of both the electric field vector, and the area vector, $\vec A$, we were easily able to take the scalar product to determine the flux. In some cases, it is easier to work with the magnitude of the vectors and the angle between them to determine the scalar product (although note that in this example, the angle between $\vec E$ and $\vec A$ is $\SI{90}{\degree}-\theta$.
\end{example}


\subsection{Non-uniform fields}
So far, we considered the flux of a uniform electric field, $\vec E$, through a surface, $S$, described by a vector, $\vec A$. In this case, the flux, $\Phi_E$, is given by:
\begin{align*}
\Phi_E=\vec E\cdot \vec A
\end{align*}
However, if the electric field is not constant in magnitude and/or in direction over the entire surface, then the flux must be integrated:
\begin{align*}
\Aboxed{\Phi_E=\int \vec E\cdot d\vec A}
\end{align*}
where $d\vec A$ is a vector representing an infinitesimal surface. That is, if the electric field is not uniform over the surface, then we subdivide the surface into infinitesimally small ``surface elements'', $dS$, each represented by an infinitesimal vector, $d\vec A$. This is illustrated in Figure \ref{fig:gauss:fluxdA}, which shows, in the left panel, a surface for which the electric field changes magnitude along the surface (as the field lines are closer in the lower left part of the surface), and, in the right panel, a scenario in which the direction (and magnitude) vary along the surface.

In order to calculate the flux through the total surface, we first calculate the flux through an infinitesimal surface, over which we assume that $\vec E$ is constant in magnitude and direction, and then, sum (integrate) the flux from all of the infinitesimal surfaces together.. Remember, the flux through a surface is related to the number of field lines that cross that surface; it is thus natural to divide the surface $S$ into smaller surfaces $dS$ and count the field lines (calculate the flux) through each $dS$ and then add them together to get the total number of field lines through $S$.
\capfig{0.9\textwidth}{figures/Gauss/fluxdA.png}{\label{fig:gauss:fluxdA} Examples of surfaces that need to be sub-divided in order to determine the net flux through them. The surface on the left must be subdivided because the electric field changes magnitude over the surface, whereas the one on the right needs to be subdivided because the angle between $\vec E$ and $d\vec A$ is not constant (and the magnitude of $\vec E$ also changes along the surface).}
\begin{example}{An electric field points in the $z$ direction everywhere in space. The magnitude of the electric field depends linearly on $x$, the $x$ position, so that the electric field vector is given $\vec E=(a-bx)\hat z$, where $a$ and $b$ are constants. What is the flux of the electric field through a square of side $L$ that is located in the $xy$ plane?}
We need to calculate the flux of the electric field through a square of side $L$ in the $xy$ plane. The electric field is always in the $z$ direction, so the angle between $\vec E$ and $d\vec A$ (the normal vector for any infinitesimal area element) will remain constant. We canc calculate the flux through the square by dividing up the square into thin strips of length $L$ in the $y$ direction and infinitesimal width $dx$ in the $x$ direction, as illustrated in Figure \ref{fig:gauss:fluxlinx}. In this case, because the electric field does not change with $y$, the dimension of the infinitesimal area element is finite ($L$). If the electric field varied both as a function of $x$ and $y$, we would start with area elements that have infinitesimal dimensions in both the $x$ and the $y$ directions. 
\capfig{0.5\textwidth}{figures/Gauss/fluxlinx.png}{\label{fig:gauss:fluxlinx} Dividing a square in the $xy$ plane into thin strips of length $L$ and width $dx$.}
%TODO The above figure can be better, probably much better!
As illustrated in Figure \ref{fig:gauss:fluxlinx}, we first calculate the flux through a thin strip of area $dA=Ldx$ located at position $x$ along the $x$ axis, and then sum the fluxes through each strip. Choosing $d\vec A$ in the direction to give a positive flux, the flux through the strip that is illustrated is given by:
\begin{align*}
d\Phi_E=\vec E\cdot d\vec A=EdA=(ax-b)Ldx
\end{align*}
where $\vec E\cdot d\vec A=EdA$, since the angle between $\vec E$ and $\vec A$ is zero. Summing together the fluxes from the strips from $x=0$ to $x=L$, the total flux is given by:
\begin{align*}
\Phi_E=\int d\Phi_E=\int_0^L(ax-b)Ldx=\frac{1}{2}aL^3-bL^2
\end{align*}
\textbf{Discussion:} In this example, we showed how to calculate the flux from an electric field that changes magnitude with position. We modelled a square of side, $L$, as being made of many thin strips of length, $L$, and width, $dx$. We then calculated the flux through each strip and added those together to obtain the total flux through the square.
\end{example}
\subsection{Closed surfaces}
\label{sec:gauss:closedsurfaces}
Gauss' Law is a relation between the flux of a field through a closed surface and the amount of charge enclosed by that surface. A surface is closed if it has a clear ``inside'' and an ``outside'', or if it could be a container with no holes that can hold a liquid. Any surface that completely encloses a volume is considered to be ``closed''. For example, the surface of a sphere, of a cube, or of a cylinder are all examples of closed surfaces. A plane, a triangle, and a disk are, on the other hand, examples of ``open surfaces''. The main difference between a closed and an open surface is that, for the closed surface, one can unambiguously define the direction of the vector $\vec A$ (or $d\vec A$) as the direction that it is perpendicular to the surface and points towards the outside. Thus, the sign of the flux is meaningful with a closed surface. The flux will be positive if there is a net number of field lines exiting the surface (since $\vec E$ and $\vec A$ will be parallel on average) and the flux will be negative if there is a net number of field lines entering the surface (as $\vec E$ and $\vec A$ will be anti-parallel on average).

When calculating the flux over a closed surface, we use a different integration symbol so show that the surface is closed:
\begin{align*}
\Phi_E=\oint \vec E\cdot d\vec A
\end{align*}
which is the same integration symbol that we used for indicating a path integral when the initial and final points are the same (see for example Section \ref{sec:potentialecons:conservative}.

%TODO Checkpoint MC: show a weird closed surface with weird field lines coming in and back out (not net enclosed charge), and ask if the net flux is positive, 0(correct), or negative. 

\begin{example}{A negative electric charge, $-Q$, is located at the origin of a coordinate system. Calculate the flux of the electric field through a spherical surface of radius, $R$, that is centred at the origin.}\label{ex:gauss:qsphere}
Figure \ref{fig:gauss:fluxsphere} shows the spherical surface of radius, $R$, centred on the origin where the charge $-Q$ is located.
\capfig{0.3\textwidth}{figures/Gauss/fluxsphere.png}{\label{fig:gauss:fluxsphere} Calculating the flux through a spherical surface.}
At all points along the surface, the electric field has the same magnitude:
\begin{align*}
E=\frac{1}{4\pi\epsilon_0}\frac{Q}{R^2}
\end{align*}
as given by Coulomb's law for a point charge. Although the vector $\vec E$ changes direction everywhere along the surface, it always makes the same angle (\SI{-180}{\degree}) with the corresponding vector $d\vec A$ at any particular location. Indeed, for a point charge, the electric field point in the radial direction (inwards for a negative charge) and is thus perpendicular to the spherical surface at all points. Since the surface is closed, the vector, $d\vec A$, points outwards anywhere on the surface. Thus, at any point on the surface, we can evaluate the flux through an infinitesimal area element, $d\vec A$:
\begin{align*}
d\Phi_E=\vec E\cdot d\vec A=EdA\cos(\SI{-180}{\degree})=-EdA
\end{align*}
where the overall minus sign comes from the fact that $\vec E$ and $d\vec A$ are anti-parallel. The total flux through the spherical surface is obtained by summing together the fluxed through each area element:
\begin{align*}
\Phi_E=\oint d\Phi_E=\oint -EdA=-E\oint dA=-E(4\pi R^2)
\end{align*}
where we factored $E$ out of the integral, since the magnitude of the electric field is constant over the entire surface (a constant distance $R$ from the charge). In the last equality, we recognized that $\oint dA$ simply means ``sum together all of the areas, $dA$, of the area elements'', which gives the total surface area of the sphere, $4\pi R^2$. The flux through the spherical surface is negative, because the charge is negative, and the field lines point towards $-Q$.

Using the value that we obtained for the magnitude of the electric field from Coulomb's Law, the total flux is given by:
\begin{align*}
\Phi_E=-E(4\pi R^2)=-\frac{1}{4\pi\epsilon_0}\frac{Q}{R^2}(4\pi R^2)=-\frac{Q}{\epsilon_0}
\end{align*}
which, surprisingly, is independent of the radius of the spherical surface. Note that we used $\epsilon_0$ instead of Coulomb's constant, $k$, since the result is cleaner without the extra factor of $4\pi$. 

\textbf{Discussion: }In this example, we calculated the flux of the electric field through a spherical surface concentric with the charge. We found the flux to be negative, which makes sense, since the field lines go towards a negative charge there is thus a net number of field lines entering the surface. Perhaps surprisingly, we found that the total flux through the surface does not depend on the radius of the surface! In fact, that statement is precisely Gauss' Law: the net flux out of a closed surface depends only on the amount of charge enclosed by that surface (and the constant, $\epsilon_0$). Gauss' Law is of course more general, and applies to surfaces of any shape, as well as charges of any shape (whereas Coulomb's Law only holds for point charges). 
\end{example}

\section{Gauss' Law}
Gauss' Law is a relation between the net flux through a closed surface and the amount of charge, $Q^{enc}$, in the volume enclosed by that surface:
\begin{align*}
\Aboxed{\oint \vec E\cdot d\vec A=\frac{Q^{enc}}{\epsilon_0}}
\end{align*}
In particular, note that Gauss' Law holds true for \textbf{any} closed surface, and the shape of that surface is not specified in Gauss' Law. That is, we \textbf{can always choose the surface to use} when calculating the flux. For obvious reasons, we often call the surface that we choose a ``gaussian surface''. But again, this surface is simply a mathematical tool, there is no actual property that makes a surface ``gaussian''; it simply means that we chose that surface in order to apply Gauss' Law. In Example \ref{ex:gauss:qsphere} above, we confirmed that Gauss' Law is compatible with Coulomb's Law for the case of a point charge and a spherical gaussian surface. 

Primarily, Gauss' Law is a useful tool to determine the magnitude of the electric field from a given charge, or charge distribution. We usually have to use symmetry to determine the direction of the electric field vector. In general, the integral for the flux is difficult to evaluate, and Gauss' Law can only be used analytically in cases with a high degree of symmetry. Specifically, the integral is easiest to evaluate if:
\begin{enumerate}
\item \textbf{The electric field makes a constant angle with the surface}. When this is the case, the scalar product can be written in terms of the cosine of the angle between $\vec E$ and $d\vec A$, which can be taken out of the integral if it is constant:
\begin{align*}
\oint \vec E\cdot d\vec A=\oint E\cos\theta dA=\cos\theta\oint EdA
\end{align*}
Ideally, one has chosen a surface such that this angle is $0$ or $\SI{180}{\degree}$.
\item \textbf{The electric field is constant in magnitude along the surface}. When this is the case, the integral can be simplified further by factor out $E$, and simply becomes an integral over $dA$ (which corresponds to the total area of the surface, $A$):
\begin{align*}
\oint \vec E\cdot d\vec A=\cos\theta\oint EdA =E\cos\theta\oint dA=EA\cos\theta 
\end{align*}
\end{enumerate}
Ultimately, the points above should dictate the choice of gaussian surface \textbf{so that} the integral for the flux is easy to evaluate. The choice of surface will depend on the symmetry of the problem. For a point (or spherical) charge, it makes sense to use a spherical gaussian surface. For a line of charge, as we will see, it makes sense to use a cylindrical surface. The steps for applying Gauss' Law are as follows:
\begin{enumerate}
\item Make a diagram showing the charge distribution.
\item Use symmetry argument to determine in which way the electric field vector points.
\item Choose a gaussian surface that, ideally, is always perpendicular or parallel to the electric field, and along which the electric field has a constant magnitude. This will make the flux easy to calculate.
\item Calculate the flux,$\oint \vec E\cdot d\vec A$.
\item Calculate the amount of charge in the volume enclosed by the surface, $Q^{enc}$.
\item Apply Gauss' Law, $\oint \vec E\cdot d\vec A=\frac{Q^{enc}}{\epsilon_0} $.
\end{enumerate}
\begin{example}{An insulating sphere of radius, $R$, contains a total charge, $Q$, which is uniformly distributed through out its volume. Determine an expression for the electric field as a function of distance, $r$, from the centre of the sphere.}
Note that this is identical, mathematically, as the derivation that is done in Section \ref{sec:gravity:gauss} for the case of gravity. 

When applying Gauss' Law, we first need to think about symmetry in order to determine the direction of the electric field vector. We also need to think about all possible regions of space in which we need to determine the electric field. In particular, for this case, we need to determine the electric field both inside ($r\leq R$) and outside ($r\geq R$) of the charged sphere.

Figure \ref{fig:gauss:spheresymmetry} shows the charged sphere of radius $R$. If we consider the direction of the electric field outside the sphere (where $\vec E_{out}$ is drawn), we realize that it can only point in the radial direction (towards or away from the centre of the sphere), as this is the only choice that preserves the symmetry of the sphere. Being a sphere, the charge looks the same from all angles; thus, the electric field must also look the same from all angles, otherwise, there would be a preferred orientation for the sphere. The same argument holds for the electric field vector inside the sphere (drawn as $\vec E_{in}$). 
\capfig{0.4\textwidth}{figures/Gauss/spheresymmetry.png}{\label{fig:gauss:spheresymmetry} For a spherical charge distribution, the electric field inside and outside must point in the radial direction, by symmetry.}

Now that we have determined the direction of the electric field, we can apply Gauss' Law. Next, we need to choose a gaussian surface that will make the flux integral easy to evaluate. Namely, we ideally need to find a surface over which the electric field makes the same angle and over which the electric field is constant in magnitude. Again, based on the symmetry of the charge distribution, it is clear that a spherical surface of radius, $r$, will satisfy these properties. We start by applying Gauss' Law outside the charge (with $r\geq R$) to determine the electric field, $\vec E_{out}$. Figure \ref{fig:gauss:spherein} shows our choice of spherical gaussian surface (labelled $S$) of radius, $r$, which is concentric with the spherical charge distribution of radius, $R$, and total charge, $+Q$.
\capfig{0.4\textwidth}{figures/Gauss/spherein.png}{\label{fig:gauss:spherein} A spherical gaussian surface to determine the electric field outside a sphere of radius, $R$, holding charge, $+Q$.}

In order to apply Gauss' Law, we need to calculate:
\begin{itemize}
\item the net flux through the surface.
\item the charge in the volume enclosed by the surface. 
\end{itemize}
The net flux through the surface is found identically as in Example \ref{ex:gauss:qsphere}, and is given by:
\begin{align*}
\Phi_E&=\oint \vec E\cdot d\vec A=\oint E dA= E\oint dA=E(4\pi r^2)
\end{align*}
where our choice of spherical surface led to $\vec E\cdot d\vec A=EdA$, since $\vec E$ and $d\vec A$ are always parallel. Furthermore, by symmetry, the electric field must be constant in magnitude along the whole surface, which allowed us to factor the $E$ out of the integral, leaving us with $\oint dA$ which is simply the area of the gaussian spherical surface, $r4\pi r^2$.

The gaussian surface with $r\geq R$ encloses the whole charged sphere, so the charge enclosed is simply the charge of the sphere, $Q^{inc}=Q$. Applying Gauss' Law allows us to determine the magnitude of the electric field:
\begin{align*}
\oint \vec E\cdot d\vec A&=\frac{Q^{enc}}{\epsilon_0} \\
E(4\pi r^2) &= \frac{Q^{enc}}{\epsilon_0}\\
\therefore E&= \frac{1}{4\pi\epsilon_0}\frac{Q}{r^2}
\end{align*}
which is the same as the electric field a distance $r$ from a point charge. 

Next, we determine the magnitude of the electric field inside the charged sphere. In this case, we choose a spherical gaussian surface of radius $r\leq R$, that is concentric with the sphere, as illustrated by the surface labelled, $S$, that is shown in Figure \ref{fig:gauss:sphereout}.
\capfig{0.4\textwidth}{figures/Gauss/sphereout.png}{\label{fig:gauss:sphereout} A spherical gaussian surface to determine the electric field outside a sphere of radius, $R$, holding charge, $+Q$.}
The flux integral is trivial again, since the electric field always makes the same angle with the gaussian surface, and the magnitude of the electric field is constant in magnitude along the surface:
\begin{align*}
\Phi_E&=\oint \vec E\cdot d\vec A=\oint E dA= E\oint dA=E(4\pi r^2)
\end{align*}
In this case, however, the charge in the volume enclosed by the gaussian surface is less than $Q$, since the whole charge is not enclosed. We are told that the charge is distributed uniformly throughout the spherical volume of radius $R$. We can thus define a volume charge density, $\rho$, (charge per unit volume) for the sphere:
\begin{align*}
\rho=\frac{Q}{V}=\frac{Q}{\frac{4}{3}\pi R^3}
\end{align*}
The volume enclosed by the gaussian surface is $\frac{4}{3}\pi r^3$, thus, the charge, $Q^{enc}$, contained in that volume is given by:
\begin{align*}
Q^{enc}=\frac{4}{3}\pi r^3 \rho=\frac{4}{3}\pi r^3 \frac{Q}{\frac{4}{3}\pi R^3}=Q\frac{r^3}{R^3}
\end{align*}
Finally, we apply Gauss' Law to find the magnitude of the electric field inside the sphere:
\begin{align*}
\oint \vec E\cdot d\vec A&=\frac{Q^{enc}}{\epsilon_0} \\
E(4\pi r^2) &= \frac{Q^{enc}}{\epsilon_0}=\frac{Q}{\epsilon_0}\frac{r^3}{R^3}\\
\therefore E&= \frac{Q}{4\pi\epsilon_0R^3}r
\end{align*}
Note that the electric field increases linearly with radius inside of the charge sphere, and then decreases with radius squared outside of the sphere. Also, note that at the centre of the sphere, the electric field has a magnitude of zero, as expected from symmetry.

\textbf{Discussion: } In this example, we showed how to use Gauss' Law to determine the electric field inside and outside of a uniformly charged sphere. We recognized the spherical symmetry of the charge distribution and chose to use a spherical surface in order to apply Gauss' Law. This, in turn, allowed the flux to be easily calculated. We found that outside the sphere, the electric field decreases in magnitude with radius squared, just as if the entire charge were contained in a point. Inside the sphere, we found that the electric field is zero at the centre, and increases linearly with radius. 
\end{example}

\begin{example}{An infinitely long straight wire carries a uniform charge per unit length, $\lambda$. What is the electric field at a distance, $R$, from the wire?}

\end{example}

%TODO Checkpoint question, show a picture of charged spherical shell and ask what the electric field is in the centre.

\section{Application of Gauss Law}
Describe charges on a conductor, etc.
\subsection{Interpretation of Gauss' Law and vector calculus}
Discuss how Gauss law is about divergence, which is a local property of that field. 

\newpage
\section{Summary}

\begin{chapterSummary}
 Something that was learned
\end{chapterSummary}

\newpage
\begin{importantEquations}
\medskip
\begin{multicols}{2}
\textbf{Momentum of a point particle:}
\begin{align*}
\vec p = m\vec v \\
\frac{d}{dt}\vec p = \sum \vec F = \vec F^{net}
\end{align*}
\columnbreak
\\
\textbf{Position of the Centre of Mass \\ of a system:}
\begin{align*}
\vec r_{CM} &=\frac{1}{M}\sum_i m_i\vec r_i 
\end{align*}
\medskip
\end{multicols}
\end{importantEquations}

\newpage
\section{Thinking about the material}

\begin{chapteractivity}{Reflect and research}
{
\item Explain
}
\end{chapteractivity}

\begin{chapteractivity}{To try at home}
{
\item Try
}
\end{chapteractivity}

\begin{chapteractivity}{To try in the lab}
{
\item Propose an experiment
}
\end{chapteractivity}

\newpage
\section{Sample problems and solutions}
\subsection{Problems}
\begin{problem}{soln:template:ballistic}{\label{prob:template:ballistic} 

}
\end{problem}

\newpage
\subsection{Solutions}
\begin{solution}{prob:template:ballistic}\label{soln:template:ballistic}

\end{solution}

