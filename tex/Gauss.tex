\section{Gauss' Law}

%%%%%%%%%%%%%%%%%%%%%%%%%%%%%%%%%%%
%%
%% Multiple Choice
%%
%%%%%%%%%%%%%%%%%%%%%%%%%%%%%%%%%%%
\subsection{Multiple Choice}


\question Which of the following closed surfaces will have the greatest net electric flux coming out of it?
\begin{checkboxes}
\choice A sphere of radius $R$ with a point charge $Q$ at its centre.
\choice A sphere of radius $R$ that is concentric with a uniformly charged sphere of radius $r<R$ carrying charge $Q$
\choice A sphere of radius $2R$ with a point charge $\frac{Q}{2}$ at its centre.
\CorrectChoice A sphere of radius $\frac{R}{2}$ with a point charge $2Q$ at its centre. \correct
\end{checkboxes}

\question A spherical metallic shell with outer radius $R$ encloses a positive point charge $+Q$ at its centre. Which one of the following statements is true?
\begin{choices}
	\choice The outer surface of the shell has a net positive charge $+Q$.
	\choice The electric field inside the shell is zero.
	\choice The electric field at the surface of the shell has magnitude $E=\frac{kQ}{R^2}$.
	\CorrectChoice All of the above are true.
	\choice None of the above are true.
\end{choices}

\question A sphere of radius $r$ has a net charge $+Q$. A spherical metallic shell with no net charge on it and of inner radius $R_1>r$ and outer radius $R_2$ is concentric with the small sphere. The outer surface of the metallic shell
\begin{checkboxes}
\choice is neutral
\choice is negatively charged
\CorrectChoice is positively charged \correct
\end{checkboxes}

%Based on Allyson Smith
\question A positive charge $+Q$ and a negative charge $-2Q$ are placed a certain distance apart from each other. A spherical gaussian surface is centred about the positive charge such that the negative charge is outside of the sphere. What is the total flux of the electric field through the surface?
\begin{checkboxes}
	\choice Negative
	\choice Zero
	\CorrectChoice Positive
	\choice Not enough information to tell
\end{checkboxes}

%Based on Allyson Smith
\question A positive charge $+2Q$ and a negative charge $-Q$ are placed a certain distance apart from each other. A spherical gaussian surface is centred about the negative charge such that the positive charge is outside of the sphere. What is the total flux of the electric field through the surface?
\begin{checkboxes}
	\CorrectChoice Negative
	\choice Zero
	\choice Positive
	\choice Not enough information to tell
\end{checkboxes}


\question An amount of charge $q$ is placed on a solid conducting sphere of radius $r$. What is the magnitude of the electric field at the centre of the sphere?
\begin{checkboxes}
\choice $\frac{q}{4\pi\epsilon_0 r}$
\choice $\frac{q}{4\pi\epsilon_0 r^2}$
\CorrectChoice Zero \correct
\choice None of the above
\end{checkboxes}

%Question submitted by Joanna Fu
\question A cylindrical Gaussian surface with length $L$ and radius $r$ surrounds an infinite line of positive charge, such that the axis of cylinder is colinear with the liner of charge. The flux $\Phi$ through the surface of the cylinder is:
\begin{checkboxes}
\CorrectChoice $2\pi rLE$ \correct
\choice $\pi r^2LE$
\choice $2\pi r(L+r)E$
\choice 0
\choice None of the above
\end{checkboxes}

%Question submitted by Madison Facchini
\question What statement is not true of conductors in electrostatic equilibrium?
\begin{checkboxes}
\choice The electric field inside a conductor is zero
\CorrectChoice The electric field is parallel to the surface of a conductor everywhere on that surface \correct
\choice Any excess charge placed on a conductor resides entirely on the surface of the conductor
\choice All of these statements are true
\end{checkboxes}

\question An electric dipole is enclosed by a Gaussian sphere. Under what conditions will the electric flux be zero? 
\begin{checkboxes}
\choice When the Gaussian sphere is centred on the dipole
\choice When the Gaussian sphere is not centred on the dipole
\choice When the Gaussian sphere is very large
\choice When the Gaussian sphere is very small (but still encloses the dipole)
\CorrectChoice All of the above \correct
\end{checkboxes}


\question If a charge $q$ is placed in the corner of a cube, how many of the six cube faces have a non-zero flux?
\begin{checkboxes}
\choice All Six
\choice Two
\CorrectChoice Three \correct
\choice All are zero
\end{checkboxes}


%Submitted by Jenna Vanker
\question Suppose you have a uniform electric field going through a plane surface of \SI{8.0}{m^2}. If it has a magnitude of \SI{14}{N/C} and the plane in inclined \SI{20}{\degree} relative to the electric field what would be the electric flux through the plane surface?
\begin{checkboxes}
\choice $\SI{22.7}{Nm^2/C} $
\CorrectChoice $\SI{105}{Nm^2/C} $ \correct
\choice $\SI{98.3}{Nm^2/C} $
\choice $\SI{120}{Nm^2/C} $
\end{checkboxes}

%Jonathan Abott
\question A point particle with charge $q$ is placed inside a cube but not at its centre. There are no other charges near the cube. The electric flux through any one side of the cube is:
\begin{choices} 
\choice Zero
\choice $\frac{q}{\epsilon_0}$
\choice $\frac{q}{6\epsilon_0}$
\CorrectChoice None of the above \correct
\end{choices}

\question An electric dipole is placed at the origin of a coordinate system. A spherical surface of radius $R$ is such that it encloses the dipole completely. Which statement is true?
\begin{checkboxes}
\choice There is a net electric flux into the surface
\choice The electric field magnitude is constant along the surface.
\CorrectChoice The net charge enclosed by the surface is zero. \correct
\choice There is a net electric flux out of the surface
\end{checkboxes}

%%%%%%%%%%%%%%%%%%%%%%%%%%%%%%%%%%%
%
% long answer
%
%%%%%%%%%%%%%%%%%%%%%%%%%%%%%%%%%%%
\subsection{Long answers}
%Giancolli -fixed
\question Suppose a cube with a side length of $L$ lies on a uniform field of magnitude $E$ such that the cube's edges align with the field lines. 
\begin{parts}
\part What is the net flux through the cube?
\part What is the flux through each of its six faces?
\part Now, imagine that we shut off the uniform field, and instead place a charge $Q$ at the center of the cube. What is the flux through one face of the cube?
\end{parts}
\begin{finalanswer}
\begin{enumerate}[(a)]
\item Zero.
\item The two faces that are perpendicular to the field will have a non-zero flux of $\Phi = EL^2$ and $\Phi = -EL^2$. The other faces will have a flux of zero. 
\item \begin{align*}
\Phi=\frac{1}{6}\frac{Q}{\epsilon_0}
\end{align*}
\end{enumerate}
\end{finalanswer}
\begin{solution}
\begin{parts}
\part Since there is no enclosed charge, the net flux out of the cube is zero.
\part The two faces that are perpendicular to the field will have a non-zero flux of $\Phi = EL^2$ and $\Phi = -EL^2$. The other faces will have a flux of zero. 
\part By symmetry each face will have the same flux going through it, corresponding to one sixth of the total flux:
\begin{align*}
\Phi=\frac{1}{6}\frac{Q}{\epsilon_0}
\end{align*}
\end{parts}
\end{solution}


\question A long cylindrical rod of radius $a$ is made from an insulating material with a uniform charge density $\rho$. Find the electric field a distance $r$ from the centre of the rod for
\begin{parts}
\part $0<r<a$
\part $r>a$
\end{parts} 
\begin{finalanswer}
\begin{enumerate}[(a)]
\item \begin{align*}
E=\frac{\rho r}{2\epsilon_0}
\end{align*}
\item \begin{align*}
E=\frac{\rho a^2}{2\epsilon_0r}
\end{align*}
\end{enumerate}
\end{finalanswer}
\begin{solution}
\begin{parts}
\part We consider a gaussian surface or radius $r$ and length $L$. The amount of charge enclosed is:
\begin{align*}
Q^{enc}=\rho \pi r^2 L
\end{align*}
The electric field is always perpendicular to the curved surface and parallel to the ends of the cylinder. The flux is thus given by:
\begin{align*}
\Phi=E2\pi r L
\end{align*}
Applying Gauss' Law:
\begin{align*}
\Phi&=\frac{Q^{enc}}{\epsilon_0}\\
E2\pi r L &= \frac{\rho \pi r^2 L}{\epsilon_0}\\
\therefore E &= \frac{\rho r}{2\epsilon_0}
\end{align*}
\part We repeat the same procedure, but note that the charge enclosed is different for a gaussian surface of radius $r>a$:
\begin{align*}
Q^{enc}=\rho \pi a^2 L
\end{align*}
The electric field is always perpendicular to the curved surface and parallel to the ends of the cylinder. The flux is thus given by:
\begin{align*}
\Phi=E2\pi r L
\end{align*}
Applying Gauss' Law:
\begin{align*}
\Phi&=\frac{Q^{enc}}{\epsilon_0}\\
E2\pi r L &= \frac{\rho \pi a^2 L}{\epsilon_0}\\
\therefore E &= \frac{\rho a^2}{2\epsilon_0 r}
\end{align*}
\end{parts}
\end{solution}

%Giancolli 22-61 -fixed
\question Consider a sphere of radius $r_0$ which has a sphere of radius $\frac{r_0}{2}$ hollowed out as shown in Figure \ref{fig:gauss:HollowSphere}. Suppose the sphere of radius $r$ has a volume charge density of $\rho$, while the hollowed out section of the sphere carries no charge density. Points A and C are located in the centres of their respective spheres.
\begin{parts}
\part What is the magnitude and direction of the electric field at point A?
\part What is the magnitude and direction of the electric field at point B?
\end{parts}
\capfig{0.3\textwidth}{figures/Gauss/HollowSphere.png}{\label{fig:gauss:HollowSphere}A uniformly charged sphere with a hollowed out cavity.}
\begin{finalanswer}
\begin{enumerate}[(a)]
\item \begin{align*}
E=\frac{\rho r_0}{6\epsilon_0}
\end{align*} which points to the right.
\item \begin{align*}
E=\frac{17\rho r_0}{54\epsilon_0}
\end{align*} 
which points to the left.
\end{enumerate}
\end{finalanswer}
\begin{solution}
Instead of a hollowed sphere, we consider this is the superposition of a sphere of radius $r_0$ centred at A with charge density $\rho$ and a sphere of radius $\frac{r_0}{2}$ centred at C with charge density $-\rho$.
\begin{parts}
\part At point A, by symmetry, the bigger sphere will contribute no electric field, whereas for the smaller sphere with the negative charge density, we use a gaussian surface centred at C with radius $\\frac{r_0}{2}$. The charge enclosed is:
\begin{align*}
Q^{enc}=\rho\frac{4}{3}\pi \frac{r_0^3}{8}
\end{align*}
The flux (which is inwards, as the field points to the right):
\begin{align*}
\Phi = E4\pi \frac{r_0^2}{4}
\end{align*}
Applying Gauss' Law:
\begin{align*}
\Phi&=\frac{Q^{enc}}{\epsilon_0}\\
E4\pi \frac{r_0^2}{4} &= \frac{\rho\frac{4}{3}\pi \frac{r_0^3}{8}}{\epsilon_0}\\
E &= \frac{\rho r_0}{6\epsilon_0}\\
\end{align*}
which points to the right.
\part We use the same principle, but this time both spheres will contribute to the electric field. First, for the bigger positive sphere, we choose a gaussian surface centred at A with radius $r_0$:
\begin{align*}
\Phi&=\frac{Q^{enc}}{\epsilon_0}\\
E^+4\pi r_0^2 &=\frac{\rho\frac{4}{3}\pi r_0^3}{\epsilon_0}\\
\therefore  E^+ &=\frac{\rho r_0}{3\epsilon_0}\\
\end{align*}
which will point to the left (outwards). For the negative smaller sphere, we choose a gaussian surface centred at C with radius $\frac{3}{2}r_0$:
\begin{align*}
\Phi&=\frac{Q^{enc}}{\epsilon_0}\\
E^-4\pi \frac{9r_0^2}{4} &= \frac{\rho\frac{4}{3}\pi \frac{r_0^3}{8}}{\epsilon_0}\\
\therefore E^- &= \frac{\rho r_0}{54\epsilon_0}\\
\end{align*}
which points to the right. The net field will point to the left and have magnitude:
\begin{align*}
E=E^+-E^-=\frac{\rho r_0}{\epsilon_0}\left( \frac{1}{3}-\frac{1}{54} \right)=\frac{17\rho r_0}{54\epsilon_0}
\end{align*}
\end{parts}
\end{solution}

%Giancolli 22-53 -fixed
\question A cube with a side length $l$ is placed such that one corner is at the origin of a coordinate system and extends along the positive $x$, $y$, and $z$ axes. What is the charge enclosed by the cube if the electric field in the region is given by $\vec E(x,y,z)=(ay+b)\hat y$, where $a$ and $b$ are positive constants.
\begin{finalanswer}
$Q=\epsilon_0al^3$
\end{finalanswer}
\begin{solution}
Since the flux is in the $y$ direction, only the two faces that are in the $xz$ plane will have net flux through them. For the face that is at $y=0$, the electric field is constant $E=b$, and the flux will be:
\begin{align*}
\Phi_1=\int \vec E\cdot d\vec A=-\int EdA=-b\int dA=-bl^2
\end{align*}
where the minus sign indicates that this is flux into to the cube. For the face that is at $y=l$, the electric field is again constant ($E=al+b$), and the flux is given by:
\begin{align*}
\Phi_2=\int \vec E\cdot d\vec A=\int EdA=al+b\int dA=al^3+bl^2
\end{align*}
The net flux is then:
\begin{align*}
\Phi=\Phi_1+\Phi_2=al^3.
\end{align*}
By Gauss' law, the charge enclosed is:
\begin{align*}
Q=\epsilon_0\Phi=\epsilon_0al^3
\end{align*}
\end{solution}

%Giancolli 23-43
\question A very large plate carries a uniform surface charge density of $\sigma=\SI{0.5e-6}{C/m^2}$. If you draw equipotentials around the plate, how far must they be spaced so that there is $\SI{100}{V}$ between equipotentials?

\textbf{Hint:} Use Gauss' Law to find the electric field above the plate!
\capfig{0.4\textwidth}{figures/Gauss/plateV.png}{\label{fig:potential:plateV} Equipotentials spaced by \SI{100}{V} above a uniformly charged plate.}
\begin{finalanswer}
	$\SI{3.54}{mm}$
\end{finalanswer}
\begin{solution}
	One can easily apply Gauss' Law to find that the electric field above a plane is:
	\begin{align*}
	E=\frac{\sigma}{2\epsilon_0}
	\end{align*}
	The electric field above a plane is constant in magnitude, and the equipotentials will be equidistant. Consider the potential difference between two points a distance $d$ apart in space on line perpendicular to the plane:
	\begin{align*}
	\Delta V=-\int_0^d Edx = -Ed=-d\frac{\sigma}{2\epsilon_0}\\
	\therefore d = \frac{2\epsilon_0\Delta V}{\sigma}=\frac{2(\SI{100}{V})(\SI{8.85e-12}{C^2N^{-1}m^{-2}})}{(\SI{0.5e-6}{C/m^2})}=\SI{3.54}{mm}
	\end{align*}
\end{solution}


