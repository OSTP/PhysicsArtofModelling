
\chapter{Gauss' Law}
\label{chapter:gauss}
In this chapter, we take a detailed look at Gauss' Law applied in the context of the electric field. We already covered Gauss' Law briefly in Section \ref{sec:gravity:gauss} when we examined the gravitational field. Since the electric force is mathematically identical to the gravitational force, we can apply the same tools to model the electric field as we do the gravitational field. This chapter provided an in depth description of Gauss' Law in the context of the electric field. 

\begin{learningObjectives}{
 \item Understand the concept of flux for a vector field.
 \item Understand how to apply Gauss' Law quantitatively to determine an electric field.
 \item Understand how to apply Gauss' Law qualitatively to discuss charges on a conductor.
 }
\end{learningObjectives}

\begin{opening}
\begin{MCquestion}{A neutral spherical conducting shell encloses a point charge, $Q$, located at the centre of the shell. Due to separation of charge, the outer surface of the shell will acquire a net positive charge. What is the magnitude of that charge? }
\item less than $Q$.
\item exactly $Q$. \correct
\item more than $Q$.
\end{MCquestion}
\end{opening}

\section{Flux of the electric field.}
Gauss' Law can be expressed as a relation between the flux of the electric field through a closed surface and the charge that is enclosed by that surface. We start by examining the concept of flux. 

Flux is always defined based on:
\begin{itemize}
\item A surface.
\item A vector field (e.g. the electric field). 
\end{itemize}
and can be thought of as a measure of many field lines cross the surface. For that reason, one usually refers to the ``flux of the electric field through a surface''. If the surface is parallel to the field, then no field lines cross that surface and we say that the flux through that surface is zero. If the surface is perpendicular to the field, then the flux through that surface is maximal. If the surface is rotated with respect to the electric field, then the flux through the surface is somewhere between zero and the maximal valus.
\capfig{0.9\textwidth}{figures/Gauss/fluxangle.png}{\label{fig:gauss:fluxangle}}

\section{Gauss' Law}
Describe Gauss Law, show how to use in simple cases



\section{Application of Gauss Law}
Describe charges on a conductor, etc.



\newpage
\section{Summary}

\begin{chapterSummary}
 Something that was learned
\end{chapterSummary}

\newpage
\begin{importantEquations}
\medskip
\begin{multicols}{2}
\textbf{Momentum of a point particle:}
\begin{align*}
\vec p = m\vec v \\
\frac{d}{dt}\vec p = \sum \vec F = \vec F^{net}
\end{align*}
\columnbreak
\\
\textbf{Position of the Centre of Mass \\ of a system:}
\begin{align*}
\vec r_{CM} &=\frac{1}{M}\sum_i m_i\vec r_i 
\end{align*}
\medskip
\end{multicols}
\end{importantEquations}

\newpage
\section{Thinking about the material}

\begin{chapteractivity}{Reflect and research}
{
\item Explain
}
\end{chapteractivity}

\begin{chapteractivity}{To try at home}
{
\item Try
}
\end{chapteractivity}

\begin{chapteractivity}{To try in the lab}
{
\item Propose an experiment
}
\end{chapteractivity}

\newpage
\section{Sample problems and solutions}
\subsection{Problems}
\begin{problem}{soln:template:ballistic}{\label{prob:template:ballistic} 

}
\end{problem}

\newpage
\subsection{Solutions}
\begin{solution}{prob:template:ballistic}\label{soln:template:ballistic}

\end{solution}

