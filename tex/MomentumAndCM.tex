\section{Centre of Mass and Momentum}

%%%%%%%%%%%%%%%%%%%%%%%%%%%%%%%%%%%
%%
%% Multiple Choice
%%
%%%%%%%%%%%%%%%%%%%%%%%%%%%%%%%%%%%
\subsection{Multiple Choice}

%easy
%Based on Natalie Dubas
\question Which of the following can best be described as an elastic collision?
\begin{checkboxes} 
	\choice A cake landing in someone's face.
	\CorrectChoice A golf club hitting a golf ball.
	\choice A meteorite striking the ground.
	\choice A car crashing into a wall.
\end{checkboxes}

\question The USS Enterprise is travelling across the galaxy at a speed $v = \frac{c}{8}$. A Romulan vessel is detected behind the starship and prepares an attack. Captain Jean-Luc Picard asks Commander Data to increase the speed of the USS Enterprise to $v = \frac{c}{4}$. What was the impulse necessary to change the speed of the starship? (Use classical mechanics.)
\begin{checkboxes}
\CorrectChoice $\frac{mc}{8}$ \correct
\choice $\frac{c}{8}$
\choice $\frac{3m}{8}$
\choice  $\frac{-mc}{8}$
\end{checkboxes}

\question A hockey puck is placed in the centre of a hockey rink, just before a game between Team Canada and Team Switzerland (which Team Switzerland obviously won). The puck suddenly explodes into 2 pieces. One piece lands \SI{2}{m} from the centre line in the direction of Team Canada's goal. Where does the second piece land, given that the second piece is twice as heavy as the first piece?
\begin{checkboxes} 
	\choice \SI{0.5}{m} in the direction of Team Switzerland's goal
	\CorrectChoice \SI{1}{m} in the direction of Team Switzerland's goal
	\choice \SI{2}{m} in the direction of Team Switzerland's goal
	\choice \SI{4}{m} in the direction of Team Switzerland's goal
\end{checkboxes}

\question Two sumo wrestlers are preparing to attack each other.  The first weighs \SI{833}{N}, and the second weighs \SI{784}{N}. The first charges at the second at \SI{3}{m/s}, while the other comes from the other direction at \SI{2.5}{m/s}.  The second one, though smaller, holds ground and remains stationary after impact and the first bounces back off.  At what speed does the first sumo wrestler bounce off the second?
\begin{checkboxes}
\choice \SI{5.5}{m/s} 
\choice \SI{0.55}{m/s}
\choice \SI{1.12}{m/s}
\CorrectChoice \SI{0.64}{m/s} \correct
\end{checkboxes}


\question Two objects with mass $m$ are moving in opposite directions with the same speed. What can we say about the total kinetic energy, $K$, and total momentum, $p$, of this system?
\begin{checkboxes}
\choice $K = 0$, $p = 0$
\choice $K > 0$, $p > 0$
\choice $K < 0$, $p = 0$
\CorrectChoice $K > 0$, $p=0$ \correct
\end{checkboxes}

%Jessica Grennan
\question Did you know that a young guanaco is called a chulengo? Betcha didn't! Two chulengos are wearing rollerblades. A stationary chulengo A has a mass of \SI{50}{kg}. She pushes chulengo B who has a mass of \SI{75}{kg} and is also stationary. After the push, chulengo B moves with a velocity of \SI{2}{m/s} to the right. What is the velocity of chulengo A?
\begin{checkboxes}
\choice \SI{2}{m/s} to the right
\choice \SI{1}{m/s} to the right
\CorrectChoice \SI{3}{m/s} to the left \correct
\choice \SI{2}{m/s} to the left
\choice \SI{1}{m/s} to the left
\end{checkboxes}


% This requires the rocket equation from the notes, which I notice is not on their formula sheet
\question A mine cart with a mass of \SI{250}{\kilo\gram} is stationary on a frictionless track. The cart is filled with \SI{1000}{\kilo\gram} of water. To drain the water, a hole is drilled in the back of the cart near the bottom. Assuming the water flows out of the hole with a constant velocity of \SI{1.0}{m/s} until the cart is empty, what will be the final speed of the mine cart?
\begin{checkboxes}
\CorrectChoice \SI{1.6}{m/s} \correct
\choice \SI{1.0}{m/s}
\choice \SI{0.22}{m/s}
\choice \SI{0}{m/s}
\end{checkboxes}

% Question submitted by Tamy Puniani
\question Two friends are moving from residence to a house in Kingston. They collectively exert a force (resulting in a net force of \SI{150}{N}) on a trolley with several of their boxes (the entire system weighs \SI{75}{kg}) for 5 seconds. Assuming the trolley is initially at rest, what is the final speed of the trolley system?
\begin{checkboxes}
\CorrectChoice \SI{10}{m/s} \correct
\choice \SI{16.67}{m/s}
\choice \SI{30}{m/s}
\end{checkboxes}

\question A large truck of mass $M$ collides head on with a car of mass $m$, causing significantly more damage to the car than to the truck. Which statement is true?
\begin{checkboxes} 
\choice The collision was elastic and the truck gave the same impulse to the car as the car did to the truck
\CorrectChoice The collision was inelastic and the truck gave the same impulse to the car as the car did to the truck \correct
\choice The collision was elastic and the truck gave a larger impulse to the car than the car did to the truck
\choice The collision was inelastic and the truck gave a larger impulse to the car than the car did to the truck
\choice Not enough information to tell
\end{checkboxes}

\question {Rhonda and John are bowling with large watermelons. By mistake, Rhonda lets go of her $\SI{3}{kg}$ watermelon at a speed of $\SI{10}{m/s}$ and it charges at John's $\SI{1}{kg}$ watermelon that was travelling at $\SI{2}{m/s}$. If the watermelons hit and start rolling together, at what speed will they be travelling?}
\begin{checkboxes}
\choice   $\SI{2}{m/s}$
\choice  $\SI{4}{m/s}$
\CorrectChoice $\SI{8}{m/s}$ 
\choice  $\SI{-6}{m/s}$
\end{checkboxes}

\question A cannon recoils after shooting a cannonball. If one assumes that the momentum of the cannon and cannonball system is conserved, and that the cannon has 100 times the mass of the cannonball, what can you say about their relative velocities?
\begin{checkboxes}
\choice The cannon ball will have have 10 times the speed of the cannon.
\choice The cannon ball will have have 25 times the speed of the cannon.
\choice The cannon ball will have have 50 times the speed of the cannon.
\CorrectChoice  The cannon ball will have have 100 times the speed of the cannon.
\end{checkboxes}


\question Which of the following could not be considered an elastic collision? (Choose the one that would be the least elastic).
\begin{checkboxes}
\CorrectChoice A train crashing into another train and the two trains sticking together and moving in unison after the collision.
\choice A billiard ball striking another.
\choice The collision between balls in Newton's Cradle.
\choice Two masses colliding with a spring in between them. 
\end{checkboxes}


\question Two cars collide. If we consider the two cars and their contents to be a system, which force is external to the system?
\begin{checkboxes}
\choice The force of gravity exerted by one car on the other. 
\CorrectChoice The normal force exerted by the road on one of the cars.
\choice The normal force exerted by the seat on one of the passengers in one of the cars.
\choice The force exerted by the bumper of one car onto the bumper of the other car. 
\end{checkboxes}


%%%%%%%%%%%%%%%%%%%%%%%%%%%%%%%%%%%
%
% long answer
%
%%%%%%%%%%%%%%%%%%%%%%%%%%%%%%%%%%%
\subsection{Long answers}
%% Heavily modified from example 9-15 in Giancolli (I'm assuming we won't need to change this one then)
\question Three astronauts, Alice, Bob, and Caroline, with masses $m_a=\SI{75}{kg}$, $m_b=\SI{62}{kg}$, and $m_c=\SI{98}{kg}$, respectively, are floating in space. I look out of my space station window and notice that they are located at the corners of a right triangle whose sides are about \SI{12.00}{m} and \SI{18.00}{m} long, as shown by Figure \ref{fig:momentumandcm:triangle}.
\begin{parts}
\part Locate the centre of mass of the systems of three astronauts using the coordinate system provided. 
\part Caroline throws a lifeline to Bob. Caroline then pulls Bob towards her. At what position do Bob and Caroline meet? (Assume that the mass of the lifeline was included in Caroline's mass quoted above).
\part Bob and Caroline attach themselves together and then throw the lifeline to Alice to real her in. Where do the three of them finally meet?
\end{parts}
\capfig{0.4\textwidth}{figures/MomentumAndCM/triangle.png}{\label{fig:momentumandcm:triangle} Astronaut positions.}

\begin{finalanswer}
\begin{enumerate}[(a)]
\item The CM position vector relative to Alice is $\vec r = (\SI{8.17}{m})\hat{i}+(\SI{7.51}{m})\hat{j}$
\item Bob and Caroline will meet at the centre of mass between the two of them. Even if Caroline recoils when throwing the life line, Bob will recoil when he catches it, maintaining the position of the initial centre of mass. The position of Bob's and Caroline's CM is:
\begin{align*}
x_{CM}&=\SI{12.0}{m}\\
y_{CM}&=\SI{11.025}{m}
\end{align*}
\item They meet at the centre of mass located in part (a). We can redundantly prove this, by considering the mass of Bob+Caroline at the location from part (b) and finding the center of mass with Alice.
\end{enumerate}
\end{finalanswer}
\begin{solution}
\begin{parts}
\part We can treat this problem as a system of discrete masses, in two dimensions:
\begin{align*}
x_{CM} &= \frac{m_ax_a + m_bx_b + m_cx_c}{m_a+m_b+m_c}\\
y_{CM} &= \frac{m_ay_a + m_by_b + m_cy_c}{m_a+m_b+m_c}
\end{align*}
We use the coordinate system given, where Alice is our astronaut at the origin. We want to find the centre of mass in both the x and y directions relative to her. We set $x_a = y_a = \SI{0}{m}$. From the diagram: $x_b = \SI{12.00}{m}$, $y_b=\SI{0}{m}$, $x_c = \SI{12.00}{m}$, and $y_c = \SI{18.00}{m}$. The x-component of the CM is:
\begin{align*}
x_{CM} &= \frac{(\SI{75}{kg})(\SI{0}{m}) + (\SI{62}{kg})(\SI{12.00}{m}) + (\SI{98}{kg})(\SI{12.00}{m})}{(\SI{75}{kg}) + (\SI{62}{kg}) + (\SI{98}{kg})}\\
&= \SI{8.17}{m}
\end{align*}
The y-component of the CM is:
\begin{align*}
y_{CM} &= \frac{(\SI{75}{kg})(\SI{0}{m}) + (\SI{62}{kg})(\SI{0}{m}) + (\SI{98}{kg})(\SI{18.00}{m})}{(\SI{75}{kg}) + (\SI{62}{kg}) + (\SI{98}{kg})}\\
&= \SI{7.51}{m}
\end{align*}
Therefore, the CM position vector relative to Alice is $\vec r = (\SI{8.17}{m})\hat{i}+(\SI{7.51}{m})\hat{j}$

\part Bob and Caroline will meet at the centre of mass between the two of them. Even if Caroline recoils when throwing the life line, Bob will recoil when he catches it, maintaining the position of the initial centre of mass. The position of Bob's and Caroline's CM is:
\begin{align*}
x_{CM}&=\SI{12.0}{m}\\
y_{CM}&= \frac{(\SI{98}{kg})(\SI{18.00}{m})}{(\SI{62}{kg}) + (\SI{98}{kg})}=\SI{11.025}{m}
\end{align*}
\part They meet at the centre of mass located in part (a). We can redundantly prove this, by considering the mass of Bob+Caroline at the location from part (b) and finding the center of mass with Alice:
\begin{align*}
x_{CM}&=\frac{(\SI{75}{kg})(\SI{0}{m}) +(\SI{160}{kg})(\SI{12}{m})}{(\SI{75}{kg}) +(\SI{160}{kg})}= \SI{8.17}{m}\\
y_{CM}&=\frac{(\SI{75}{kg})(\SI{0}{m}) +(\SI{160}{kg})(\SI{11.025}{m}}{(\SI{75}{kg}) +(\SI{160}{kg})}= \SI{7.51}{m}\\
\end{align*}

\end{parts}
\end{solution}

%Based on Giancolli 9-89
\question A bullet of mass $m$ is fired vertically into a block of mass $M$ and embeds itself into the block (Figure \ref{fig:momentumandcm:Bullet}). 
\begin{parts}
\part If the block with the bullet embedded in it rises by a height $h$, what was the speed of the bullet just before hitting the block?
\part What is the ratio of the kinetic energy after the collision over the kinetic energy before the collision?
\end{parts}
\capfig{0.3\textwidth}{figures/MomentumAndCM/Bullet.png}{\label{fig:momentumandcm:Bullet}Bullet fired vertically into a block.}

\begin{finalanswer}
\begin{enumerate}[(a)]
\item $\sqrt{2gh}$
\item $\frac{m}{m+M}$
\end{enumerate}
\end{finalanswer}
\begin{solution}
\begin{parts}
\part Conservation of momentum for the inelastic collision gives:
\begin{align*}
mv_1 =(m+M)v_2
\end{align*}
where $v_1$ is the speed of the bullet before the collision and $v_2$ is the speed of the bullet+block after the collision. After the collision, energy is conserved:
\begin{align*}
(M+m)gh &= \frac{1}{2}(m+M)v_2^2\\
\therefore v_2=\sqrt{2gh}
\end{align*}
which gives:
\begin{align*}
v_1=\frac{m+m}{m}\sqrt{2gh}
\end{align*}
\part The ratio of kinetic energy after the collision to before the collision is:
\begin{align*}
\frac{E_2}{E_1}&=\frac{\frac{1}{2}(m+M)v_2^2}{\frac{1}{2}mv_1^2}\\
&=\frac{(m+M)2gh}{m\left(\frac{m+m}{m}\right)^22gh}=\frac{m}{m+M}
\end{align*}
\end{parts}
\end{solution}

%Kate Fenwick
\question Although I am not very skilled in the game of pool, my friends still convince me to join them in a match. If I break, the pool balls will go flying off the table so my partner, Elizabeth, and I make the collective decision to let her break. 
\begin{parts}
\part I observe Elizabeth line up her pool cue to hit the \SI{0.165}{kg} cue ball. I estimate that the cue ball moves at about \SI{10}{m/s} after being hit. Elizabeth is a skilled player, so she can stop the motion of the cue stick after it has moved the short distance of \SI{2}{cm}. What is the impulse that Elizabeth delivers to the cue ball during her break?
\part  What is the average force associated with the impulse calculated in part (a) if I estimate the impulse to occur over the course of roughly \SI{4}{ms}?
\part After a few rounds, Elizabeth and I are in the lead! Elizabeth lines up a difficult shot, and I watch her carefully as she hits the cue ball in the direction of her target (the solid yellow ball, with the same mass as the cue ball). She causes the \SI{165}{g} cue ball to move directly towards her target, hitting it at a speed of \SI{11}{m/s}. After the collision, both balls move in the same direction as the initial velocity of the cue ball; however, the cue ball moves much more slowly than the yellow ball, at a speed of about \SI{3}{m/s}. What is the speed of the yellow ball after the collision? Was this collision elastic or inelastic?
\end{parts}

\begin{finalanswer}
\begin{enumerate}[(a)]
\item \SI{1.65}{kgm/s}
\item \SI{412.5}{N}
\item \SI{8}{m/s}, inelastic
\end{enumerate}
\end{finalanswer}
\begin{solution}
\begin{parts}
\part We approach this problem using the momentum-impulse relationship. The cue ball's speed changes from \SI{0}{m/s} to \SI{10}{m/s}. The impulse, {J}, can be calculated by finding the change in momentum, $\Delta p$:
\begin{align*}
J &= \Delta p = (\SI{0.165}{kg})(\SI{10}{m/s} - \SI{0}{m/s}) = \SI{1.65}{kg*m /s}
\end{align*}
\part We can use the definition of impulse, $J = F_{avg}\Delta t$ , to calculate the average force imparted on the cue ball:
\begin{align*}
F_{avg} &= \frac{J}{\Delta t} \\
&= \frac{(\SI{1.65}{kg m/s})}{(\SI{0.004}{s})}\\
&= \SI{412.5}{N}
\end{align*}
\part We use conservation of linear momentum to solve this problem, where $m$ is the mass of both the cue ball and the target yellow ball. $v_{c1}$ is the initial velocity of the cue ball, $v_{y1}$ is the initial velocity of the yellow ball, $v_{c2}$ is the final velocity of the cue ball, and $v_{y2}$ is the final velocity of the yellow ball:
\begin{align*}
m v_{c1} + m v_{y1} &= m v_{c2} + m v_{y2}
\end{align*}
Mass can be divided out from both sides. Setting $v_{y1} = 0$, we get:
\begin{align*}
v_{c1} &= v_{c2} + v_{y2} \\
\SI{11}{m/s} &= \SI{3}{m/s} + v_{y2}
\end{align*}
And calculate $v_{y2} = \SI{8}{m/s}$. 

To determine if the collision was elastic or inelastic, we use energy conservation:
\begin{align*}
\frac{1}{2}m v_{c1}^2 + \frac{1}{2}m v_{y1}^2 &= \frac{1}{2}m v_{c2}^2 + \frac{1}{2}m v_{y2}^2
\end{align*}
We can simplify to:
\begin{align*}
v_{c1}^2 &= v_{c2}^2 + v_{y2}^2 \\
(\SI{11}{m/s})^2 &= (\SI{3}{m/s})^2 + (\SI{8}{m/s})^2
\end{align*}
In which case the equality does not hold, so therefore the collision is inelastic. 
\end{parts}
\end{solution}


%% Taken from: Example 9-11 in Giancolli! -fixed
\question A pendulum of length $l$ has a wooden block of mass \SI{1.20}{kg} attached at its end. A bullet with a mass of \SI{7.4}{g} is fired at the block. The bullet connects with the wooden block and stops moving at its centre. Consequently, the wooden block (now with the bullet inside of it) swings up to a height of \SI{0.62}{m} as shown in Figure \ref{fig:momentumandcm:ballpend}. Given this information, what is the initial velocity of the bullet?

\capfig{0.7\textwidth}{figures/MomentumAndCM/ballpend.png}{\label{fig:momentumandcm:ballpend} Ballistic pendulum.}

\begin{finalanswer}
\SI{474.81}{m/s}
\end{finalanswer}
\begin{solution}
We can analyse the process by dividing it into two parts: (1) the time interval from just before to just after the collision itself, and (2) the subsequent time interval in which the pendulum moves from the vertical hanging position to its maximum height. 


In interval (1) we assume the collision time is very short, so that the projectile comes to rest in the block before the block has moved significantly from its rest position directly below its support. Thus, there is effectively no net external forces, and we can apply conservation of momentum to this completely inelastic collision. In time interval (2), we can use conservation of mechanical energy since gravity is a conservative force. The kinetic energy immediately after the collision is changed entirely to gravitational potential energy when the pendulum reaches its maximum height.

In interval (1) momentum is conserved:
\begin{align*}
mv_1 &= (m+M)v_2
\end{align*}
where $v_2$ is the speed of the block and embedded bullet just after the collision, before they have moved significantly. 

In interval (2), mechanical energy is conserved. We choose $y=0$ when the pendulum is at the bottom, and $y=\SI{0.62}{m}$ when the system reaches its maximum height. We can write:
\begin{align*}
\frac{1}{2}(m+M)v_2^2 &= (m+M)gh
\end{align*}
Rearranging to find $v_2 = \sqrt{2gh}$. We then insert this expression for $v_2$ back into our momentum conservation equation above to find:
\begin{align*}
mv_1 &= (m+M)\sqrt{2gh}\\
v_1 &= \frac{(m+M)}{m}\sqrt{2gh}\\
&=\frac{ (\SI{1.0074}{kg})}{(\SI{0.0074}{kg})}\sqrt{(2)(\SI{9.81}{m/s^2})(\SI{0.62}{m})}\\
&= \SI{474.81}{m/s}
\end{align*}

\end{solution}


%% Taken from: Example 9-16 in Giancolli! -fixed
\question Suppose there is a rod of length $l$ and mass $M$.
\begin{parts}
\part Show that the centre of mass of this rod is at its centre if the mass of the rod is uniformly distributed.
\part  Determine the centre of mass of the rod if its mass is not uniformly distributed, but instead its linear mass density begins as $\lambda_0$ at the bottom of the rod and increases linearly to a maximum of $\lambda = 2\lambda_0$ at the top of the rod.
\end{parts}

\begin{finalanswer}
\begin{enumerate}[(a)]
\item N/A
\item $x_{CM}=\frac{5}{9}l$
\end{enumerate}
\end{finalanswer}
\begin{solution}
We choose a coordinate system so that the rod lies on our x-axis, with the left end at $x=0$, so that $y_{CM}=0$ and $z_{CM}=0$. 
\begin{parts}
\part The rod is uniform, so its mass per unit length is constant and we can write it as $\lambda = \frac{M}{l}$. We now imagine the rod as divided into infinitesimal elements of length $dx$, each of which has mass $dm = \lambda dx$. We can then solve for the centre of mass by integrating:
\begin{align*}
x_{CM} &= \frac{1}{M}\int_{x=0}^{l} x dm\\
&= \frac{1}{M}\int_{x=0}^{l} \lambda x dx\\
&= \Big[\frac{\lambda}{M}\frac{x^2}{2}\Big]^l_0\\
&= \frac{\lambda l^2}{2M}\\
&= \frac{l}{2}
\end{align*}
where we have used $\lambda = M/l$. $x_{CM}$ is at the center of the rod, as expected.
\part Now we have $\lambda(x=0)=\lambda_0$ and we are told that $\lambda(x)$ increases linearly to $\lambda(x=l)=2\lambda_0$. We first need to find an equation for $\lambda(x)$. We are told that the dependence is linear, so we have:
\begin{align*}
\lambda(x) &= (\textit{slope})x +\lambda_0\\
\lambda(x) &= \frac{2\lambda_0-\lambda_0}{l}x+\lambda_0\\
\therefore \lambda(x)&= \lambda_0(\frac{x}{l}+1)
\end{align*}
Again, we integrate:
\begin{align*}
x_{CM}&= \frac{1}{M}\int_{x=0}^{l} \lambda(x) x dx\\
&= \frac{1}{M}\lambda_0\int_{x=0}^{l} \Big(\frac{x}{l}+1\Big) x dx\\
&= \frac{\lambda_0}{M} \Big[\frac{x^2}{2}+\frac{x^3}{3l}\Big]^l_0\\
&= \frac{5}{6}\frac{\lambda_0}{M}l^2
\end{align*}
However, we still need to find the total mass of the rod, $M$ in terms of $\lambda_0$ and $l$. We use:
\begin{align*}
M&= \int_{x=0}^{l} dm\\
&=  \int_{x=0}^{l} \lambda(x) dx\\
&=  \lambda_0 \int_{x=0}^{l} \Big(\frac{x}{l}+1\Big) dx  \\
&= \lambda_0 \Big[x+\frac{x^2}{2l}\Big]^l_0\\
&= \frac{3}{2}\lambda_0 l
\end{align*}
Then, we find:
\begin{align*}
x_{CM}&= \frac{5}{6}\frac{\lambda_0}{M}l^2\\
&= \frac{5}{9}l
\end{align*}
\end{parts}
\end{solution} 


\question A basketball of mass $m_b$ is dropped from a height $h$ above the Earth. Assume that the collision between the basketball and the Earth is elastic. Show that the basketball will rebound to a height $h$.

\textit{Note that you have to model this as a collision with the Earth, and show that because the mass of the basketball is much smaller than the mass of the Earth, $m_E$, the final height of the basketball will be $h$. When you want to show that something is true when one quantity is much bigger than the other, it is useful to write the quantities as a ratio (if $m_E>>m_b$, then $\frac{m_b}{m_E}\sim 0$).}
\begin{solution}
The speed of the basketball just before it hits the Earth is given by conservation of energy:
\begin{align*}
\frac{1}{2}m_bv_b^2&=m_bgh\\
\therefore v_b &=\sqrt{2gh}
\end{align*}
Just before the collision, we can approximate that the Earth has a speed of zero.

Modelling the collision as elastic, conservation of momentum gives:
\begin{align*}
m_bv_b+0&=m_bv_b'+m_Ev_E'\\
m_b(v_b-v_b')&=m_Ev_E'
\end{align*}
and conservation of energy gives:
\begin{align*}
\frac{1}{2}m_bv_b^2 + 0 &= \frac{1}{2}m_bv_b^{'2}+\frac{1}{2}m_Ev_E^{'2}\\
m_b(v_b^2-v_b^{'2}) &=m_Ev_E^{'2}\\
m_b(v_b+v_b') (v_b-v_b')&=m_Ev_E^{'2}\\
\end{align*}
If we divide this by the momentum equation, we find:
\begin{align*}
\frac{m_b(v_b+v_b') (v_b-v_b')}{m_b(v_b-v_b')}&=\frac{m_Ev_E^{'2}}{m_Ev_E'}\\
v_b+v_b' &=v_E'
\end{align*}
At this point, we have not yet taken into account that $m_E >>m_b$. We can get rid of the $v_E'$ by using the momentum equation again:
\begin{align*}
v_b+v_b' &=v_E'=\frac{m_b}{m_E}(v_b-v_b')\\
m_Ev_b+m_Ev_b'&=m_bv_b-m_bv_b'\\
v_b(m_b-m_E)&=v_b'(m_b+m_E)\\
v_b' &=\frac{m_b-m_E}{m_b+m_E}v_b\\
v_b' &=\frac{\frac{m_b}{m_E}-1}{\frac{m_b}{m_E}+1}v_b\\
\end{align*}
Now, we can say that when $m_E>>m_b$, then $\frac{m_b}{m_E} \sim 0$, and find that:
\begin{align*}
v_b' = -v_b
\end{align*}
Since the ball has the opposite velocity after the collision and the Earth is effectively stationary, it will rebound to the same height:
\begin{align*}
\frac{1}{2}mv_b^{'2}&=mgh'\\
\frac{1}{2}mv_b^{2}&=mgh'\\
\therefore h' &= h
\end{align*}
\end{solution}


\question A baby guanaco, with a mass of \SI{30}{kg}, is in the middle of a $L=\SI{10}{m}$ long wooden plank that weighs \SI{100}{kg} and is floating on lake Titicaca, on the Peruvian side. The plank is lined up so that it is perpendicular to the shore line (i.e. the plank points to the shore). You whistle from the shore, and the guanaco, intrigued, starts to walk towards you, until it reaches the end of the plank. How much closer is the guanaco to you? 

\begin{finalanswer}
\SI{3.846}{m}
\end{finalanswer}
\begin{solution}
As the guanaco walks towards you, the plank moves backwards (the momentum of the plank+gunanaco system is conserved, as no external forces act on the system; this also means that the centre of mass of the system is immobile). The situation is illustrated in Figure \ref{fig:momentumandcm:GuanacoPlank}, showing the position of the guanaco's centre of mass ($d_G$) and of the plank's centre of mass ($d_p$) after the guanaco has moved, as measured from the reference frame of the water. 
\capfig{0.4\textwidth}{figures/MomentumAndCM/GuanacoPlank.png}{\label{fig:momentumandcm:GuanacoPlank} Guanaco and plank system.}
If we set the origin to be the location of the centre of mass, then we have:
\begin{align*}
d_{CM}=\frac{m_pd_p-m_Gd_G}{m_p+m_G}&=0\\
\therefore m_pd_p-m_Gd_G&=0
\end{align*}
The centre of mass of the plank is located at the midpoint of the plank, so we must have:
\begin{align*}
d_G+d_p=\frac{1}{2}L
\end{align*}
Combining the two equations, we have:
\begin{align*}
m_pd_p-m_Gd_G &=0\\
d_G&=\frac{m_p}{m_G}d_p\\
d_G&=\frac{m_p}{m_G}(\frac{1}{2}L-d_G)\\
\frac{m_G}{m_p}d_G+d_G&=\frac{1}{2}L\\
\therefore d_G&=\frac{L}{2\left(1+\frac{m_G}{m_p}\right)}=\frac{(\SI{10}{m})}{2\left(1+\frac{(\SI{30}{kg})}{(\SI{100}{kg})}\right)}\\
&=\SI{3.846}{m}
\end{align*}
\end{solution}

%based on Giancolli 9-85
\question Figure \ref{fig:momentumandcm:Pool} shows a shot that you are trying to make in pool. You are trying to sink the red ball into the pocket in the upper left corner by hitting it with the cue ball which sits directly below. Both balls have the same mass and dimensions, the collision is elastic, and you can assume that the initial velocity of the cue ball is parallel to the long side of the pool table. You are worried that if you sink the red ball, the cue ball will go into the upper right pocket, and your opponent will then make fun of you for not realizing this. Should you be concerned? Show us why or why not!

\capfig{0.3\textwidth}{figures/MomentumAndCM/Pool.png}{\label{fig:momentumandcm:Pool} A pool shot. Dimensions are in feet.}

\begin{finalanswer}
Yes, you should be worried. The direction of the cue ball after the balls collide is directly towards the pocket. 
\end{finalanswer}
\begin{solution}
We refer to Figure \ref{fig:momentumandcm:Pool_vec} to define the velocity vectors of the cue ball ($\vec v_c$) and of the red ball ($\vec v_b$), before (subscript $1$) and after (subscript $2$) the collision. 

\capfig{0.3\textwidth}{figures/MomentumAndCM/Pool_vec.png}{\label{fig:momentumandcm:Pool_vec} Velocity vectors of the pool balls before and after the collision.}

Note that from the dimensions in the diagram in Figure \ref{fig:momentumandcm:Pool}, we can easily find the angle $\theta$ for the red ball to go into the pocket:
\begin{align*}
\cos\theta&=\frac{\sqrt 3}{2}\\
\sin\theta&=\frac{1}{2}
\end{align*}

There is an elegant way to solve this problem (see below), but we start with the ``brute force'' solution.

\textbf{Brute force solution:} We write the conservation of momentum in the $x$ direction, noting that the masses cancel and that $\sin\theta=\frac{1}{2}$:
\begin{align*}
0&=mv_{c2}\sin\phi-mv_{b2}\sin\theta\\
\therefore v_{b2}&=2v_{c2}\sin\phi
\end{align*}

We write the conservation of momentum in the $y$ direction:
\begin{align*}
mv_{c1}&=mv_{c2}\cos\phi+mv_{b2}\cos\theta\\
v_{c1}&=v_{c2}\cos\phi+v_{b2}\frac{\sqrt 3}{2}\\
\end{align*}

We can substitute the expression for $v_{b2}$ from the $x$ component into the $y$ component to find:
\begin{align*}
v_{c1}&=v_{c2}\cos\phi+v_{c2}\sin\phi\sqrt 3\\
\therefore v_{c1}&=v_{c2}(\cos\phi+\sin\phi\sqrt 3)\\
\end{align*}

The conservation of energy equation is:
\begin{align*}
\frac{1}{2}mv_{c1}^2&=\frac{1}{2}mv_{c2}^2+\frac{1}{2}mv_{b2}^2\\
v_{c1}^2&=v_{c2}^2+v_{b2}^2\\
&=v_{c2}^2+(2v_{c2}\sin\phi)^2\\
\therefore v_{c1}^2&=v_{c2}^2(1+4\sin^2\phi)\\
\end{align*}
where in the second line, we replaced $v_{b2}$ by the expression that comes from the $x$ component of momentum.

We can now square the previous expression that we obtained for $v_{c1}$ and equate it to what we have above:
\begin{align*}
\left( v_{c2}(\cos\phi+\sin\phi\sqrt 3) \right)^2&=v_{c2}^2(1+4\sin^2\phi)\\
\cos^2\phi+2\sqrt 3\cos\phi\sin\phi+3\sin^2\phi &= 1+4\sin^2\phi
\end{align*}
We note that $\cos^2\phi=1-\sin^2\phi$:
\begin{align*}
1-\sin^2\phi+2\sqrt 3\cos\phi\sin\phi+3\sin^2\phi &= 1+4\sin^2\phi\\
-2\sin^2\phi+2\sqrt 3\cos\phi\sin\phi&=0
\end{align*}
We now divide by $2\sin\phi\cos\phi$, noting that $\frac{\sin\phi}{\cos\phi}=\tan\phi$:
\begin{align*}
-\tan\phi+\sqrt 3&=0\\
\therefore \tan\phi&=\sqrt 3
\end{align*}

Again, if we look at the diagram in Figure \ref{fig:momentumandcm:Pool_vec}, we can see that $\tan\phi=\sqrt 3$, is precisely the angle $\phi$ that corresponds to the cue ball going into the upper right pocket, so you should indeed be worried!

\textbf{Elegant solution:}
Since the masses of the two balls are the same, we can work with the vectors directly. The masses cancel in both the conservation of momentum and the conservation of energy equations. We thus have for conservation of momentum and energy, respectively:
\begin{align*}
\vec v_{c1}&=\vec v_{c2}+\vec v_{b2}\\
v_{c1}^2 &= v_{c2}^2  + v_{b2}^2 \\
\end{align*}
The first equation is just an addition of vectors that we illustrate in Figure \ref{fig:momentumandcm:Pool_vec2}. 
\capfig{0.15\textwidth}{figures/MomentumAndCM/Pool_vec2.png}{\label{fig:momentumandcm:Pool_vec2} Velocity vectors of the pool balls before and after the collision.}
The vectors thus form a triangle. The second equation (conservation of energy) is a relation between the length of the sides of the triangle. By Pythagoras' theorem (or the cosine rule), it is clear that the angle between $\vec v_{c2}$ and $\vec v_{b2}$ must be \SI{90}{\degree}.

Referring to Figure \ref{fig:momentumandcm:Pool}, you can see that the red ball and the two upper pockets form a right angle triangle. Thus, if the red ball goes into the pocket at the upper left, then the cue ball will go into the pocket in the upper right.


\end{solution}


%Zaremba 2002 Final exam - This question already included in QLibrary in electric potential.
\question A proton (mass $m$) makes an elastic head-on collision with a nucleus at rest and rebounds with a speed that is nine-tenths of its initial speed. What is the mass, $M$, of the nucleus?

\begin{finalanswer}
$M=19m$
\end{finalanswer}
\begin{solution}
We use conservation of energy and momentum in one dimension. If the initial velocity of the proton is $v$, then its final velocity is $-\frac{9}{10}v$. Let the nucleus' final velocity be $v'$. Conservation of momentum gives:
\begin{align*}
mv &= -\frac{9}{10}mv + Mv'\\
mv \left(1+\frac{9}{10}\right)&=Mv'\\
\end{align*}
Conservation of energy gives:
\begin{align*}
\frac{1}{2}mv^2&= \frac{1}{2}m\left(\frac{9}{10}v\right)^2 + \frac{1}{2}Mv'^2\\
\therefore mv^2(1-\left(\frac{9}{10}\right)^2)&=Mv'^2\\
mv^2\left(1-\frac{9}{10}\right)\left(1+\frac{9}{10}\right)&=Mv'^2
\end{align*}
Dividing this by the momentum equation gives:
\begin{align*}
v\left(1-\frac{9}{10}\right)=v'
\end{align*}
Substituting the above expression for $v'$ back into the momentum equation, we find
\begin{align*}
mv \left(1+\frac{9}{10}\right)&=Mv'\\
mv \frac{19}{10}&=Mv\left(1-\frac{9}{10}\right)\\
m \frac{19}{10}&=M\frac{1}{10}\\
\therefore M&=19m
\end{align*}
\end{solution}


\question You have always wanted to be a ninja turtle. You are thus interested in spinning a slice of pizza on the tip of your finger (the slice of pizza is in the horizontal plane and balanced on your vertical finger). The slice of pizza can be modelled as an isosceles triangle of uniform density and thickness with a base length of $b$, and a height of $h$, as in Figure \ref{fig:momentumandcm:Triangle}.
\begin{parts}
\part Where is the centre of mass of the slice of pizza?
\part Where is the centre of mass of the slice of pizza if, instead, it can be modelled as an equilateral triangle of side $b$? 
\end{parts}

\capfig{0.2\textwidth}{figures/MomentumAndCM/TriangleCM.png}{\label{fig:momentumandcm:Triangle} An isosceles triangle.}

\begin{finalanswer}
\begin{enumerate}[(a)]
\item We chose the origin as the vertex of the triangle, with the x axis pointing up from the vertex down the centre of the slice. By symmetry, the centre of mass will be along the  $x_{CM}=\frac{2}{3}h$ measured from the vertex.
\item CM measured from the vertex is $\frac{1}{\sqrt{3}}b$
\end{enumerate}
\end{finalanswer}
\begin{solution}
\begin{parts}
\part To find the centre of mass, we model the triangle as being made of small rods of mass $dm$, length $l(x)$ at position $x$ that are parallel to the base. We choose the origin as the vertex of the triangle, as shown in Figure \ref{fig:momentumandcm:Triangle_dm}. By symmetry, the centre of mass will be along the $x$ axis, and each rod will have its center of mass on the $x$ axis. 

\capfig{0.2\textwidth}{figures/MomentumandCM/Triangle_dm.png}{\label{fig:momentumandcm:Triangle_dm} Mass element on the isosceles triangle.}

The surface density of the rod is:
\begin{align*}
\sigma = \frac{M}{A}=\frac{2M}{bh}
\end{align*}

The length of each rod as a function of $x$ is linear and easily found to be:
\begin{align*}
l(x)=\frac{b}{h}x
\end{align*}
which will have a length of $l(x=h)=b$. Each little rod will have a mass $dm$ of:
\begin{align*}
dm &= \sigma l(x)dx=\frac{2M}{bh}\frac{b}{h}xdx\\
&=\frac{2M}{h^2}xdx\\
\end{align*}
The $x$ position of the centre of mass (as measured from the vertex) is given by:
\begin{align*}
x_{CM}=\frac{1}{M}\int xdm=\frac{1}{M}\int_0^h x\frac{2M}{h^2}xdx
&=\frac{2}{h^2}\int_0^h x^2dx=\frac{2}{h^2}\frac{h^3}{3}=\frac{2}{3}h
\end{align*}
which corresponds to $\frac{1}{3}h$ if measured from the base.
\part If the triangle is equilateral, then:
\begin{align*}
h=\sin(\SI{60}{\degree})b=\frac{\sqrt 3}{2}b 
\end{align*} 
and using our formula from above, the center mass as measured from the vertex is located at:
\begin{align*}
x_{CM}=\frac{2}{3}h=\frac{2}{3}\frac{\sqrt 3}{2}b=\frac{\sqrt 3}{3}b=\frac{1}{\sqrt 3}b
\end{align*}
or $\frac{\sqrt 3}{6}b=\frac{1}{2\sqrt 3}b$ if measured from the base. This could have been found from geometry (intersection of the bi-sectors) as well. 
\end{parts}
\end{solution}


%From Emma - need to include more details in the solution.
\question Find the centre of mass of a uniform half disk of radius $R$.
\begin{solution}
First, we should draw a diagram with appropriate axis.

\capfig{0.40\textwidth}{figures/MomentumAndCM/halfdisk.png}{\label{fig:momentumandcm:halfdisk} A uniform half-disk.}

Let us say that the mass of a semicircular ring is m. Considering a slice of this half-disk with a width of dy and mass dm,

\begin{align*}
\sigma &=\frac{mass}{area} \\
\sigma &=\frac{m}{\frac{1}{2}\pi r^{2}} \\
\therefore \sigma &=\frac{2m}{\pi r^{2}} \\
\end{align*}

\begin{align*}
dm&=\sigma dA\\
&=\frac{2m}{\pi r^{2}}\pi ydy\\
\therefore&=\frac{2m}{r^{2}} ydy\\
\end{align*}

The y coordinate of the position of the centre of mass will be located at $x=0$ and $y=\frac{2r}{\pi}$ by symmetry.

Using this information, we can use the centre of mass formula for the y coordinate to determine the centre of mass for the half-disk.

\begin{align*}
y_{CM}&=\frac{1}{m}\int y dm\\
&=\frac{1}{m}\int_{0}^{r} (\frac{2y}{\pi})\frac{2m}{r^{2}}ydy\\
&=\frac{4}{\pi r^{2}}\int_{0}^{r}y^{2}dy\\
&=\frac{4}{\pi r^{2}}(\frac{R^{3}}{3})\\
&=\frac{4r}{3\pi}\\
\therefore y_{CM}&=\frac{4r}{3\pi}\\
\end{align*}

Therefore, the centre of mass of a uniform half-disk is $x_{CM}=0$, $y_{CM}=\frac{4r}{3\pi}$.
\end{solution}


%Original
\question A building that is \SI{10}{m} tall and \SI{5}{m} wide is being struck by a \SI{120}{km/h} wind head on. If you assume that the air comes to rest upon striking the building, determine the force exerted by the wind on the building, and show that it scales with the square of the wind speed. The density of air is \SI{1.2}{kg/m^3}.

\begin{solution}
We need to determine the rate at which the building is changing the momentum of the air that is striking the building, which will correspond to the force (Newton's Second Law written with momentum). We can consider the amount of force that is required to stop a given mass, $m$, of air that has speed $v$ when it hits the building and a speed of zero after hitting the building. 

First, we determine the mass of air, $m$, that strikes the building over a period of time $\Delta t$. In a period of time $\Delta t$, the wind will travel a distance $d=v\Delta t$, and the mass of air that strikes the building will be given by:
\begin{align*}
 m = \rho V=\rho A v\Delta t
\end{align*}
where $\rho$ is the density of air, $ V$, is the volume of air, and $A$ is the area of the building. Before striking the building, the mass of air will have momentum (in the direction towards the building):
\begin{align*}
p = m v=\rho A v\Delta t v=\rho A v^2\Delta t
\end{align*}
The change in momentum of that mass of air is related to force exerted on the mass of air:
\begin{align*}
F&=\frac{\Delta p}{\Delta t}=\frac{0-\rho A v^2\Delta t}{\Delta t}=-\rho A v^2\\
\end{align*}
where the minus sign indicates that the force exerted on the air is in the opposite direction of the momentum. The force exerted on the building is in the same direction as the momentum, and is given by:
\begin{align*}
F&=\rho A v^2=(\SI{1.2}{kg/m^3})(\SI{10}{m})(\SI{5}{m})(\SI{33.3}{m/s})^2=\SI{6.65e4}{N}
\end{align*}
which increases with the square of the speed of the wind.
\end{solution}

\question A cannonball is launched with a speed $v=\SI{100}{m/s}$ by a cannon that aims at an angle of $\SI{45}{\degree}$ from the horizontal. At some point during its trajectory the cannonball breaks up into two pieces of equal mass. The first piece then lands at a distance $D=\SI{800}{m}$ from the cannon, in the same direction that the original cannonball was fired. How far from the cannon does the second piece land? Assume that drag is negligible and that the fragments land on the ground at the same height as the cannonball was fired.
\begin{solution}
While the cannonball is in the air, the only force on the cannonball is gravity. Its centre of mass will thus follow a parabolic trajectory, and would land a distance, say, $R$, from the cannon. If the cannonball breaks up, the two fragments of equal mass will be equidistant from the centre of mass. Thus, the second fragment will land a distance $R+(R-D)$ from the cannon. We can find the distance $R$ by finding the range for a projectile fired with a speed $v$ and an angle $\theta=\SI{45}{\degree}$. The time that it takes for the projectile to cover that distance is given by:
\begin{align*}
y_f&=y_i+v_{0y}t-\frac{1}{2}gt^2\\
0&= v\sin\theta t-\frac{1}{2}gt^2\\
\therefore t&=\frac{2v\sin\theta}{g}
\end{align*}
where we defined a coordinate system with a origin where the cannonball was fired, with $y$ positive upwards and $x$ positive in the horizontal direction of the cannonball. The horizontal displacement in that time is given by:
\begin{align*}
R&=v\cos\theta t=v\cos\theta\frac{2v\sin\theta}{g}\\
\therefore R&=\frac{v^2}{g}=\frac{(\SI{100}{m/s})^2}{(\SI{9.8}{m/s^2})}=\SI{1020.41}{m}
\end{align*}
where we used the fact that $\cos\theta=\sin\theta=\frac{1}{\sqrt 2}$ for $\theta=\SI{45}{\degree}$. Thus, the centre of mass of the cannonball ``lands'' a distance $R$ from the cannon. The second cannonball will thus land at a distance:
\begin{align*}
D' = R+(R-D)=\frac{2v^2}{g} - D = \frac{2(\SI{100}{m/s})^2}{(\SI{9.8}{m/s^2})}-(\SI{800}{m})= \SI{1240.82}{m}
\end{align*}
\end{solution}

\question A bullet of mass $m$ is moving horizontally just before it impact the bob of mass $M$ of a pendulum that consists of the mass $M$ attached to a mass-less string of length $L$. The bullet embeds itself in the bob. What is the minimum speed of the bullet required for the pendulum to make a complete circle after the impact? Assume that drag and friction are negligible.
\capfig{0.3\textwidth}{figures/MomentumAndCM/ballistic.png}{\label{fig:momentumandcm:ballisitic}A bullet colliding and embedding itself in a pendulum.}
\begin{solution}
In order for the pendulum to make a complete revolution, the speed at the top of the trajectory must be that such that the centripetal force is provided by gravity (the tension in the string goes to zero at the top of the trajectory if the pendulum has the minimum speed to make a full revolution). Note that after the collision, the bob as a mass of $M+m$. At the top of the trajectory, the speed is given by:
\begin{align*}
(M+m)\frac{v^2_{top}}{L}&=(M+m)g\\
\therefore v^2_{top}&=gL
\end{align*}
After the collision with the bullet, the mechanical energy of the pendulum is conserved. By comparing the energy of the pendulum at the top of its trajectory and at the bottom (just after the bullet was embedded), we can find the speed of the pendulum just after the collision. If we define gravitational potential energy to be zero at the bottom of the trajectory, conservation of energy gives:
\begin{align*}
\frac{1}{2}(m+M)v_{bottom}^2&=\frac{1}{2}(m+M)v^2_{top}+(m+M)g(2L)\\
v_{bottom}^2&=gL+4gL\\
\therefore v_{bottom}&=\sqrt{5gL}
\end{align*}
We can apply conservation of momentum to find the speed of the bullet before the collision:
\begin{align*}
mv &= (m+M)v_{bottom}\\
\therefore v &= \frac{m+M}{m}\sqrt{5gL}
\end{align*}
\end{solution}



\question A mass $m$ is attached to a mass-less string of length $L=\SI{4.20}{m}$ and can swing in the vertical plane. The mass $m$ is released from a horizontal position as shown by the dotted lines in Figure \ref{fig:PendulumMass}. At the bottom of the trajectory, it collides elastically head-on with a block of mass $M=2m$ which lies on a frictionless surface and is constrained to move in one dimension (the horizontal direction in the diagram).
\capfig{0.4\textwidth}{figures/MomentumAndCM/PendulumMass.png}{\label{fig:PendulumMass} A mass on a pendulum hits a block elastically.}
\begin{parts}
	\part Show that the speed of mass $m$ just before the collision is $v_m=\SI{9.07}{m/s}$
	\part What is the velocity of the block just after the collision?
	\part To what height above the ground does the mass $m$ rebound after the collision? 
\end{parts}


\begin{solution}
\begin{parts}
\part We can use conservation of energy, and choose the ground as the location of zero potential gravitational energy. Initially, the mechanical energy is $E_1=mgL$, and at the bottom, it is $E_2=\frac{1}{2}mv^2$. Equating the two gives the speed just before the collision:
\begin{align*}
v_{m1} = \sqrt{2gL}=\sqrt{2(\SI{9.8}{m/s^2})(\SI{4.20}{m})}=\SI{9.07}{m/s}
\end{align*}
\part The collision is elastic and one dimensional. We denote $v_{m1}$ as the initial speed of mass $m$, $v_{m2}$ its final speed, and $v_{M2}$ the speed of the block after the collision. Conservation of momentum gives:
\begin{align*}
mv_{m1} &= mv_{m2} + 2mv_{M2}\\
v_{m1} - v_{m2} &= 2v_{M2}
\end{align*}
Conservation of energy gives:
\begin{align*}
\frac{1}{2}mv^2_{m1} &= \frac{1}{2}mv^2_{m2} + \frac{1}{2}2mv^2_{M2}\\
v^2_{m1} &= v^2_{m2} + 2m^2_{M2}\\
v^2_{m1} - v^2_{m2} &= 2m^2_{M2}\\
(v_{m1} - v_{m2})(v_{m1} + v_{m2}) &= 2v^2_{M2}\\
\end{align*}
Dividing this equation by that from momentum conservation, we find:
\begin{align*}
v_{m1} + v_{m2}&= v_{M2}
\end{align*}
Isolating $v_{m2}$ in each equation, and equating them allows us to solve for $v_{M2}$
\begin{align*}
v_{m1} - v_{m2} &= 2v_{M2}\\
\therefore v_{m2} &=v_{m1}-2v_{M2}\\
v_{m1} + v_{m2}&= v_{M2}\\
\therefore v_{m2} &=v_{M2}-v_{m1} \\
v_{m1}-2v_{M2} &= v_{M2}-v_{m1}\\
2v_{m1} &= 3v_{M2}\\
\therefore v_{M2} &=\frac{2}{3}v_{m1}=\frac{2}{3}\sqrt{2gL}=\SI{6.05}{m/s}
\end{align*}
\part Here, we need the velocity of $m$ after the collision, and then apply conservation of energy. From above:
\begin{align*}
v_{m2} &= v_{m1}-2v_{M2}\\
&=\sqrt{2gL}-\frac{4}{3}\sqrt{2gL}\\
&=-\frac{1}{3}\sqrt{2gL}
\end{align*}
Applying conservation of energy:
\begin{align*}
\frac{1}{2}mv_{m2}^2&=mgh\\
\frac{1}{2}\frac{1}{9}2gL&=gh\\
\therefore h&=\frac{L}{9}=\SI{0.47}{m}
\end{align*}
\end{parts}
\end{solution}


%written by Frankie (Emily) Polak -make sure this question is fine.
\question A bullet of mass $m_b$ strikes a block of wood of mass $m_w$ which is at rest. The bullet is embedded in the wood, causing the bullet and the block of wood to move together. Use conservation of momentum to prove that this collision is inelastic.
\begin{solution}
Suppose $v_1$ is the initial speed of the bullet and $v_2$ is the speed of the wood+bullet system. $v_2$ can be written in terms of $v_1$ 
\begin{align*}
m_bv_{v1}&= v_2(m_b+m_w)\\
v_2&=\frac{m_bv_1}{m_b+m_w}
\end{align*}
Next, we must make the (incorrect) assumption that the energy is conserved in order to prove our assumption wrong.
\begin{align*}
\frac{1}{2}m_bv_1^2&=\frac{1}{2}(m_b+m_w)v_2^2\\
v_2^2&=\frac{m_bv_1^2}{m_b+m_w}\\
v_2&=\sqrt{\frac{m_bv_1^2}{m_b+m_w}}\\
\end{align*}
Now, to finish the incorrect assumption, we will state the final result of using conservation of momentum and conservation of energy is a collision of this type.
\begin{align*}
\sqrt{\frac{m_bv_1^2}{m_b+m_w}}=\frac{m_bv_1}{m_b+m_w}\\
\end{align*}
The above statement is only true for two cases, those being the case where the bullet or the wood is massless, which would of course mean that no collision occurred. Therefore, we can say that this collision must be inelastic.
\end{solution}

%% TODO: Question Library question: Give a function F(t), calculate the change in momentum resulting from that force over a certain range of time (i.e. they need to take an integral). 


