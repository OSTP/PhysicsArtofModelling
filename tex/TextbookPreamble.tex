\usepackage{mathtools} % for \Aboxed
\usepackage{paralist}
\usepackage{calc}
\usepackage{subfig}
\usepackage{setspace}
\usepackage{amssymb}
\usepackage{amsmath}
\usepackage{amstext}
\usepackage[font={small,it}]{caption}
\usepackage[pdftex]{graphicx} %Does not work in pressbooks!!!
\usepackage{fancyhdr,lastpage}
\usepackage{url}
\usepackage{longtable}
\usepackage{comment}
\usepackage{ifthen}
\usepackage{color}
\usepackage[colorlinks=true,linkcolor=blue,urlcolor=blue]{hyperref}
\usepackage[explicit]{titlesec}
\usepackage{lmodern}
\usepackage{listings}
\usepackage{parskip}
\usepackage[table]{xcolor}
%\usepackage{enumitem}
\usepackage{wrapfig}
\usepackage[framemethod=TikZ]{mdframed}
\usepackage{titlesec} %for spacing around titles
\usepackage{caption}
\usepackage[separate-uncertainty = true]{siunitx}
\usepackage{float}
\restylefloat{table}
\usepackage{multicol}
\usepackage[shortlabels]{enumitem}
%\lstset{language=Python,showstringspaces=false,commentstyle=}
\usepackage{tikz}
\usepackage[american,smartlabels]{circuitikz}%for drawing circuits

\usetikzlibrary{arrows}

%%Some math and other shortcuts
\newcommand{\chloens}{Chlo\"e}
\newcommand{\chloe}{Chlo\"e~}
\newcommand{\die}[2]{\frac{\partial #1}{\partial #2}}
\newcommand{\ddt}{\frac{d}{dt}}
\newcommand{\lagd}{\mathcal{L}}
\newcommand{\code}[1]{\texttt{#1}}
\newcommand{\angstrom}{\textup{\AA}}
\newcommand{\ampere}{Amp\`ere}
\newcommand{\amperesp}{Amp\`ere~}

\newcommand{\pvec}[1]{\vec{#1}\mkern2mu\vphantom{#1}} % for a vector of a primed quantity, e.g. \vec p ', should be \pvec p'

%Colours
\definecolor{mygreen}{rgb}{0.2,0.6,0}
\definecolor{TBgreen}{rgb}{0.0,0.5,0.0}
\definecolor{TBblue}{rgb}{0.0, 0.58, 0.71}
\definecolor{TBred}{rgb}{0.8, 0.25, 0.33}
\definecolor{TBorange}{rgb}{0.89, 0.35, 0.13}
\definecolor{TBPurple}{rgb}{0.6,0.4,0.8}

%Stuff for writing code:

\lstset{ %
  belowskip=0pt,
  aboveskip=0pt,
  caption=\relax,
  backgroundcolor=\color{white},   % choose the background color; you must add \usepackage{color} or \usepackage{xcolor}
  basicstyle=\footnotesize,        % the size of the fonts that are used for the code
  breakatwhitespace=false,         % sets if automatic breaks should only happen at whitespace
  breaklines=true,                 % sets automatic line breaking
  captionpos=t,                    % sets the caption-position to bottom
  commentstyle=\color{mygreen},    % comment style
  deletekeywords={...},            % if you want to delete keywords from the given language
  escapeinside={(*}{*)},          % if you want to add LaTeX within your code
  extendedchars=true,              % lets you use non-ASCII characters; for 8-bits encodings only, does not work with UTF-8
  frame=none,	                   % adds a frame around the code
  keepspaces=true,                 % keeps spaces in text, useful for keeping indentation of code (possibly needs columns=flexible)
  keywordstyle=\color{blue},       % keyword style
  language=Python,                 % the language of the code
  otherkeywords={*,...},           % if you want to add more keywords to the set
  numbers=none,                    % where to put the line-numbers; possible values are (none, left, right)
  numbersep=5pt,                   % how far the line-numbers are from the code
  numberstyle=\tiny\color{black}, % the style that is used for the line-numbers
  rulecolor=\color{black},         % if not set, the frame-color may be changed on line-breaks within not-black text (e.g. comments (green here))
  showspaces=false,                % show spaces everywhere adding particular underscores; it overrides 'showstringspaces'
  showstringspaces=false,          % underline spaces within strings only
  showtabs=false,                  % show tabs within strings adding particular underscores
  stepnumber=1,                    % the step between two line-numbers. If it's 1, each line will be numbered
  stringstyle=\color{red},     % string literal style
  tabsize=2,	                   % sets default tabsize to 2 spaces
  title=\lstname                   % show the filename of files included with \lstinputlisting; also try caption instead of title
}

%Environments for writing code
\DeclareCaptionFont{white}{\color{white}}
\DeclareCaptionFormat{listing}{\colorbox{gray}{\parbox{\textwidth}{#1#2#3}}}

\captionsetup[lstlisting]{format=listing,labelfont=white,textfont=white}
\renewcommand{\lstlistingname}{Python Example}
\renewcommand{\lstlistlistingname}{List of \lstlistingname s}

\lstnewenvironment{python}[1][]{
  \lstset{#1, language=Python}%
  \renewcommand\lstlistingname{Python Code}
}{}

\lstnewenvironment{poutput}{
 \lstset{caption=\mbox{}, language=,aboveskip=-3pt}
 \addtocounter{lstlisting}{-1}
 \renewcommand\lstlistingname{Output}
}{}


%%Pretty chapter headings:
\newlength\chapnumb
\setlength\chapnumb{4cm}

\titleformat{\chapter}[block]
{\normalfont\sffamily}{}{0pt}
{\parbox[b]{\chapnumb}{%
   \fontsize{120}{110}\selectfont\thechapter}%
  \parbox[b]{\dimexpr\textwidth-\chapnumb\relax}{%
    \raggedleft%
    \hfill{\LARGE#1}\\
    \rule{\dimexpr\textwidth-\chapnumb\relax}{0.4pt}}}
\titleformat{name=\chapter,numberless}[block]
{\normalfont\sffamily}{}{0pt}
{\parbox[b]{\chapnumb}{%
   \mbox{}}%
  \parbox[b]{\dimexpr\textwidth-\chapnumb\relax}{%
    \raggedleft%
    \hfill{\LARGE#1}\\
    \rule{\dimexpr\textwidth-\chapnumb\relax}{0.4pt}}}


%%%spacing around titles
%\setlength{\parindent}{0pt}
%\parskip = \baselineskip

%spacing around captions (e.g. caption after a table)
\captionsetup{belowskip=6pt,aboveskip=4pt}

\titlespacing*{\chapter}
{0pt}{0ex}{0ex}
\titlespacing*{\section}
{0pt}{1ex}{1.5ex}
\titlespacing*{\subsection}
{0pt}{1ex}{1.5ex}
\titlespacing*{\subsubsection}
{0pt}{1ex}{1.5ex}

%%% Spacing in lists:
\setlist{nosep}

%%Verticall spacing between table rows
\renewcommand{\arraystretch}{1.5}

\setlength{\intextsep}{12pt}

%space before itemized list:
%\setlength{\topsep}{-10pt} %does nothing?

%%Simplifed figure environment:

\newenvironment{capfig}[3]{\begin{center}\includegraphics[width=#1]{#2}\captionof{figure}{#3}\end{center}}{}


%Wrap figure environments (right or left). Argument #1 (default value 12, specified as optional), is the number of
%lines that the figure should take.
%space around wrap figures:
%\setlength{\intextsep}{20pt}%
%\setlength{\columnsep}{5pt}%
\newenvironment{Rwcapfig}[4][0]{
\begingroup
%\setlength{\intextsep}{0pt}%
\setlength{\columnsep}{10pt}%
\begin{wrapfigure}[#1]{R}{#2}\centering\includegraphics[width=#2]{#3}\caption{#4}\end{wrapfigure}}{\endgroup}

\newenvironment{rwcapfig}[4][0]{
\begingroup
%\setlength{\intextsep}{0pt}%
\setlength{\columnsep}{10pt}%
\begin{wrapfigure}[#1]{r}{#2}\centering\includegraphics[width=#2]{#3}\caption{#4}\end{wrapfigure}}{\endgroup}

\newenvironment{Lwcapfig}[4][0]{
\begingroup
%\setlength{\intextsep}{0pt}%
\setlength{\columnsep}{10pt}%
\begin{wrapfigure}[#1]{L}{#2}\centering\includegraphics[width=#2]{#3}\caption{#4}\end{wrapfigure}}{\endgroup }


\newenvironment{lwcapfig}[4][0]{
\begingroup
%\setlength{\intextsep}{0pt}%
\setlength{\columnsep}{10pt}%
\begin{wrapfigure}[#1]{l}{#2}\centering\includegraphics[width=#2]{#3}\caption{#4}\end{wrapfigure}}{\endgroup }

\newenvironment{lwfig}[3][0]{
\begingroup
\setlength{\intextsep}{0pt}%
\setlength{\columnsep}{10pt}%
\begin{wrapfigure}[#1]{l}{#2}\centering\includegraphics[width=#2]{#3}\end{wrapfigure}}{\endgroup }

%Environments (in the works) %%%%%%%%%%%%%%%%%%%%%%%%%%%%%%%%%%%%%%%%%%%%%%%%%%%%%%%%%%%%%%%%%%%%%%%%%%%%%%%%%%%%%%%%%%%%%%%%%%%%%%%%%%%%%%%%%%%%%%%%%%%%%%%%%%%%%%%%%%%%%%%%%%%%%%%
%% Environments to Keep
%%Learning Objectives:
\newenvironment{learningObjectives}[1]{%
\mdfsetup{%
    frametitle={%
        \tikz[baseline=(current bounding box.east),outer sep=0pt]
        \node[anchor=west,rectangle,fill=TBblue, minimum width=5cm, rounded corners=0.2cm]
        {\strut Learning Objectives};},
    frametitlefont=\color{white}\sffamily\bfseries,
    innertopmargin=0pt,linecolor=TBblue,%
    linewidth=1pt,topline=true,%
    frametitleaboveskip=\dimexpr-\ht\strutbox\relax%
}
\vspace{10pt}
\begin{mdframed}\begin{itemize}[label=\textcolor{TBblue}{\textbullet}] #1 \end{itemize}}{\end{mdframed}}


%%Opening Question:
\newenvironment{opening}[1]{%
\mdfsetup{%
    frametitle={%
        \tikz[baseline=(current bounding box.east),outer sep=0pt]
        \node[anchor=west,rectangle,fill=TBblue, minimum width=5cm,rounded corners=0.2cm]
        {\strut Think About It};},
    frametitlefont=\color{white}\sffamily\bfseries,
    innertopmargin=0pt,linecolor=TBblue,backgroundcolor=TBblue!5,%
    linewidth=1pt,topline=true,%
    frametitleaboveskip=\dimexpr-\ht\strutbox\relax%
}
\vspace{10pt}
\begin{mdframed}[nobreak=true]\relax #1}{%
\end{mdframed}}

%Problems
\newcounter{problem}[chapter]
\def\theproblem{\thechapter-\arabic{problem}}

\newenvironment{problem}[2]
  {\refstepcounter{problem}\textbf{Problem \theproblem: }#2 (\hyperref[#1]{Solution})} %
  {\vspace{2ex}}

\newenvironment{problemParts}[2]
  {\refstepcounter{problem}\textbf{Problem \theproblem: }#2 (\hyperref[#1]{Solution}) %
  \begin{enumerate}[label=\alph*),topsep=-10pt]}%
   {\end{enumerate}}
  {\vspace{2ex}}

\newenvironment{MCquestion}[1]{#1%
   \begin{enumerate}[label=\Alph*),topsep=-10pt]}{%
   \end{enumerate}}

%Example Box with counter
\newcounter{example}[chapter]
\def\theexample{\thechapter-\arabic{example}}
\newenvironment{example}[2]{%
\refstepcounter{example}%
\mdfsetup{%
    frametitle={%
        \tikz[baseline=(current bounding box.east),outer sep=0pt]
        \node[anchor=west,rectangle,fill=TBred, minimum width=5cm,rounded corners=0.2cm]
        {\strut Example~\theexample};},
    frametitlefont=\color{white}\sffamily\bfseries,
    skipabove=\strutbox,
    innertopmargin={0.5cm},
    linecolor=TBred,%
    linewidth=1pt,topline=true,%
    frametitleaboveskip=\dimexpr-\ht\strutbox\relax%
    %frametitleaboveskip=-\strutbox
}
\begin{mdframed}[]\relax #1 \leavevmode \\ \newline { \color{TBred}\textsf{\large Solution}} \\ {\color{TBred}\rule[10pt]{\textwidth}{1pt}}  #2}{%
\end{mdframed}}

%Review Box
\newenvironment{review}[1]{%
\mdfsetup{%
    frametitle={%
        \tikz[baseline=(current bounding box.east),outer sep=0pt]
        \node[anchor=west,rectangle,fill=TBorange, minimum width=5cm,rounded corners=0.2cm]
        {\strut Review Topics};},
    frametitlefont=\color{white}\sffamily\bfseries,
    innertopmargin=0pt,linecolor=TBorange,backgroundcolor=TBorange!5,%
    linewidth=1pt,topline=true,%
    frametitleaboveskip=\dimexpr-\ht\strutbox\relax%
}
\vspace{10pt}
\begin{mdframed}[nobreak=true]\relax #1}{%
\end{mdframed}}

%Reflect and Research
\newenvironment{chapteractivity}[2]{%
\mdfsetup{%
    frametitle={%
        \tikz[baseline=(current bounding box.east),outer sep=0pt]
        \node[anchor=west,rectangle,fill=TBgreen!80, minimum width=5cm,rounded corners=0.2cm]
        {\strut #1};},
    frametitlefont=\color{white}\sffamily\bfseries,
    innertopmargin=0pt,linecolor=TBgreen!80,backgroundcolor=TBgreen!5,%
    linewidth=1pt,topline=true,%
    frametitleaboveskip=\dimexpr-\ht\strutbox\relax%
}
\vspace{10pt}
\begin{mdframed}\begin{enumerate}[itemsep=1ex] #2
   \end{enumerate}}
{\end{mdframed}}



%%Checkpoint question in a box, with counter:
\newcounter{checkpoint}[chapter]
\def\thecheckpoint{\thechapter-\arabic{checkpoint}}

\newenvironment{checkpoint}[1]{%
\refstepcounter{checkpoint}%
\mdfsetup{%
    frametitle={%
        \tikz[baseline=(current bounding box.east),outer sep=0pt]
        \node[anchor=west,rectangle,fill=TBgreen!80, minimum width=5cm,rounded corners=0.2cm]
        {\strut Checkpoint~\thecheckpoint};},
    frametitlefont=\color{white}\sffamily\bfseries,
    innertopmargin=0pt,linecolor=TBgreen!80,backgroundcolor=TBgreen!5,%
    linewidth=1pt,topline=true,%
    frametitleaboveskip=\dimexpr-\ht\strutbox\relax%
}
\vspace{10pt}
\begin{mdframed}[nobreak=true]\relax #1}{%
\end{mdframed}}


%Student Opinion with option for student name
\newenvironment{studentOpinion}[2]{%
\mdfsetup{%
    frametitle={%
        \tikz[baseline=(current bounding box.east),outer sep=0pt]
        \node[anchor=west,rectangle,fill=TBorange, minimum width=5cm,rounded corners=0.2cm]
        {\strut #1's Thoughts};},
    frametitlefont=\color{white}\sffamily\bfseries,
    linecolor=TBorange,%
    linewidth=1pt,topline=true,
    skipabove=\strutbox,
    innertopmargin={0.5cm},%
    frametitleaboveskip=\dimexpr-\ht\strutbox\relax%
}
\vspace{10pt}
\begin{mdframed}[]\relax #2}{%
\end{mdframed}}

%%Important Equations:
\newenvironment{importantEquations}[1]{%
\mdfsetup{%
    frametitle={%
        \tikz[baseline=(current bounding box.east),outer sep=0pt]
        \node[anchor=west,rectangle,fill=TBblue, minimum width=5cm,rounded corners=0.2cm]
        {\strut Important Equations};},
    frametitlefont=\color{white}\sffamily\bfseries,
    innertopmargin=0pt,linecolor=TBblue,%
    linewidth=1pt,topline=true,%
    frametitleaboveskip=\dimexpr-\ht\strutbox\relax%
}
\vspace{10pt}
\begin{mdframed}[nobreak=false]\relax #1}{%
\end{mdframed}}

%%Definitions
\newenvironment{definitions}[1]{%
	\mdfsetup{%
		frametitle={%
			\tikz[baseline=(current bounding box.east),outer sep=0pt]
			\node[anchor=west,rectangle,fill=TBPurple, minimum width=5cm,rounded corners=0.2cm]
			{\strut Important Definitions};},
		frametitlefont=\color{white}\sffamily\bfseries,
		innertopmargin=0pt,linecolor=TBPurple,%
		linewidth=1pt,topline=true,%
		frametitleaboveskip=\dimexpr-\ht\strutbox\relax%
	}
	\vspace{10pt}
	\begin{mdframed}[nobreak=false]\relax #1}{%
\end{mdframed}}

%%Key Takeaways:
\newenvironment{chapterSummary}[1]{%
\mdfsetup{%
    frametitle={%
        \tikz[baseline=(current bounding box.east),outer sep=0pt]
        \node[anchor=west,rectangle,fill=TBblue, minimum width=5cm,rounded corners=0.2cm]
        {\strut Key Takeaways};},
    frametitlefont=\color{white}\sffamily\bfseries,
    innertopmargin=0pt,linecolor=TBblue,backgroundcolor=TBblue!5,%
    linewidth=1pt,topline=true,%
    frametitleaboveskip=\dimexpr-\ht\strutbox\relax%
}
\vspace{10pt}
\begin{mdframed}[]\relax #1}{%
\end{mdframed}}

\newcounter{solution}[chapter]
\def\thesolution{\thechapter-\arabic{solution}}

\newenvironment{solution}[2]{\refstepcounter{solution}\textbf{Solution to problem \ref{#1}:} #2}
{\vspace{2ex}}

\newenvironment{solutionParts}[2]{\refstepcounter{solution}\textbf{Solution to problem \ref{#1}:}
  \begin{enumerate}[label=\alph*),topsep=-10pt]
   #2
  \end{enumerate}}
  {\vspace{2ex}}

%%Reflect and Research Questions with Counter
\newcounter{tQuestion}[chapter]
\def\therQuestion{\thechapter-\arabic{tQuestion}}

\newenvironment{tQuestion}[1]{\refstepcounter{tQuestion}%
    \textbf{Activity~\therQuestion: }#1}

\usepackage[paper=letterpaper,
            %includefoot, % Uncomment to put page number above margin
            marginparwidth=.0in,     % Length of section titles
            marginparsep=.05in,       % Space between titles and text
            margin=1in,               % 1 inch margins
            includemp]{geometry}

\setcounter{secnumdepth}{2}
\setcounter{tocdepth}{3}
