\section{Electric Potential}

%%%%%%%%%%%%%%%%%%%%%%%%%%%%%%%%%%%
%%
%% Multiple Choice
%%
%%%%%%%%%%%%%%%%%%%%%%%%%%%%%%%%%%%
\subsection{Multiple Choice}
%original
\question An electron is placed at $x=\SI{1}{m}$ in a region where the electric potential is given by $V(x)=(\SI{10}{V/m})x$. Which statement is true?
\begin{checkboxes}
\choice The electric field in this region is constant as a function of $x$ and points in the positive $x$ direction
\choice The electric field in this region increases linearly with $x$ and points in the negative $x$ direction
\choice The electron will move with constant velocity in the negative $x$ direction
\CorrectChoice The electron will move with constant acceleration in the positive $x$ direction \correct
\end{checkboxes}

\question Which will have the highest final speed?
\begin{checkboxes}
	\CorrectChoice An electron accelerated from rest across a potential difference of \SI{200}{V}.
	\choice A proton accelerated from rest across a potential difference of \SI{200}{V}.
	\choice A single ionized Helium atom accelerated from rest across a potential difference of \SI{200}{V}.
	\choice All three will have the same speed.
\end{checkboxes}

\question A potential difference is applied along a thin metallic strip with a width of $w=\SI{2}{cm}$ and a cross sectional area of $A=\SI{0.1}{cm^2}$, so that a current $I=\SI{2.5}{A}$ goes through the strip, as shown in Figure \ref{fig:MagneticForce:HallProbe}. The strip is made of copper (with a free electron density, $n=\SI{8.48e22}{cm^{-3}}$) and is immersed in a uniform magnetic field. The maximum Hall voltage across the strip is measured to be $\Delta V_{Hall}=\SI{0.1}{V}$. What is the strength of the magnetic field?

\capfig{0.4\textwidth}{figures/Potential/HallProbe.png}{\label{fig:MagneticForce:HallProbe} The Hall potential across a Hall probe in a magnetic field.}
\begin{choices}
	\choice \SI{2.7e-1}{T}
	\choice \SI{2.7e1}{T}
	\choice \SI{2.7e3}{T}
	\CorrectChoice \SI{2.7e5}{T}
\end{choices}

%Maaike de Lint (improved precision of wording)
\question Which of the following is a valid definition for the electric potential difference between two points in space?
\begin{checkboxes}
	\CorrectChoice The change in potential energy of a charge $q$ moved between those two points and then divided by the charge.
	\choice The potential energy stored in a system of charged objects due to the charges at those points.
	\choice  The potential energy per one unit charge at that point in space.
	\choice The difference in electrical potential energy of a charge $q$ between those two points. 
\end{checkboxes}

%Tashifa Imtiaz
\question Two uniformly charged plates placed \SI{1}{m} apart have a potential difference of \SI{120}{V} between them. An electron (charge $q_e=-e$, mass $m_e$) is released from the negative plate. What is the speed of the electron right as it reaches the positive plate?
\begin{checkboxes}
	\choice $v=\sqrt{\frac{q_e(\SI{120}{V})}{m_e}}$
	\choice $v=\sqrt{\frac{-q_e(\SI{120}{V})}{m_e}}$
	\CorrectChoice  $v=\sqrt{\frac{-2q_e(\SI{120}{V})}{m_e}}$
	\choice $v=\sqrt{\frac{2q_e(\SI{120}{V})}{m_e}}$
\end{checkboxes}

%Camren Oakes
\question The voltage between two large parallel plates separated by a distance of \SI{0.5}{m} is \SI{200}{V}. The magnitude of the electric field between the plates is:
\begin{checkboxes}
	\choice \SI{50}{N/C}
	\choice \SI{100}{N/C}
	\choice  \SI{200}{N/C}
	\CorrectChoice \SI{400}{N/C}
\end{checkboxes}


%original
\question Which statement is true of an electron or proton in a region of space where the only force on the particle is from the electric field?
\begin{checkboxes}
\CorrectChoice An electron will reduce its electrical potential energy by moving to a region of high electric potential \correct
\choice An electron will increase its electrical potential energy by moving to a region of high electric potential
\choice Both a proton and an electron will reduce their electrical potential energy by moving to a region of lower electric potential
\choice A proton will increase its electrical potential energy by moving to a region of lower electric potential
\end{checkboxes}


%From Stephanie
\question What potential difference is needed to stop an electron that has an initial velocity $v=\SI{4.2e6}{m/s}$?
\begin{checkboxes}
\choice $\Delta V = 30.1 V$
\CorrectChoice $\Delta V = -50.2 V$ \correct
\choice $\Delta V = -10.0 V$
\choice $\Delta V = 50.2 V$
\end{checkboxes}

%From Stephanie
\question The electric field between two parallel plates connected to a \SI{65}{V} battery is $\SI{1750}{V/m}$. How far apart are the plates? 
\begin{checkboxes}
\CorrectChoice $d = \SI{3.71e-2}{m}$ \correct %%Note use of \SI for units
\choice $d = \SI{3.5e-2}{m}$
\choice $d = \SI{4.71e-2}{m}$
\choice $d = \SI{3.21e-2}{m}$
\end{checkboxes}


%Question submitted by Emily Darling
\question How far away from the center of a \SI{1}{cm} diameter solid metal sphere (which has a -\SI{3.00}{nC} static charge) will the voltage be -\SI{270}{V} if \SI{0}{V} is defined to be at infinity? 
\begin{checkboxes}
\choice \SI{0.1}{cm}
\choice \SI{1.0}{cm}
\CorrectChoice \SI{10}{cm} \correct
\choice \SI{100}{cm}
\end{checkboxes}

%From Troy
\question What causes lightning to strike the ground during a thunderstorm?
\begin{checkboxes}
\choice The electric field generated by the Earth's molten metal core
\CorrectChoice A large potential difference between the Earth's surface and the cloud \correct
\choice A high local electron flux
\choice Global warming
\end{checkboxes}

%Question submitted by Adam McCaw
\question How much energy is given to an electron accelerated through a potential difference of \SI{1000}{V}?
\begin{checkboxes}
\choice \SI{1000}{J}
\CorrectChoice \SI{1.6e-16}{J} \correct
\choice \SI{1.6e19}{J}
\choice \SI{1.6e-19}{J}
\end{checkboxes}


%Question from Ceaira Heimstra
\question Consider eight equally spaced locations along the circumference of a circle of radius $R$. If a charge $q$ is then placed at seven of the eight locations, what is the electric potential at the center of the circle?
\begin{checkboxes}
\CorrectChoice $ k \frac{7q}{R}$ \correct
\choice $ k \frac{q}{8 R}$
\choice $ k \frac{7q}{8 R}$
\choice $ k \frac{q}{7 R}$
\end{checkboxes}

%Qiqi Zhang (modified for clarity)
\question A uniform electric field with strength \SI{100}{V/m} points to the right. Two points, A and B, are separated by a distance of \SI{10}{cm} along a line that makes a \SI{60}{\degree} angle with the field, as shown in Figure \ref{fig:potential:LineField}. What is the  electric potential difference, $\Delta V=V_B-V_A$.
\capfig{0.3\textwidth}{figures/Potential/LineField.png}{\label{fig:potential:LineField}Two points in a uniform electric field.}
\begin{choices} 
\choice $\Delta V=\SI{5}{V}$
\CorrectChoice  $\Delta V=\SI{-5}{V}$ \correct
\choice $\Delta V=\SI{10}{V}$
\choice $\Delta V=\SI{-10}{V}$
\end{choices}

\question Two charges, $q_1$ and $q_2$, are held at a distance $d$ apart. Which statement is correct?
\begin{checkboxes}
\choice The magnitude of the stored electrical potential energy is bigger if the charges have the same sign 
\choice The magnitude of the stored electrical potential energy is bigger if the charges have opposite signs
\CorrectChoice  The magnitude of the stored electrical potential energy is independent of the relative sign of the charges \correct
\end{checkboxes}

\question A charge $q_1=1.5\times 10^{-9}$\,C is placed on the x-axis at a position $x=30$\,cm and a second charge, $q_2=-2.5\times 10^{-9}$\,C, is placed on the y-axis at $y=40$\,cm. If we define electric potential at infinity to be $V=1$\,V, what is the electric potential at point $P=(30,40,0)$\,cm?
\begin{checkboxes}
\choice -42.25\,V
\CorrectChoice -40.25\,V \correct
\choice 107.75\,V
\choice 109.75\,V
\end{checkboxes}

%%%%%%%%%%%%%%%%%%%%%%%%%%%%%%%%%%%
%
% long answer
%
%%%%%%%%%%%%%%%%%%%%%%%%%%%%%%%%%%%
\subsection{Long answers}
%Stephanie
\question A charge, $q=\SI{-4.20e-6}{C}$, is placed at rest at point $a$. An external force, $F$, does work, $W_F=\SI{6.00e-4}{J}$, to accelerate the charge from point $a$ to point $b$. At point $b$, the charge has a kinetic energy of \SI{1.7e-4}{J}. What is the electric potential difference between points $a$ and $b$, $\Delta V=V_b-V_a$, if there are no other forces acting on the charge other than then external force and the electric field?
\begin{finalanswer}
$V=\SI{-102.38}{V}$
\end{finalanswer}
\begin{solution}
The total work done on the charge is equal to the change in kinetic energy:
\begin{align*}
W_{tot}=\Delta K=\SI{1.7e-4}{J}
\end{align*}
The total work is that done by the electric field, $W_E$, plus that done by the external force, $W_F$:
\begin{align*}
W_F+W_E&=W_{tot}\\
\therefore W_E&=W_{tot}-W_F=(\SI{1.7e-4}{J})-(\SI{6.00e-4}{J})=\SI{-4.3e-4}{J}
\end{align*}
The work done by the electric field is thus negative (so the negative charge is moving in the same direction as the electric field, from high electric potential to low electric potential). Because this is a negative charge, the electric potential difference is thus negative.

The work done by the electric field is given by the charge times the change in electric potential:
\begin{align*}
W_E&=-\Delta U=-q\Delta V\\
\therefore \Delta V&=V_b-V_a=\frac{-W_E}{q}=\frac{(\SI{4.3e-4}{J})}{(\SI{-4.20e-6}{C})}=\SI{-102.38}{V}
\end{align*}
and the electric potential difference is indeed negative.
\end{solution}

%AB McLean originally
\question The inner conductor of two coaxial cylindrical conductors carries a net positive charge per unit of length $\lambda$. Let the radius of the inner conductor equal $a$ and the inner radius of the outer conductor equal $b$, as shown in Figure \ref{fig:potential:coax}.
\begin{parts}
\part Give an expression for the electric field as a function of radius between the two cylinders, $E(r)$.
\part Give an expression for the for the potential difference between the two conductors, $V_b-V_a$.
\end{parts}
\capfig{0.2\textwidth}{figures/Potential/coax.png}{\label{fig:potential:coax} Cross-section of two coaxial cylindrical conductors.}
\begin{finalanswer}
\begin{enumerate}[(a)]
\item \begin{align*}
E=\frac{\lambda}{2\epsilon_0\pi r}
\end{align*}
\item \begin{align*}
E=\frac{\lambda}{2\epsilon_0\pi}ln\frac{a}{b}
\end{align*}
\end{enumerate}
\end{finalanswer}
\begin{solution}
\begin{parts}
\part We use Gauss' Law with a cylindrical surface of length $l$ and radius $r$ ($a<r<b$). The charge enclosed is $\lambda l$, and the electric flux is $2\pi r l E(r)$. The electric field is thus:
\begin{align*}
\Phi&=\frac{Q^{enc}}{\epsilon_0}\\
2\pi r l E(r) &= \frac{\lambda l}{\epsilon_0}\\
\therefore E(r)&=\frac{\lambda}{2\epsilon_0\pi r}
\end{align*}
The electric field points radially outwards since the inner conductor is charged.
\part The potential difference is found by integrating the electric field:
\begin{align*}
\Delta V&= V_b - V_a=-\int_a^bE(r)dr\\
&=-\int_a^b\frac{\lambda}{2\epsilon_0\pi r}dr=-\frac{\lambda}{2\epsilon_0\pi}\int_a^b\frac{1}{r}dr\\
&=-\frac{\lambda}{2\epsilon_0\pi}[\ln(r)]_a^b=-\frac{\lambda}{2\epsilon_0\pi}\ln\left(\frac{b}{a}\right)\\
&=\frac{\lambda}{2\epsilon_0\pi}\ln\left(\frac{a}{b}\right)
\end{align*}
\end{parts}

\end{solution}

%Giancolli problem
\question  A cathode-ray tube is commonly used in television sets to display images. The cathode-ray tube steers electrons (mass $m_e = 9.11\times 10^{-31}$kg, charge $e = 1.6\times 10^{-19}$C) towards the screen. Consider a cathode-ray tube with a potential difference of $\Delta V_1 = 5500$V that accelerates an electron from rest. The electron moves through two charged plates with a potential difference of $\Delta V_2 = 250$V. What is the angle $\theta$ of the velocity vector once the electron leaves the space between the two plates?
\capfig{0.4\textwidth}{figures/Potential/eplates.png}{\label{fig:potential:eplates}An electron deflected by charged plates.}
\begin{finalanswer}
$\SI{6.48}{\degree}$
\end{finalanswer}
\begin{solution}
The electron is accelerated by a potential difference, $\Delta V_1=\SI{5500}{V}$, allowing us to find the horizontal component of its velocity vector:
\begin{align*}
e\Delta V_1=\frac{1}{2}m_ev_1^2\\
\therefore v_1=\sqrt{\frac{2e\Delta V_1}{m_e}}
\end{align*}
Once the electron is between the horizontal plates, it will experience a constant vertical force upwards, just like in projectile motion. The horizontal component of its velocity will remain unchanged. The vertical electric field between the plates is given by:
\begin{align*}
E=\frac{\Delta V_2}{h}
\end{align*}
and the vertical component of the acceleration, $a_y$, is thus:
\begin{align*}
F_y&=m_ea_y=eE=e\frac{\Delta V_2}{h}\\
\therefore a_y&=e\frac{\Delta V_2}{hm_e}
\end{align*}

To find the angle of the velocity vector when the electron exits the plates, we need to know the vertical component of the velocity vector at that point, which requires knowing for how long the electron was accelerated upwards. We can find how long the electron was between the plates by using the horizontal component of the velocity:
\begin{align*}
t=\frac{d}{v_1}=d\sqrt{\frac{m_e}{2e\Delta V_1}}
\end{align*}

The vertical component of the velocity vector when exiting the plates is given by:
\begin{align*}
v_{2y}&=a_yt=e\frac{\Delta V_2}{hm_e}d\sqrt{\frac{m_e}{2e\Delta V_1}}\\
&=\Delta V_2\frac{d}{h}\sqrt{\frac{e}{2m_e\Delta V_1}}
\end{align*}

The angle is then given by:
\begin{align*}
\tan\theta&=\frac{v_{2y}}{v_1}=\Delta V_2\frac{d}{hv_1}\sqrt{\frac{e}{2m_e\Delta V_1}}\\
&=\Delta V_2\frac{d}{h}\sqrt{\frac{m_e}{2e\Delta V_1}}\sqrt{\frac{e}{2m_e\Delta V_1}}\\
&=\frac{d}{2h}\frac{\Delta V_2}{\Delta V_1}\\
&=\frac{(\SI{6.5}{cm})}{2(\SI{1.3}{cm})}\frac{(\SI{250}{V})}{(\SI{5500}{V})}\\
&=0.1136\\
\therefore \theta &= \SI{6.48}{\degree}
\end{align*}
\end{solution}
%Giancolli 23-14 -- This is not actually complete!!! Need to revise, clean up, etc!!!

%Gotta fix this one up. Was left behind.

%\question A 32 cm diameter conducting sphere is charged to 680V relative to V=0 (at infinity). What is the surface charge density on the sphere? At what distance from the centre of the sphere will the potential be 25V?
%\begin{solution}
%\begin{equation}
%V_b - V_a = -\int_{r_a}^{r_b}\vec{E}\cdot d\vec{l}
%\end{equation}

%and using Gauss's law,

%\begin{equation}
%V_b - V_a = -\frac{Q}{4\pi\epsilon_o} \int_{r_a}^{r_b} \frac{dr}{r}
%\end{equation}

%\begin{equation}
%V_o = \frac{Q}{4\pi\epsilon_o r_o}
%\end{equation}

%\begin{equation}
%Q = 4\pi\epsilon_o r_oV_o
%\end{equation}
%\begin{equation}
%\sigma = \frac{Q}{A} = \frac{Q}{4\pi r_o^2}
%\end{equation}
%\begin{equation}
%\sigma = \frac{V_o \epsilon_o}{r_o} = \frac{(680V)(8.85\times10^{-12}\mathrm{C}^2\mathrm{Nm}^2}{(0.16\mathrm{m})}
%\end{equation}
%\begin{equation}
%\sigma = 3.761\times 10^{-8} \mathrm{C}/\mathrm{m}^2
%\end{equation}
%\end{solution}


\question Suppose that the end of your finger is charged after dragging your feet along the carpet. 
\begin{parts}
\part Estimate the breakdown voltage in air for your finger. Air will breakdown and conduct current if the electric field exceeds \SI{3e6}{N/C}
\part About what surface charge density would have to be on your finger at this estimated voltage?
\end{parts}
\begin{finalanswer}
\begin{enumerate}[(a)]
\item \SI{15000}{V}
\item $\SI{2.66e-5}{C/m^2}$
\end{enumerate}
\end{finalanswer}
\begin{solution}
\begin{parts}
\part The width of the end of a finger is about \SI{1}{cm}, and so consider the fingertip to be a part of a sphere of diameter \SI{1}{cm}. We assume that the electric field at the surface of the sphere is the minimum value that will produce breakdown in air.
The electric field at the surface of a charged sphere of radius $R$ carrying charge $Q$ is given by:
\begin{align*}
E = \frac{1}{4\pi\epsilon_0}\frac{Q}{R^2}
\end{align*}
The voltage at the surface of the sphere (assume \SI{0}{V} at infinity) is given by:
\begin{align*}
V = \frac{1}{4\pi\epsilon_0}\frac{Q}{R} = ER
\end{align*}
Thus, the breakdown voltage at the surface of your finger can be related to the breakdown electric field strength:
\begin{align*}
V_{surface}=RE_{breakdown}=(\SI{0.005}{m})(\SI{3e6}{V/m})=\SI{15000}{V}\\
\end{align*} 
Note: Since this is just an estimate, we might expect anywhere from \SI{10000}{V} to \SI{20000}{V}.

\part Since we know the potential (or the electric field), if we assume that all of the charge is on the surface of your finger, we have:
\begin{align*}
V &= \frac{1}{4\pi\epsilon_0}\frac{Q}{R}\\
\therefore Q &= 4\pi\epsilon_0 VR
\end{align*}
Since surface charge density is the charge divided by the surface of the sphere:
\begin{align*}
\sigma &= \frac{Q}{4\pi R^2}=\frac{\epsilon_0V}{R}\\
&=\frac{(\SI{8.85e-12}{C^2/N/m^2})(\SI{15000}{V})}{(\SI{0.005}{m})}=\SI{2.66e-5}{C/m^2}
\end{align*}
\end{parts}
 
\end{solution}

%Very modified past 104/106 problem, AB McLean
\question A Geiger counter is used to detect x-rays and gamma rays. The counter is a gas-filled metal cylinder with a diameter $D = \SI{2.0}{cm}$. Along the axis of the cylinder, there is a thin wire that is stretched so that it is taut. The wire has a diameter $d = \SI{0.013}{cm}$. When an x-ray or gamma ray enters the tube, passing through the walls, and ionizes a gas molecule, the electron moves towards the wire, which is biased at a positive potential relative to the cylinder, and the molecule moves towards the cylinder. The electron is accelerated by the field and it ionizes more gas molecules on its way to the wire, causing a pulse that can easily be measured. 
\begin{parts}
\part If we apply a positive voltage difference $\Delta V$ between the wire and the cylinder (the wire is at higher potential, with positive charge on it, the cylinder is at ground, \SI{0}{V}), show that the electric field between the wire and cylinder, at a radial distance $r$ from the centre of the wire, is given by:
\begin{align*}
\vec E(r)&=\frac{\Delta V}{r\ln\left(\frac{D}{d} \right)}\hat r
\end{align*}
\part If \SI{850}{V} are applied between the wire and the cylinder, what is the magnitude of the electric field just above the surface of the wire?
\part If \SI{850}{V} are applied between the wire and the cylinder, what is the charge per unit length on the wire?
\end{parts}
\begin{finalanswer}
\begin{enumerate}[(a)]
\item N/A
\item $E=\SI{2.596712e6}{N/C}$
\item $\lambda=\SI{9.39e-9}{C/m}$
\end{enumerate}
\end{finalanswer}
\begin{solution}
\begin{parts}
\part Let the wire carry positive charge per unit length equal to $\lambda$. We can easily use Gauss' law to obtain the field as a function of radius. Consider a cylindrical surface centred on the wire with length $l$ and radius $r$. The flux through the end caps is zero, and the electric field is constant and perpendicular to the curved surface. The charge enclosed is $\lambda l$. Gauss' law gives:
\begin{align*}
E(r)(2\pi r l) &= \frac{\lambda l}{\epsilon_0}\\
\therefore E(r)&=\frac{\lambda}{2\pi\epsilon_0 r}
\end{align*}
Using the electric field, we can now find the potential difference between points at $r=\frac{d}{2}$ and $r=\frac{D}{2}$. We choose to integrate from small radius to large radius, and note that $\vec E$ and $d\vec r$ are in the same direction, and so the potential difference between the large radius and the small radius is given by:
\begin{align*}
V\left(\frac{D}{2}\right) - V\left(\frac{d}{2}\right) &=-\int_\frac{d}{2}^\frac{D}{2}\vec E(r)\cdot d\vec r= -\int_\frac{d}{2}^\frac{D}{2}E(r)dr=-\int_\frac{d}{2}^\frac{D}{2}\frac{\lambda}{2\pi\epsilon_0 r}dr\\
&=-\frac{\lambda}{2\pi\epsilon_0}\int_\frac{d}{2}^\frac{D}{2}\frac{1}{r}=-\frac{\lambda}{2\pi\epsilon_0}\left[\ln(r)\right]_\frac{d}{2}^\frac{D}{2}=-\frac{\lambda}{2\pi\epsilon_0}\ln\left(\frac{D}{d} \right)\\
\end{align*}
Since $D>d$, the logarithm is positive and the voltage is thus negative. This is the opposite of the voltage that we want, namely, $\Delta V=V(\frac{d}{2})-V(\frac{D}{2})$, so we just remove the negative sign to get the \textit{positive} voltage between the wire and the cylinder:
\begin{align*}
\Delta V=V\left(\frac{d}{2}\right)-V\left(\frac{D}{2}\right)&=\frac{\lambda}{2\pi\epsilon_0}\ln\left(\frac{D}{d} \right)
\end{align*}
We can isolate the unknown charge density $\lambda$ in the expressions for the electric field and for the voltage and equate them:
\begin{align*}
\lambda = E(r)(2\pi\epsilon_0 r) &=\Delta V\left( \frac{2\pi\epsilon_0}{\ln\left(\frac{D}{d} \right)}\right)\\
\therefore E(r)&=\frac{\Delta V}{r\ln\left(\frac{D}{d} \right)}
\end{align*}
\part This is just a matter of plugging numbers into the formula from part a):
\begin{align*}
E(r=d)&=\frac{\Delta V}{d\ln\left(\frac{D}{d} \right)}=\frac{(\SI{850}{V})}{(\SI{0.000065}{m})\ln\left(\frac{(\SI{0.01}{m})}{(\SI{0.000065}{m})} \right)}&=\SI{2.596712e6}{N/C}
\end{align*}
\part From part a, we have:
\begin{align*}
\lambda&= \Delta V\left( \frac{2\pi\epsilon_0}{\ln\left(\frac{D}{d} \right)}\right)\\
&=(\SI{850}{V})\left( \frac{2\pi(\SI{8.85e-12}{C^2/N/m^2})}{\ln\left(\frac{(\SI{0.01}{m})}{(\SI{0.000065}{m})} \right)}\right)\\
&=\SI{9.39e-9}{C/m}
\end{align*}
\end{parts}
\end{solution}

%eport 18
\question A solid sphere of radius $R_1$ carries a total positive charge $+Q$. A thin metallic spherical shell of radius $R_2$ ($R_2 = 2 R_1$) surrounds the sphere ($R_2>R_1$) and carries a total positive charge $+2Q$. An electron is placed halfway between the sphere and the shell. What speed will the electron have when it collides with the sphere or the shell, and which object will it collide with?
\begin{finalanswer}
The electron will collide with the sphere,
\begin{align*}
v=\sqrt{\frac{eQ}{6\pi\epsilon_0R_1m_e}}
\end{align*}
\end{finalanswer}
\begin{solution}
The field between the sphere and shell only depends on the sphere (Gauss’ Law, charge enclosed). The potential in that region is identical to that from a point charge. The electron will \textbf{collide with the sphere} and its potential energy will change by:
\begin{align*}
\Delta U &= U_2-U_1=\frac{eQ}{4\pi\epsilon_0}\left(\frac{1}{R_1}-\frac{1}{\frac{3}{2}R_1} \right)\\
&=\frac{eQ}{12\pi\epsilon_0R_1}
\end{align*}
This is equal to the change in kinetic energy:
\begin{align*}
\Delta U &= \Delta K\\
\frac{eQ}{12\pi\epsilon_0R_1} &= \frac{1}{2}m_ev^2\\
\therefore v &=\sqrt{\frac{eQ}{6\pi\epsilon_0R_1m_e}}
\end{align*}
\end{solution}
 %%% Not complete beyond here!!!

\question The volume charge density $\rho(r)$ within a sphere of radius $R$ is distributed in accordance with the following spherically symmetric relation:
\begin{align*}
\rho(r)=\rho_0\left(1-\frac{r^{2}}{R^{2}}\right)\\
\end{align*}
where $r$ is measured from the centre of the sphere and $\rho_0$ is a constant.
\begin{parts}
\part Find the electric field as a function of distance $r$ from the centre of the sphere inside and outside of the sphere.
\part Find the electric potential as a function of distance $r$ from the centre of the sphere inside and outside of the sphere, assuming that the surface of the sphere is at \SI{100}{V}.
\end{parts}
\begin{finalanswer}
\begin{enumerate}[(a)]
\item \begin{align*}
\vec E(r)&=\frac{2}{15\epsilon_0r^2}\rho_0R^3\hat r \qquad (r\geq R)
\vec E(r)&=\frac{\rho_0}{\epsilon_0}\left( \frac{r}{3}-\frac{r^3}{5R^2}\right) \hat r \qquad (r\leq R)
\end{align*}
\item \begin{align*}
V(r) &=(\SI{100}{V})+\frac{2}{15 \epsilon_0}\rho_0R^3\left( \frac{1}{r}-\frac{1}{R}\right)\quad(r\geq R)\\
V(r) &=(\SI{100}{V})+\frac{\rho_0}{\epsilon_0}\left(\frac{7R^2}{60} -\frac{r^2}{6}+\frac{r^{4}}{20R^{2}}    \right)\quad(r< R)
\end{align*}
\end{enumerate}
\end{finalanswer}
\begin{solution}
\begin{parts}
\part The electric field outside of the sphere is easily found using Gauss' Law and the total charge of the sphere. The total charge of the sphere can be found by considering a thin spherical shell at radius $r$ with thickness $dr$ and volume $dV=4\pi r^2 dr$. The total charge of the sphere is the sum of the charges of each spherical shell, $dq=\rho dV$:
\begin{align*}
Q&=\int dq = \int \rho(r)dV=\int_0^R\rho_0\left(1-\frac{r^{2}}{R^{2}}\right)4\pi r^2 dr\\
&=4\pi\rho_0\int_0^R\left(r^2-\frac{r^{4}}{R^{2}}\right)dr=4\pi\rho_0\left[\frac{r^3}{3}-\frac{r^{5}}{5R^{2}}\right]_0^R\\
&=4\pi\rho_0\left(\frac{R^3}{3}-\frac{R^{3}}{5}\right)=\frac{8}{15}\pi\rho_0R^3
\end{align*}
The electric field outside the sphere, for $r\geq R$, is thus given by:
\begin{align*}
\vec E(r)=\frac{Q}{4\pi\epsilon_0r^2}\hat r= \frac{1}{4\pi \epsilon_0r^2}\frac{8}{15}\pi\rho_0R^3\hat r=\frac{2}{15 \epsilon_0r^2}\rho_0R^3\hat r\quad(r\geq R)
\end{align*}
Inside the sphere ($r<R$), we use the same strategy, but the charge enclosed at some radius $r$ will be smaller, so we simply change the limits of integration (and use $r'$ for the integration variable):
\begin{align*}
Q^{enc}(r)&=\int_0^r\rho_0\left(1-\frac{r'^{2}}{R^{2}}\right)4\pi r'^2 dr'\\
&=4\pi\rho_0\left(\frac{r^3}{3}-\frac{r^{5}}{5R^{2}}\right)
\end{align*}
Again, using a Gaussian spherical surface of radius $r$, the electric field inside the sphere is given by:
\begin{align*}
\vec E(r)&=\frac{Q^{enc}}{4\pi\epsilon_0r^2}\hat r=\frac{1}{4\pi\epsilon_0r^2}4\pi\rho_0\left(\frac{r^3}{3}-\frac{r^{5}}{5R^{2}}\right)\hat r\\
&=\frac{\rho_0}{\epsilon_0}\left(\frac{r}{3}-\frac{r^{3}}{5R^{2}}\right)\hat r\quad(r< R)
\end{align*}
\part For the electric potential, we can calculate the potential difference between a point at position $r$ and the surface of the sphere, which is at $V_S=\SI{100}{V}$. In general, this is given by:
\begin{align*}
\Delta V = -\int \vec E(r)\cdot d\vec r
\end{align*}
For $r\geq R$, the field was given above, and we integrate from the surface of the sphere ($r'=R$) to the position $r'=r$, so that that $\vec E$ and $d\vec r$ are parallel:
\begin{align*}
\Delta V &=V(r)-V_S=-\int_R^r \vec E(r')\cdot d\vec r'=-\int_R^r E(r')dr'\\
&=-\int_R^r \frac{2}{15 \epsilon_0r'^2}\rho_0R^3 dr'=-\frac{2}{15 \epsilon_0}\rho_0R^3\int_R^r \frac{1}{r'^2} dr'\\
&=\frac{2}{15 \epsilon_0}\rho_0R^3\left[ \frac{1}{r'}\right]_R^r=\frac{2}{15 \epsilon_0}\rho_0R^3\left( \frac{1}{r}-\frac{1}{R}\right)\\
\therefore V(r)&=V_S+\frac{2}{15 \epsilon_0}\rho_0R^3\left( \frac{1}{r}-\frac{1}{R}\right)\\
&=(\SI{100}{V})+\frac{2}{15 \epsilon_0}\rho_0R^3\left( \frac{1}{r}-\frac{1}{R}\right)\quad(r\geq R)
\end{align*}
For $r<R$, we use the electric field inside of the sphere and integrate from $r<R$ to $R$, so that the electric field is parallel to the direction of integration:
\begin{align*}
\Delta V &=V_S-V(r)=-\int_r^R \vec E(r')\cdot d\vec r'=-\frac{\rho_0}{\epsilon_0}\int_r^R \left(\frac{r'}{3}-\frac{r'^{3}}{5R^{2}}\right)dr'\\
&=-\frac{\rho_0}{\epsilon_0} \left[\frac{r'^2}{6}-\frac{r'^{4}}{20R^{2}}\right]_r^R = -\frac{\rho_0}{\epsilon_0}\left(\frac{R^2}{6}-\frac{R^2}{20} -\frac{r^2}{6}+\frac{r^{4}}{20R^{2}}    \right)\\
&=-\frac{\rho_0}{\epsilon_0}\left(\frac{7R^2}{60} -\frac{r^2}{6}+\frac{r^{4}}{20R^{2}}    \right)\\
\therefore V(r) &= V_S+\frac{\rho_0}{\epsilon_0}\left(\frac{7R^2}{60} -\frac{r^2}{6}+\frac{r^{4}}{20R^{2}}    \right)\\
&=(\SI{100}{V})+\frac{\rho_0}{\epsilon_0}\left(\frac{7R^2}{60} -\frac{r^2}{6}+\frac{r^{4}}{20R^{2}}    \right)\quad(r< R)
\end{align*}
\end{parts}
\end{solution}

%Zaremba 2002 Final exam
\question A proton makes a head-on elastic collision with a nucleus at rest and rebounds with a speed which is nine-tenths of its initial speed. The mass of the proton is $m = \SI{1.673e-27}{kg}$ and its charge is $e=\SI{1.6e-19}{C}$.
\begin{parts}
\part What is the mass of the nucleus?
\part  If the nucleus has a charge +9e and the proton has an initial energy of \SI{1}{MeV}, what is the distance of closest approach between the proton and nucleus?
\end{parts}
\begin{finalanswer}
\begin{enumerate}[(a)]
\item $M=\SI{3.18e-26}{kg}$
\item $d=\SI{1.363e-14}{m}$
\end{enumerate}
\end{finalanswer}
\begin{solution}
\begin{parts}
\part We use conservation of energy and momentum in one dimension. If the initial velocity of the proton is $v$, then its final velocity is $-\frac{9}{10}v$. Let the nucleus' final velocity be $v'$. Conservation of momentum gives:
\begin{align*}
mv &= -\frac{9}{10}mv + Mv'\\
mv \left(1+\frac{9}{10}\right)&=Mv'\\
\end{align*}
Conservation of energy gives:
\begin{align*}
\frac{1}{2}mv^2&= \frac{1}{2}m\left(\frac{9}{10}v\right)^2 + \frac{1}{2}Mv'^2\\
\therefore mv^2(1-\left(\frac{9}{10}\right)^2)&=Mv'^2\\
mv^2\left(1-\frac{9}{10}\right)\left(1+\frac{9}{10}\right)&=Mv'^2
\end{align*}
Dividing this by the momentum equation gives:
\begin{align*}
v\left(1-\frac{9}{10}\right)=v'
\end{align*}
Substituting the above expression for $v'$ back into the momentum equation, we find
\begin{align*}
mv \left(1+\frac{9}{10}\right)&=Mv'\\
mv \frac{19}{10}&=Mv\left(1-\frac{9}{10}\right)\\
m \frac{19}{10}&=M\frac{1}{10}\\
\therefore M&=19m=19(\SI{1.673e-27}{kg})=\SI{3.18e-26}{kg}
\end{align*}
\part Initially, we can consider that the proton is very far away from the nucleus at rest. As the proton approaches the nucleus, the proton will slow down, and the nucleus will start to be repelled by the proton. At the point of closest approach, there is a frame of reference where it will look as if the two particles are at rest. This means that in the original frame of reference, they will have the same velocity, say $v'$, which corresponds to the speed of the frame of reference where the two particles appear momentarily at rest. 

The initial kinetic energy of the proton (initial speed $v$) will be converted into the kinetic energy of the proton and nucleus (moving at the same speed, $v'$) and the electric potential energy between the two particles:
\begin{align*}
\frac{1}{2}mv^2=\frac{1}{2}(m+M)v'^2+\frac{1}{4\pi\epsilon_0}\frac{9e^2}{d}
\end{align*}
Conservation of momentum gives:
\begin{align*}
mv &= (m+M)v'\\
\therefore v'&=\frac{1}{20}v
\end{align*}
where we used the fact that $M=19m$, from part a. Subbing this into the conservation of energy equation:
\begin{align*}
\frac{1}{2}mv^2&=\frac{1}{2}(m+M)v'^2+\frac{1}{4\pi\epsilon_0}\frac{9e^2}{d}\\
&=10m\left(\frac{v}{20}\right)^2+\frac{1}{4\pi\epsilon_0}\frac{9e^2}{d}\\
&=\frac{1}{40}mv^2+\frac{1}{4\pi\epsilon_0}\frac{9e^2}{d}\\
\therefore d&= \frac{40}{19}\frac{1}{4\pi\epsilon_0}\frac{9e^2}{mv^2}\\
&= \frac{5}{19}\frac{1}{\pi\epsilon_0}\frac{9e^2}{\left(\frac{1}{2}mv^2\right)}\\
&=\frac{45}{19}\frac{1}{\pi\epsilon_0}\frac{e^2}{\left(\frac{1}{2}mv^2\right)}\\
\end{align*}
where we factored out the kinetic energy of the proton, since it is given as $\frac{1}{2}mv^2=\SI{1}{MeV}=(\SI{1}{MeV})(\SI{1.6e-13}{J/MeV})$. Plugging in the values to get the distance of closest approach:
\begin{align*}
d&=\frac{45}{19}\frac{1}{\pi\epsilon_0}\frac{e^2}{\left(\frac{1}{2}mv^2\right)}\\
&=\frac{45}{19}\frac{1}{\pi(\SI{8.85e-12}{C^2/N/m^2})}\frac{(\SI{1.6e-19}{C})^2}{(\SI{1}{MeV})(\SI{1.6e-13}{J/MeV})}\\
&=\SI{1.363e-14}{m}
\end{align*}
\end{parts}
\end{solution}

%eport 18
\question The electric potential in a region of space is given as:
\begin{align*}
V(x,y,z) = \frac{by}{a^2+y^2}
\end{align*}
where $a$ and $b$ are constants.
\begin{parts}
\part Give an expression for the electric field vector in that region of space.
\part If $a=\SI{1}{m}$ and $b=\SI{1}{Vm}$, plot the magnitude of the electric field and the value of the electric potential between $y=0$ and $y=\SI{10}{m}$. Make sure that your plots have labels and units on the axes.
\part If a charge is placed in this field ($a=\SI{1}{m}$ and $b=\SI{1}{Vm}$), will there be a point where it is in equilibrium? If yes, where is the equilibrium point and what type of equilibrium is it?
\end{parts}
\begin{finalanswer}
\begin{enumerate}[(a)]
\item $E_y=\frac{b(y^2-a^2)}{(a^2+y^2)^2}$
\item \capfig{0.8\textwidth}{figures/Potential/EVPlots.png}{\label{fig:potential:EVPlots2}Electric field and potential.}
\item Unstable equilibrium at $y=\SI{1}{m}$
\end{enumerate}
\end{finalanswer}
\begin{solution}
\begin{parts}
\part Since the potential only depends on $y$, the electric field will be in the $y$ direction, with magnitude given by:
\begin{align*}
E_y &= -\frac{\partial}{\partial y} V(x,y,z)\\
&=-\frac{b(a^2+y^2)-2by^2}{(a^2+y^2)^2}=\frac{b(y^2-a^2)}{(a^2+y^2)^2}
\end{align*}
\part We can easily plot this in python, which is shown in Figure \ref{fig:potential:EVPlot}
\begin{verbatim}
import pylab as pl
import numpy as np

#We define functions to plot the potential
#and electric field (with a=1, b=1)
def V(y):
    return y/(y**2+1)

def E(y):
    return (y**2-1)/(y**2+1)**2

#define 100 values of y between 0 and 10
y = np.linspace(0,10,100)

#plot both on the same figure as sub plots
pl.figure(figsize=(12,4))
pl.subplot('121')
pl.plot(y,E(y))
pl.ylabel('E [V/m]')
pl.xlabel('position [m]')
pl.title('Electric Field')
pl.subplot('122')
pl.plot(y,V(y))
pl.ylabel('V [V]')
pl.xlabel('position [m]')
pl.title('Electric Potential')
pl.show()
\end{verbatim}
\capfig{0.8\textwidth}{figures/Potential/EVPlots.png}{\label{fig:potential:EVPlots}Electric field and potential.}
\part As can be seen in the plots, there is a point where the electric field is zero and the slope of the potential is zero. Since this is a maximum in the potential, it is an \textbf{unstable} equilibrium. The point is given by setting the electric field to zero:
\begin{align*}
E_y=\frac{b(y^2-a^2)}{(a^2+y^2)^2}&=0\\
(\SI{1}{Vm})\frac{(y^2-(\SI{1}{m}))^2}{((\SI{1}{m}))^2+y^2)^2}&=0\\
\therefore y&=\SI{1}{m}
\end{align*}
\end{parts}
\end{solution}

\question \label{q:potential:E_field_from_rod} A rod made of an insulating material has a length, $L$, and carries a total charge $+Q$. The rod is placed along the x-axis, such that one end is at the origin and the other end is at a position $x=-L$ (Figure \ref{fig:potential:ChargedRod}).
\capfig{0.3\textwidth}{figures/Potential/ChargedRod.png}{\label{fig:potential:ChargedRod} Charged rod of length $L$ carrying charge $Q$, Question \ref{q:potential:E_field_from_rod}.}
\begin{parts}
\part  Give an expression for the electric field vector at a point $x>0$ along the x-axis, as shown.
\part  How much work is required to bring a positive charge $q$ from infinity and place it at a position $x>0$ along the x-axis?
\end{parts}
\textbf{Hint:} If you need to integrate $\int_a^b (u+c)^n du$, where $c$ is a constant, you can use substitution: for example, let $v = u+c$ (so that $dv=du$), change the limits of the integral, and integrate $\int_{a+c}^{b+c}v^ndv$.
\begin{finalanswer}
\begin{enumerate}[(a)]
\item \begin{align*}
E=k\frac{Q}{L}\left[\frac{1}{x}-\frac{1}{L+x}\right]
\end{align*}
\item The work done is positive and equal to \begin{align*}
W=k\frac{qQ}{L}\ln\left( \frac{L+x}{x} \right)
\end{align*}
\end{enumerate}
\end{finalanswer}
\begin{solution}
a) The field is in the positive x-direction, so we calculate its magnitude. Let $a=0$ at $x=0$ and $a=+L$ when $x=-L$ (positive $a$ in the negative $x$ direction), and use $a$ as our variable of integration:
\begin{align*}
E = \int_{0}^{L}k\frac{\lambda da}{(a+x)^2}
\end{align*}
Integrate by letting $b=a+x$, $da=db$, change limits of integration:
\begin{align*}
E = \int_{x}^{L+x}k\frac{\lambda db}{b^2}=k\lambda \left[-\frac{1}{b}\right]_{x}^{L+x}=k\frac{Q}{L}\left[\frac{1}{x}-\frac{1}{L+x}\right]
\end{align*}
b) Calculate potential at x in basically the same way, assuming $V=0$ at infinity:
\begin{align*}
V &= \int_{0}^{L}k\frac{\lambda da}{(a+x)} = \int_{x}^{L+x}k\frac{\lambda db}{b}\\
&=k\frac{Q}{L}\left[\ln(b)\right]_{x}^{L+x}=k\frac{Q}{L}\ln\left( \frac{L+x}{x} \right)
\end{align*}
The work done is positive, and equal to:
\begin{equation*}
W=qV=k\frac{qQ}{L}\ln\left( \frac{L+x}{x} \right)
\end{equation*}

\end{solution}

\question \label{q:potential:dipole_design} You need to build a demonstration of an electric dipole oscillating in a uniform electric field.
\begin{parts}
\part  In order to create a uniform electric field, you use two very large parallel metal plates that are $d=5$\,cm apart and connect them to a DC generator that can provide $\Delta V=110$\,V. How strong will the electric field be between the plates?
\part  You build the dipole using two spherical masses (each with mass $m_s=1$\,g and radius $R=2.5$\,mm) connected by a thin rod of mass $m_r=2$\,g and length $L=1$\,cm (the rod is inserted into the spheres, so that the centres of the spheres are 1\,cm apart). What is the moment of inertia, $I$, of the dipole about an axis that goes through the middle of the rod and is perpendicular to the rod?
\part The centre of the rod connecting the two spheres is held fixed, allowing the dipole to rotate freely about the centre of the rod. For small angles between the dipole vector and the electric field, the electric dipole will undergo simple harmonic motion. What is the magnitude of the charge on each sphere for the frequency of the small angle oscillations of the dipole to be 5\,Hz?
\end{parts}
\begin{finalanswer}
\begin{enumerate}[(a)]
\item $E=\SI{2200}{V/m}$
\item $I=\SI{7.167e-8}{kgm^2}$
\end{enumerate}
\end{finalanswer}
\begin{solution}
a)$E=\frac{V}{d}=\frac{110\,V}{0.05\,m}=2200\,V/m$\\
b)For the two spheres, using parallel axis theorem:
\begin{align*}
I_s&=2 \left(\frac{2}{5}m_s R^2+m_s \frac{L^2}{4}\right)\\
&=2\left(\frac{2}{5}(0.001\,kg) (0.0025\,m)^2+(0.001\,kg) \frac{(0.01\,m)^2}{4}\right)\\
&=5.5\times 10^{-8}\,kg m^2
\end{align*}
Adding that to the moment of inertia of the rod:
\begin{align*}
I_{tot}&=\frac{1}{12}m_r L^2+I_s\\
&=\frac{1}{12}(0.002\,kg) (0.01)^2+5.5\times 10^{-8}\,kg m^2\\
&=7.167\times 10^{-8}\,kg m^2
\end{align*}
c)The torque provided by electric field tries to reduce the angle (negative angular acceleration)
\begin{align*}
\tau = pE\sin\theta = QLE\sin\theta = - I\frac{d^2\theta}{dt^2}\\
\end{align*}
For small angles, this is SHM:
\begin{align*}
QlE\sin\theta &\approx QLE\theta = -I\frac{d^2\theta}{dt^2}\\
\therefore \frac{d^2\theta}{dt^2}&=-\frac{QLE}{I}\theta
\end{align*}
with angular frequency:
\begin{align*}
\omega^2=\frac{QLE}{I}
\end{align*}
Re-arranging for $Q$ and using $\omega=2\pi f$:
\begin{align*}
4\pi^2 f^2&=\frac{QLE}{I}\\
Q &= \frac{4\pi^2 f^2 I}{LE}=\frac{4\pi^2 (5\,Hz)^2 (7.167\times 10^{-8}\,kg m^2)}{(0.01\,m)(2200V/m)}\\
&=3.215 \times 10^{-6}\,C
\end{align*}
\end{solution}

\question \label{q:potential:E_field_from_rod} A semi-infinite rod made of an insulating material has a charge per unit length $\lambda$. The rod is placed along the x-axis, such that one end is at the origin and the other end is at negative infinity (Figure \ref{fig:potential:ChargedInfiniteRod}).
\capfig{0.3\textwidth}{figures/Potential/ChargedInfiniteRod.png}{\label{fig:potential:ChargedInfiniteRod} Semi-infinite charged rod, Question \ref{q:potential:E_field_from_rod}.}
\begin{parts}
\part  Give an expression for the electric field vector at a point $x>0$ along the x-axis, as shown.
\part  How much work is required to move a positive charge, $q$, from a position $x=a$ to a position $x=b$ along the x-axis? Note that both $a$ and $b$ are positive.
\end{parts}
\textbf{Hint:} If you need to integrate $\int_a^b (u+c)^n du$, where $c$ is a constant, you can use substitution: for example, let $v = u+c$ (so that $dv=du$), change the limits of the integral, and integrate $\int_{a+c}^{b+c}v^ndv$.
\begin{finalanswer}
\begin{enumerate}[(a)]
\item \begin{align*}
E=k\lambda\left[\frac{1}{x}\right]
\end{align*}
\item \begin{align*}
W=kq\lambda\ln\left( \frac{b}{a} \right)
\end{align*}
\end{enumerate}
\end{finalanswer}
\begin{solution}
a) The field is in the positive x-direction, so we calculate its magnitude. Let $a=0$ at $x=0$ and $a=+\infty$ when $x=-\infty$ (positive $a$ in the negative $x$ direction), and use $a$ as our variable of integration:
\begin{align*}
E = \int_{0}^{L}k\frac{\lambda da}{(a+x)^2}
\end{align*}
Integrate by letting $b=a+x$, $da=db$, change limits of integration:
\begin{align*}
E = \int_{x}^{\infty}k\frac{\lambda db}{b^2}=k\lambda \left[-\frac{1}{b}\right]_{x}^{+\infty}=k\lambda\left[\frac{1}{x}\right]
\end{align*}
b) Calculate the work done using the electric field from above:
\begin{align*}
W &= \int_{a}^{b}\vec F\cdot d\vec x=\int_{a}^{b}kq\lambda\frac{1}{x} dx= kq\lambda\ln\left( \frac{b}{a} \right)
\end{align*}
\end{solution}

\question An electron (mass, $m_e$, charge, $-e$) is placed at rest at a distance $d$ from the midpoint of a uniformly negatively charged rod of length $L$ holding total charge $-Q$, as shown in Figure \ref{fig:potential:erod}. What will be the speed of the electron once it reaches infinity (i.e. an infinite distance away from the rod)?
\capfig{0.2\textwidth}{figures/Potential/erod.png}{\label{fig:potential:erod}An electron about to be accelerated by a uniformly charged negative rod.}
\begin{solution}
We can model this problem using electric potential and conservation of energy. We set the potential to be zero at infinity, and calculate the electric potential from the rod at the position where the electron starts. From there, we will be able to determine the change in potential energy between that position and infinity in order to determine the corresponding change in kinetic energy.

We can determine the potential by modelling the rod as a collection of point charges, $dq$, each of length $dy$, as shown in Figure \ref{fig:potential:erod_sol}.
\capfig{0.2\textwidth}{figures/Potential/erod_sol.png}{\label{fig:potential:erod_sol}Modelling the potential from a charged rod.}
Each charge can be described in terms of the (negative) charge per unit length of the rod, $\lambda$, and an infinitesimal length, $dy$, along the rod:
\begin{align*}
dq=\frac{-Q}{L}dy=\lambda dy
\end{align*}
The potential at the position of the electron from that infinitesimal charge $dq$ at position $y$ is given by:
\begin{align*}
dV=k\frac{dq}{r}=\frac{kdq}{\sqrt{d^2+y^2}}
\end{align*}
The total potential is then given by:
\begin{align*}
V&=\int dV=k\lambda\int_0^L\frac{dy}{\sqrt{d^2+y^2}}\\
&=k\lambda\left[\ln\left(\sqrt{y^2+d^2} +y \right)  \right]_0^L\\
&=-k\frac{Q}{L}\ln\left(\frac{\sqrt{L^2+d^2} +L}{d} \right) 
\end{align*}
We can now calculate the speed of the electron using conservation of energy:
\begin{align*}
\Delta K &= -\Delta U\\
\frac{1}{2}m_ev^2 &= -e (-\Delta V) = \frac{kQe}{L}\ln\left(\frac{\sqrt{L^2+d^2} +L}{d} \right)   \\
\therefore v &= \sqrt{\frac{2kQe}{Lm_e}\ln\left(\frac{\sqrt{L^2+d^2} +L}{d} \right)}
\end{align*}

\end{solution}

\question An electron is accelerated horizontally from rest by a potential difference of $\Delta V_1=\SI{5000}{V}$. It then passes between two horizontal plates with a length, $L=\SI{20}{cm}$, and a distance, $d=\SI{2.0}{cm}$, apart that have a potential difference of $\Delta V_2=\SI{100}{V}$ between them, as shown in Figure \ref{fig:potential:eplates2}. What is the vertical displacement of the electron, $h$, as it exits the region of electric field? Ignore gravity.
\capfig{0.4\textwidth}{figures/Potential/eplates2.png}{\label{fig:potential:eplates2}An electron deflected by charged plates.}
\begin{finalanswer}
	$\SI{1.0}{cm}$
\end{finalanswer}
\begin{solution}
	The electron is accelerated by a potential difference, $\Delta V_1=\SI{5000}{V}$, allowing us to find the horizontal component of its velocity vector:
	\begin{align*}
	e\Delta V_1=\frac{1}{2}m_ev_1^2\\
	\therefore v_1=\sqrt{\frac{2e\Delta V_1}{m_e}}
	\end{align*}
	Once the electron is between the horizontal plates, it will experience a constant vertical force upwards, just like in projectile motion. The horizontal component of its velocity will remain unchanged. The vertical electric field between the plates is given by:
	\begin{align*}
	E=\frac{\Delta V_2}{d}
	\end{align*}
	and the vertical component of the acceleration, $a_y$, is thus:
	\begin{align*}
	F_y&=m_ea_y=eE=e\frac{\Delta V_2}{d}\\
	\therefore a_y&=e\frac{\Delta V_2}{dm_e}
	\end{align*}
	
	To find the vertical displacement when the electron exits the plates, we need to know for how long the electron was accelerated upwards. We can find how long the electron was between the plates by using the horizontal component of the velocity:
	\begin{align*}
	t=\frac{L}{v_1}=L\sqrt{\frac{m_e}{2e\Delta V_1}}
	\end{align*}
	
	The horizontal displacement, $h$, is then given by:
	\begin{align*}
	h &= \frac{1}{2}a_yt^2=\frac{1}{2}e\frac{\Delta V_2}{dm_e}L^2\frac{m_e}{2e\Delta V_1}\\
	&=\frac{1}{4}\frac{\Delta V_2}{\Delta V_1}\frac{L^2}{d}=\frac{1}{4}\frac{(\SI{100}{V})}{(\SI{5000}{V})}\frac{(\SI{0.2}{m})^2}{(\SI{2.0}{cm})}=\SI{1.0}{cm}
	\end{align*}
\end{solution}

\question A conducting sphere of radius $R_1$ carries a total charge $Q$ on its surface. A second, neutral, metallic sphere of radius $R_2$ is then connected to the first sphere using a conducting wire of negligible dimensions, so that some charge is transferred to the second sphere, as shown in Figure \ref{fig:potential:twospheres}. How much charge is left on each sphere after charges have transferred from the sphere of radius $R_1$ to that or radius $R_2$? Assume that the two spheres are far enough from each other that the charges on each sphere are distributed uniformly and that only negligible charge is left on the wire.
\textbf{Hint:} The spheres form an equipotential when connected together. 
\capfig{0.4\textwidth}{figures/Potential/twospheres.png}{\label{fig:potential:twospheres}A sphere of radius $R_1$ initially carrying charge $Q$ is connected to a sphere of radius $R_2$.}
\begin{finalanswer}
	\\
	$Q_1 &= \frac{Q}{1+\frac{R_2}{R_1}}$\\
	$Q_2 = Q\left( 1 -  \frac{1}{1+\frac{R_2}{R_1}} \right)$\\
	
\end{finalanswer}
\begin{solution}
	If we define electric potential to be zero at infinity, then the electric potential at the surface of a sphere of radius $r$ carrying charge $q$ is given by:
	\begin{align*}
	V=\frac{kq}{r}
	\end{align*}
	as this is the same as the potential a distance $r$ from a point charge at the centre of the sphere (Gauss' Law). If $Q_1$ is the charge on the sphere of radius $R_1$, and $Q_2$ the charge on that of radius $R_2$, then we require that:
	\begin{align*}
	Q_1 + Q_2 = Q
	\end{align*}
	Furthermore, since the two sphere must be at the same potential when connected, we have:
	\begin{align*}
	\frac{kQ_1}{R_1}&=\frac{kQ_2}{R_2}\\
	\therefore \frac{Q_1}{R_1}&=\frac{Q_2}{R_2}
	\end{align*}
	We can use the equation above to solve for $Q_2$:
	\begin{align*}
	Q_2 = Q_1\frac{R_2}{R_1}
	\end{align*}
	and substitute that into the first equation:
	\begin{align*}
	Q_1 \left(1+\frac{R_2}{R_1}\right)&=Q\\
	\therefore Q_1 &= \frac{Q}{1+\frac{R_2}{R_1}}
	\end{align*}
	Similarly (we really just need to invert the fraction $R_2/R_1$):
	\begin{align*}
	Q_2 &= \frac{Q}{1+\frac{R_1}{R_2}}
	\end{align*}
	or, equivalently:
	\begin{align*}
	Q_2 = Q-Q_1 =Q\left( 1 -  \frac{1}{1+\frac{R_2}{R_1}} \right)
	\end{align*}
	
\end{solution}

