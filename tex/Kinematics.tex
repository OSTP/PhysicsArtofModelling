\section{Kinematics}

%%%%%%%%%%%%%%%%%%%%%%%%%%%%%%%%%%%
%%
%% Multiple Choice
%%
%%%%%%%%%%%%%%%%%%%%%%%%%%%%%%%%%%%
\subsection{Multiple Choice}

\question The acceleration vector must point in the same direction as the velocity vector.
\begin{checkboxes}
\choice True
\CorrectChoice False \correct
\end{checkboxes}

\question A motor is rotating at \SI{420}{rpm} (rotations per minute). What does this correspond to in \si{Hz} (rotations per second)?
\begin{checkboxes}
\choice \SI{1.1}{Hz}
\CorrectChoice \SI{7}{Hz} \correct
\choice \SI{17.5}{Hz}
\choice \SI{25200}{Hz}
\end{checkboxes}

\question A block is moving at \SI{3}{m/s} westward.  It begins accelerating at \SI{0.75}{m/s^2} northward.  There is a wall \SI{16}{m} to the west, and a wall \SI{8}{m} to the north.  Which wall does it reach first?
\begin{checkboxes}
\choice Western wall
\CorrectChoice Northern wall \correct
\choice Same time for both walls
\end{checkboxes}


%Submitted by Emily Darling
\question A child, Alice, is dropped from rest at from a height of \SI{10}{m} while an identical child, Brittany, is thrown horizontally (from the same height) with a velocity of $v = (\SI{15}{m/s})\hat{i} + (\SI{24}{m/s})\hat{j}$, where the positive $y$ direction is defined as North, the positive $x$ direction is defined as East, and the positive $z$ direction is upward. Which child will hit the ground first?
\begin{checkboxes}
\choice Alice
\choice Brittany
\CorrectChoice Neither, they will hit the ground at the same time \correct
\choice I should not have children
\end{checkboxes}


\question Consider an object moving in 3 dimensions, whose acceleration in $x$ is given by $a_x (t)$. Integration of $a_x (t)$ with respect to $t$  yields (choose all that apply)
\begin{checkboxes}
\choice The $x$-component of the object's velocity.
\choice The $x$-distance covered by the object.
\CorrectChoice The change in the $x$-component of the \correct object's velocity.
\choice The time elapsed during the object's motion.
\CorrectChoice The $x$-velocity of the object, to within a constant of integration. \correct
\end{checkboxes}

\question The position vector of an object under uniform circular motion is given by \\ $\vec r(t)= \cos(\omega t) \hat i + \sin(\omega t)\hat{j}$. If it takes \SI{5}{s} to complete a single rotation and the object is along the $\hat{i}$ axis at $t=\SI{0}{s}$, what is its position after \SI{13.75}{s}?  
\begin{checkboxes}
\choice $-\hat{i}+0\hat{j}$
\CorrectChoice $0\hat{x}-\hat{y}$
\choice $-\hat{i}-\hat{j}$
\choice $0\hat{i}+\hat{j}$
\end{checkboxes}

\question A ball is thrown vertically upwards, then falls back down and hits the ground. The acceleration of the ball:
\begin{checkboxes}
\choice Points upwards when the ball is released from the hand throwing it, then points downwards as it falls to the ground.
\CorrectChoice Always points downwards. \correct
\choice Always points downwards, except at the vertex of the ball's trajectory where there is no acceleration because the ball is not moving.
\end{checkboxes}

\question An archer shoots an arrow straight upwards. He observes that it takes the arrow \SI{10.2}{s} to reach its maximum height before it begins to fall back down. How much time does he now have to move out of the way before the arrow hits him? Assume acceleration due to gravity is constant at \SI{9.8}{m/s} and that air resistance is negligible.
\begin{checkboxes}
\choice More than $10.2$ seconds
\CorrectChoice $10.2$ seconds \correct
\choice Less than $10.2$ seconds
\end{checkboxes}


%Amy Van Nest
\question A swimmer is standing on the South shore of a wide river that is flowing East at \SI{1}{m/s}. She wants to swim across the river and land directly across from where she started. She can swim with a speed of \SI{2}{m/s}. In which direction does she have to swim?
\begin{checkboxes} 
\choice \SI{60}{\degree} West of North
\choice \SI{45}{\degree} West of North
\CorrectChoice \SI{30}{\degree} West of North \correct
\choice \SI{60}{\degree} West of South
\end{checkboxes}


\question A feather and a hammer are dropped at the same time from the same height above the surface of a planet. They land on the surface of the planet at the same time.
\begin{checkboxes}
\choice The hammer had a larger acceleration than the feather.
\choice The feather had a larger acceleration than the hammer.
\CorrectChoice The hammer and feather had the same acceleration.
\choice Not enough information to tell.
\end{checkboxes}

\question A feather and a hammer are dropped at the same time from the same height above the surface of a planet. The hammer lands on the surface of the planet before the feather.
\begin{checkboxes}
\CorrectChoice The hammer had a larger acceleration than the feather.
\choice The feather had a larger acceleration than the hammer.
\choice The hammer and feather had the same acceleration.
\choice Not enough information to tell.
\end{checkboxes}

\question A rock and a feather are dropped at the same time from a height of \SI{1}{m} from the ground. The rock is assumed to fall with the acceleration due to gravity ($g=\SI{9.8}{m/s^2}$), whereas the feather has half of that acceleration. How much time elapses between when the rock and feather hit the ground?
\begin{choices} 
	\CorrectChoice \SI{0.187}{s}
	\choice  \SI{0.204}{s}
	\choice \SI{0.408}{s}
	\choice \SI{0.639}{s}
\end{choices}

\question Two bugs are sitting on a rotating disk (the disk is rotating about its axis of symmetry), at different distances from the center of the disk.
\begin{checkboxes}
\choice Both bugs have the same velocity.
\choice Both bugs have the same speed.
\CorrectChoice Both bugs have the same angular velocity.
\choice Both bugs have the same centripetal acceleration.
\end{checkboxes}

%%%%%%%%%%%%%%%%%%%%%%%%%%%%%%%%%%%
%
% long answer
%
%%%%%%%%%%%%%%%%%%%%%%%%%%%%%%%%%%%
\subsection{Long answers}

\question An airplane is flying at a speed of \SI{100}{\knot} (knots) relative to the air and pointing in the North-West direction (a heading of \SI{315}{\degree} if North is defined \SI{0}{\degree}). A strong wind, blowing towards the east at a speed of \SI{30}{\knot} (knots), changes the trajectory of the plane, with respect to the ground.
\begin{parts}
 \part What is the magnitude (in knots) and direction of the plane's velocity vector relative to the ground?
 \part Given that \SI{1}{\knot} corresponds to \SI{1.852}{km/h}, how much distance does the plane cover on the ground in \SI{10}{min}.
 \part A second plane flies with the same speed relative to the air, but with a heading of \SI{45}{\degree} (pointing North - East). The angle between the velocity vectors of the two planes as seen in the air is \SI{90}{\degree}. What is the angle between the velocity vectors of the two planes when the velocity is measured from the ground?
 
\end{parts}

\begin{finalanswer}
\begin{enumerate}[(a)]
\item \SI{81.6}{\knot}, \SI{29.9}{\degree} West of North (a heading of \SI{330.1}{\degree})
\item \SI{25.19}{km}
\item \SI{84.86}{\degree}
\end{enumerate}
\end{finalanswer}
\begin{solution}
\textbf{a)}
If you make North the direction of the positive y-axis, and East the direction of the positive x-axis, then the velocity of the plane, relative to the air, is:
\begin{align*}
\vec v_p^a = (\SI{100}{\knot})\cos(\SI{135}{\degree})\hat i + (\SI{100}{\knot})\sin(\SI{135}{\degree})\hat j
\end{align*}
The wind has a velocity vector, relative to the ground:
\begin{align*}
\vec v_w^g = (\SI{30}{\knot})\cos(\SI{0}{\degree})\hat i
\end{align*}
The velocity of the plane relative to the ground is found by adding these two vectors:
\begin{align*}
\vec v_p^g &= \vec v_p^a + \vec v_w^g\\
&=[(\SI{100}{\knot})\cos(\SI{135}{\degree})+(\SI{30}{\knot}) ]\hat i + (\SI{100}{\knot})\sin(\SI{135}{\degree})\hat j\\
&= (-\SI{40.71}{\knot})\hat i + (\SI{70.71}{\knot})\hat j
\end{align*}
The magnitude and angle, with respect to the x-axis are:
\begin{align*}
|| \vec v_p^g ||&= \sqrt{(\SI{40.71}{\knot})^2+(\SI{70.71}{\knot})^2} = \SI{81.6}{\knot}\\
\theta &= \tan^{-1}\left(\frac{(\SI{70.71}{\knot})}{(-\SI{40.71}{\knot})}\right) = \SI{119.93}{\degree}
\end{align*}
or, \SI{29.9}{\degree} West of North (a heading of \SI{330.1}{\degree}).

\textbf{b)} The plane's speed in \si{km/h} is $(\SI{81.6}{\knot})(\SI{1.852}{\knot^{-1}km/h})=\SI{151.12}{km/h}$. In \SI{10}{min}, the plane will cover a distance of $\frac{\SI{151.12}{km}}{6}=\SI{25.19}{km}$ (one sixth of the distance it would cover in \SI{1}{\hour}).

\textbf{c)}First, we need to find the velocity vector of the second plane:
\begin{align*}
\vec v_{p2}^g &= \vec v_{p2}^a + \vec v_w^g\\
&=[(\SI{100}{\knot})\cos(\SI{45}{\degree})+(\SI{30}{\knot}) ]\hat i + (\SI{100}{\knot})\sin(\SI{45}{\degree})\hat j\\
&= (\SI{100.71}{\knot})\hat i + (\SI{70.71}{\knot})\hat j
\end{align*}
The magnitude and angle, with respect to the x-axis are:
\begin{align*}
|| \vec v_{p2}^g ||&= \sqrt{(\SI{100.71}{\knot})^2+(\SI{70.71}{\knot})^2} = \SI{123.06}{\knot}\\
\theta &= \tan^{-1}\left(\frac{(\SI{70.71}{\knot})}{(\SI{100.71}{\knot})}\right) = \SI{35.07}{\degree}
\end{align*}
Since we know the angles, we can simply subtract them, to get:
\begin{align*}
\theta = \SI{119.93}{\degree} - \SI{35.07}{\degree} = \SI{84.86}{\degree}
\end{align*}
\end{solution}


\question Chlo\"e is out attempting to take a picture of an elusive and shy vicu\~na. Chlo\"e is jogging along a path in the North direction at a speed of \SI{10}{km/h}, while the vicu\~na is walking in the East direction at a speed of \SI{5}{km/h}. At one instant in time, $t=\SI{0}{s}$, the vicu\~na is directly North of Chlo\"e, at a distance of \SI{1}{km}. Assume that both Chlo\"e and the vicu\~na maintain constant velocity vectors.
\begin{parts}
\part At which time will the distance between Chlo\"e and the vicu\~na be minimal?
\part What will be the distance between Chlo\"e and the vicu\~na at that time?
\end{parts}

\begin{finalanswer}
\begin{enumerate}[(a)]
\item \SI{288}{s}
\item \SI{0.447}{km}
\end{enumerate}
\end{finalanswer}
\begin{solution}
\textbf{a)}
We can define positive $y$ as the direction in which Chlo\"e walks (North) and $x$ as the direction in which the vicu\~na walks (East). Furthermore, we choose to define the origin as the position of the vicu\~na at $t=\SI{0}{s}$. The positions vectors of Chlo\"e and the vicu\~na are:
\begin{align*}
\vec r^C(t)&=y^C(t)\hat j=[y^C_0+v^C_{y0}t]\hat j=[(\SI{-1.0}{km})+(\SI{10}{km/h})t]\hat j\\
\vec r^v(t)&=x^v(t)\hat i[v^v_{x0}t]\hat i=[(\SI{5}{km/h})t]\hat i\\
\end{align*}

We can define a displacement vector, between Chlo\"e and the vicu\~na:
\begin{align*}
\vec d(t) &=\vec r^v(t)-\vec r^C(t)\\
&=x^v(t)\hat i-y^C(t)\hat j
\end{align*}
whose magnitude is given by:
\begin{align*}
d(t)&=\sqrt{(x^v(t))^2+(y^C(t))^2}\\
&=\sqrt{[(\SI{5}{km/h})t]^2+[(\SI{-1.0}{km})+(\SI{10}{km/h})t]^2}
\end{align*}
The distance will be minimal when the derivative of $d(t)$ with respect to time is zero. Using the Chain Rule (twice), we have:
\begin{align*}
\frac{d}{dt}d(t)&=\frac{1}{2\sqrt{(x^v(t))^2+(y^C(t))^2}}\frac{d}{dt}\left[(x^v(t))^2+(y^C(t))^2\right]\\
&=\frac{1}{2\sqrt{(x^v(t))^2+(y^C(t))^2}}\left( 2x^v(t)\frac{d}{dt}x^v(t)+2y^C(t)\frac{d}{dt}y^C(t) \right)\\
&=\frac{1}{2\sqrt{(x^v(t))^2+(y^C(t))^2}}\left( 2x^v(t)v^v_{x0}+2y^C(t)v^C_{y0} \right)
\end{align*}
We only need to consider when the numerator is zero (the part in parenthesis), giving:
\begin{align*}
 0&=2x^v(t)v^v_{x0}+2y^C(t)v^C_{y0}\\
 &=2(\SI{5}{km/h})t(\SI{5}{km/h})+2((\SI{-1.0}{km})+(\SI{10}{km/h})t)(\SI{10}{km/h})\\
 &=(\SI{50}{km^2/h^2})t-(\SI{20}{km^2/h})+(\SI{200}{km^2/h})t\\
 &=(\SI{250}{km^2/h^2})t-(\SI{20}{km^2/h})\\
 \therefore t&=\frac{(\SI{20}{km^2/h})}{(\SI{250}{km^2/h^2})}=\SI{0.08}{\hour}=\SI{4.8}{min}=\SI{288}{s}
\end{align*}
\textbf{b)} We just use the value determined above to evaluate $d(t)$:
\begin{align*}
d(t=\SI{0.08}{hr})&=\sqrt{[(\SI{5}{km/h})(\SI{0.08}{hr})]^2+[(\SI{-1.0}{km})+(\SI{10}{km/h})(\SI{0.08}{hr})]^2}\\
&=\SI{0.447}{km}
\end{align*}
\end{solution}

%Giancolli 3-61 (modified) 
\question A child, who is \SI{45}{m} from the bank of a river is being carried by the river's current at \SI{1.0}{m/s}. A lifeguard jumps into the water just as the child passes by (at this point in time, the child is \SI{45}{m} from the life guard). The life  guard can swim at a speed of \SI{2.0}{m/s} relative to the water. 
\begin{parts}
\part How long does it take the lifeguard to reach the child?
\part What is the magnitude of the displacement vector of the lifeguard once she reaches the child?
\end{parts}

\begin{finalanswer}
\begin{enumerate}[(a)]
\item \SI{22.5}{s}
\item \SI{50.3}{m}
\end{enumerate}
\end{finalanswer}
\begin{solution}
\textbf{a)}
In the river, the life guard must cover \SI{45}{m} at a speed of \SI{2.0}{m/s}, which will take $t=(\frac{\SI{45}{m}})({\SI{2.0}{m/s}})=\SI{22.5}{s}$.

\textbf{b)} In \SI{22.5}{s}, the lifeguard has travelled $y=(\SI{1.0}{m/s})(\SI{22.5}{s})=\SI{22.5}{m}$ downstream due to the river's current.

The total displacement vector magnitude is thus $\sqrt{(\SI{45}{m})^2+(\SI{22.5}{m})^2}=\SI{50.3}{m}$
\end{solution}

%Giancolli 2-62 -fixed
\question Curious about how quickly you can cause water to spray from your garden hose, you decide to conduct an experiment to find out. You place your thumb over the hose, then hold it vertically upwards \SI{0.5}{m} above the ground. You shut off the water instantaneously (such that the stream does not slow before stopping) and count \SI{2.5}{s} seconds before you hear the water stop hitting the ground. Given these measurements, determine the speed of the water as it left the nozzle of the hose.

\begin{finalanswer}
\SI{12.1}{m/s}
\end{finalanswer}
\begin{solution}
If upwards is positive $y$, $y_0=\SI{0}{m}$, the initial position of the nozzle, $v_{y0}$ the initial speed, $a_y=\SI{-9.8}{m/s^2}$, the acceleration, and $y_f=\SI{-0.5}{m}$, and the time is $t=\SI{2.5}{s}$. We have simply:
\begin{align*}
y_f &= y_0 + v_{y0}t+\frac{1}{2}at^2\\
\therefore v_{y0} &= \frac{y_f-y_0-\frac{1}{2}at^2}{t}\\
&=\frac{(\SI{-0.5}{m})-(\SI{0}{m})-\frac{1}{2}(\SI{-9.8}{m/s^2})(\SI{2.5}{s})^2}{(\SI{2.5}{s})}=\SI{12.1}{m/s}
\end{align*}
\end{solution}

%Giancolli 2-70 -fixed
\question A young banjo player is trying to get from Texas to Tennessee. She decides to jump onto the cart of a freight train that travels at a constant speed of \SI{5.0}{m/s}. The banjo player notices an empty cart just as it begins to pass her and she begins to accelerate from rest at \SI{1.4}{m/s^2} (in the direction of the train) to a maximum speed of \SI{5.6}{m/s}.
\begin{parts}
\part How long does it take the banjo player to reach the box car?
\part What distance did the banjo player travel to reach the box car?
\end{parts}

\begin{finalanswer}
\begin{enumerate}[(a)]
\item \SI{18.7}{s}
\item \SI{93.3}{m}
\end{enumerate}
\end{finalanswer}
\begin{solution}
We choose the origin to be where the banjo player starts with the positive $x$ direction in the direction of the train. We have to consider the case before and after the banjo player reaches her maximum speed. The banjo player will reach her maximum speed at $t=\frac{\SI{5.6}{m/s}}{\SI{1.4}{m/s^2}}=\SI{4.0}{s}$

The condition for the banjo player to reach the box car is that their positions are equal. First consider their positions at $t=\SI{4.0}{s}$, where $t=0$ is when the cart passes the banjo player:

\begin{align*}
x^b(t)=\frac{1}{2}at^2&=\frac{1}{2}(\SI{1.4}{m/s^2})(\SI{4.0}{s})^2=\SI{11.2}{m}\\
x^c(t)&=v^bt=(\SI{5.0}{m/s})(\SI{4.0}{s})=\SI{20}{m}
\end{align*}
where $x^b$ is the position of the banjo player and $x^c$ is the position of the train cart. At $t=\SI{4.0}{s}$, the banjo has not caught up. It is most convenient to now re-formulate the problem and have $t=0$ be when the banjo player reaches maximum speed (but we will have to remember to add back in the \SI{4}{s} that have already elapsed). Keeping the same origin, we now have different initial positions for the banjo player and the train cart. Their positions as a function of time are given by:

\begin{align*}
x^b(t)&=(\SI{11.1}{m})+v^bt=(\SI{11.2}{m})+(\SI{5.6}{m/s})t\\
x^c(t)&=(\SI{20}{m})+v^ct=(\SI{20}{m})+(\SI{5.0}{m/s})t
\end{align*}

We can equate the two positions to find out the corresponding time:
\begin{align*}
(\SI{11.2}{m})+v^f t&= (\SI{20}{m})+v^ct\\
\therefore t&= \frac{(\SI{20}{m})-(\SI{11.2}{m})}{v^b-v^c}=\frac{(\SI{20}{m})-(\SI{11.2}{m})}{(\SI{5.6}{s})-(\SI{5.0}{m/s})}=\SI{14.7}{s}
\end{align*}
We must remember to add back the \SI{4}{s}, so the banjo player catches up to the box car \SI{18.7}{s} after the car first passed her.

\textbf{part b)} Using the time $t=\SI{18.7}{s}$, the box car (and thus the fugitive), has travelled a distance of $(\SI{5}{m/s})(\SI{18.7}{s})=\SI{93.3}{m}$.
\end{solution}



%Giancolli 3-46 (modified)
\question A projectile is shot from the edge of a cliff \SI{115}{m} above ground level with an initial speed of \SI{65}{m/s} at an angle of \SI{35.0}{\degree} from the horizontal. Determine:
\begin{parts}
\part The distance at which the projectile lands from the base of the cliff
\part The maximum height above the ground that the projectile reaches
\part The velocity vector of the projectile just before it hits the ground (give the components and the magnitude)
\end{parts}

\begin{finalanswer}
\begin{enumerate}[(a)]
\item \SI{530.53}{m}
\item \SI{185.91}{m}
\item \SI{80.5}{m/s} (and it makes an angle of \SI{48.6}{\degree} below the horizontal)
\end{enumerate}
\end{finalanswer}
\begin{solution}
First, we setup a coordinate system, choosing positive $x$ along the $x$ component of the initial velocity vector, $y$ positive upwards, and the origin at the base of the cliff. The initial velocity vector is thus:
\begin{align*}
\vec v_0 &= v_{0x}\hat i + v_{0y}\hat j\\
&=(\SI{65}{m/s}\cos\SI{35}{\degree})\hat i + (\SI{65}{m/s}\sin\SI{35}{\degree}) \hat j \\
&=(\SI{53.24}{m/s})\hat i + (\SI{37.28}{m/s}) \hat j 
\end{align*}
\textbf{a)} First, we find how long the motion takes using the $y$ component to find when it hits the ground. We can then use that time to find the distance in $x$ where it lands.
\begin{align*}
y(t) &= y_0 +  v_{0y}t+\frac{1}{2}a_yt^2\\
(\SI{0}{m})&= (\SI{115}{m}) + (\SI{37.28}{m/s})t + \frac{1}{2}(\SI{-9.8}{m/s^2})t^2
\end{align*}
This is a quadratic equation for $t$, the positive solution is $t=\SI{9.964}{m/s^2}$. The distance $x$ from the base of the cliff is then, $x=v_{0x}t=(\SI{53.24}{m/s})(\SI{9.964}{s})=\SI{530.53}{m}$


\textbf{b)} We find the time at which the vertical component of velocity is zero, and then calculate the vertical position at that time. The time is given by:
\begin{align*}
v_y(t) &= v_{y0}+a_yt\\
t &= \frac{v_y(t) - v_{y0}}{a_y}=\frac{(\SI{0}{m/s})-(\SI{37.28}{m/s})}{(\SI{-9.8}{m/s^2})}=\SI{3.80}{s}
\end{align*}
The $y$ position at this time is:
\begin{align*}
y(t) &= y_0 +  v_{0y}t+\frac{1}{2}a_yt^2\\
&=(\SI{115}{m}) + (\SI{37.28}{m/s})(\SI{3.80}{s}) + \frac{1}{2}(\SI{-9.8}{m/s^2})(\SI{3.80}{s})^2=\SI{185.91}{m}
\end{align*}


\textbf{c)} The $x$ component is unchanged, whereas the $y$ component is given by:
\begin{align*}
v_y(t) &= v_{y0}+a_yt\\
&=(\SI{37.28}{m/s})+(\SI{-9.8}{m/s^2})(\SI{9.964}{s})=\SI{-60.36}{m/s}
\end{align*}

The velocity vector is thus:
\begin{align*}
\vec v(t=\SI{9.964}{s})=(\SI{53.24}{m/s})\hat i - (\SI{60.36}{m/s}) \hat j
\end{align*}
with a magnitude of \SI{80.5}{m/s} (and it makes an angle of \SI{48.6}{\degree} below the horizontal).
\end{solution}


\question Chlo\"e is riding a toy guanaco on a merry-go-round (carousel). The guanaco that Chlo\"e is riding is located \SI{2.5}{m} from the centre of the merry-go-round, which rotates twenty times per minute. You decide to model Chlo\"e's position using a coordinate system whose origin coincides with the centre of the merry-go-round, the $x$ axis points East, the $y$ axis points North, and the $z$ axis points upwards. At time $t=0$, Chlo\"e's guanaco crosses the $x$ axis. 
\begin{parts}  
\part What are Chlo\"e's angular velocity (in \si{rad/s}) and centripetal acceleration?
\part What is Chlo\"e's velocity vector $\vec v(t)$ (as a function of time) and its magnitude? What is the velocity vector at $t=\SI{5.0}{s}$? (give the $x$ and $y$ components of the vector)
\part What is Chlo\"e's acceleration vector $\vec a(t)$ (as a function of time) and its magnitude? What is the acceleration vector at $t=\SI{5.0}{s}$? (give the $x$ and $y$ components of the vector)

Suddenly, to everyone's surprise, you realize that the whole merry-go-round is on wheels and that Steve, the merry-go-round thief, has attached the merry-go-round to his truck and started driving South at a constant speed of \SI{10}{m/s}, taking the whole thing with him. Assume that Steve started driving away at $t=0$, when Chlo\"e's guanaco crossed the $x$ axis, and that he accelerated in a negligible amount of time. The merry-go-round, being powered by batteries, continues to rotate.

\part What is Chlo\"e's velocity vector (as a function of time) relative to the coordinate system that you originally defined (i.e. relative to a fixed coordinate system whose origin is where the centre of the merry-go-round used to be)? What is the velocity vector at $t=\SI{5.0}{s}$? (give the $x$ and $y$ components of the vector)
\end{parts}

\begin{finalanswer}
\begin{enumerate}[(a)]
\item \SI{2.09}{rad/s}, \SI{10.97}{m/s^2}
\item $(\SI{4.44}{m/s})\hat i - (\SI{2.70}{m/s})\hat j$
\item $(\SI{5.69}{m/s^2})\hat i + (\SI{9.38}{m/s^2})\hat j$
\item $\vec v'(t) = (\SI{5.2}{m/s})\left(-\sin((\SI{2.09}{rad/s})t) \hat i + \cos( (\SI{2.09}{rad/s}) t) \hat j\right)+(\SI{-10}{m/s})\hat j$, $\vec v'(t=\SI{5.0}{s})=(\SI{4.44}{m/s})\hat i - (\SI{12.70}{m/s})\hat j$
\end{enumerate}
\end{finalanswer}
\begin{solution}
\textbf{a)} We need to convert the \SI{20}{rpm} to \si{rad/s}, to get the angular velocity:
\begin{align*}
\omega &=\SI{20}{rpm}=(20\frac{\si{rotation}}{\si{min}})(2\pi\frac{\si{rad}}{\si{rotation}})(\frac{1}{60}\frac{\si{min}}{\si{s}})=\SI{2.09}{rad/s}
\end{align*}
The centripetal acceleration is given by:
\begin{align*}
a_c=\omega^2 R=(\SI{2.09}{rad/s})(\SI{2.5}{m})=\SI{10.97}{m/s^2}
\end{align*}

\textbf{b)} For uniform circular motion, the position vector is given by:
\begin{align*}
\vec r(t) = R\cos(\omega t)\hat i+R\sin(\omega t)\hat j
\end{align*}
We can differentiate with respect to time to get the velocity vector:
\begin{align*}
\vec v(t) &=\frac{d}{dt}\vec r(t)\\
&=\frac{d}{dt}R\cos(\omega t)\hat i+\frac{d}{dt}R\sin(\omega t)\hat j\\
&=-R\omega\sin(\omega t)\hat i + R\omega \cos(\omega t) \hat j\\
&=R\omega\left(-\sin(\omega t)\hat i + \cos(\omega t) \hat j\right) \\
&=(\SI{2.5}{m})\left(\SI{2.09}{rad/s})(-\sin((\SI{2.09}{rad/s})t) \hat i + \cos( (\SI{2.09}{rad/s}) t)\hat j\right)\\
&=(\SI{5.2}{m/s})\left(-\sin((\SI{2.09}{rad/s})t) \hat i + \cos( (\SI{2.09}{rad/s}) t) \hat j\right)
\end{align*}
We have already factored out the magnitude, which is $v=\SI{5.2}{m/s}$ ($\omega R$) and is constant.

Evaluating the vector at $t=\SI{5.0}{s}$:
\begin{align*}
\vec v(t=\SI{5.0}{s})&=(\SI{5.2}{m/s})\left(-\sin((\SI{2.09}{rad/s})(\SI{5.0}{s})) \hat i + \cos( (\SI{2.09}{rad/s}) (\SI{5.0}{s})) \hat j\right)\\
&=(\SI{4.44}{m/s})\hat i - (\SI{2.70}{m/s})\hat j
\end{align*}

\textbf{c)} For the acceleration vector, we differentiate the velocity vector:
\begin{align*}
\vec a(t) &=\frac{d}{dt}\vec v(t)\\
&=\frac{d}{dt}R\omega\left(-\sin(\omega t)\hat i + \cos(\omega t) \hat j\right) \\
&=R\omega^2\left(-\cos(\omega t)\hat i - \sin(\omega t) \hat j\right) \\
&=a_c\left(-\cos(\omega t)\hat i - \sin(\omega t) \hat j\right)\\
&=(\SI{10.97}{m/s^2})\left(-\cos((\SI{2.09}{rad/s})t) \hat i -\sin( (\SI{2.09}{rad/s}) t) \hat j\right)
\end{align*}
The magnitude is constant and equal to the centripetal acceleration, \SI{10.97}{m/s^2}.

Evaluating the vector at $t=\SI{5.0}{s}$:
\begin{align*}
\vec a(t=\SI{5.0}{s})&=(\SI{10.97}{m/s^2})\left(-\cos((\SI{2.09}{rad/s})(\SI{5.0}{s})) \hat i -\sin( (\SI{2.09}{rad/s}) (\SI{5.0}{s})) \hat j\right)\\
&=(\SI{5.69}{m/s^2})\hat i + (\SI{9.38}{m/s^2})\hat j
\end{align*}

\textbf{d)} Relative to the original coordinate system, we have to add Steve's velocity vector, $\vec v^S$, to the original velocity vector.
\begin{align*}
\vec v^S &= (\SI{-10}{m/s})\hat j\\
\vec v'(t) &= \vec v(t) + \vec v^S\\
&=(\SI{5.2}{m/s})\left(-\sin((\SI{2.09}{rad/s})t) \hat i + \cos( (\SI{2.09}{rad/s}) t) \hat j\right)+(\SI{-10}{m/s})\hat j
\end{align*}
To evaluate it at $t=\SI{5.0}{s}$, we can simply subtract \SI{10}{m/s} from the $y$ component of the answer in part b):
\begin{align*}
\vec v'(t=\SI{5.0}{s})&=(\SI{4.44}{m/s})\hat i - (\SI{2.70}{m/s})\hat j+(\SI{-10}{m/s})\hat j \\
&=(\SI{4.44}{m/s})\hat i - (\SI{12.70}{m/s})\hat j
\end{align*}

\end{solution}

%Giancolli 3-77 -fixed
\question Romeo is attempting to wake Juliet up by tossing pebbles at her window. The pebbles thrown by Romeo leave his hand hand \SI{4.0}{m} away from the side of Juliet's house and \SI{10.0}{m} below her window. At what speed do the pebbles hit Juliet's window if they are only moving with a horizontal component of velocity?

\begin{finalanswer}
$\frac{(\SI{4.0}{m})}{(\SI{1.43}{s})}=\SI{2.8}{m/s}$
\end{finalanswer}
\begin{solution}
We can choose the origin where the pebbles leave Romeo's hand and positive $y$ to be up and positive $x$ to be towards the window. First, we can find the $y$ component of the initial velocity, knowing that the $y$ component of the final velocity is zero after a distance of \SI{10.0}{m}:
\begin{align*}
v_{yf}^2-v_{y0}^2&=2a_y(y_f-y_0)\\
\therefore v_{y0}&=\sqrt{-2a_y(y_f-y_0)}=\sqrt{-2(\SI{-9.8}{m/s^2})(\SI{10.0}{m})}=\SI{14.0}{m/s}
\end{align*}
We can now find how much time it takes for the pebbles to reach the window:
\begin{align*}
v_y(t)&=v_{y0}(t)+at\\
\therefore t&= \frac{v_y(t)-v_{y0}(t)}{a}=\frac{(-\SI{14.0}{m/s})}{(\SI{-9.8}{m/s^2})}=\SI{1.43}{s}
\end{align*}
Finally, we can find the $x$ speed required to cover \SI{4.0}{m} in \SI{1.43}{s}:
\begin{align*}
v_x = \frac{(\SI{4.0}{m})}{(\SI{1.43}{s})}=\SI{2.8}{m/s}
\end{align*}
This corresponds to the speed of the pebbles as they hit the window, since the $y$ component is zero.
\end{solution}


\question Model a long jump. Make precise sketches of position, velocity, and acceleration as a function of time (x and y components) for a long jumper, from when she starts at rest to when she lands and stops in the sand pit. Make sure that the six graphs are well labeled and that the various features are explained in your answer. You should come up with reasonable numerical estimates for all variables that you need to describe the motion (how fast she can run, how long she runs, how far she jumps, how high she jumps, her acceleration when running, etc). Although the graphs do not need to be exact, they should be close sketches to what you would get if you plotted your model with a computer (which of course, you are welcome to do). For example, the axes should be labelled and show the correct maximum height of the jump, even if they are not perfectly to scale. 

\begin{finalanswer}
\capfig{0.9\textwidth}{figures/Kinematics/longjump.png}{\label{fig:kinematics:longjumpanswer} Example graphs: Kinematic variables for a long jump. The dotted line shows when the jump phase begins.} 
\end{finalanswer}
\begin{solution}
We divide this into a running phase, and a jumping phases. During the running phase, all the motion is in $x$, whereas during the jump, there is motion in both $x$ and $y$ (parabolic motion).

A quick google search shows that a long jumper gets up to a speed of about \SI{10.0}{m/s} and that they have not usually reached their full speed when they jump (sprinters take a longer distance to reach maximal speed than the distance that long jumpers have to run). Therefore, we can model the running phase as constant acceleration, and guess that it lasts about \SI{5.0}{s}. This leads to an acceleration of \SI{2.0}{m/s^2} in the x direction for \SI{5.0}{s}, after which we can assume that the speed in the x-direction is constant (i.e. that air friction during the jump phase is negligible).

For the jump, we can guess that the jumper reaches a height of approximately \SI{1.0}{m}, with a constant negative acceleration of $g=\SI{9.8}{m/s^2}$. We can assume that the jumper almost instantaneously accelerates upwards with a velocity $v_{y0}$ such that she will reach a maximum height of about one meter. Rounding things off a little bit, if the initial upwards speed is $v_{y0}=\SI{4.5}{m/s}$, with a constant downwards acceleration of $g$, she we will reach a height of approximately \SI{1}{m}. We thus have:
\begin{align*}
vmax_x &= \SI{10.0}{m/s}\\
t_{run} &= \SI{5.0}{s}\\
a_x &= \frac{vmax_x}{t_{run}}=\SI{2.0}{m/s^2}\\
g &= \SI{9.8}{m/s^2}\\
v_{y0} &= \SI{4.5}{m/s}\\
t_{jump} &= 2\frac{v_{y0}}{g}\\
\end{align*}
Which leads to the following model:
\begin{align*}
x(t) &= \frac{1}{2}a_xt^2\;\;(t<t_{run})&
x(t) & = \frac{1}{2}a_xt_{run}^2 + vmax_x(t-t_{run})\;\;(t\geq t_{run})\\
y(t) &= 0\;\;(t<t_{run})&
y(t) & =v_{y0}(t-t_{run})- \frac{1}{2}g(t-t_{run})^2 \;\;(t\geq t_{run})\\
v_x(t) &= a_xt\;\;(t<t_{run})&
v_x(t) & = vmax_x\;\;(t\geq t_{run})\\
v_y(t) &= 0\;\;(t<t_{run})&
v_y(t) & = v_{y0}-g(t-t_{run})\;\;(t\geq t_{run})\\
a_x(t) &= \SI{2.0}{m/s^2}\;\;(t<t_{run})&
a_x(t) &= 0\;\;(t\geq t_{run})\\
a_y(t) &= 0\;\;(t<t_{run})&
a_y(t) & = -g\;\;(t\geq t_{run})\\
\end{align*}
and is plotted in Figure \ref{fig:kinematics:longjump}. The dotted line shows when the jump phase begins. The total jump distance is about \SI{10.0}{m}, which is reasonable (the world record is around \SI{9.0}{m}). After the running phase, the $x$ distance continues to increase linearly (constant speed), whereas the $x$ speed remains constant and the $x$ acceleration drops to zero. The $y$ position increases and then decreases back to zero during the jump, the $y$ speed decreases at a constant rate from a positive value (jumping up) to a negative value (when she hits the ground). The $y$ acceleration is constant during the jump phase and 0 otherwise.

\capfig{0.9\textwidth}{figures/Kinematics/longjump.png}{\label{fig:kinematics:longjump} Kinematic variables for a long jump. The dotted line shows when the jump phase begins.} 

\end{solution}

%From Midyear Makeup Exam F17
%Zaremba 2006 April final
\question Answer the following:
\begin{parts}
\part A projectile is launched from a horizontal surface and lands a distance $R$ away. What is the \textbf{minimum} speed with which the projectile should be launched in order to land at the distance $R$?
\part A child wishes to throw a ball to a friend on the other side of a building of height $h$ and width $d$ as shown in Figure \ref{fig:kinematics:Building}. They experiment by trying different launch speeds, angles of projection, and starting points of the launch. What is the \textbf{minimum speed} with which the ball can be thrown to clear the building and reach the other side? \textbf{Hint:} You may find the solution to part a) useful in this question.
\capfig{0.3\textwidth}{figures/Kinematics/Building.png}{\label{fig:kinematics:Building}Trajectory of ball thrown over the building.}
\end{parts}

\begin{finalanswer}
\begin{enumerate}[(a)]
\item $\sqrt{gR}$
\item $\sqrt{gd+2gh}$
\end{enumerate}
\end{finalanswer}
\begin{solution}
\begin{parts}
\part Suppose that the projectile is launched with an angle $\theta$ with respect to the horizontal. The question is equivalent to finding the value of $\theta$ that minimize the speed $v$ with which the projectile is launch and lands at a distance $R$. We define a coordinate system with the origin at the location where the object is launched, with positive $x$ in the horizontal direction of motion and positive $y$ upwards. If the projectile is launched at $t=0$, and lands at some time $t$, we can write the $x$ and $y$ positions as:
\begin{align*}
R&=v\cos\theta t\\
0&=0+v\sin\theta t -\frac{1}{2}gt^2
\end{align*}
Solving for $t$ in the second equation and substituting into the first:
\begin{align*}
t &= \frac{2v\sin\theta}{g}\\
R &=\frac{2v^2\cos\theta\sin\theta}{g}\\
\therefore v&=\sqrt{\frac{gR}{2\cos\theta\sin\theta}}
\end{align*}
The value of $v$ is minimized when $\sin\theta\cos\theta$ is maximal (since it appears in the denominator). We can find this value by taking the derivative:
\begin{align*}
\frac{d}{d\theta}\sin\theta\cos\theta&=\cos^2\theta-\sin^2\theta = 0\\
\therefore \sin\theta=\cos\theta\to\theta=\frac{\pi}{4}
\end{align*}
That is, the angle is \SI{45}{\degree}. Now that we know the angle, we can find $v$:
\begin{align*}
v&=\sqrt{\frac{gR}{2\cos\theta\sin\theta}}\\
&=\sqrt{\frac{gR}{2\frac{1}{\sqrt 2}\frac{1}{\sqrt 2}}}\\
&=\sqrt{gR}
\end{align*}
\part For the ball to clear the building, we can consider the trajectory at the top of the building, where the ball must skim both corners, as it covers a distance $d$. The trajectory between the two top corners is similar to what happens in part a), as we want the ball to barely clear the building. From part a), we know that as the ball arrives at the top left corner, it must have a velocity, that makes and angle of \SI{45}{\degree} with the horizontal, with a magnitude of:
\begin{align*}
v_{top\;left}=\sqrt{gd}
\end{align*}
This gives the horizontal component of the velocity as:
\begin{align*}
v_x=v_{top\;left}\cos(\SI{45}{\degree})=\frac{1}{\sqrt 2}\sqrt{gd}
\end{align*}
The $y$ component at the top left corner will be given by:
\begin{align*}
v_y=v_{top\;left}\sin(\SI{45}{\degree})=\frac{1}{\sqrt 2}\sqrt{gd}
\end{align*}
Using the kinematic equations in the vertical direction, we can find $v_{y0}$ the $y$ component of the velocity when it left the ground based on the height, $h$, that it covered on the way up:
\begin{align*}
v_y^2 - v_{0y}^2 &= 2ah =-2gh\\
\therefore v_{0y} &= \sqrt{v_y^2+2gh}\\
&=\sqrt{\frac{1}{2}gd+2gh}
\end{align*}
The total speed at the bottom is then:
\begin{align*}
v_0&=\sqrt{v_{0x}^2+v_{0y}^2}=\sqrt{\frac{1}{2}gd+\frac{1}{2}gd+2gh}\\
&=\sqrt{gd+2gh}
\end{align*}
\end{parts}
\end{solution}

\question A satellite is in a circular orbit around a spherical planet. The satellite is at a height $H$ from the surface of the planet and executes one full orbit around the planet at constant speed in a time $T$. The planet has a radius $R$ and does not rotate about itself. You wish to intercept the satellite by launching a rocket straight up from the surface of the planet. The rocket has a vertical acceleration given by:
\begin{align*}
a(t) = jt^2
\end{align*}
where $j$ is a constant and $t$ is the amount of time since the launch. At some time, $t=0$, the satellite passes right above the launch point of the rocket, as illustrated in Figure \ref{fig:kinematics:rocketsatellite}. Assume that the rocket's acceleration is given by the above equation throughout its trajectory.
\begin{parts}
\part How much time should you wait before launching the rocket so that it intercepts the satellite during its next orbit? Give your answer in terms of $T$, $j$, $H$, and $R$.
\part What will be the speed of the rocket when it intercepts the satellite?
\end{parts}
\capfig{0.3\textwidth}{figures/Kinematics/rocketsatellite.png}{\label{fig:kinematics:rocketsatellite}At time $t=0$, a satellite is just above a rocket at the surface of a planet. How much time should one wait in order to fire the rocket such that it will intercept the satellite on its next orbit?}
\begin{solution}
\begin{parts}
\part We need to determine how long it will take the rocket to reach an altitude of $H$, and then wait an amount of time that is $T$ minus the time it takes the rocket to reach the altitude $H$. Since the acceleration is not constant, we must derive a new equation for position versus time. We model the motion of the rocket as one-dimensional, with an initial position of zero, and an initial velocity of zero. The velocity of the rocket as a function of time is given by:
\begin{align*}
\frac{dv}{dt}&=a(t)\\
\therefore v(t)&=\int_0^t a(t) dt = \int_0^tjt^2dt=\frac{1}{3}jt^3
\end{align*}
The position as a function of time is given by:
\begin{align*}
x(t)=\int_0^tv(t)dt=\int_0^t\frac{1}{3}jt^3dt=\frac{1}{12}jt^4
\end{align*}
We can then easily find the time that it takes to get to a position $x=H$:
\begin{align*}
H&=\frac{1}{12}jt^4\\
\therefore t&=\sqrt[4]{12\frac{H}{j}}
\end{align*}
One should thus wait an amount of time:
\begin{align*}
T-\sqrt[4]{12\frac{H}{j}}
\end{align*}
to launch the rocket after the satellite has passed just above the rocket.
\part Using the formula that we derived above, the speed of the rocket evaluated at the time when the rocket reaches a position $H$ is given by:
\begin{align*}
v\left(t=\sqrt[4]{12\frac{H}{j}}\right)=\frac{1}{3}j\left( 12\frac{H}{j}  \right)^{\frac{3}{4}}
\end{align*}
\end{parts}
\end{solution}

%written by Emma Neary
\question You are standing at a train station in Toronto saying good-bye to your little cousin. As the train carrying your cousin begins to pull away at \SI{2.00}{m/s[E]}, your cousin runs \SI{4.20}{m/s[W]} in the opposite direction, up the aisle of the train.
\begin{parts}
\part What is your cousin's velocity relative to the ground?
\part If you now begin to walk away at \SI{1.50}{m/s[W]}, and your cousin continues to run as before, what is his relative velocity to you?
\part Your cousin realizes that he forgot to give you the chocolate egg he bought for you. He quickly throws it horizontally out the train window at \SI{5.00}{m/s[S]}, landing on the ground \SI{0.75}{s later}. From how high did the cousin throw the egg?
\end{parts}

\begin{finalanswer}
	\begin{parts} 
	\part Your cousin's velocity relative to the ground is $\SI{2.20}{m/s[W]}$.
	\part Your cousin's velocity relative to you is $\SI{1.70}{m/s[W]}$.
	\part The egg was thrown an initial height of \SI{2.75}{m} above the ground.
	\end{parts}
\end{finalanswer}

\begin{solution}
	\begin{parts}
		\part Defining the positive y-axis as North, and the positive x-axis as East, then your cousin’s velocity relative to the ground is:
		\begin{align*}
		\vec v_c^g &= \vec v_c^t + \vec v_t^g\\
		&=(\SI{-4.20}{m/s})i + (\SI{2.00}{m/s})i\\
		&=(\SI{-2.20}{m/s})i
		\end{align*}
		
		Therefore, the cousin's velocity relative to the ground is $\SI{2.20}{m/s[W]}$
		
		\part First, we must find the velocity of the train relative to you.
		\begin{align*}
		\vec v_c^y &= \vec v_c^g + \vec v_t^y\\
		&=(\SI{-2.20}{m/s})i+(\SI{0.50}{m/s})i\\
		&=(\SI{-0.50}{m/s})\\
		\end{align*}
		
		Now, we can find the cousin's velocity relative to you
		
		\begin{align*}
		\vec v_c^y &= \vec v_c^g \vec v_t^y\\
		&= (\SI{-2.20}{m/s})i+(\SI{0.50}{m/s})i\\
		&= (\SI{-1.7}{m/s})i\\
		\end{align*}
		Therefore, your cousin's velocity relative to you is \SI{1.7}{m/s[W]}.
		\part
		We can imagine the chocolate egg as a projectile in the x-direction and y-direction (it is travelling at a constant speed in the East and South directions), but is being acted upon by gravity in the z-direction.
		
		Modelling the egg as a projectile in the z-direction and defining the ground as our zero point:
		
		\begin{align*}
		\vec v_{z}& = \SI{0}{m/s}\\
		\vec a_z& = \SI{9.80}{m/s^2}\\
		t& = \SI{0.75}{s}
		\end{align*}
		With this information, we can use kinematics to solve for the initial position of the egg:
		\begin{align*}
		\vec z(t)&= \vec z_0 + \vec v_{0t} + \frac{1}{2}\vec a_z(t)^2\\
		&0 = \vec z_0 \frac{1}{2}(\SI{9.80}{m/s^2})(\SI{0.75}{s})^2\\
		&\vec z_0 = \SI{2.76}{m}
		\end{align*}
		Therefore, the egg was thrown an initial height of \SI{2.76}{m} above the ground.
	\end{parts}
\end{solution}