\chapter{Source of magnetic field}
\label{chapter:magneticsource}
In this chapter, we develop the tools to model the magnetic field that is produced by an electric current. 

\begin{learningObjectives}{
 \item Understand how to apply the Biot-Savart Law to determine the magnetic field from an electric current.
 \item Understand how to apply Ampère's Law.
 \item Understand how to model a solenoid.
 }
\end{learningObjectives}

\begin{opening}
\begin{MCquestion}{QUESTION }
\item a choice
\item another choice \correct
\end{MCquestion}
\end{opening}

\section{The Biot-Savart Law}
The Biot-Savart law allows us to determine the magnetic field at some position in space that is due to an electric current. More precisely, the Biot-Savart law allows us to calculate the infinitesimal magnetic field, $d\vec B$, that is created by a small section of wire, $d\vec l$, carrying current, $I$:
\begin{align*}
\Aboxed{d\vec B = \frac{\mu_0 I}{4\pi}\frac{d\vec l\times \hat r}{r^2}}
\end{align*}
where, $\vec r$, is the vector from the element of wire, $d\vec l$, to the point where we would like to determine the magnetic field, as illustrated in Figure \ref{fig:magneticsource:biotsavart}. $\mu_0$ is a constant of proportionality called the ``permeability of free space'', and has the value $\mu_0=\SI{4\pi e-7}{T\cdot m/A}$.
\capfig{0.4\textwidth}{figures/MagneticSource/biotsavart.png}{\label{fig:magneticsource:biotsavart}The infinitesimal magnetic field, $d\vec B$, that is created by an infinitesimal section of wire, $d\vec l$, carrying current $I$.}

The Biot-Savart Law has some similarities with the Coulomb Law to calculate the electric field, as the magnitude of the magnetic field decreases as the inverse of the square distance between the source and the field. However, it can only be expressed in differential form (i.e. as an infinitesimal), and it requires working in three dimensions, because of the cross product. It is usually more convenient to use the Biot-Savart Law in the form:
\begin{align*}
d\vec B = \frac{\mu_0 I}{4\pi}\frac{d\vec l\times \vec r}{r^3}
\end{align*}
where the unit vector $\hat r$ was replaced by $\vec r/r$.

The procedure for applying the Biot-Savart Law is as follows:
\begin{enumerate}
\item Make a really good diagram, as you will have to include some 3D aspects.
\item Choose an infinitesimal section of wire, $d\vec l$.
\item Determine the vector $\vec r$.
\item Determine the cross-product, $d\vec l \times \vec r$, which will point in the direction of the magnetic field from that infinitesimal section of wire.
\item Write out the infinitesimal vector $d\vec B$, and determine its components.
\item Think about symmetry! As you sum the $d\vec B$, will some components cancel? If yes, you do not need to do those integrals.
\item Determine the total magnetic field, component by component, by summing (integrating) the components of $d\vec B$ over the wire.
\end{enumerate}

\subsection{Magnetic field from a straight current-carrying wire}
In this section, we use the Biot-Savart Law to determine the magnetic field a distance, $h$, from the centre of a straight wire carrying current, $I$, as illustrated in Figure \ref{fig:magneticsource:bswire}.
\capfig{0.6\textwidth}{figures/MagneticSource/bswire.png}{\label{fig:magneticsource:bswire}Setting up the model to use the Biot-Savart Law to calculate the magnetic field a distance $h$ from the centre of a current-carrying wire of length $L$.}

We start by choosing an infinitesimal element of wire, $d\vec l$, a distance $y$ above the centre of the wire, as shown (we choose the origin to be located at the centre of the wire). The vector $d\vec l$ is thus given by:
\begin{align*}
d\vec l = dl\hat y
\end{align*}
The vector, $\vec r$, from $d\vec l$ to the point at which we would like to know the magnetic field is given by:
\begin{align*}
\vec r &= r\cos\theta\hat x -r\sin\theta\hat y\\
r &=\sqrt{h^2+y^2} =\frac{h}{\cos\theta}
\end{align*}
The cross-product between $d\vec l$ and $\vec r$ is easily found with the right-hand rule to point into the page (corresponding to the negative $z$ direction). The magnitude of the cross-product is given by:
\begin{align*}
||d\vec l \times \vec r||=dl r \sin\phi
\end{align*}
where $\phi=\pi/2+\theta$ is the angle between $d\vec l$ and $\vec r$, so that $\sin\phi=\cos\theta$. The cross-product can thus be written in terms of $\theta$ as:
\begin{align*}
d\vec l \times \vec r=-dl r \cos\theta \hat z
\end{align*}
Note that we can also determine the cross-product algebraically instead of using the right-hand rule and the magnitude:
\begin{align*}
d\vec l \times \vec r &= (dl\hat y) \times (r\cos\theta\hat x -r\sin\theta\hat y)\\
&=dlr\cos\theta (\hat y \times\hat x) - rdl\sin\theta(\hat y \times \hat y)\\
&=-dlr\cos\theta \hat z 
\end{align*}
The infinitesimal magnetic field element, $d\vec B$, is given by:
\begin{align*}
d\vec B = \frac{\mu_0 I}{4\pi}\frac{d\vec l\times \vec r}{r^3}=-\frac{\mu_0 I}{4\pi}\frac{dl\cos\theta}{r^2}\hat z
\end{align*}
Any segment along the wire will result in a magnetic field that is into the page (negative $z$ direction), thus there will be no cancellations due to any symmetries. We can now proceed to perform the integral. The two obvious choices of variable to use to label each segment for the integration are $\theta$ and $y$. We will choose to integrate over $\theta$, requiring us to express $dl$ and $r$ in terms of $\theta$ and constants (as those are the only quantities in the integrand that change with $\theta$). In order to express $dl$ in terms of $d\theta$, we first relate $\theta$ to $y$, the position of the wire element:
\begin{align*}
y = h\tan\theta\quad \to \quad
dl = dy = \frac{dy}{d\theta}d\theta = \frac{h}{\cos^2\theta}d\theta
\end{align*}
whereas $r$ is given by:
\begin{align*}
r=\frac{h}{\cos\theta}\quad \to \quad
\frac{1}{r^2}&=\frac{\cos^2\theta}{h^2}
\end{align*}
Putting this altogether into $d\vec B$:
\begin{align*}
d\vec B &=-\frac{\mu_0 I}{4\pi}\frac{dl\cos\theta}{r^2}\hat z = -\frac{\mu_0 I}{4\pi}\left(\frac{h}{\cos^2\theta}d\theta\right) \left( \frac{\cos^2\theta}{h^2} \right)\cos\theta\hat z=-\frac{\mu_0 I}{4\pi h}\cos\theta d\theta \hat z=dB_z\hat z
\end{align*}
We can define the angle, $\theta_0$, to be the maximum amplitude of the angle $\theta$ when integrating over the wire (see Figure \ref{fig:magneticsource:bswire}):
\begin{align*}
B_z&=\int_{-\theta_0}^{+\theta_0}dB_z= -\frac{\mu_0 I}{4\pi h} \int_{-\theta_0}^{+\theta_0}\cos\theta d\theta=-\frac{\mu_0 I}{4\pi h}(2\sin\theta_0) =-\frac{\mu_0 I}{2\pi h}\sin\theta_0
\end{align*}
Using the given dimensions:
\begin{align*}
\sin\theta_0&=\frac{L/2}{\sqrt{h^2+\frac{L^2}{4}}}\\
\end{align*}
Thus, the magnetic field, $\vec B$, a distance, $h$, from the centre of a wire of length, $L$, carrying current, $I$, is given by:
\begin{align*}
\Aboxed{\vec B &= -\frac{\mu_0 I}{2\pi h}\frac{L/2}{\sqrt{h^2+\frac{L^2}{4}}}\hat z}\quad\text{(finite wire)}
\end{align*}
The magnetic field must be rotationally symmetric; that is, if the wire is vertical, the magnetic field at a distance $h$ must look the same regardless of the angle from which we view the vertical wire (we should always see the magnetic field going into the page at the point that we use in Figure \ref{fig:magneticsource:bswire}). Thus, the magnetic field lines form circles around the wire, as illustrated in Figure \ref{fig:magneticsource:wirefield}. Note that the direction of the magnetic field is given by the right-hand rule for axial vectors; when you align your thumb with the current, your fingers curl in the direction of the magnetic field.
%TODO I think this figure can be improved (or changed to be from a different angle), as the field circles don't quite look like they live in the plane.
\capfig{0.4\textwidth}{figures/MagneticSource/wirefield.png}{\label{fig:magneticsource:wirefield}The magnetic field from a current-carrying wire forms concentric circles centred on the wire.}

It is of particular interest to investigate the limiting case of an infinitely long wire, in the limit of $L\to\infty$, or equivalently, $\theta_0\to\frac{\pi}{2}$. The latter is easiest to evaluate, since $\sin\theta_0\to 1$. The magnetic field, $\vec B$, a distance, $h$, from an infinite wire carrying current, $I$, is given by:
\begin{align*}
\Aboxed{\vec B &= -\frac{\mu_0 I}{2\pi h}}\quad\text{(infinite wire)}
\end{align*}
In general, one can often make the approximation that the wire is infinite in length, when the distance $h$, is small compared to the length of the wire. 
\subsection{Magnetic field from circular current-carrying wire}
In this section, we examine the magnetic field that is created by a circular current carrying wire. We can determine the shape of the magnetic field, by considering small sections of current as straight wires, that have circular magnetic field lines around them. As we move closer to the centre of the ring, those fields sum together in, as illustrated in Figure \ref{fig:magneticsource:ringfield}. Note that the magnetic field from a ring of current, is very similar to that from a bar magnet (in particular, in the plane perpendicular to the ring that goes through its centre).   
\capfig{0.4\textwidth}{figures/MagneticSource/ringfield.png}{\label{fig:magneticsource:ringfield}The magnetic field from a current-carrying loop of wire can be thought of as the sum of the field from small straight sections of wire.}
Below, we use the Biot-Savart Law to derive an expression for the magnitude of the magnetic field at a distance, $h$, from the centre of a ring of radius, $R$, along its axis of symmetry. While the mathematics are much easier than the case for the straight wire, the challenge in this case is to visualize the calculation in three dimensions! Figure \ref{fig:magneticsource:bsring} shows the loop of current, as well as our choice of coordinate system. In particular, we wish to calculate the magnetic field at a distance, $h$, along the $z$ axis. The $x$ axis goes into the page. 
\capfig{0.4\textwidth}{figures/MagneticSource/bsring.png}{\label{fig:magneticsource:bsring}Diagram to apply the Biot-Savart Law in order to determine the magnetic field along the symmetry axis of a ring carrying current, $I$. The $x$ axis goes into the page.}
In order to apply the Bio-Savart Law, we choose an element, $d\vec l$, of wire at the top of the ring, as illustrated. At this position, the element, $d\vec l$, points in the positive $x$ direction (into the page):
\begin{align*}
d\vec l = dl \hat x
\end{align*}
The vector, $\vec r$, from the wire element to the point where we wish to determine the magnetic field is given by:
\begin{align*}
\vec r =  - r\sin\theta \hat y+r\cos\theta \hat z
\end{align*}
Their cross-product can be evaluated algebraically:
\begin{align*}
d\vec l \times \vec r &= (dl \hat x) \times ( - r\sin\theta \hat y+r\cos\theta \hat z)\\
&=-rdl\sin\theta (\hat x \times \hat y) - rdl\cos\theta (\hat x \times \hat z)\\
&=-rdl\sin\theta\hat z - rdl\cos\theta (-\hat y)\\
&=-rdl\sin\theta\hat z + rdl\cos\theta \hat y
\end{align*}
so that the element of magnetic field, $d\vec B$, corresponding to that choice of $d\vec l$, will lie in the $y-z$ plane, as illustrated in Figure \ref{fig:magneticsource:bsring}.
\section{Force between two current-carrying wires}

\section{Ampère's Law}
%The Law
%Field from an infinite wire
%Field in a solenoid
%Field in a toroid

\newpage
\section{Summary}

\begin{chapterSummary}
 Something that was learned
\end{chapterSummary}

\newpage
\begin{importantEquations}
\medskip
\begin{multicols}{2}
\textbf{Momentum of a point particle:}
\begin{align*}
\vec p = m\vec v \\
\frac{d}{dt}\vec p = \sum \vec F = \vec F^{net}
\end{align*}
\columnbreak
\\
\textbf{Position of the Centre of Mass \\ of a system:}
\begin{align*}
\vec r_{CM} &=\frac{1}{M}\sum_i m_i\vec r_i 
\end{align*}
\medskip
\end{multicols}
\end{importantEquations}

\newpage
\section{Thinking about the material}

\begin{chapteractivity}{Reflect and research}
{
\item Explain
}
\end{chapteractivity}

\begin{chapteractivity}{To try at home}
{
\item Try
}
\end{chapteractivity}

\begin{chapteractivity}{To try in the lab}
{
\item (Simulation) Calculate the magnetic field from a loop of current at all positions in space (not only on the axis of symmetry). 
}
\end{chapteractivity}

\newpage
\section{Sample problems and solutions}
\subsection{Problems}
\begin{problem}{soln:template:ballistic}{\label{prob:template:ballistic} 

}
\end{problem}

\newpage
\subsection{Solutions}
\begin{solution}{prob:template:ballistic}\label{soln:template:ballistic}

\end{solution}

