\section{Sources of magnetic fields}

%%%%%%%%%%%%%%%%%%%%%%%%%%%%%%%%%%%
%%
%% Multiple Choice
%%
%%%%%%%%%%%%%%%%%%%%%%%%%%%%%%%%%%%
\subsection{Multiple Choice}

%Question submitted by Emily Darling
\question A long, thick wire has a radius of \SI{5}{cm} and a current of \SI{500}{A} that runs from West to East along a relatively straight shoreline. If you are walking North towards the beach, what is the strength of the magnetic field that you feel when you are \SI{8}{m} away from the wire?
\begin{checkboxes}
\choice \SI{2.00E-3}{T}
\CorrectChoice \SI{1.25E-5}{T} \correct
\choice \SI{2.00E-5}{T}
\choice \SI{1.25E-3}{T}
\end{checkboxes}

\question A generator is made of a single coil with resistance $R=\SI{5}{\Omega}$ and area $A=\SI{0.1}{m^2}$ rotated at a frequency of $f=\SI{60}{Hz}$ in a magnetic field of $B=\SI{5.0}{T}$. What is the maximum torque that needs to be applied to the coil for the generator to rotate at constant speed? Note that the torque required to rotate the generator varies with time, we want to know its maximal value.
\begin{choices}
	\choice $\tau=\SI{3.77}{N\cdot m}$
	\CorrectChoice $\tau=\SI{18.8}{N\cdot m}$
	\choice $\tau=\SI{188}{N\cdot m}$
	\choice $\tau=\SI{471}{N\cdot m}$
\end{choices}


%Q12
\question A rectangular loop of current is moved from a region with no magnetic field into a region where the magnetic field is constant and uniform out of the page, as shown in Figure \ref{fig:Induction:LoopFlux}. In which way is the current in the loop induced?
\capfig{0.2\textwidth}{figures/MagneticSource/LoopFlux.png}{\label{fig:Induction:LoopFlux}A rectangular loop moving into a region of uniform magnetic field out of the page.}
\begin{checkboxes}
	\choice counter-clockwise.
	\CorrectChoice clockwise.
	\choice no current is induced.
\end{checkboxes}


\question A rectangular loop of current is moved from a region where the magnetic field is constant and uniform out of the page, as shown in Figure \ref{fig:Induction:LoopFlux2}, to a region with no magnetic field. In which way is the current in the loop induced?
\capfig{0.2\textwidth}{figures/MagneticSource/LoopFlux2.png}{\label{fig:Induction:LoopFlux2}A rectangular loop moving out of a region of uniform magnetic field out of the page.}
\begin{checkboxes}
	\CorrectChoice counter-clockwise.
	\choice clockwise.
	\choice no current is induced.
\end{checkboxes}


\question A rectangular loop of current is located in a region of uniform magnetic field, as shown in Figure \ref{fig:Induction:LoopFlux3}. If the magnetic field increases with time, in which way is the current in the loop induced?
\capfig{0.15\textwidth}{figures/MagneticSource/LoopFlux3.png}{\label{fig:Induction:LoopFlux3}A rectangular loop in a regio of uniform magnetic field out of the page.}
\begin{checkboxes}
	\choice counter-clockwise.
	\CorrectChoice clockwise.
	\choice no current is induced.
\end{checkboxes}


\question Two infinitely long  vertical wires are separated by a distance of $\SI{1}{m}$. One wire carries an upwards going current of $\SI{2}{A}$, while the other carries the same current in the opposite direction. What is the magnitude of the magnetic field half way between the two wires?
\begin{checkboxes}
	\choice \SI{0}{T}
	\choice \SI{8e-7}{T}
	\CorrectChoice \SI{1.6e-6}{T}
	\choice \SI{3.2e-6}{T} 
\end{checkboxes}


\question Two infinitely long vertical wires are separated by a distance of $\SI{1}{m}$. Both wires carry an upwards going current of $\SI{2}{A}$. What is the magnitude of the magnetic field half way between the two wires?
\begin{checkboxes}
	\CorrectChoice \SI{0}{T}
	\choice \SI{8e-7}{T}
	\choice \SI{1.6e-6}{T}
	\choice \SI{3.2e-6}{T} 
\end{checkboxes}

\question Two infinitely long vertical wires are separated by a distance of $\SI{1}{m}$. One wire carries an upwards going current of $\SI{1}{A}$, while the other carries the same current in the opposite direction. What is the magnitude of the magnetic field half way between the two wires?
\begin{checkboxes}
	\choice \SI{0}{T}
	\choice \SI{4e-7}{T}
	\CorrectChoice \SI{8e-7}{T}
	\choice \SI{1.6e-6}{T} 
\end{checkboxes}


%Question submitted by Adam McCAw
\question Two parallel identical wires are beside each other with currents going in opposite directions. The magnetic force between the wires is
\begin{checkboxes}
\choice Attractive
\CorrectChoice Repulsive \correct
\choice Zero
\choice Cannot be determined
\end{checkboxes}

\question Two parallel identical wires are beside each other with currents going in same direction. The magnetic force between the wires is
\begin{checkboxes}
\CorrectChoice Attractive \correct
\choice Repulsive
\choice Zero
\choice Cannot be determined
\end{checkboxes}

%Based on question from Emma Lanciault
\question A straight wire carries current I and results in a magnetic field at a distance $r$ from the wire. What happens to the direction of the magnetic field if the current in the wire is reversed?
\begin{checkboxes}
\CorrectChoice It is reversed \correct
\choice It stays the same
\choice It goes to zero
\end{checkboxes}

%Maya Gibb
\question A uniform magnetic field B is created by an electric current I flowing through a long solenoid with n turns per unit length and radius R. What is the magnitude of the magnetic field inside the solenoid?
\begin{checkboxes}
\choice $\mu_0 n I R$
\CorrectChoice $\mu_0 n I$ \correct
\choice $\mu_0 n I R^2$
\choice $\mu_0 n I/R$
\choice $\mu_0 n I/R^2$
\end{checkboxes}

%Damara Gagnier, modified
\question A long solenoid of radius $R$ is made of $N$ loops. When a current $I$ goes through the solenoid, the magnetic field at the centre the solenoid has magnitude $B$. If one doubles the number of loops in the solenoid, what will be the magnetic field at the centre of the solenoid?
\begin{choices} 
\choice $\frac{B}{2}$
\choice $B$
\CorrectChoice $2B$ \correct
\choice or not $2B$, that is the question
\end{choices}
%%%%%%%%%%%%%%%%%%%%%%%%%%%%%%%%%%%
%
% long answer
%
%%%%%%%%%%%%%%%%%%%%%%%%%%%%%%%%%%%
\subsection{Long answers}
%Ryan U, heavily modifided part c to make it awesome
\question  Two vertical infinite wires are separated by a distance $r$. 
\begin{parts}
\part If both wires carry the same positive charge per unit length, $\lambda$, show that one wire will feel a repulsive electric force per unit length:
\begin{align*}
\frac{F_E}{l}=\frac{\lambda^2}{2\pi\epsilon_0 r}
\end{align*}  from the other wire.
\part Now, suppose that the two wires each carry an upwards current $I$.  Show that the attractive magnetic force per unit length between the wires is:
\begin{align*}
\frac{F_B}{l}=\frac{\mu_0I^2}{2\pi r}
\end{align*}
\part For the case in part (a), assume that you are moving downwards with such a speed that it appears that the charge density on the wire is moving upwards with the same current as in part (b). In part (a), you argued that the wires repelled each other, but now, it would appear that because of the current they might attract (although, the wires still have a net charge on them, so there is also an electrostatic repulsion). Is the force between the wires still the same as in part (a) when you are moving downwards? Explain what is going on!
\textbf{Hint}: look up special relativity.
\end{parts}
\begin{finalanswer}
\begin{enumerate}[(a)]
\item N/A
\item N/A
\item Yes, the force between the wires must still be the same as in part (a). When you move relative to the wires, you see a current and thus a magnetic field that results in an attractive force between the wires. However, due to Special Relativity, you see the wires contract lengthwise (they shorten). You however still see the same number of charges on the wire, so if the wire contracts, then the charge density on the wire increases. Because of the increase in charge density, the electrostatic repulsion between the wires is larger than before, by exactly the amount to balance the attractive force from the magnetic field!
\end{enumerate}
\end{finalanswer}
\begin{solution}
\begin{parts}
\part We can use Gauss' Law to find the electric field a distance $r$ from one of the wires. Using a cylindrical surface with length $l$ and radius $r$, coaxial with the wire, the flux out of the surface is $\Phi=E(r)2\pi r l$ and the charge enclosed is $Q^{enc}=\lambda l$. The field is thus:
\begin{align*}
\Phi&=\frac{Q^{enc}}{\epsilon_0}\\
\therefore E&=\frac{\lambda}{2\pi\epsilon_0 r}
\end{align*}

The electric force on a section of wire of length $l$, is given by the charge on that section, $q=\lambda l$ multiplied by the electric field:
\begin{align*}
F_E&=qE=\lambda l \frac{\lambda}{2\pi\epsilon_0 r}\\
\therefore\frac{F_E}{l} &=\frac{\lambda^2}{2\pi\epsilon_0 r}
\end{align*}
The force is repulsive.

\part In this case, we use Amp\`ere's Law to find the force a distance $r$ away from one of the wires carrying current $I$. By symmetry, the field lines form concentric circles around the wire, so we choose an Amp\`erian loop that is a circle centred about the wire. Amp\`ere's Law gives:
\begin{align*}
\oint \vec B \cdot d\vec l&=\mu_0I\\
B(r)2\pi r &= \mu_0I\\
\therefore B(r)&=\frac{\mu_0I}{2\pi r}
\end{align*}

The magnetic force on a section of wire of length $l$ is then given by:
\begin{align*}
F_B&=IlB=Il\frac{\mu_0I}{2\pi r}\\
\therefore \frac{F_B}{l}&=\frac{\mu_0I^2}{2\pi r}
\end{align*}
The force is attractive.

\part Yes, the force between the wires must still be the same as in part (a). When you move relative to the wires, you see a current and thus a magnetic field that results in an attractive force between the wires. However, due to Special Relativity, you see the wires contract lengthwise (they shorten). You however still see the same number of charges on the wire, so if the wire contracts, then the charge density on the wire increases. Because of the increase in charge density, the electrostatic repulsion between the wires is larger than before, by exactly the amount to balance the attractive force from the magnetic field! We show below how you can figure out that length contraction must occur if you are moving relative to the wires.

Let us assume that the force between the two wires is independent of our frame of reference. Thus, when we are moving downwards with velocity $v$, we should observe the same force between the wires that we found in part (a). Furthermore, let us assume that the equations for the electric and magnetic forces are valid in all frame of references (which is not obvious, since the magnetic force depends on the observer's velocity, $q\vec v \times \vec B$, $I\vec L \times \vec B$). If the electric and magnetic forces are the same, then let us assume that it is the electric charge density that depends on our reference frame.

Let $\lambda_0$ be the linear charge density on the wires as measured when we are at rest with respect to the wires, and $\lambda$ be the linear charge density as measured when we move downwards with speed $v$. In the frame of the wires ($0$), the electric force per unit distance on one of the wires is repulsive and given by (part (a)):
\begin{align*}
\frac{F_E^0}{l} &=\frac{\lambda_0^2}{2\pi\epsilon_0 r}
\end{align*}
If we move downwards with speed $v$, we see an upwards current, $I$, given by:
\begin{align*}
I=\frac{dQ}{dt}=\frac{d}{dt}\lambda l=\lambda \frac{dl}{dt}=\lambda v
\end{align*}
where $\lambda$ is the charge density that we see on the wire when we move relative to the wire. There is thus an attractive magnetic force per unit length on each wire given by (part (b)):
\begin{align*}
\frac{F_B}{l}&=-\frac{\mu_0I^2}{2\pi r}=-\frac{\mu_0\lambda^2v^2}{2\pi r}
\end{align*}
where the minus sign indicates that it is in the attractive direction (in the opposite direction from the electric force). In the moving frame of reference, there must still be a repulsive electric force on each wire given by:
\begin{align*}
\frac{F_E}{l} &=\frac{\lambda^2}{2\pi\epsilon_0 r}
\end{align*}
If the net force on one of the wires is independent of our frame of reference, then:
\begin{align*}
\frac{F_E^0}{l} &= \frac{F_E}{l}+\frac{F_B}{l}\\
\therefore \frac{\lambda_0^2}{2\pi\epsilon_0 r} &= \frac{\lambda^2}{2\pi\epsilon_0 r}-\frac{\mu_0\lambda^2v^2}{2\pi r}
\end{align*}
which we can simplify to express $\lambda$, the linear charge density that we measure in the moving frame, in terms of $\lambda_0$, the linear charge density that we measure in the rest frame:
\begin{align*}
\frac{\lambda_0^2}{\epsilon_0} &= \frac{\lambda^2}{\epsilon_0}-\mu_0\lambda^2v^2\\
\lambda_0^2&=\lambda^2-\mu_0\epsilon_0\lambda^2v^2\\
\lambda_0^2&=\lambda^2(1-\mu_0\epsilon_0v^2)\\
\therefore \lambda&=\lambda_0\frac{1}{\sqrt{1-\mu_0\epsilon_0v^2}}
\end{align*}
Thus, we find that if (1) the force should be the same in both frames of references, and that, (2), the constants $\mu_0$ and $\epsilon_0$ are independent of our frame of reference, then the charge density must depend on our frame of reference. Since we would count the same number of charges on a wire independent of our frame of reference, then the only possibility that is consistent with our two assumptions is that the length of the wire appears shorter in the moving frame of reference!!! And this turns out to be the case in Nature, as far as we can tell.

Finally, note that the combination $\mu_0\epsilon_0$ is related to the speed with which electromagnetic waves propagate in vacuum, $\mu_0\epsilon_0=\frac{1}{c^2}$, where $c$ is of course the speed of light. Our 2 assumptions above are of course equivalent to Einstein's famous postulates. In terms of $c$, we find the length contraction formula:
\begin{align*}
\lambda&=\lambda_0\frac{1}{\sqrt{1-\frac{v^2}{c^2}}}
\end{align*}

\end{parts}
\end{solution}


\question Two identical rectangular loops, with width \SI{3}{m}, and height \SI{2}{m}, as shown in Figure \ref{fig:magneticsource:currentloops}, carry a clockwise current, $I=\SI{1}{A}$. One loop is \SI{1}{m} directly to the right of the other. Point $P$ is located \SI{1}{m} away from the left edge of the right loop midway between the horizontal sections. What is the magnetic field vector at point $P$?


\capfig{0.8\textwidth}{figures/MagneticSource/currentloops.png}{\label{fig:magneticsource:currentloops} Two identical rectangular wires carrying current $I$.}
\begin{finalanswer}
\SI{-4.868e-7}{T}$\hat z$
\end{finalanswer}
\begin{solution}
We have to use the Biot-Savart law to find the magnetic field from each section of wire in each loop. The two horizontal sections of wire from one loop will both contribute the same amount to a magnetic field into the page. All of the vertical sections of wire will contribute differently to the total magnetic field at point $P$. As shown in Figure \ref{fig:magneticsource:currentloops}, the positive $z$ direction is out of the page, and all of the sections of the loops will create magnetic fields that are in the $z$ direction. 


We can calculate the magnetic field from a section of wire carrying current $I$, a distance $D$ away from point $P$, as shown in Figure \ref{fig:magneticsource:BS_wire}. Rather than use the length of the wire, we use the angles that it subtends to indicate its extent ($\theta_1$ and $\theta_2$ in Figure \ref{fig:magneticsource:BS_wire}).
\capfig{0.4\textwidth}{figures/MagneticSource/BS_wire.png}{\label{fig:magneticsource:BS_wire}Magnetic field element from a small element of wire.}

The small section of wire, $d\vec l$, in Figure \ref{fig:magneticsource:BS_wire}, with upwards going current in the figure, creates an element of magnetic field given by:
\begin{align*}
d\vec B&=\frac{\mu_0I}{4\pi} \frac{d\vec l\times \hat r}{r^2}\\
&=\frac{\mu_0I}{4\pi} \frac{dl\cos\theta(-\hat z)}{r^2}
\end{align*}
since the cross-product between $d\vec l$ and $\hat r$ takes the component of, say, $d\vec l$, that is perpendicular to $\hat r$ ($dl_\perp=dl\cos\theta$), and we know from the right hand rule that it will point in the negative $z$ direction.

If we want to integrate over $\theta$, then we need to express both $d l$ and $r^2$ in terms of $\theta$. We have:
\begin{align*}
r&=\frac{D}{\cos\theta}\\
\therefore \frac{1}{r^2}&=\frac{\cos^2\theta}{D^2}\\
l&=D\tan\theta\\
\therefore dl&=\frac{D}{\cos^2\theta}d\theta
\end{align*}
Putting this altogether, and noting that the wire extends from $\theta_1$ to $\theta_2$:
\begin{align}
\label{eng:BS_general}
\vec B&=\int d\vec B=\int \frac{dl\cos\theta}{r^2}(-\hat z)\nonumber\\
&=\frac{\mu_0I}{4\pi} \int\cos\theta\left( \frac{\cos^2\theta}{D^2}\right) \left( \frac{D}{\cos^2\theta}d\theta\right)(-\hat z)\nonumber\\
&=\frac{\mu_0I}{4\pi D}\int_{\theta_1}^{\theta_2}\cos\theta d\theta(-\hat z)\nonumber\\
&=\frac{\mu_0I}{4\pi D}[\sin\theta]_{\theta_1}^{\theta_2}(-\hat z)\nonumber\\
&=\frac{\mu_0I}{4\pi D}(\sin\theta_2-\sin\theta_1)(-\hat z)
\end{align}
For the vertical sections of wire, where point $P$ is on the wire's bisector, we have $\theta_1=-\theta_2=-\theta_0$ (where $\theta_0$ is half of the angle subtended by the whole wire), so for those sections, the magnetic field is give by:
\begin{align*}
\vec B&=\frac{\mu_0I}{2\pi D}\sin\theta_0(-\hat z)
\end{align*}

If we start by the horizontal sections (H), we have that $P$ is a distance $D=\SI{1}{m}$ from the sections of wire. For the left loop, the two horizontal sections contribute:
\begin{align*}
\sin\theta_1^{left}&=-\frac{5}{\sqrt 26}\\
\sin\theta_2^{left}&=-\frac{2}{\sqrt 5}\\
\therefore\vec B_H^{left}&=\frac{\mu_0I}{2\pi D}(\sin\theta_2^{left}-\sin\theta_1^{left})(-\hat z)\\
&=\frac{\mu_0I}{2\pi D}\left(\frac{-2}{\sqrt 5}+\frac{5}{\sqrt 26}\right)(-\hat z)
\end{align*}
where, by the right hand rule, the field is in the negative $z$ direction (into the page), and we multiplied the result from equation \ref{eng:BS_general} by two.

For the right loop, the two horizontal sections contribute:
\begin{align*}
\sin\theta_1^{right}&=-\frac{1}{\sqrt 2}\\
\sin\theta_2^{right}&=\frac{2}{\sqrt 5}\\
\therefore\vec B_H^{right}&=\frac{\mu_0I}{2\pi D}(\sin\theta_2^{right}-\sin\theta_1^{right})(-\hat z)\\
&=\frac{\mu_0I}{2\pi D}\left(\frac{2}{\sqrt 5}+\frac{1}{\sqrt 2}\right)(-\hat z)
\end{align*}

The total magnetic field from the horizontal sections of wire is thus:
\begin{align*}
\vec B_H&=\vec B_H^{left}+\vec B_H^{right}\\
&=\frac{\mu_0I}{2\pi D}\left(  \frac{-2}{\sqrt 5}+\frac{5}{\sqrt 26}+\frac{2}{\sqrt 5}+\frac{1}{\sqrt 2}\right)(-\hat z)\\
&=\frac{\mu_0I}{2\pi D}\left(  \frac{5}{\sqrt 26}+\frac{1}{\sqrt 2}\right)(-\hat z)\\
&=\frac{4\pi(\SI{1e-7}{NA^{-2}})(\SI{1}{A})}{2\pi(\SI{1}{m})}\left(  \frac{5}{\sqrt 26}+\frac{1}{\sqrt 2}\right)(-\hat z)\\
&=\SI{-3.375e-7}{T}\hat z
\end{align*}

We can now add the contributions of each section of vertical wire. We label the wires as $a$, $b$, $c$, $d$, with $a$ being the left most wire, and $d$ being the right most wire. We can determine the distance from each wire to $P$ as well as the value of $\sin\theta_0$ for each wire:
\begin{align*}
D_a&=\SI{5}{m}& D_b&=\SI{2}{m}& D_c&=\SI{1}{m}& D_d&=\SI{2}{m}\\
\sin\theta_a&=\frac{1}{\sqrt{26}} &\sin\theta_b&=\frac{1}{\sqrt{5}} &\sin\theta_c&=\frac{1}{\sqrt{2}} &\sin\theta_d&=\frac{1}{\sqrt{5}}
\end{align*}

We also have to take into account that the direction of the magnetic field will be different in each case. If $B_i$ refers to the magnitude of the magnetic field in the $z$ direction created by wire $i$, then the total magnetic field at $P$ from the vertical (V) sections is:
\begin{align*}
\vec B_V&=(-B_a+B_b-B_c-B_d)\hat z\\
&=\frac{\mu_0I}{2\pi}\left( -\frac{\sin\theta_a}{D_a}+\frac{\sin\theta_b}{D_b}-\frac{\sin\theta_c}{D_c}-\frac{\sin\theta_d}{D_d}  \right)\hat z\\
&=\frac{\mu_0I}{2\pi}\left( -\frac{1}{(\SI{5}{m})\sqrt{26}}+ \frac{1}{(\SI{2}{m})\sqrt{5}} - \frac{1}{(\SI{1}{m})\sqrt{2}} - \frac{1}{(\SI{2}{m})\sqrt{5}}\right )\hat z\\
&=\frac{4\pi(\SI{1e-7}{NA^{-2}})(\SI{1}{A})}{2\pi}\left( -\frac{1}{(\SI{5}{m})\sqrt{26}} - \frac{1}{(\SI{1}{m})\sqrt{2}}\right )\hat z\\
&=\SI{-1.493e-7}{T}\hat z
\end{align*}

Adding together the magnetic fields from the vertical and horizontal sections, we find:
\begin{align*}
\vec B_{tot}&=\vec B_V+\vec B_H=(\SI{-3.375e-7}{T})\hat z+ (\SI{-1.493e-7}{T})\hat z=\SI{-4.868e-7}{T}\hat z
\end{align*}

\end{solution}

%Original
\question A wire is shaped into a regular polygon with $N$ sides ($N\geq 3$), that is inscribed in a circle of radius $R$. If the wire carries current $I$, show that the magnetic field at the centre of the inscribing circle is: 
\begin{align*}
B=N\frac{\mu_0I}{2\pi R}\tan\left(\frac{\pi}{N}\right)
\end{align*}
Furthermore, show that in the limit of $N\to\infty$, one recovers the expression for the magnetic field at the centre of a circular loop with current $I$.
\begin{solution}
Consider the case for $N=8$, as illustrated in Figure \ref{fig:magneticsource:Octagon}.
\capfig{0.3\textwidth}{figures/MagneticSource/Octagon.png}{\label{fig:magneticsource:Octagon}A wire in the shape of an octagon carrying current $I$.}
The polygon can be thought of being made of $N$ sections of wires that are a distance $d$ from the centre of the circle. Each wire subtends an angle $2\theta$, where $2\theta=\frac{2\pi}{N}$. From the Biot-Savart Law, the magnitude of the magnetic field a distance $d$ away from a wire that subtends an angle $2\theta$ is given by:
\begin{align*}
B_i=\frac{\mu_0I}{2\pi d}\sin\theta
\end{align*}
For the polygon, each segment contributes to the magnetic field by the same amount, and in the same direction (into the page for the current shown in Figure \ref{fig:magneticsource:Octagon}). The distance $d$ is given by $d=R\cos\theta$. The total magnetic field is thus given by:
\begin{align*}
B&=NB_i=N\frac{\mu_0I}{2\pi d}\sin\theta\\
&=N\frac{\mu_0I}{2\pi R\cos\theta}\sin\left(\frac{\pi}{N}\right)\\
&=N\frac{\mu_0I}{2\pi R\cos\left(\frac{\pi}{N}\right)}\sin\left(\frac{\pi}{N}\right)\\
&=N\frac{\mu_0I}{2\pi R}\tan\left(\frac{\pi}{N}\right)\\
\end{align*}
We need to take the limit of $N\to\infty$ of :
\begin{align*}
N\tan\left(\frac{\pi}{N}\right)
\end{align*}
We can write this as a fraction and use L'Hospital's Rule:
\begin{align*}
\lim_{x\to a}\frac{f(x)}{g(x)}=\lim_{x\to a}\frac{f'(x)}{g'(x)}
\end{align*}
We have:
\begin{align*}
N\tan\left(\frac{\pi}{N}\right)=\frac{\tan\left(\frac{\pi}{N}\right)}{\frac{1}{N}}
\end{align*}
Thus, identifying $f(N)=\tan\left(\frac{\pi}{N}\right)$ and $g(N)=\frac{1}{N}$, we have:
\begin{align*}
\lim_{N\to \infty}\frac{\tan\left(\frac{\pi}{N}\right)}{\frac{1}{N}}&=\lim_{N\to \infty}\frac{\frac{1}{\cos^2\left(\frac{\pi}{N}\right)}\left( \frac{-\pi}{N^2}\right)}{\frac{-1}{N^2}}\\
&=\pi\lim_{N\to \infty}\frac{1}{\cos^2\left(\frac{\pi}{N}\right)}=\pi
\end{align*}
Finally, we recover the expression for the magnetic field from a circular loop:
\begin{align*}
\lim_{N\to \infty}N\frac{\mu_0I}{2\pi R}\tan\left(\frac{\pi}{N}\right)=\frac{\mu_0I}{2\pi R}\pi=\frac{\mu_0I}{2 R}
\end{align*}
\end{solution}



% Modified by Troy question from phone app called "Brilliant"

\question Two concentric semi-circles of radii $R_1$ and $R_2$ carry currents $I_1$ and $I_2$, respectively, as shown in Figure \ref{fig:magneticsource:2loops}. Find an expression for the magnitude and direction of the magnetic field at the centre of the two circles (point $P$ in Figure \ref{fig:magneticsource:2loops}).

\capfig{0.4\textwidth}{figures/MagneticSource/2loops.png}{\label{fig:magneticsource:2loops} Two current carrying semicircles}
\begin{finalanswer}
\begin{align*}
\vec B=\frac{\mu_0}{4}\left( \frac{I_1}{R_1}-\frac{I_2}{R_2} \right)(\hat z)
\end{align*}
\end{finalanswer}

\begin{solution}
The straight sections of wire will not contribute to the magnetic field at the center of the semi circles, since they are in line with that point.  We can use the Biot-Savart law to calculate the field from each semi-circle, and then add those together.

Figure \ref{fig:magneticsource:B_semicircle} shows an element of wire, $d\vec l$, and the corresponding magnetic field element, $d\vec B$, from the Biot-Savart law.
\capfig{0.35\textwidth}{figures/MagneticSource/B_semicircle.png}{\label{fig:magneticsource:B_semicircle} Magnetic field element from an element of wire on a semi-circle}

The magnetic field element, $d\vec B$ is given by:
\begin{align*}
d\vec B&=\frac{\mu_0I}{4\pi} \frac{d\vec l\times \hat r}{r^2}\\
&=\frac{\mu_0I}{4\pi r^2}dl (\hat z)
\end{align*}
where we noted that $d\vec l$ (tangent to the circle) and $\hat r$ (radial) are always perpendicular, and we introduced $\hat z$ as the direction out of the page. We can integrate this over the semi circle, noting that everything in the expression of $d\vec B$ is independent of position:
\begin{align*}
\vec B&=\int d\vec B=\int_0^{\pi r}\frac{\mu_0I}{4\pi r^2}dl (\hat z)=\frac{\mu_0I}{4\pi r^2}\int_0^{\pi r}dl (\hat z)\\
&=\frac{\mu_0I}{4\pi r^2}(\pi r)(\hat z)=\frac{\mu_0I}{4 r}(\hat z)
\end{align*}

Noting that the magnetic fields for wire 1 and wire 2 will be in opposite directions, we have:
\begin{align*}
\vec B&=\vec B_1 + \vec B_2\\
&=\frac{\mu_0I_1}{4 R_1}(\hat z)-\frac{\mu_0I_2}{4 R_2}(\hat z)\\
&=\frac{\mu_0}{4}\left( \frac{I_1}{R_1}-\frac{I_2}{R_2} \right)(\hat z)
\end{align*}

\end{solution}


%Adapted by Troy from question 47 in chapter 20 of Giancoli 7th ed (numbers and wording changed) --This was commented out. Any reason why?
\question A \SI{20}{\centi\metre} long and \SI{3}{\centi\metre} in diameter solenoid with 600 turns is aligned along the $x$ axis. The current in its coils is \SI{40}{\ampere}. A straight wire aligned with the $y$ axis passes directly through the center of the solenoid. The current in the wire is \SI{20}{\ampere} and travels in the $-y$ direction. What is the force on the section of wire inside the solenoid assuming the magnetic field of the solenoid points in the $+x$ direction?

\begin{finalanswer}
	\SI{0.09}{N}\vec{z}
\end{finalanswer}

\begin{solution}
We already know the direction of the magnetic field in the solenoid, so we just need to find its magnitude using the general equation for the magnetic field strength inside a solenoid,

\begin{align*}
B = \mu_0 n I_s,
\end{align*}

where $n$ is the number of turns per unit length and $I_s$ is the current flowing through the coils. So the magnetic field can be written as

\begin{align*}
\vec B = \mu_0 n I_s \hat{x}.
\end{align*}

The force acting on a current carrying wire in a uniform magnetic field, $\vec B$ is given as

\begin{align*}
\vec F= I_w \vec{l} \times \vec B,
\end{align*}

where $I_w$ is the current flowing through the wire  and $\vec{l}$ is the length vector of the affected part of the wire, which can be written as $l(-\hat y)$. Substituting in the expression for the magnetic field inside the solenoid gives us

\begin{align*}
\vec F = (I_w l)(\mu_0 n I_s)(-\hat{\mathbf{y}} \times \hat{\mathbf{x}}),\\
\vec F = \mu_0 I_w I_s l n \hat{z}.
\end{align*}

Now simply substitute values:

\begin{align*}
\vec F  = \mu_0 (\SI{20}{A}) (\SI{40}{A}) (\SI{3}{cm}) \left( \frac{600}{\SI{20}{cm}}\right) \hat{z}\\
\vec F  = \SI{0.09}{N}\hat{z}
\end{align*}

\end{solution}

%Question 43 in chapter 20 of Giancoli 7th ed -fixed
\question Two very long wires are placed perpendicular to one another. When the wires are closest, they are \SI{15}{cm} apart, as shown in Figure \ref{fig:magneticsource:PerpWires}. If the top wire in the figure has a current of \SI{14}{A} and the bottom wire has a current of \SI{18}{A}, what is the magnetic field midway between them at their closest point.

\capfig{0.4\textwidth}{figures/MagneticSource/PerpWires.png}{\label{fig:magneticsource:PerpWires} Two perpendicular current carrying wires.}
\begin{finalanswer}
\SI{6.08e-5}{T}
\end{finalanswer}
\begin{solution}
The magnetic field from wire 1 will be to the right and the magnetic field from wire 2 will be out of the page. Since the two wires create magnetic fields that are perpendicular to each other, the magnitude of the total field will be the quadrature sum of the two magnetic fields (as shown in Figure \ref{fig:magneticsource:Quadrature}).
\capfig{0.2\textwidth}{figures/MagneticSource/Quadrature.png}{\label{fig:magneticsource:Quadrature} View from above of the fields from the wires shown in Figure \ref{fig:magneticsource:PerpWires}.}

One can easily find the magnetic field a distance $R$ away from a wire carrying current $I$ using Amp\`ere's law:
\begin{align*}
\oint \vec B(R) \cdot d\vec l&=\mu_0I\\
B2\pi R&=\mu_0 I\\
B&=\frac{\mu_0 I}{2\pi R}
\end{align*}

The magnetic field from wire 1:
\begin{align*}
B_1 = \frac{\mu_0 (\SI{14.0}{A})}{2\pi (\SI{7.5}{cm})}\\ 
= \SI{3.73e-5}{T}\\
\end{align*}
and from wire 2
\begin{align*}
B_2 = \frac{\mu_0 (\SI{18.0}{A})}{2\pi (\SI{7.5}{cm})}\\ 
= \SI{4.8e-5}{T}\\
\end{align*}

Giving the total magnetic field:
\begin{align*}
B_{tot} = \sqrt{(\SI{3.73e-5}{T})^2 + (\SI{4.8e-5}{T})^2}\\
= \SI{6.08e-5}{T}
\end{align*}

\end{solution}

%Giancolli 28-31 -fixed
\question A coaxial cable is made up of three shells of radius $R_1, R_2,$ and $R_3$ respectively. The outermost shell does not carry a current, but the inner most shell and the middle shell carry equal and opposite currents which are uniformly distributed, as shown in Figure \ref{fig:magneticsource:Coax}. Determine the magnetic field a distance $r$ from the axis of the cable for each of the following cases:
\begin{parts}
\part $r<R_1$
\part $R_1<r<R_2$
\part $R_2<r<R_3$
\part $R_3<r$
\part Plot the magnitude of the magnetic field as a function of $r$ (choose values for the various quantities).
\end{parts}
\capfig{0.4\textwidth}{figures/MagneticSource/Coax.png}{\label{fig:magneticsource:Coax} A coaxial wire.}
\begin{finalanswer}
\begin{enumerate}[(a)]
\item \begin{align*}
B(r)=\frac{\mu_0 I r}{2\pi R_1^2}\quad (r<R_1)
\end{align*}
\item \begin{align*}
B(r)=\frac{\mu_0 I}{2\pi r}\quad (R_1<r<R_2)
\end{align*}
\item \begin{align*}
B(r)=\frac{\mu_0I}{2\pi r}\frac{(R_3^2-r^2)}{(R_3^2-R_2^2)}\quad (R_2<r<R_3)
\end{align*}
\item \begin{align*}
B(r) = 0 \quad (R_3<r)
\end{align*}
\item \capfig{0.7\textwidth}{figures/MagneticSource/BofR.png}{\label{fig:magneticsource:BofR2} Magnetic field in a coax cable.}
\end{enumerate}
\end{finalanswer}
\begin{solution}
In all cases, the magnetic field will form concentric circles about the axis of the cable. We can use Amp\`ere's Law to determine the magnetic field depending on the amount of current enclosed. 
\begin{parts}
\part In the centre conductor, we need to use the current density to determine the current enclosed by an Amp\`erian loop of radius $r<R_1$. The current density is given by:
\begin{align*}
j=\frac{I}{\pi R_1^2}
\end{align*}
An Amp\`erian loop of radius $r<R_1$ will thus enclose a current:
\begin{align*}
I(r)=\pi r^2 j = I\frac{r^2}{R_1^2}
\end{align*}
Applying Amp\`ere's Law:
\begin{align*}
\oint \vec B(r) \cdot d\vec l&=\mu_0I^{enc}\\
B(r)2\pi r &= \mu_0 I\frac{r^2}{R_1^2}\\
\therefore B(r)&=\frac{\mu_0 I r}{2\pi R_1^2}\quad (r<R_1)
\end{align*}
\part Between the inner conductor and the shield, the enclosed current is that on the central conductor, namely ($I$), so the magnetic field is given by:
\begin{align*}
\oint \vec B(r) \cdot d\vec l&=\mu_0I^{enc}\\
B(r)2\pi r &= \mu_0 I\\
\therefore B(r)&=\frac{\mu_0 I}{2\pi r}\quad (R_1<r<R_2)
\end{align*}
\part Inside the shield, we need to use the current density of the shield to determine the current enclosed at some radial position inside the shield. The current density in the shield, $j_s$ is given by:
\begin{align*}
j_s=-\frac{I}{\pi(R_3^2-R_2^2)}\\
\end{align*} 
where the minus sign accounts for the different direction of the current in the shield. At some radial position $R_2<r<R_3$ the total enclosed current is $I$ from the inner conductor plus the fraction of the current enclosed in the shield:
\begin{align*}
I(r) &= I+j_s\pi(r^2-R_2^2)=I\left(1-\frac{(r^2-R_2^2)}{(R_3^2-R_2^2)} \right)\\
&=I\frac{(R_3^2-r^2)}{(R_3^2-R_2^2)}
\end{align*}
Again, using Amp\`ere's Law:
\begin{align*}
\oint \vec B(r) \cdot d\vec l&=\mu_0I^{enc}\\
B(r)2\pi r &= \mu_0I\frac{(R_3^2-r^2)}{(R_3^2-R_2^2)}\\
\therefore B(r)&=\frac{\mu_0I}{2\pi r}\frac{(R_3^2-r^2)}{(R_3^2-R_2^2)}\quad (R_2<r<R_3)
\end{align*}
\part Outside the shield, the total enclosed current is zero, so the magnetic field is zero (hence the name shield for the outer conductor). 
\begin{align*}
\therefore B(r) = 0 \quad (R_3<r)
\end{align*}
\part We can make the plot shown in Figure \ref{fig:magneticsource:BofR} in python using the following code:
\begin{verbatim}
import pylab as pl
import numpy as np

#Define the radii
R1, R2, R3 = 1, 2, 2.2

#Define the magnetic field as a function of radius
#(assume mu_0I/2/pi = 1)
def BofR(r):
    if r < R1:
        return r/R1**2
    elif r >= R1 and r < R2:
        return 1/r
    elif r >= R2 and r < R3:
        return 1/r*(R3**2-r**2)/(R3**2-R2**2)
    else: return 0
    
#define radial values, and evalue B(r)
rvals = np.linspace(0,1.1*R3, 100)
bvals = np.zeros(rvals.size)
for i in range(rvals.size):
    bvals[i] = BofR(rvals[i])

#plot it
pl.plot(rvals,bvals)
pl.ylabel("Magnetif field")
pl.xlabel("Radius")
pl.title("B field in coax with R1={}, R2={}, R3={}".format(R1,R2,R3))
pl.show()
\end{verbatim}
\capfig{0.7\textwidth}{figures/MagneticSource/BofR.png}{\label{fig:magneticsource:BofR} Magnetic field in a coax cable.}
\end{parts}
\end{solution}

\question The magnetic field at the surface of the Earth is approximately uniform with a magnitude of $B=\SI{25e-6}{T}$ and points North. If you have a circular loop of wire with radius $r=\SI{20}{cm}$, what is the minimum frequency with which you must rotate it at the surface of the Earth in order to induce a maximal voltage (emf) of $\Delta= V\SI{5}{V}$ across the loop?
\begin{finalanswer}
	\SI{253.3}{Hz}
\end{finalanswer}
\begin{solution}
	We can write the flux as a function of time as:
	\begin{align*}
	\Phi_B = AB\cos(\omega t)
	\end{align*}
	where $A=\pi r^2$ is the area of the loop. The induced emf is thus:
	\begin{align*}
	\Delta V = -\frac{d\Phi_B}{dt}=\omega AB \sin(\omega t)
	\end{align*}
	The maximum voltage is given by:
	\begin{align*}
	\omega A B &= \SI{5}{V}\\
	\therefore \omega &= \frac{(\SI{5}{V})}{AB}
	\end{align*}
	Allowing us to determine the frequency:
	\begin{align*}
	\therefore f &= \frac{\omega}{2\pi}=\frac{(\SI{5}{V})}{2\pi^2(\SI{0.2}{m})^2(\SI{25e-6}{T})}=\SI{253.303}{kHz}
	\end{align*}
\end{solution}

\question The magnetic field at the surface of the Earth is approximately uniform with a magnitude of $B=\SI{25e-6}{T}$ and points North. You have a piece of wire that you shape into a coil containing $N$ concentric circular loops of radius $r=\SI{20}{cm}$. If you rotate the coil with a frequency of $f=\SI{100}{Hz}$, what is the minimum number of loops, $N$, required in the coil in order to have a maximal induced voltage (emf) across the coil of $\Delta V=\SI{5}{V}$?
\begin{finalanswer}
	2534
\end{finalanswer}
\begin{solution}
	We can write the flux as a function of time as:
	\begin{align*}
	\Phi_B = AB\cos(\omega t)
	\end{align*}
	where $A=\pi r^2$ is the area of the loop. The induced emf is thus:
	\begin{align*}
	\Delta V = -N\frac{d\Phi_B}{dt}=\omega NAB \sin(\omega t)
	\end{align*}
	The maximum voltage is given by:
	\begin{align*}
	\omega  NAB &= \SI{5}{V}\\
	\therefore N&=\frac{(\SI{5}{V})}{AB\omega}=\frac{(\SI{5}{V})}{AB2\pi f}\\
	&=\frac{(\SI{5}{V})}{2\pi^2(\SI{100}{Hz})(\SI{0.2}{m})^2(\SI{25e-6}{T})}=2533.03
	\end{align*}
	In principle, we should round up to get the minimum number of coils, which is 2534.
\end{solution}

\question Answer the following:
\begin{parts}
	%Question from Dana Fahey, textbook application 2019
	\part Two infinitely long wires separated by a distance $R$ carry currents $I_1$ and $I_2$ as shown in Figure \ref{fig:MagneticSource:1loop1wire}. The wire carrying current $I_1$ has a section that is bent into a semi-circle of radius R and centred at point $P$ (a distance $R$ from the wire carrying current $I_2$). What is the total magnetic field vector at point $P$?
	\capfig{0.5\textwidth}{figures/MagneticSource/1loop1wire.png}{\label{fig:MagneticSource:1loop1wire}Two infinitely long wires carrying current.}
	\part A thin insulating charged disk of radius $R$ carries a total charge $+Q$ spread uniformly over its surface. The disk rotates with a constant angular velocity, $\omega$, as shown in Figure \ref{fig:MagneticSource:chargeddisk}, and has a small hole (of negligible dimensions) right at its centre. What is the magnitude of the magnetic field created by the disk inside the centre of the hole? Assume that the thickness of the disk is negligible and that the charge can be treated as a uniform surface charge density $\sigma=\frac{Q}{\pi R^2}$.
	\capfig{0.3\textwidth}{figures/MagneticSource/chargeddisk.png}{\label{fig:MagneticSource:chargeddisk}A thin charged rotating disk with a hole at its centre.}
\end{parts}
\begin{finalanswer}
	\begin{parts}
		\part $\left(\frac{\mu_0I_2}{2\pi R}-\frac{\mu_0I_1}{4R}\right)\hat z$
		\part $\frac{\mu_0 Q\omega}{2\pi R}$
	\end{parts}
\end{finalanswer}
\begin{solution}
	\begin{parts}
		\part We can sum together the fields from the circular wire and the straight wire. The field from the straight wire at point $P$ (a distance $R$ from the wire carrying current $I_2$) is easily found using Amp\`ere's Law:
		\begin{align*}
		\vec B_2 = \frac{\mu_0I_2}{2\pi R}\hat z
		\end{align*}
		and comes out of the page (positive z).
		
		We use the Biot-Savart Law to find the field at point P from the circular section of wire carrying current $I_1$. Consider a small element of wire of length $dl$, as illustrated in Figure \ref{fig:MagneticSource:1loop1wire_sol}.
		\capfig{0.3\textwidth}{figures/MagneticSource/1loop1wire_sol.png}{\label{fig:MagneticSource:1loop1wire_sol}Two wires carrying current.}
		The field from that piece of wire is given by the Biot-Savart Law:
		\begin{align*}
		d\vec B_1 = \frac{\mu_0I_1}{4\pi}\frac{d\vec l\times \vec r}{r^3}
		\end{align*}
		Since the vectors $d\vec l$ and $\vec r$ are always perpendicular, and $\vec r$ always has a magnitude $R$, the magnetic field will be into the page (negative z) direction:
		\begin{align*}
		d\vec B_1 &= -\frac{\mu_0I_1}{4\pi}\frac{dl}{R^2}\hat z
		\end{align*}
		We can sum these together trivially, since the $dB$ do not depend on the particular location of the $d\vec l$:
		\begin{align*}
		\vec B_1 &= -\int\frac{\mu_0I_1}{4\pi}\frac{dl}{R^2}(\hat z)=-\frac{\mu_0I_1}{4\pi R^2}\int dl (\hat z)=-\frac{\mu_0I_1}{4\pi R^2}(\pi R) (\hat z)\\
		&=-\frac{\mu_0I_1}{4R}\hat z
		\end{align*}
		This indeed corresponds to half of the magnitude of the magnetic field at the centre of a loop of radius $R$ carrying the same current.
		
		The straight sections of wire 1 do not contribute to the magnetic field at point $P$ (since the vector $\vec r$ would be parallel to the vector $d\vec l$). Thus, the total magnetic field at point $P$ is given by:
		\begin{align*}
		\vec B = \vec B_1 + \vec B_2 =  \left(\frac{\mu_0I_2}{2\pi R}-\frac{\mu_0I_1}{4R}\right)\hat z
		\end{align*}
		
		\part We break up the disk into infinitesimal rings of radius $r$ and thickness $dr$, and use the result that the magnetic field at the centre of a ring carrying current $I$ is given by (e.g. twice the result from part (b) above):
		\begin{align*}
		B = \frac{\mu_0I}{2 R}
		\end{align*}
		A specific ring of radius $r$ and thickness $dr$, will carry a charge $dq$ given by:
		\begin{align*}
		dq = \sigma (2\pi r) dr
		\end{align*}
		That charge will go around in a period of time $T$ given by:
		\begin{align*}
		T=\frac{2\pi}{\omega}
		\end{align*}
		which corresponds to an infinitesimal current:
		\begin{align*}
		dI=\frac{dq}{T}=\sigma (2\pi r) dr \frac{\omega}{2\pi}=\sigma\omega r dr
		\end{align*}
		The magnetic field from that ring of radius $r$, at the centre, is then given by:
		\begin{align*}
		dB = \frac{\mu_0dI}{2r}=\frac{\mu_0 (\sigma\omega r dr)}{2r} =\frac{\mu_0 \sigma\omega}{2}dr 
		\end{align*}
		Summing the fields from all the rings:
		\begin{align*}
		B &= \int dB = \int_0^R\frac{\mu_0 \sigma\omega}{2}dr =\frac{\mu_0 \sigma\omega R}{2}\\
		&=\frac{\mu_0 Q\omega}{2\pi R}
		\end{align*}
		
	\end{parts}
\end{solution}

%TODO: Calculate the acceleration of a railgun (which involves calculating the magnetic field in a square loop, and potentially integrating)