\section{Gravity}

%%%%%%%%%%%%%%%%%%%%%%%%%%%%%%%%%%%
%%
%% Multiple Choice
%%
%%%%%%%%%%%%%%%%%%%%%%%%%%%%%%%%%%%
\subsection{Multiple Choice}

%Stephanie
\question The Earth is closest to the Sun in December. When is the Earth's gravitational potential energy with respect to the Sun the greatest? 
\begin{checkboxes}
\CorrectChoice When the Northern Hemisphere has Summer. \correct
\choice When the Northern Hemisphere has Winter.
\choice Not enough information.
\end{checkboxes}

%James
\question  Which of the following planets has the lowest value of the acceleration from gravity at its surface?
\begin{checkboxes}
\choice A planet with mass $2M$ and radius $1/2$ $R$
\choice A planet with mass $4M$ and radius $1/\sqrt{2}$ $R$
\choice A planet with mass $M$ and radius $1/3$ $R$
\CorrectChoice A planet with mass $6M$ and radius $R$ \correct
\end{checkboxes}

%Question submitted by Josh Rinaldo
\question Jack is standing on top of a tall ladder, and Jill is on the ground. The gravitational acceleration felt by Jack is $\rule{1cm}{0.15mm}$ the gravitational acceleration acting on Jill. 
\begin{checkboxes}
\CorrectChoice less than \correct
\choice greater than
\choice the same as
\end{checkboxes}

\question A satellite of mass $m$ is in a circular orbit of radius $r$ about a spherical planet of mass $M$, which can be considered fixed in space. If the gravitational potential energy, $U$, of $m$ is defined as $U=-G\frac{Mm}{r}$, what can you say about the kinetic energy, $K$, of $m$?
\begin{choices} 
	\choice $K=\frac{1}{2}U$
	\CorrectChoice  $K=-\frac{1}{2}U$
	\choice  $K=U$
	\choice $K=-U$
\end{choices}
\begin{solution}
	In a circular orbit, the centripetal acceleration is provided by the force of gravity:
	\begin{align*}
	m\frac{v^2}{r}=G\frac{Mm}{r^2}
	\end{align*}
	If we multiply each side by $\frac{1}{2}r$, we find:
	\begin{align*}
	\frac{1}{2}mv^2&=G\frac{Mm}{2r}\\
	K&=-\frac{1}{2}U
	\end{align*}
	where:
	\begin{align*}
	U=-G\frac{Mm}{r}
	\end{align*}
\end{solution}

%Ryan U
\question Two astronauts are floating freely in space with no means of propulsion.  Each astronaut, with their suit on, has a mass of \SI{100}{kg}.  They are holding onto each other with their arms extended keeping them \SI{.1}{m} apart, so that they don't drift apart.  They let go for a second, and push off on each other a little.  What is the maximum speed they can move apart if their bodies are ever to be reunited under their mutual gravitational attraction?
\begin{checkboxes}
\choice \SI{0.37}{m/s}
\choice \SI{0.037}{m/s}
\choice \SI{0.0037}{m/s}
\CorrectChoice \SI{0.00037}{m/s} \correct
\end{checkboxes}

%Troy
\question The Earth experiences tidal forces from both the Sun and the Moon. Which of the following statements is true about the tidal forces acting on the Earth?
\begin{checkboxes}
\choice The Sun is responsible for the larger of the two forces because of its much larger mass
\choice The Moon is responsible for the larger of the two forces because of its much closer proximity
\choice The Sun is responsible for the larger of the two forces because it has a larger difference between forces on the near and far side of the Earth
\CorrectChoice The Moon is responsible for the larger of the two forces because it has a larger difference between forces on the near and far side of the Earth \correct
\choice Two of the above are correct
\end{checkboxes}

%Submitted by Daniel Tazbaz
\question A guanaco and a vicu\~na are trying to escape Earth's gravity. The guanaco has mass of \SI{85}{kg}, and the mass of the vicu\~na is \SI{60}{kg}. Which animal will require a greater initial velocity to escape Earth's gravitational pull, never to return? (Neglect air resistance.)
\begin{checkboxes}
\choice The guanaco requires a greater initial velocity.
\choice The vicu\~na requires a greater initial velocity.
\CorrectChoice Both animals will require the same initial velocity. \correct
\end{checkboxes}

%Submitted by Adam McCaw
\question If a planet has a mass 10 times that of the Earth, and a radius 2 times that of the Earth, the acceleration due to gravity on the surface of the planet in terms of $g$ would be:
\begin{checkboxes}
\CorrectChoice 2.5g \correct
\choice g
\choice g/5
\choice 5g
\choice g/2.5
\end{checkboxes}


%Joanna Fu
\question You are on an airplane travelling with a constant velocity at an altitude of \SI{20000}{m} above the surface of the Earth. What is the acceleration of gravity at that altitude? (The radius of Earth is \SI{6.37e6}{m}, and its mass is \SI{5.97e24}{kg}).
\begin{checkboxes} 
\choice \SI{9.70}{m/s^2}
\CorrectChoice \SI{9.75}{m/s^2} \correct
\choice \SI{9.80}{m/s^2}
\choice \SI{9.85}{m/s^2}
\end{checkboxes}

\question A rocketship follows a circular orbit of radius $r$ around Earth. The rocket quickly accelerates to twice its initial speed (so the acceleration is essentially instantaneous). What will happen to the period of the rocket's orbit?
\begin{checkboxes}
\choice It will be the same as its initial period.
\CorrectChoice It will be greater than its initial period. \correct
\choice It will be less than its initial period. 
\end{checkboxes}

\question In general, the gravitational potential energy of a mass $m$ a distance $r$ away from a mass $M$
\begin{checkboxes}
\choice is equal to the work done by gravity in moving $m$ from an infinite distance away from $M$ to the current position.
\choice is equal to the work done by a person in moving $m$ from an infinite distance away from $M$ to the current position.
\CorrectChoice can be positive or negative.
\choice is none of the above.
\end{checkboxes}


\question A mass $m$ is located in vacuum a distance $r$ away from another mass $M$ which can be considered fixed in space. If we move mass $m$ so that it is infinitely far away from $M$, what is the work done by the force of gravity on $m$?
\begin{checkboxes}
\CorrectChoice $W=-G\frac{Mm}{r}$
\choice  $W=G\frac{Mm}{r}$
\choice not enough information to tell
\end{checkboxes}

\question A mass $m$ is located in vacuum a distance $r$ away from another mass $M$ which can be considered fixed in space. If we move mass $m$ so that it is closer to mass $M$ (decrease $r$), what can you say about the gravitational potential energy of $m$?
\begin{checkboxes}
\CorrectChoice It decreases
\choice  It increases
\choice not enough information to tell
\end{checkboxes}

%%%%%%%%%%%%%%%%%%%%%%%%%%%%%%%%%%%
%
% long answer
%
%%%%%%%%%%%%%%%%%%%%%%%%%%%%%%%%%%%
\subsection{Long answers}


\question Answer the following:
\begin{parts}
\part Show that the escape velocity a distance $R$ from the centre of an object of mass $M$ is given by:
\begin{align*}
v_{esc}=\sqrt{\frac{2GM}{R}}
\end{align*}
\part Determine the escape velocity from the Sun for an object at the Sun's surface ($R_\odot=\SI{7.0e5}{km}$, $M_\odot=\SI{2.0e30}{kg})$. 
\part Determine the escape velocity from the Sun for an object at the average distance of the Earth ($d_\oplus=\SI{1.5e8}{km}$). Compare this with the average orbital speed of the Earth around the Sun (assume a circular orbit to determine the Earth's orbital speed).
\end{parts}
\begin{finalanswer}
\begin{enumerate}[(a)]
\item N/A
\item $\SI{6.2e5}{m/s}$
\item $v_{esc}=\SI{4.2e4}{m/s}$. The escape velocity is $\frac{v_{esc}}{v}=\sqrt{2} \sim 1.4$ times bigger than the average orbital speed of the Earth around the Sun, so the Earth is going to stay in orbit around the Sun for the time being.
\end{enumerate}
\end{finalanswer}

\begin{solution}
\begin{parts}
\part The escape velocity is given by the condition that an object will have zero kinetic energy at infinity. If we write gravitational potential energy of an object of mass $m$, choosing 0 at infinity:
\begin{align*}
U(r)=-G\frac{Mm}{r}
\end{align*}
Then, at position $R$, the object would have the following mechanical energy:
\begin{align*}
E(R)=\frac{1}{2}mv^2-G\frac{Mm}{R}
\end{align*}
At infinity, it will have zero kinetic and zero potential energy. Thus, the total energy above must equal zero:
\begin{align*}
\frac{1}{2}mv^2-G\frac{Mm}{R}&=0\\
\therefore v_{esc}=\sqrt{\frac{2GM}{R}}
\end{align*}
as required
\part At the surface of the Sun, this gives:
\begin{align*}
v_{esc}=\sqrt{\frac{2GM_{\odot}}{R_{\odot}}}=\sqrt{\frac{2(\SI{6.67e-11}{Nm^2/kg^2})(\SI{2e30}{kg})}{(\SI{7.0e8}{m}})}=\SI{6.2e5}{m/s}\\
\end{align*}
\part At the average distance to the Earth, this gives:
\begin{align*}
v_{esc}=\sqrt{\frac{2GM_{\odot}}{d_\oplus}}=\sqrt{\frac{2(\SI{6.67e-11}{Nm^2/kg^2})(\SI{2e30}{kg})}{(\SI{1.5e11}{m}})}=\SI{4.2e4}{m/s}\\
\end{align*}
The orbital velocity of the Earth can be found by equating the force of gravity from the Sun on the Earth, to the force required to keep the Earth in a circular orbit of radius $d_\oplus$:
\begin{align*}
m_\oplus \frac{v^2}{d_\oplus}&=G\frac{M_\odot m_\oplus}{d_\oplus^2}\\
\therefore v&=\sqrt{\frac{GM}{d_\oplus}}
\end{align*}
The escape velocity is $\frac{v_{esc}}{v}=\sqrt{2} \sim 1.4$ times bigger than this, so the Earth is going to stay in orbit around the Sun for the time being.
\end{parts}
\end{solution}

\question A sphere of radius $R_2$ has a concentric spherical cavity of radius $R_1$, resulting in a spherical shell. This spherical has a uniform density throughout and has a total mass of $M$. Show that the force experienced by a mass $m$ at a distance $r$ from the centre of the shell is given by:
\begin{align*}
F(r) = \begin{cases}
0 &\;\; r \leq R_1\\
-\frac{GMm}{r^2}\frac{(r^3-R_1^3)}{(R_2^3-R_1^3)}  &\;\; R_1<r\leq R_2\\
-G\frac{Mm}{r^2} &\;\; R_2<r
\end{cases}
\end{align*}
where a minus sign indicates that the force points towards the centre of the shell.
\begin{solution}
We use Gauss' Law, which tells us that, at a distance $r$ from the centre of a spherical mass distribution, only the mass at radial values less than $r$ contribute to the gravitational force, and that it is equivalent to concentrating all of that mass at the centre.

For $R_2<r$, all of the mass of the shell can be modelled as being concentrated at the centre, so that the gravitational force is given by:
\begin{align*}
F(r) &= -G\frac{Mm}{r^2}
\end{align*}


When  $R_1<r\leq R_2$, we need to know the density of the shell to know how much mass is between $R_1$ and $r$. The total volume of the shell is given by subtracting the volume of the cavity from the volume of the sphere of radius $R_2$:
\begin{align*}
V^{shell}=\frac{4}{3}\pi R_2^3-\frac{4}{3}\pi R_1^3=\frac{4}{3}\pi (R_2^3-R_1^3)
\end{align*}
The density of the shell is then:
\begin{align*}
\rho=\frac{M}{V}=\frac{3M}{4\pi (R_2^3-R_1^3)}
\end{align*}
The mass, $M_s(r)$, of a shell with outer radius $r$ and inner radius $R_1$ is given by the volume of that shell times the density:
\begin{align*}
M_s(r)&=\rho V_s=\rho \frac{4}{3}\pi (r^3-R_1^3)=\frac{3M}{4\pi (R_2^3-R_1^3)}\frac{4}{3}\pi (r^3-R_1^3)\\
&=M\frac{(r^3-R_1^3)}{(R_2^3-R_1^3)} 
\end{align*}
Gauss' Law tells us that at a radius $r$, the net gravitational force:
\begin{itemize}
\item Comes only from the mass at smaller radii ($M_s(r)$)
\item Can be calculated by assuming that all of that mass is located at $r=0$
\end{itemize}
Thus, at a radius $R_1<r\leq R_2$, the gravitational force inside the shell is:
\begin{align*}
F(r) = -\frac{GmM_s(r)}{r^2} = -\frac{GMm}{r^2}\frac{(r^3-R_1^3)}{(R_2^3-R_1^3)} 
\end{align*}
where the minus sign indicates that the force is in the negative radial direction (towards $r=0$).

Finally, the gravitational force for a mass $m$ at a radius $R_1<r\leq R_2$ is zero, since there is no mass enclosed at a smaller radius.

\end{solution}
\question  A sphere of radius $R_2$ has a concentric spherical cavity of radius $R_1$, resulting in a spherical shell. This spherical shell has a uniform density throughout and has a total mass of $M$. Show that the gravitational potential energy of a mass $m$ at a distance $r$ from the centre of the shell is given by:
\begin{align*}
U(r) = \begin{cases}
\frac{GMm}{(R_2^3-R_1^3)} \left(  \frac{3}{2}R_1^2 \right) -G\frac{Mm}{R_2}-\frac{GMm}{(R_2^3-R_1^3)} \left(  \frac{1}{2}R_2^2+\frac{R_1^3}{R_2}  \right) &\;\; r \leq R_1\\
\frac{GMm}{(R_2^3-R_1^3)} \left(  \frac{1}{2}r^2+\frac{R_1^3}{r}  \right) -G\frac{Mm}{R_2}-\frac{GMm}{(R_2^3-R_1^3)} \left(  \frac{1}{2}R_2^2+\frac{R_1^3}{R_2}  \right)  &\;\; R_1<r\leq R_2\\
-G\frac{Mm}{r} &\;\; R_2<r
\end{cases}
\end{align*}
where we have chosen 0 potential energy when $r\to\infty$.
\begin{solution}
We use Gauss' Law, which tells us that, at a distance $r$ from the centre of a spherical mass distribution, only the mass at radial values less than $r$ contribute to the gravitational force, and that it is equivalent to concentrating all of that mass at the centre.

For $R_2<r$, all of the mass of the shell can be modelled as being concentrated at the centre, so that the potential energy is given by:
\begin{align*}
U(r) &= -G\frac{Mm}{r}
\end{align*}
This has the correct choice of constant so that potential energy will be equal to zero when $r\to\infty$.

When  $R_1<r\leq R_2$, we need to know the density of the shell to know how much mass is between $R_1$ and $r$. The total volume of the shell is given by subtracting the volume of the cavity from the volume of the sphere of radius $R_2$:
\begin{align*}
V^{shell}=\frac{4}{3}\pi R_2^3-\frac{4}{3}\pi R_1^3=\frac{4}{3}\pi (R_2^3-R_1^3)
\end{align*}
The density of the shell is then:
\begin{align*}
\rho=\frac{M}{V}=\frac{3M}{4\pi (R_2^3-R_1^3)}
\end{align*}
The mass, $M_s(r)$, of a shell with outer radius $r$ and inner radius $R_1$ is given by the volume of that shell times the density:
\begin{align*}
M_s(r)&=\rho V_s=\rho \frac{4}{3}\pi (r^3-R_1^3)=\frac{3M}{4\pi (R_2^3-R_1^3)}\frac{4}{3}\pi (r^3-R_1^3)\\
&=M\frac{(r^3-R_1^3)}{(R_2^3-R_1^3)} 
\end{align*}
Gauss' Law tells us that at a radius $r$, the net gravitational force:
\begin{itemize}
\item Comes only from the mass at smaller radii ($M_s(r)$)
\item Can be calculated by assuming that all of that mass is located at $r=0$
\end{itemize}
Thus, at a radius $R_1<r\leq R_2$, the gravitational force inside the shell is:
\begin{align*}
F(r) = -\frac{GmM_s(r)}{r^2}=-Gm\frac{1}{r^2}M\frac{(r^3-R_1^3)}{(R_2^3-R_1^3)} = -\frac{GMm}{(R_2^3-R_1^3)}\frac{(r^3-R_1^3)}{r^2}
\end{align*}
where the minus sign indicates that the force is in the negative radial direction (towards $r=0$). To calculate the change in potential energy in going from $R_1$ to $R_1<r\leq R_2$, we calculate the work done by gravity:
\begin{align*}
-\Delta U &=W =\int_{R_1}^{r}Fdr=-\int_{R_1}^{r}\frac{GMm}{(R_2^3-R_1^3)}\frac{(r^3-R_1^3)}{r^2}dr\\
&=-\frac{GMm}{(R_2^3-R_1^3)} \int_{R_1}^{r}\frac{(r^3-R_1^3)}{r^2}dr\\
&=-\frac{GMm}{(R_2^3-R_1^3)} \left[  \frac{1}{2}r^2+\frac{R_1^3}{r}  \right]_{R_1}^{r}
\end{align*}
We can define a potential energy function inside the shell as:
\begin{align*}
U(r) = \frac{GMm}{(R_2^3-R_1^3)} \left(  \frac{1}{2}r^2+\frac{R_1^3}{r}  \right) + C
\end{align*}
since this will give the correct value of work in going from $R_1$ to $r$ somewhere inside the shell. The constant $C$ needs to be chosen so that the potential energy function matches the function outside the shell at $r=R_2$, namely, we require:
\begin{align*}
U(R_2) &= -G\frac{Mm}{R_2} \\
\frac{GMm}{(R_2^3-R_1^3)} \left(  \frac{1}{2}R_2^2+\frac{R_1^3}{R_2}  \right) + C &= -G\frac{Mm}{R_2}\\
\therefore C&=-G\frac{Mm}{R_2}-\frac{GMm}{(R_2^3-R_1^3)} \left(  \frac{1}{2}R_2^2+\frac{R_1^3}{R_2}  \right)
\end{align*} 
Finally, rhe potential energy for a mass $m$ at a radius $R_1<r\leq R_2$ is thus given by:
\begin{align*}
U(r) = \frac{GMm}{(R_2^3-R_1^3)} \left(  \frac{1}{2}r^2+\frac{R_1^3}{r}  \right) -G\frac{Mm}{R_2}-\frac{GMm}{(R_2^3-R_1^3)} \left(  \frac{1}{2}R_2^2+\frac{R_1^3}{R_2}  \right)
\end{align*}

When $r\leq R_1$, there is no mass enclosed, since it is an empty cavity. The force for $r\leq R_1$ is zero, and the potential has a constant value. The constant value  inside the cavity must equal the potential energy at the inner surface of the shell, namely, the above expression evaluated at $r=R_1$:
\begin{align*}
U(R_1)&=\frac{GMm}{(R_2^3-R_1^3)} \left(  \frac{3}{2}R_1^2 \right) -G\frac{Mm}{R_2}-\frac{GMm}{(R_2^3-R_1^3)} \left(  \frac{1}{2}R_2^2+\frac{R_1^3}{R_2}  \right)
\end{align*}

The potential is plotted in Figure \ref{fig:gravity:UofRShell}.
\capfig{0.4\textwidth}{figures/Gravity/UofRShell.png}{\label{fig:gravity:UofRShell} Potential as a function of radius for a shell.}

\end{solution}


\question A robotic lander with an Earth weight of \SI{3430}{N} is sent to Mars, which has a radius of $R_M = \SI{3.40e6}{m}$ and a mass of $M_M =\SI{6.42e23}{kg}$.
\begin{parts}
\part Find the acceleration due to gravity on the surface of Mars, $g_M$
\part Find the weight $F_g$ of the lander on the Martian surface.
\part Since the mass of Mars is about 10 times less than that of Earth, comment on whether the results in parts a) and b) make sense.
\end{parts}
\begin{finalanswer}
\begin{enumerate}[(a)]
\item $\SI{3.7}{m/s^2}$
\item \SI{1296.50}{N}
\item The answers make sense since although the mass of Mars is an order of magnitude smaller than that of Earth, the gravitational acceleration is less significantly affected because of the inverse square law, as Mars' radius being around half the size of Earth's. 
\end{enumerate}
\end{finalanswer}
\begin{solution}
\begin{parts}
\part At the surface of the planet, the acceleration due to gravity is:
\begin{align*}
F_g&=mg_M=\frac{Gm M_M}{R_M^2}\\
\therefore g_M&=\frac{GM_M}{R_M^2}=\frac{(\SI{6.67e-11}{N m^2/kg^2})(\SI{6.42e23}{kg})}{(\SI{3.40e6}{m})^2}&=\SI{3.7}{m/s^2}\\
\end{align*}
\part
First, we need to know the mass of the rover, which we get from its weight on Earth.
\begin{align*}
m &= \frac{W}{g}=\frac{\SI{3430}{N}}{\SI{9.80}{m/s^2}}=\SI{350}{kg}\\
\end{align*}
which we then use with the value for the acceleration at Mars' surface found above:
\begin{align*}
F_g &=mg_M=(\SI{350}{kg})(\SI{3.70}{m/s^2})=\SI{1296.50}{N}
\end{align*}
\part The answers make sense since although the mass of Mars is an order of magnitude smaller than that of Earth, the gravitational acceleration is less significantly affected because of the inverse square law, as Mars' radius being around half the size of Earth's. 
\end{parts}
\end{solution}


\question You wish to put a \SI{1000}{kg} satellite into a circular orbit \SI{300}{km} above the Earth's surface.
\begin{parts}
\part What speed, period, and radial acceleration will it have?
\part What is the minimum amount of work that needs to be done to the satellite to put it in orbit if launched from the Earth's equator? What if it is launched from the North pole?
\part How much additional work would have to be done to make the satellite escape the Earth's gravity?
\end{parts}
\textit{Assume that the Earth has a mass of $M_\oplus=\SI{5.97e24}{kg}$ and a radius of $R_\oplus=\SI{6.38e6}{m}$}.
\begin{finalanswer}
\begin{enumerate}[(a)]
\item Speed: $\SI{7.72e3}{m/s}$, Period: \SI{90}{min}, Acceleration: $\SI{8.92}{m/s^2}$
\item Equator: $\SI{3.25e10}{J}$, North Pole :$\SI{3.26e10}{J}$
\item $\SI{2.98e10}{J}$
\end{enumerate}
\end{finalanswer}
\begin{solution}
\begin{parts}
\part The radius of the satellite's orbit will be $d=\SI{6380}{km}+\SI{300}{km}=\SI{6680}{km}$. The speed for an object in a circular orbit is given by requiring that the gravitational force be equal to the mass of the satellite times its centripetal acceleration:
\begin{align*}
F_g &=\frac{GM_\oplus m}{d^2}=m\frac{v^2}{d}\\
\therefore v&= \sqrt{\frac{GM_\oplus}{d}}= \sqrt{\frac{(\SI{6.67e-11}{N m^2/kg^2})(\SI{5.97e24}{kg})}{(\SI{6.68e6}{m})}}\\
&=\SI{7.72e3}{m/s}\\
\end{align*}
If the satellite completes one orbit of length $2\pi d$ in a period of time $T$, with the above speed, we have:
\begin{align*}
T=\frac{2\pi d}{v}= \frac{2\pi(\SI{6.68e6}{m})}{(\SI{7.72e3}{m/s})}=\SI{5440}{s}\sim\SI{90}{min}
\end{align*}
The centripetal acceleration will be:
\begin{align*}
a_c=\frac{F_g}{m}=\frac{GM_\oplus}{d^2}=\frac{(\SI{6.67e-11}{N m^2/kg^2})(\SI{5.97e24}{kg})}{(\SI{6.68e6}{m})^2} =\SI{8.92}{m/s^2}
\end{align*}

\part The work required is the difference between $E_2$, the total mechanical energy when the satellite is in orbit, and $E_1$, the total mechanical energy when the satellite was at rest on the launch pad. If the satellite is at the equator, it already has kinetic energy due to the Earth's rotation. If it is at the North pole, it has no initial kinetic energy.

The energy when in orbit is:
\begin{align*}
E_2 &= K_2+U_2= \frac{1}{2}mv^2 -\frac{GM_\oplus m}{d}\\
&=\frac{1}{2}m \frac{GM_\oplus }{d} -\frac{G M_\oplus m}{d}\\
&=-\frac{GM_\oplus m}{2d}\\
&= -\frac{(\SI{6.67e-11}{N m^2/kg^2})(\SI{5.97e24}{kg})(\SI{1000}{kg})}{(\SI{6.68e6}{m})}\\
&= \SI{-2.98e10}{J}\\
\end{align*}
At the Earth's equator, the satellite has kinetic energy given by the Earth's rotation speed:
\begin{align*}
v &= \frac{2\pi R_\oplus}{\SI{24}{hr}}=\frac{2\pi (\SI{6.38e6}{m})}{(\SI{86400}{s})}=\SI{463.97}{m/s}
\end{align*}
The satellite's initial energy is thus:
\begin{align*}
E_1&= K_1+U_1=\frac{1}{2}mv^2+\left(-\frac{GM_\oplus m}{R_\oplus}\right)\\
&=\frac{1}{2}(\SI{1000}{kg})(\SI{463.97}{m/s})^2-\frac{(\SI{6.67e-11}{N m^2/kg^2})(\SI{5.97e24}{kg})(\SI{1000}{kg})}{(\SI{6.38e6}{m})}\\
&=\SI{-6.23e-10}{J}
\end{align*}
From the equator, the required work to get into orbit is:
\begin{align*}
W &= E_2 - E_1\\
&=\SI{-2.98e10}{J} - \SI{-6.23e-10}{J}=\SI{3.25e10}{J}
\end{align*}

At the north pole, the satellite kinetic energy is zero on the launch pad $(r=R_E)$, so
\begin{align*}
E_1&= K_1+U_1=0+\left(-\frac{GM_\oplus m}{R_\oplus}\right)\\
&= -\frac{(\SI{6.67e-11}{N m^2/kg^2})(\SI{5.97e24}{kg})(\SI{1000}{kg})}{(\SI{6.38e6}{m})}\\
&= \SI{-6.24e10}{J}\\
\end{align*}
The work done on the satellite to place it in orbit is thus given by:\\
\begin{align*}
W &= E_2-E_1=(\SI{-2.98e10}{J})-(\SI{-6.24e10}{J})\\
&= \SI{3.26e10}{J}
\end{align*}
It thus takes approximately $\SI{1e8}{J}$ less energy to launch a satellite from the equator (which corresponds to the kinetic energy at the equator). 

c)If we define potential energy to be zero at infinity, the satellite needs to have zero total mechanical energy in order to escape (at infinity, it will have zero kinetic and zero potential energy). We found that the mechanical energy was \SI{-2.98e10}{J} when the satellite is in orbit. It thus requires another \SI{2.98e10}{J} of work in order to escape.
\end{parts}
\end{solution}

\question Comet Halley moves in an elliptical orbit around the Sun ($M_\odot=\SI{1.99e30}{kg}$). It's distances from the sun at perihelion and aphelion are $d_p=\SI{8.75e7}{km}$ and $d_a=\SI{5.26e9}{km}$, respectively.
\begin{parts}
\part Find the orbital semi-major axis.
\part Find the eccentricity of the orbit (defined as the distance between the foci of the ellipse divided by the length of the major axis).
\part Find the period of the orbit.
\end{parts}
\begin{finalanswer}
\begin{enumerate}[(a)]
\item $\SI{2.67e9}{km}$
\item 0.969
\item \SI{75.5}{years}
\end{enumerate}
\end{finalanswer}
\begin{solution}
\begin{parts}
\part The length of the semi major-axis, $a$, is the average of the aphelion and perihelion distances so
\begin{align*}
a=\frac{d_p+d_a}{2}=\frac{(\SI{8.75e7}{km})+(\SI{5.26e9}{km})}{2}=\SI{2.67e9}{km}\\
\end{align*}
\part Half of the distance between the foci, $c$, is given by the length of the semi-major axis minus the perihelion distance:
\begin{align*}
c=a-d_p=\frac{d_p+d_a}{2}-d_p=\frac{d_a-d_p}{2}=\frac{(\SI{5.26e9}{km})-(\SI{8.75e7}{km})}{2}=\SI{2.58e9}{km}
\end{align*}
The eccentricity is then:
\begin{align*}
e=\frac{2c}{2a}=\frac{c}{a}=0.969
\end{align*}

\part Kepler's Third Law says that the square of the orbital period divided by the cube of the semi-major axis is a constant for all planets:
\begin{align*}
\frac{T^2}{a^3}=k
\end{align*}
We can easily calculate that constant for a circular orbit of radius $R=a$. A circular orbit is given by the condition that gravity provide the force required for uniform circular motion:
\begin{align*}
GM_\odot m/R^2&=m\frac{v^2}{R}\\
\therefore v&=\frac{GM_\odot}{R}
\end{align*}
The period (squared) for an orbit is given by:
\begin{align*}
T^2=\left(\frac{2\pi R}{v}\right)^2=\frac{4\pi^2R^3}{GM_\odot}
\end{align*}
If we divide this by $R^3$:
\begin{align*}
k=\frac{T^2}{R^3}=\frac{4\pi^2}{GM_\odot}
\end{align*}
which is indeed a constant. Using this for our orbit:
\begin{align*}
T&=\sqrt{ka^3}\\
&=\sqrt{\frac{4\pi^2a^3}{GM_\odot}}\\
&=\frac{2\pi a^{\frac{3}{2}}}{\sqrt{GM_\odot}}\\
&=\frac{2\pi(\SI{2.67e12}{m})^{3/2}}{\sqrt{(\SI{6.67e-11}{N m^2/kg^2})(\SI{1.99e30}{kg})}}\\
&= \SI{2.38e9}{s} \\
&= \SI{75.5}{years}
\end{align*}
\end{parts}
\end{solution}


\question A thin ring of mass $M$ and radius $R$ lies in the $yz$ plane, as shown in Figure \ref{fig:gravity:RingGravity}. A mass $m$ is placed at a distance $x$ away from the origin at point $P$, along the $x$ axis. Give an expression for the gravitational force vector exerted by the ring on the small mass $m$. 

\textit{Hint: break the ring up into small mass elements $dM$ and sum the forces, $d\vec F$, from each mass element. Think about the symmetry when summing the contributions from each mass element.}
\capfig{0.3\textwidth}{figures/Gravity/RingGravity.png}{\label{fig:gravity:RingGravity}A ring of mass $M$.}
\begin{finalanswer}
\begin{align*}
\vec F=-\frac{GMmx}{(R^2+x^2)^{3/2}}\hat i
\end{align*}
\end{finalanswer}
\begin{solution}
All of the mass elements are the same distance, $r$, from $m$, namely:
\begin{align*}
r = \sqrt{R^2+x^2}
\end{align*}
The force vector from a particular mass element $dM$ on $m$:
\begin{align*}
d\vec F=-\frac{GmdM}{r^2}\hat r=-\frac{GmdM}{R^2+x^2}\hat r
\end{align*}
where the vector $\hat r$ goes from the mass element $dM$ to $m$. For a given mass element, there will be another mass element on the opposite side of the ring exerting a force whose $y$ and $z$ components will cancel that of the given mass element. Thus, only the $x$ components of the forces from each mass element will contribute to the total force. If $\theta$ is the opening angle of the cone created by the point $P$ and the circle, then the $x$ component of each force element is:
\begin{align*}
dF_x=dF\cos\theta=dF\frac{x}{r}=dF\frac{x}{\sqrt{R^2+x^2}}
\end{align*}
The total force is thus given by:
\begin{align*}
\vec F&=F_x\hat i=\int dF_x \hat i \\
&=\int dF\frac{x}{\sqrt{R^2+x^2}} \hat i\\
&=-\int \frac{GmdM}{R^2+x^2}\frac{x}{\sqrt{R^2+x^2}} \hat i\\
&=\int \frac{Gmx}{(R^2+x^2)^{\frac{3}{2}}}dM \hat i\\
\end{align*}
Now, note that everything in the integrand is constant (it does not depend on \textit{which} mass element we are considering), so everything can come out of the integral:
\begin{align*}
\vec F&=-\int \frac{Gmx}{(R^2+x^2)^{\frac{3}{2}}}dM \hat i\\
&=-\frac{Gmx}{(R^2+x^2)^{\frac{3}{2}}}\int dM \hat i\\
&=-\frac{GMmx}{(R^2+x^2)^{\frac{3}{2}}}\hat i
\end{align*}
where we have recognized that $\int dM=M$, since it is just the sum of the mass elements $dM$, which must add up to the mass of the ring.
\end{solution}

%based on Giancolli 6-59
\question You are interested in exploring planet Camelid, but are worried that the force of gravity on this gigantic planet would crush you. You plan to visit the planet at the equator, as you will have the greatest chance to observe a space llama there, as indicated by a previous un-manned probe that visited the planet. The planet rotates about itself once every \SI{12}{hr}, has a radius of $R=\SI{7e4}{km}$, and a mass of $M=\SI{2e27}{kg}$. What is the value of your apparent weight at the equator on the surface of the planet? Give your answer as a multiple of $mg$, your weight at the Earth's surface ($g=\SI{9.8}{m/s^2}$).

\textit{Remember that the apparent weight is the value of the normal force that you feel. You need to take the centripetal acceleration from being at the equator into consideration. Newton's Universal Constant for Gravity is $G=\SI{6.67e-11}{Nm^2/kg^2}$.}

\begin{finalanswer}
2.62 times that on Earth.
\end{finalanswer}
\begin{solution}
The magnitude of the gravitational force minus the magnitude of the gravitational force will be equal to the centripetal force:
\begin{align*}
F_g-N&=m\frac{v^2}{R}\\
N&=F_g-m\frac{v^2}{R}
\end{align*}
The force of gravity at the surface of the planet is:
\begin{align*}
F_g=G\frac{Mm}{r^2}
\end{align*}
The rotation speed of the planet is:
\begin{align*}
v=\frac{2\pi R}{T}=\frac{2\pi (\SI{7e7}{m})}{(\SI{12}{hr})}=\SI{10181}{m/s}
\end{align*}
We thus have the normal force given by:
\begin{align*}
N&=F_g-m\frac{v^2}{R}\\
&= G\frac{Mm}{R^2}-m\frac{v^2}{R}\\
&= m\left( G\frac{M}{R^2}-\frac{v^2}{R}  \right)
\end{align*}

Since we want this in terms of $mg$, we divide by $mg$ to find the apparent weight as a multiple of $mg$:
\begin{align*}
\frac{N}{mg} &=\frac{1}{g}\left( G\frac{M}{R^2}-\frac{v^2}{R}  \right)\\
&=\frac{1}{g}\left( G\frac{M}{R^2}-\frac{v^2}{R}  \right)\\
&=\frac{1}{(\SI{9.8}{m/s^2})}\left( (\SI{6.67e-11}{Nm^2/kg^2})\frac{(\SI{2e27}{kg})}{(\SI{7e7}{m})^2}-\frac{(\SI{10181}{m/s})^2}{(\SI{7e7}{m})}  \right)\\
&=2.62
\end{align*}
So your apparent weight is about 2.6 times that on Earth. 
\end{solution}

%based on Giancolli 6-64
\question Observations of a galaxy far far away seem to imply that there is a black hole at its centre. By using spectroscopic measurements, astronomers were able to measure the speed of gases orbiting \SI{50}{ly} from the centre of the galaxy (\SI{1}{ly} is the distance that light travels in a year, at a speed of \SI{3e8}{m/s}). Astronomers measured that the speed of those gases was approximately \SI{750}{km/s}.
\begin{parts}
\part Estimate the mass of the black hole at the centre of the galaxy in terms of the number of solar masses (the mass of the Sun is \SI{1.98e30}{kg}). Clearly state any assumption that you make.
\part Find the radius of circular orbits around the black hole whose orbital speed is the speed of light.
\end{parts}

\textit{Newton's Universal Constant for Gravity is $G=\SI{6.67e-11}{Nm^2/kg^2}$.}

\begin{finalanswer}
\begin{enumerate}[(a)]
\item $\num{2.01e9}$ solar masses, assuming the gas follows a circular orbit of radius $R=\SI{50}{ly}$.
\item $\SI{2.95e12}{m}$
\end{enumerate}
\end{finalanswer}
\begin{solution}
\begin{parts}
\part We assume that the gas follows a circular orbit of radius $R=\SI{50}{ly}$ about the centre of the galaxy. We note that \SI{1}{ly} corresponds to \SI{9.46e15}{m}. The orbital speed is given by:
\begin{align*}
G\frac{Mm}{R^2}&=m\frac{v^2}{R}\\
\therefore M &= \frac{v^2R}{G}\\
&=\frac{(\SI{750e3}{m/s})^2(\SI{50}{ly})}{(\SI{6.67e-11}{Nm^2/kg^2})}\\
&=\frac{(\SI{750e3}{m/s})^2(50)(\SI{9.46e15}{m}) }{(\SI{6.67e-11}{Nm^2/kg^2})}\\
&=\SI{3.99e39}{kg}
\end{align*}
which corresponds to $\frac{\SI{2.36e31}{kg}}{\SI{1.98e30}{kg}}=\num{2.01e9}$ solar masses.
\part For a circular orbit, the centripetal acceleration is given by the acceleration from gravity.:
\begin{align*}
G\frac{Mm}{R^2}&=m\frac{v^2}{R}\\
\therefore R&= \frac{GM}{v^2}=\frac{GM}{c^2}\\
&=\frac{(\SI{6.67e-11}{Nm^2/kg^2})(\SI{3.99e39}{kg})}{(\SI{3e8}{m/s})^2}=\SI{2.95e12}{m}
\end{align*}
where we substituted the speed of light for the orbital velocity. 

\end{parts}
\end{solution}


%From Midyear Makeup Exam F17
\question A satellite of mass $m=\SI{1e3}{kg}$ is in a circular orbit of radius $2R$, where $R=\SI{6.23e6}{m}$ is the radius of the Earth, and $M=\SI{5.97e24}{kg}$ is the mass of the Earth.
\begin{parts}
\part What is the speed of the satellite in this orbit?
\part What is the minimum amount of net work that was done to place the satellite in this orbit if the satellite started from the surface of the Earth?
\end{parts}
\begin{finalanswer}
\begin{enumerate}[(a)]
\item \SI{5653.16}{m/s}
\item $\SI{4.78e10}{J}$
\end{enumerate}
\end{finalanswer}
\begin{solution}
\begin{parts}
\part The speed in a circular orbit is given by the condition that the centripetal force equals the force from gravity. If the satellite:
\begin{align*}
m\frac{v^2}{2R}&=G\frac{mM}{4R^2}\\
\therefore v &=\sqrt{\frac{GM}{2R}}=\sqrt{\frac{(\SI{6.67e-11}{Nm^2/kg^2})(\SI{5.97e24}{kg})}{2(\SI{6.23e6}{m})}}\\
&=\SI{5653.16}{m/s}
\end{align*}
\part We can use the change in mechanical energy between the surface of the Earth and the orbit to find the net work that was done on the satellite. For the work to be a minimum, the satellite would be launched from the equator of the Earth, thus having already a net speed due to the Earth's rotation. The initial velocity is given by:
\begin{align*}
v_0=\frac{2\pi R}{(\SI{24}{h})}=\SI{453.1}{m/s}
\end{align*}
The difference between the mechanical energy in orbit and the mechanical energy at the surface gives the net work done:
\begin{align*}
W&=E-E_0=\frac{1}{2}mv^2-G\frac{Mm}{2R}-\frac{1}{2}mv_0^2+G\frac{Mm}{R}\\
&=\frac{1}{2}m\frac{GM}{2R}-G\frac{Mm}{2R}-\frac{1}{2}mv_0^2+G\frac{Mm}{R}\\
&=\frac{GMm}{4R}-G\frac{Mm}{2R}-\frac{1}{2}mv_0^2+G\frac{Mm}{R}\\
&=\frac{3GMm}{4R}-\frac{1}{2}mv_0^2\\
&=\frac{3(\SI{6.67e-11}{Nm^2/kg^2})(\SI{5.97e24}{kg})(\SI{1e3}{kg})}{4(\SI{6.23e6}{m})}-\frac{1}{2}(\SI{1e3}{kg})(\SI{453.1}{m/s})^2\\
&=\SI{4.78e10}{J}
\end{align*}
\end{parts}
\end{solution}

%Based off Giancolli, solution Olivia W
\question It is possible for a small satellite of mass $m$ to orbit the Sun with the same period $T$ as the Earth. This is only possible if the satellite is located at one of five points that are called ``Lagrange points.'' Assume that the Earth follows a circular orbit of radius $r_E$. The first Lagrange point is located between the Earth and the Sun. As the satellite orbits the Sun, it will always be a distance $d$ from the Earth's centre. Hint: You can use the binomial expansion $(1+x)^n\approx 1+nx$ if $x<<1$.
\begin{parts}
\part Kepler's third law says that for an object orbiting the Sun, the period will change with the distance to the Sun. How is it possible for the satellite to orbit the Sun with the same period as the Earth if they don't have the same orbital radius?
\part Find $d$, the distance from the Earth's centre to the satellite, in terms of the mass of the Earth $M_E$, the mass of the sun $M_S$, and the radius of the Earth's orbit, $r_E$.
\end{parts}
\begin{finalanswer}
\begin{enumerate}[(a)]
\item Kepler's third law refers to objects orbiting the Sun that are only under the influence of the Sun's gravity. The satellite is influenced by the gravity of Earth in addition to the Sun. 
\item $d=\left(\frac{M_E}{3M_S}\right)^{1/3}r_E$
\end{enumerate}
\end{finalanswer}
\begin{solution}
\begin{parts}
\part Kepler's third law refers to objects orbiting the Sun that are only under the influence of the Sun's gravity. The satellite is influenced by the gravity of Earth in addition to the Sun. 
\part The two forces acting on the satellite are the force of gravity due to the Sun, and the force of gravity due to the Earth. The forces will point in opposite directions. The net force acting on the satellite is then:
\begin{align*}
F_{net}=\frac{GM_Sm}{(r_E-d)^2}-\frac{GM_Em}{d^2}
\end{align*}
where we used that fact that the satellite will always be a distance $d$ from the Earth and therefore a distance $r_E-d$ from the sun. The satellite follows a circular orbit of radius $(r_E-d)$, so it is undergoing centripetal motion:
\begin{align*}
\frac{mv^2}{r_E-d}&=\frac{GM_Sm}{(r_E-d)^2}-\frac{GM_Em}{d^2}\\
\frac{v^2}{r_E-d}&=\frac{GM_S}{(r_E-d)^2}-\frac{GM_E}{d^2}
\end{align*}
We do not know the velocity, but we do know that the satellite has the same period as the Earth, so we rewrite our equation in terms of the period $T$:
\begin{align*}
\left(\frac {2\pi (r_E-d)}{T}\right)\frac {1}{r_E-d}&=\frac{GM_S}{(r_E-d)^2}-\frac{GM_E}{d^2}\\
\frac{4\pi^2(r_E-d)}{T^2}&=\frac{GM_S}{(r_E-d)^2}-\frac{GM_E}{d^2}
\end{align*}
Our goal is to solve for $d$. This will be difficult to do with our current equation, so let's see if we can use a binomial expansion to simplify it. We can use a binomial expansion on a term $(1+x)^2$ if $x$ is much less than one. Right now, the terms we want to expand are $(r_E-d)$ and $(r_E-d)^2$. Neither $r_E$ nor $d$ is less than one, but if we factor out $r_E$, we can write our equation as:
\begin{align*}
\frac{4\pi^2r_E}{T^2}\left(1-\frac{d}{r_E}\right)=\frac{GM_S}{r_E^2}\left(1-\frac{d}{r_E}\right)^{-2}-\frac{GM_E}{d^2}
\end{align*}
We can use a binomial expansion,$(1+x)^n\approx 1+nx$,  on the terms including $1-d/r_E$:
\begin{align*}
\frac{4\pi^2r_E}{T^2}\left(1-\frac{d}{r_E}\right)&=\frac{GM_S}{r_E^2}\left(1-\frac{d}{r_E}\right)^{-2}-\frac{GM_E}{d^2}\\
\frac{4\pi^2r_E}{T^2}\left(1-\frac{d}{r_E}\right)&=\frac{GM_S}{r_E^2}\left(1+2\frac{d}{r_E}\right)-\frac{GM_E}{d^2}\\
\end{align*}
We don't know the period of the satellite's motion, but we do know that the period is the same is that of the Earth's. Since the mass of the satellite is so small, the force it exerts on the Earth will be far less than the force due to the sun. Newton's second law for the Earth in its circular orbit can be written as:
\begin{align*}
F=\frac{M_Ev^2}{r_E}&=\frac{GM_EM_S}{r_E^2}\\
\therefore \frac{GM_S}{r_E^2}&=\frac{4\pi^2r_E}{T^2}
\end{align*}
where again we rewrote the velocity in terms of the period. This expression allows us to simplify our equation for the satellite. Substituting in $GM_s/r_E^2$ results in:
\begin{align*}
\frac{GM_S}{r_E^2}\left(1-\frac{d}{r_E}\right)&=\frac{GM_S}{r_E^2}\left(1+2\frac{d}{r_E}\right)-\frac{GM_E}{d^2}
\end{align*}
This simplifies to:
\begin{align*}
\frac{GM_S}{r_E^2}-\frac{GM_S}{r_E^2}\frac{d}{r_E}&=\frac{GM_S}{r_E^2}+\frac{2GM_S}{r_E^2}\frac{d}{r_E}-\frac{GM_E}{d^2}\\
\frac{GM_E}{d^2}&=\frac{3GM_S}{r_E^2}\frac{d}{r_E}\\
\end{align*}
Finally, we can rearrange for $d$ and get:
\begin{align*}
d=\left(\frac{M_E}{3M_S}\right)^{1/3}r_E
\end{align*}
\end{parts}
\end{solution}

%Olivia W
\question What is the field due to a thin uniform rod of mass $M$ at a point $P$ located a distance $d$ along the axis of the rod (Figure \ref{fig:gravity:rodfieldaxis})?

\capfig{0.7\textwidth}{figures/Gravity/rodfieldaxis.png}{\label{fig:gravity:rodfieldaxis}A thin rod of mass $M$ and length $L$ produces a gravitational field at a point $P$ located along the axis of the rod.}
\begin{finalanswer}
\begin{align*}
\vec g(\vec x)=\frac{GM}{L}\left[\frac{1}{d}-\frac{1}{L+d}\right]\hat{x}
\end{align*}
\end{finalanswer}
\begin{solution}
\capfig{0.7\textwidth}{figures/Gravity/rodfieldaxissoln.png}{\label{fig:gravity:rodfieldaxissoln}A thin rod of mass $M$ and length $L$ produces a gravitational field at a point $P$ located along the axis of the rod. Each segment $dx$ makes a contribution to the field}
The gravitational field due to a body of mass $M$ is given by:
\begin{align*}
\vec g(\vec r)=-\frac{GM}{r^2}\hat{r}
\end{align*}

Our strategy will be to break the rod into very small segments of length $dx$. Each segment, of mass $dM$, will make a small contribution, $d\vec g$, to the gravitational field, as shown in Figure \ref{fig:gravity:rodfieldaxissoln}. The gravitational field due to each segment is given by:
\begin{align*}
d\vec g&=-\frac{GdM}{r^2}\hat{r}\\
d\vec g&=\frac{GdM}{x^2}\hat{x}
\end{align*}
where we have noted that, using our coordinate system, the vector $\vec r$ from each segment to point $P$ is just $-\vec x$. The gravitational field will therefore point in the $\hat{x}$ direction, which makes sense since any mass placed at point $P$ will be attracted to the rod. (Note that if we placed $P$ on the other side of the rod, the field would point in the $-\hat{x}$ direction).\\

We can find the magnitude of the field at $P$ by taking the sum of the contributions from each segment, using integration:
\begin{align*}
g(x)=\int dg=\int_{x_1}^{x_2} \frac{G dM}{x^2}
\end{align*}
Since we have an $x$ in our equation for $dg$, we want to be able to integrate over $dx$. We can easily write $dM$ in terms of $dx$ if we define $\lambda$ to be the linear density, so that $\lambda=M/L$. The mass of each segment will then be $dM=\lambda dx$. Our integral then becomes:
\begin{align*}
g(x)=\int_{x_1}^{x_2} \frac{G\lambda}{x^2}dx
\end{align*}
We have set up our axis so that $P$ is located at $x=0$. The end of the rod that is nearest to $P$ is located at $x=d$. The far end of the rod is located at $L+d$. These become our limits of integration, since each segment of the rod must be between $x=d$ and $x=L+d$:
\begin{align*}
g(x)=\int_{d}^{L+d} \frac{G\lambda}{x^2}dx
\end{align*}
Now we can integrate to find the magnitude of the field, $g(x)$:
\begin{align*}
g(x)&=\int_{d}^{L+d} \frac{G\lambda}{x^2}dx\\
&=G\lambda\int_{d}^{L+d} \frac{1}{x^2}dx\\
&=G\lambda\int_{d}^{L+d} x^{-2}dx\\
&=G\lambda[-x^{-1}]_d^{L+d}\\
&=G\lambda\left[-\frac{1}{L+d}+\frac{1}{d}\right]\\
&=G\lambda\left[\frac{1}{d}-\frac{1}{L+d}\right]
\end{align*}
Remembering that we set $\lambda=M/L$, we get:
\begin{align*}
g(x)&=\frac{GM}{L}\left[\frac{1}{d}-\frac{1}{L+d}\right]\\
\therefore \vec g(\vec x)&=\frac{GM}{L}\left[\frac{1}{d}-\frac{1}{L+d}\right]\hat{x}
\end{align*}
\end{solution}

\question An object of mass $m$ orbits a much larger spherical object of mass $M$ in a circular orbit. Show that the kinetic energy, $K$, of $m$ is given by:
\begin{align*}
K=-\frac{1}{2}U
\end{align*}
where $U=U(r)$, is the gravitational potential energy of $m$ due to $M$:
\begin{align*}
U(r)=-G\frac{Mm}{r}
\end{align*}
This is a special case of the ``Virial Theorem''.
\begin{solution}
In a circular orbit of radius $r$, the centripetal force is provided by gravity:
\begin{align*}
m\frac{v^2}{r}&=G\frac{Mm}{r^2}\\
\therefore \frac{1}{2}mv^2 &=G\frac{Mm}{2r}
\end{align*}
The term on the right is indeed half of the negative of the potential energy.
\end{solution}

\question You are designing the landing rockets for a probe of mass $m=\SI{1e3}{kg}$ to land on a spherical planet of mass $M=\SI{1e24}{kg}$ and radius $R=\SI{1e6}{m}$ with no atmosphere (i.e. no drag on the probe as it lands). The rockets will fire and exert a total constant force $F=\SI{1e8}{N}$ in the direction opposite to the velocity of the probe. At what altitude from the surface of the planet do the landing rockets have to fire for the probe to land with a speed of zero relative to the surface of the planet? Assume that the probe originated at an infinite distance from the planet, where it had a speed of zero, and that it is moving directly towards the centre of the planet.
\begin{solution}
Since the probe had a kinetic energy of zero at an infinite distance from the planet, the mechanical energy, $E$, of the probe is identically zero before the rockets fire. The rockets will fire and do (negative) work until the kinetic energy of the probe is zero right at the surface of the planet. The rockets will do an amount of work given by:
\begin{align*}
W=-Fh
\end{align*}
where $h$ is the altitude above the ground at which the rockets fire. At the surface of the planet the energy of the probe is given by:
\begin{align*}
E=K+U=-G\frac{Mm}{R}
\end{align*}
The amount of work done by the rockets must be equal to the change in energy of the probe, where the initial energy is 0:
\begin{align*}
W&=\Delta E=-G\frac{Mm}{R}-0\\
Fh &= G\frac{Mm}{R}\\
\therefore h &= G\frac{Mm}{RF}= (\SI{6.67e-11}{N\cdot m^2/kg^2})\frac{(\SI{1e24}{kg})(\SI{1e3}{kg})}{(\SI{1e6}{m})(\SI{1e8}{N})}\\
&=\SI{667}{m}
\end{align*}
\end{solution}



%Written by Tristan Menard
\question The Falcon Heavy rocket was first launched on February 6, 2018, as the world’s most powerful
operational rocket. It has a mass of $\SI{1.421e6}{kg}$, and additionally, carried a payload (Elon
Musk’s Tesla Roadster) of approximately \SI{1300}{kg} during its maiden flight which brought it to
a low Earth orbit of nearly \SI{2000}{km} in altitude.
\begin{parts}
	\part Knowing that the Earth has a mass of \SI{5.972e24}{kg} and a radius of \SI{6371}{km}, determine
	the work done by gravity as the Falcon Heavy traveled from sea level to its low Earth
	orbit. Recall that the gravitational constant is $G = \SI{6.674e-11}{m^3kg^{-1}s^{-2}}$.
	\part An object in a circular geosynchronous orbit remains above the same position on Earth
	at all times. Based on this information, determine the work done by gravity if the Falcon
	Heavy were to reach geosynchronous orbit starting from sea level while carrying a
	payload of \SI{2.5e4}{kg}.
\end{parts}

\begin{finalanswer}
	\begin{parts}
	\part The total work done by gravity on the Falcon Heavy ship is $\SI{-2.125e13}{J}$.
	\part The work done by gravity while launching Falcon Heavy into geosynchronous orbit is $\SI{-7.864e13}{J}$.
	\end{parts}
\end{finalanswer}

\begin{solution}
	\begin{parts}
		\part We can find the total work done by integrating the force due to gravity, beginning at the distance between Earth's center and Earth's crust, $r_E$, and ending at $r_o$, which is $r_E$ plus the distance from Earth's crust to Falcon Heavy's orbit.
		\begin{align*}
		W_G&= \int_{r_E}^{r_o}-\frac{GM_Em_F}{r}dr\\
		&=-GM_Em_F\int{r_E}{r_o}\frac{1}{r^2}dr\\
		&=-GM_Em_F(-\frac{1}{r})|_r_E^r_o\\
		&-GM_E(\SI{1.421e6}{kg}+\SI{1300}{kg})(-\frac{1}{\SI{2.0e6}{m}+\SI{6.371e6}{m}}+\frac{1}{\SI{6.371e6}{m}})\\
		&=\SI{-2.125e13}{J}
		\end{align*}
		
		Therefore, the total work done on the Falcon Heavy ship by gravity is $\SI{-2.125e13}{J}$
		
		\part First, we must find the radius from the Earth at which a circular geosynchronous orbit is possible.
		This will only be physically possible when the gravitational force cancels the centripetal force.
		Let the radius of the geosynchronous orbit be $r_g$
		\begin{align*}
		ma_r&=F_G\\
		m_F\frac{v^2}{r_g}&=\frac{G M_E m_F}{r_g^2}\\
		v^2&=\frac{GM_E}{r_g}\\
		\end{align*}
		
		The tangential velocity v can be calculated using the fact that an object in geosynchronous
		orbit follows the same orbital period as the Earth’s rotation. Therefore, the Falcon
		Heavy must complete one full revolution or period every 24 hours, which is equal to \SI{86,400}{s}.
		
		\begin{align*}
		v &= \frac{2\pi r_g}{T}=\frac{\pi}{\SI{43200}{s}}r_g\\
		\end{align*}
		We can now substitute this $v$ into our previous equation in order to obtain $r_g$
		
		\begin{align*}
		\frac{\pi^2}{(\SI{43200}{s})^2}r_g^2&=\frac{GM_E}{r_g}\\
		r_g^3= \frac{(\SI{43200}{s})^2GM_E}{\pi^2}\\
		r_g= \SI{4.224e7}{m}
		\end{align*}
		
		We can now apply the same method to calculate work as used in the previous part. The method will be the same, but we will replace $r_o$ with $r_g$
		\begin{align*}
		W_G&= \int_{r_E}^{r_g}-\frac{GM_Em_F}{r}dr\\
		&=-GM_Em_F\int{r_E}{r_g}\frac{1}{r^2}dr\\
		&=-GM_Em_F(-\frac{1}{r})|_r_E^r_g\\
		&-GM_E(\SI{1.421e6}{kg}+\SI{1300}{kg})(-\frac{1}{\SI{4.224e7}{m}+\SI{6.371e6}{m}}+\frac{1}{\SI{6.371e6}{m}})\\
		&=\SI{-7.864e13}{J}
		\end{align*}
		
		Which gives our final answer of $\SI{-7.864e13}{J}$
	
	\end{parts}
\end{solution}



%Zaremba -needs solution??
\question A satellite is in a circular Earth orbit of radius $2R$, where $R=\SI{6.38e6}{m}$ is the radius of the Earth.
\begin{parts}
\part What is the speed of the satellite in this orbit?
\part A rocker thruster on the satellite is fired for a brief burst to slow the satellite down. After the burst, the satellite is moving in the same direction but at a reduced speed. What change in speed results in an orbit that just grazes the surface of the Earth? Ignore the effect of the Earth's atmosphere.
\part What is the elapsed time between the burst and the instant when the satellite is closest to Earth?
\end{parts}


