\section{Waves}

%%%%%%%%%%%%%%%%%%%%%%%%%%%%%%%%%%%
%%
%% Multiple Choice
%%
%%%%%%%%%%%%%%%%%%%%%%%%%%%%%%%%%%%
\subsection{Multiple Choice}

\question If a wave has a velocity of \SI{7}{m/s} and a period of \SI{3}{s}, what is its wavenumber?
\begin{checkboxes}
\choice \SI{0.299}{m}
\choice \SI{2.693}{m^{-1}}
\choice \SI{2.693}{m}
\CorrectChoice \SI{0.299}{m^{-1}} \correct
\end{checkboxes}

\question A guitar string with uniform mass density has a length of $\SI{1}{m}$ and a mass of \SI{3.0}{g}. The string is installed such that there is a distance of $\SI{0.7}{m}$ over which the string is taught (the remaining length of the string is used to attach it). Under what tension should the guitar string be so that its fundamental frequency corresponds to the A note for a guitar (\SI{110}{Hz})? 
\begin{checkboxes}
	\choice \SI{57}{N}
	\CorrectChoice \SI{71}{N}
	\choice \SI{145}{N}
	\choice \SI{924}{N}
\end{checkboxes}

%Ian Mclean
\question  If a wave has frequency \SI{10}{Hz} and a wavelength of \SI{5}{m}, with what speed is the wave travelling?
\begin{checkboxes}
	\choice \SI{0.5}{m/s}
	\choice \SI{1.0}{m/s}
	\choice \SI{10.0}{m/s}
	\CorrectChoice \SI{50.0}{m/s} 
\end{checkboxes}

\question A rope acting as a leash for a Guanaco is chewed on by a fellow Guanaco trying to aid his friend escape to freedom.  While trying to save his friend, the Guanaco moves the string up and then down (the string is fixed to wall), creating a series of pulses.  The reflected pulses (after hitting the wall), compared to the original are:
\begin{checkboxes}
	\CorrectChoice inverted.
	\choice in the same orientation.
\end{checkboxes}

%Nathan Wilson
\question[1] A wave passes from a section of heavy rope to a section of lighter rope; what will happen to the speed of the wave?
\begin{checkboxes}
	\choice It will not transfer to the lighter rope.
	\choice It will slow down.
	\CorrectChoice It will speed up.
	\choice There will be twice as many waves in the lighter rope.
\end{checkboxes}

%Question submitted by Noah Rowe
\question Which of the following parameters of a wave do not change when a wave transfers mediums?
\begin{checkboxes}
\choice Velocity
\CorrectChoice Frequency \correct
\CorrectChoice Period \correct
\choice Wavelength
\end{checkboxes}

%Question from http://www.stmary.ws/HighSchool/Physics/home/marys_java/wave/waves_mc1.htm
\question Wave motion in a medium transfers:
\begin{checkboxes}
\CorrectChoice Energy \correct
\choice Mass
\choice Both energy and mass
\choice Neither mass or energy
\end{checkboxes}


\question Richard plays his guitar.  The length of the strings is \SI{64.8}{cm}, and the mass of one of his strings is \SI{1.2}{g}.  He plays a note on this string, which can be heard as a G note, at \SI{196}{Hz}.  To do this he places his finger on one of the strings, shortening its length to \SI{50}{cm}.  Assume the note is the fundamental frequency. What is the tension in the string?
\begin{checkboxes}
\choice \SI{1.32}{N}
\CorrectChoice \SI{71.1}{N} \correct
\choice  \SI{91.2}{N}
\choice \SI{541}{N}
\end{checkboxes}
%Solution v=wavelength*frequency= 0.5*2 *196=196m/s
%T=v^2*linear density =196 *.0012/.648= .363 N


\question Two otherwise identical standing waves are out of phase by \SI{180}{degrees}. What will happen if you allow these waves to interfere?
\begin{checkboxes}
\choice Entirely constructive interference
\CorrectChoice Entirely destructive interference \correct
\choice A mix of constructive and destructive interence
\choice No interference, both waves will continue on their separate ways
\end{checkboxes}

\question How does the energy in one wavelength of a standing wave, $E_s$, compare to the energy in one wavelength of a travelling wave, $E_T$, if both waves have the same maximum amplitude?
\begin{checkboxes}
\choice $E_s=1/4 E_T$
\CorrectChoice $E_s=1/2 E_T$
\choice $E_s=E_T$
\choice $E_s=2E_T$
\choice $E_s=4E_T$
\end{checkboxes}


%Derived from Question by Zoe Macmillan
\question Which of the following waves have the same velocity? Note that this refers to the velocity of the wave itself and not the medium in which it travels. 
\begin{align*}
&1)\ y = 2\sin{(3x-6t+5)} \\
&2)\ y = 4\sin{(3x-4t-5)} \\
&3)\ y = 2\cos{(2x-6t+5)} \\
&4)\ y = 4\cos{(2x-4t-5)}
\end{align*}
\begin{checkboxes}
\choice $1$ \& $3$
\choice $3$ \& $4$
\choice $1$ \& $2$ 
\CorrectChoice $1$ \& $4$ \correct
\end{checkboxes}

\question Two guitar strings made of the same material have the same (volume) mass density, the same tension and the same length (between points that are held fixed). String 1 has twice the radius of string 2. What is the ratio, $\frac{f_1}{f_2}$, of the fundamental frequency of string 1 divided by the fundamental frequency of string 2?
\begin{choices} 
\choice 0.25
\CorrectChoice 0.5 \correct
\choice 2.0
\choice 4.0
\end{choices}


%Original - difficult
% Make this into a long answer question with different tensions in each string (72N in high E and 78N in low E)
\question Two guitar strings are made of the same material. In order for the strings to make different sounds, their diameters are different, but their volume density is the same. In addition, the tension in both strings is the same, and the distance between points where the strings are held fixed on the guitar is the same as well. The low E string has a fundamental frequency of \SI{82}{Hz} and the high E string a fundamental frequency of \SI{330}{Hz}. If the diameter of the low E string is \SI{1.2}{mm} what is the diameter of the high E string?
\begin{choices} 
\choice \SI{0.15}{mm}
\CorrectChoice \SI{0.3}{mm} \correct
\choice \SI{0.6}{mm}
\choice \SI{4.8}{mm}
\end{choices}
\begin{solution}
The fundamental wavelength is given by the requirement that $\lambda=2L$, where $L$ is the distance between the points where the string is held fixed. We thus have:
\begin{align*}
\lambda f&=v\\
2L f&=\sqrt{\frac{T}{\mu}}\\
\therefore 4L^2&=\frac{T}{f^2\mu}
\end{align*}
where $f$ is the fundamental frequency, $v$ is the speed of the wave, $T$ is the tension in the string, and $\mu$ is the linear mass density of the string. The quantity on the left is independent of the string (it's just related to the distance between where the string is held fixed), so we can relate the quantities on the right for each string:
\begin{align*}
\frac{T_1}{f^2_1\mu_1} &=\frac{T_2}{f_2^2\mu_2}\\
\therefore \frac{\mu_2}{\mu_1}&=\frac{f_1^2}{f_2^2}
\end{align*}
where we took into account that $T_1=T_2$. The linear mass density, $\mu$, is related to the volume mass density, $\rho$, by $\mu=\pi r^2 \rho$. Since both strings are made of the same material, the volume density cancels in the ratio of the mass densities, and we have:
\begin{align*}
\frac{r_2^2}{r_1^2}&=\frac{f_1^2}{f_2^2}\\
\therefore r_2&=r_1\frac{f_1}{f_2}
\end{align*}
Finally, in terms of the diameters, we have:
\begin{align*}
d_2&=d_1\frac{f_1}{f_2}=(\SI{1.2}{mm})\frac{(\SI{82}{Hz})}{(\SI{330}{Hz})}=\SI{0.298}{mm}
\end{align*}
where string 2 is the high E string.
\end{solution}

%Shona Birkett
\question The winter holidays came to an end and the Grinch found himself so happy that he actually made a new year's resolution. His resolution was to tone his arms a little more. Instead of going out and buying a new workout toy, the Grinch attached a rope to a fixed point on a wall on one end and held the other end in his hand. Playing his favourite 80's workout tunes, he moved his arm up and down creating a wave on the rope. The mass of the rope was \SI{2.00}{kg} and he pulled on the rope with a tension of \SI{550}{N}. If the waves travelled with a speed of \SI{20}{m/s}, what was the length of the rope?
\begin{choices} 
\CorrectChoice \SI{1.45}{m} \correct
\choice \SI{2.25}{m}
\choice \SI{2.75}{m} 
\choice \SI{55}{m}
\end{choices}

\question A string on a cello is twice as long as a string on a violin (between points where the string is held fixed). If the strings have the same mass per unit length and are under the same tension, what can you say about their fundamental frequency?
\begin{choices} 
\choice The fundamental frequency of the violin string is a quarter of that of the cello
\choice The fundamental frequency of the violin string is half of that of the cello
\CorrectChoice The fundamental frequency of the violin string is double that of the cello \correct
\choice The fundamental frequency of the violin string is quadruple that of the cello
\end{choices}

\question If we double the tension in a stretched rope, waves that propagate along the rope will travel:
\begin{checkboxes}
\choice 2 times faster
\choice 4 times faster
\CorrectChoice $\sqrt{2}$ times faster \correct
\choice 2 times slower
\end{checkboxes}

\question Two identical loudspeakers face each other and emit a constant tone with a frequency of 500\,Hz (the sound waves travel at a speed of 343\,m/s). The speakers are turned on at exactly the same time (such that they would be in phase if placed next to each other and facing the same direction).  You wish to set the distance between the speakers such that one cannot hear the tone from either speaker anywhere along the line joining them. How far away should you place the speakers?
\begin{checkboxes}
\CorrectChoice Any multiple of 0.686\,m \correct
\choice 0.686\,m
\choice 0.343\,m
\choice Any multiple of 0.343\,m
\end{checkboxes}
\newpage 

%%%%%%%%%%%%%%%%%%%%%%%%%%%%%%%%%%%
%
% long answer
%
%%%%%%%%%%%%%%%%%%%%%%%%%%%%%%%%%%%
\subsection{Long answers}
%From 
%Giancolli? No good, since students didn't practice energy
%\question Traveling waves are created on a rope that has a mass per length \SI{0.200}{kg/m} that is stretched with a tension of \SI{400}{N}.  If the end of the rope is moved up and down with simple harmonic motion with an amplitude of \SI{30.0}{cm} and a period \SI{1.40}{s}, how much work is done in one period of the motion?  

%Are we not putting this one in then? Could potentially add it somewhere else


%\begin{solution}

%\end{solution}

%Giancolli TestBank chapter 15 quantitative Q1 -fixed (is it ok to use this figure?)
\question Figure \ref{fig:waves} shows two graphs, both describing the same wave. One graph shows the displacement of the wave at some position as a function of time and the other graph shows the displacement of the wave at some time as a function of position. What is the speed of the wave?
\capfig{0.5\textwidth}{figures/Waves/waves.png}{\label{fig:waves}Displacement as a function of position and time for a wave.} 
\begin{finalanswer}
\SI{0.75}{m/s}
\end{finalanswer}
\begin{solution}
The speed of the wave is given by:
\begin{align*}
v = \lambda f
\end{align*}
and we can determine the wavelength ($\lambda$) and the frequency $f$, from the graphs. The wavelength is $\lambda=\SI{3}{m}$ and the period is $T=\SI{4}{s}$. The speed of the wave is thus:
\begin{align*}
v &= \lambda f=\lambda \frac{1}{T}=(\SI{3}{m})\frac{1}{(\SI{4}{s})}=\SI{0.75}{m/s}
\end{align*}
\end{solution}

%Giancolli TestBank chapter 15 quantitative Q12 -fixed
\question When an earthquake strikes an area, the type of wave that typically does the most damage is called a secondary wave. Secondary waves are named as such because a quicker, less damaging wave called a primary wave will move through the area as a precursor to the secondary wave. Secondary waves travel transversely by shearing the Earth, while primary waves travel longitudinally as a pressure wave. Primary waves are often undetectable by human senses, but many animals can sense a primary wave. Suppose an earthquake strikes and area. The density of the material the primary and secondary waves move through is \SI{3.2e3}{kg/m^3}, the Young's bulk modulus is \SI{8.0e10}{N/m^2}, and the shear modulus is \SI{2.0e10}{N/m^2}. A woman is walking her dog, and notices that her dog begins barking 20 seconds before she felt the earthquake. Determine the approximate distance between the dogwalker and the earthquake.

\begin{finalanswer}
\SI{100}{km}
\end{finalanswer}
\begin{solution}
The dog would have felt the primary wave before the secondary wave. Since it is a pressure wave, its propagation speed depends on the bulk (Young's) modulus:
\begin{align*}
v_P=\sqrt{\frac{B}{\rho}}=\sqrt{\frac{(\SI{8.0e10}{N/m^2})}{(\SI{3.2e3}{kg/m^3})}}=\SI{5000}{m/s}
\end{align*} 
The person would feel the secondary waves, since they are the strongest. Their propagation speeds depends on the shear modulus:
\begin{align*}
v_S=\sqrt{\frac{S}{\rho}}=\sqrt{\frac{(\SI{2.0e10}{N/m^2})}{(\SI{3.2e3}{kg/m^3})}}=\SI{2500}{m/s}
\end{align*}
Both waves have travelled the same distance, $d$, and started at the same time. If the pressure wave took a time $t_P$ to reach the person and their dog, we can write $d$ as:
\begin{align*}
d &= v_Pt_P=v_S(t_P+\Delta t)
\end{align*}
where $\Delta t=\SI{20}{s}$. Solving for $t_P$ and finding $d$:
\begin{align*}
v_Pt_p-v_St_p&=v_S\Delta t\\
\therefore t_P &= v_S\frac{\Delta t}{v_P-v_S}\\
\therefore d &=  v_Pt_P = v_Pv_S\frac{\Delta t}{v_P-v_S}\\
&=(\SI{5000}{m/s})(\SI{2500}{m/s})\frac{(\SI{20}{s})}{(\SI{5000}{m/s})-(\SI{2500}{m/s})}\\
&=\SI{1e5}{m}=\SI{100}{km}
\end{align*}
\end{solution}

%Giancolli 15-18 -fixed
\question A seismologist measures the intensity of a spherical Earthquake to be \SI{1.6e5}{J/m^2/s}. The measurement of intensity is taken \SI{95}{km} from the hypocentre of the Earthquake.
\begin{parts}
\part What was the intensity of the earthquake when it passed a point \SI{2.0}{km} from the hypocentre?
\part At what rate did energy pass through an area of \SI{5.0}{m^2} if the area was \SI{2.0}{km} from the hypocentre?
\end{parts}
\begin{finalanswer}
\begin{enumerate}[(a)]
\item $\SI{3.6e8}{J/m^2/s}$
\item $\SI{1.81e9}{W}$
\end{enumerate}
\end{finalanswer}
\begin{solution}
\begin{parts}
\part For a spherical wave, the intensity decreases as one over distance squared:
\begin{align*}
I\propto \frac{1}{d^2}
\end{align*}
The intensity at distance $d_1$ compared to the intensity at distance $d_2$ is given by:
\begin{align*}
\frac{I_1}{I_2}&=\frac{d_2^2}{d_1^2}\\
\therefore I_1 &= \frac{d_2^2}{d_1^2}I_2=\frac{(\SI{95}{km})^2}{(\SI{2}{km})^2}(\SI{1.6e5}{J/m^2/s})\\
&=\SI{3.61e8}{J/m^2/s}
\end{align*}
\part The power is given by the intensity times the area:
\begin{align*}
P=IA=(\SI{3.61e8}{J/m^2/s})(\SI{5.0}{m^2})=\SI{1.81e9}{W}
\end{align*}
\end{parts}
\end{solution}

%Giancolli TestBank chapter 15 quantitative Q25. Is there a reason that this question is commented out? I have not altered it yet
%\question Two waves are travelling through a medium simultaneously.  The first wave is described by $y_1(x,t) = A \cos(kx-\omega t)$ and the second wave is described by $y_2(x,t) = 2A \sin(kx-\omega t)$.  If the result of the superposition is written in the form $y(x,t) = B cos(kx-\omega t + \phi)$, determine $B$ in terms of $A$.  
%\begin{solution}
%To find the equation for the superposition, we add the two waves that are given:
%\begin{align*}
%y(x,t) &= y_1(x,t)+y_2(x,t)\\
%&= A \cos(kx-\omega t) + 2A \sin(kx-\omega t) \\
%&= A \sin(kx-\omega t+\frac{\pi}{2}) + 2A \sin(kx-\omega t) 
%\end{align*}
%where the two waves have the same wavelength and frequency, but different amplitudes and are out of phase by $\phi=\frac{\pi}{2}$.

%\end{solution}


%Giancolli 15-50 -fixed
\question A guitarist plucks the string of their guitar and produces a note. The string \SI{92.5}{cm} long, has a mass of \SI{5.42}{g}, the distance from the bridge to the support post is \SI{64.8}{cm}, and the string is under a tension of \SI{65.9}{N}. What are the frequencies of the fundamental and first two overtones of this string? Which guitar string is this?
\begin{finalanswer}
\SI{80.3}{Hz}, \SI{160.7}{Hz}, \SI{241.0}{Hz}. The fundamental overtone is closest to an open low E.
\end{finalanswer}
\begin{solution}
The wavelength of the first fundamental is given by the condition that the ends of the string are both fixed, and are thus half a wavelength apart. The distance between which the points are fixed is the distance between the bridge and support. The resonant wavelength is thus:
\begin{align*}
\lambda_0 = 2L=2(\SI{65.9}{cm})=\SI{1.32}{m}
\end{align*}
To relate this to frequency, we need the speed of the wave, which we can find from the tension and linear density. The linear density is given by the mass of the string divided by the total length:
\begin{align*}
\mu=\frac{m}{L}=\frac{(\SI{5.42}{g})}{(\SI{92.5}{cm})}=\SI{5.86e-3}{kg/m}
\end{align*}
\begin{align*}
v = \sqrt{\frac{F_T}{\mu}}=\sqrt{\frac{(\SI{65.9}{N})}{(\SI{5.86e-3}{kg/m})}}=\SI{106}{m/s}
\end{align*}
The resonant frequency is thus given by:
\begin{align*}
v &=\lambda_0 f \\
\therefore f&=\frac{v}{\lambda_0}\\
&=\frac{(\SI{106}{m/s})}{(\SI{1.32}{m})}=\SI{80.34}{Hz}
\end{align*}
The next two overtones will have double and triple this frequency, respectively, namely \SI{160.7}{Hz} and \SI{241.0}{Hz}
\end{solution}


%Giancolli 15-38 -fixed
\question A cord is driven sinusoidally. The cord is split into two sections, one with a lighter linear mass density, $\mu_L$, and one with a heavier linear mass density, $\mu_H$ as in Figure \ref{fig:HeavyLight}.
\capfig{0.3\textwidth}{figures/Waves/HeavyLight.png}{\label{fig:HeavyLight}Displacement as a function of position and time for a wave.} 
\begin{parts}
\part Determine the ratio of the speeds of the wave in the two sections, $\frac{v_H}{v_L}$
\part Determine the ratio of the wavelengths of the wave in the two section$\frac{\lambda_H}{\lambda_L}$
\part Which section has a longer wavelength?
\end{parts}
\begin{finalanswer}
\begin{enumerate}[(a)]
\item $\sqrt{\frac{\mu_L}{\mu_H}}$
\item $\sqrt{\frac{\mu_L}{\mu_H}}$
\item The wavelength is longer in the lighter side.
\end{enumerate}
\end{finalanswer}
\begin{solution}
\begin{parts}
\part The tension must be the same in both sides of the cord. The ratio of the speeds is thus:
\begin{align*}
\frac{v_H}{v_L}&=\frac{\sqrt{\frac{F_T}{\mu_H}}}{\sqrt{\frac{F_T}{\mu_L}}}=\sqrt{\frac{\mu_L}{\mu_H}}
\end{align*}
\part The frequency in both sides of the cord needs to be the same, or the joint would not be able to move consistently with both sides of the cord. The ratio of the wavelengths is thus:
\begin{align*}
\frac{\lambda_H}{\lambda_L} &= \frac{\frac{v_H}{f}}{\frac{v_L}{f}}=\frac{v_H}{v_L}=\sqrt{\frac{\mu_L}{\mu_H}}
\end{align*}
\part The wavelength is longer in the lighter side.
\end{parts}
\end{solution}

%Giancolli 15-8
\question A sailor wishes to test the depth of the water below his ship. The sailor strikes the side of his ship just below the surface of the water and measures how long it takes to hear the echo of his ship. The echo of the wave reflected from the ocean floor is heard \SI{3.4}{s} later. How deep is the ocean at this point?
\begin{finalanswer}
\SI{2374.7}{m}
\end{finalanswer}
\begin{solution}
We need the bulk modulus and density of sea water to know the speed of the wave. A quick internet search gives $B=\SI{2e9}{N/m^2}$ and $\rho=\SI{1.035e3}{kg/m^3}$. The speed of the wave is thus:
\begin{align*}
v=\sqrt{\frac{B}{\rho}}=\sqrt{\frac{(\SI{2e9}{N/m^2})}{(\SI{1.035e3}{kg/m^3})}}=\SI{1396.87}{m/s}
\end{align*}
The wave must travel twice the distance to the ocean floor, so the distance is:
\begin{align*}
d = \frac{1}{2}vt=\frac{1}{2}(\SI{1396.87}{m/s})(\SI{3.4}{s})=\SI{2374.7}{m}
\end{align*}
\end{solution}

%Olivia W
\question Show that if $D_1$ and $D_2$ satisfy the one dimensional wave equation, so does $a_1D_1+a_2D_2$, where $a_1$ and $a_2$ are constants and $D_1$ and $D_2$ are functions that depend on $x$ and $t$.
\begin{solution}
The 1d wave equation is the condition:
\begin{align*}
\die{^2D}{x^2}=\frac{1}{v^2}\die{^2D}{t^2}
\end{align*}
For the sum $D=a_1D_1+a_2D_2$, this becomes:
\begin{align*}
\die{^2}{x^2}(a_1D_1+a_2D_2)=\frac{1}{v^2}\die{^2}{t^2}(a_1D_1+a_2D_2)
\end{align*}
Since a property of derivatives is that they're distributive, we can write:
\begin{align*}
a_1\die{^2}{x^2}D_1+a_2\die{^2}{x^2}D_2&=\frac{1}{v^2}\left( a_1\die{^2}{t^2}D_1+a_2\die{^2}{t^2}D_2\right)\\
&=a_1\frac{1}{v^2}\die{^2}{t^2}D_1+a_2\frac{1}{v^2}\die{^2}{t^2}D_2
\end{align*}
where we pulled the constants $a_1$ and $a_2$ out in front of the derivatives. If we know that $D_1$ and $D_2$ satisfy the wave equation, then we know:
\begin{align*}
\die{^2}{x^2}D_1&=\frac{1}{v^2}\die{^2}{t^2}D_1\\
\textrm{and} \quad \die{^2}{x^2}D_2&=\frac{1}{v^2}\die{^2}{t^2}D_2
\end{align*}
We can therefore write:
\begin{align*}
a_1\left( \frac{1}{v^2}\die{^2}{t^2}D_1\right)+a_2\left( \frac{1}{v^2}\die{^2}{t^2}D_1\right)
&=a_1\frac{1}{v^2}\die{^2}{t^2}D_1+a_2\frac{1}{v^2}\die{^2}{t^2}D_2
\end{align*}
The left side is equal to the right side, so we have proven that the linear combination $a_1D_1(x,t)+a_2D_2(x,t)$ satisfies the one dimensional wave equation.
\end{solution}

%Olivia W
\question
When the source of a wave and an observer move either towards or away from each other, the observed frequency of the wave will be different from the frequency at which the wave is emitted. For example, if you are standing by the side of the road and a police car with a siren passes by, the frequency of the sound you hear will be higher when the car is moving towards you and lower when it is moving away from you. This is called the Doppler Effect. 

Consider a stationary source that emits waves at some constant frequency $f_s$. The waves move at a constant speed $v$ through the medium. If that source starts to move at a constant speed $v_s$ (relative to the medium) towards an observer, it will``catch up'' with the wave crests that it emits, so that the wave crests will be ``closer together'' (the wavelength will be shorter) and the observed frequency, $f_o$, will be higher. 
\begin{parts}
\part Show that the frequency observed when a wave source moves either towards or away from a stationary observer with speed $v_s$ relative to the medium is given by:
\begin{align*}
f_o=\frac{v}{v\pm v_s}f_s
\end{align*}
Determine when the sign will be positive or negative.
\part An observer moves towards/away from a source with speed $v_o$ relative to the medium. Show that the observed frequency is given by:
\begin{align*}
f_o=\frac{v\pm v_o}{v}f_s
\end{align*}
Determine when the sign will be positive or negative.\\
\textbf{Hint:} Think relative velocities!


\end{parts}
\begin{solution}
\begin{parts}
\part It is easiest to approach this problem by considering how the wavelength changes when the source is moving towards/away from an observer. The wavelength at the source is $\lambda_s$. We know that the wavelength is equal to:
\begin{align*}
\lambda_s=\frac{v}{f}=vT_s
\end{align*}
where $T_s$ is the period. The wavelength is thus equal to the velocity of the wave multiplied by the difference in time between when two waves are emitted. When the source is moving at a velocity $v_s$, the source will travel a distance $d=\lambda T_s$ in the time between emitting two wave crests. So, the observed wavelength becomes $\lambda\pm d$ when the source is moving towards or away from the observer:
\begin{align*}
\lambda_o&=\lambda_s\pm d\\
\lambda_o&=\lambda_s\pm v_sT_s\\
\lambda_o&=\lambda_s\pm \frac{v_s}{f_s}\\
\end{align*}
writing the wavelength in terms of the frequency and the speed of the wave:
\begin{align*}
\lambda_o&=\lambda_s\pm \frac{v_s}{f_s}\\
\frac{v}{f_o}&=\frac{v}{f_s}\pm \frac{v_s}{f_s}\\
\frac{v}{f_o}&=\frac{v\pm v_s}{f_s}\\
\therefore f_o&=\frac{v}{v\pm v_s}f_s
\end{align*}
The frequency will be higher when the source is moving towards the observer, so the sign will be negative when the source is moving towards the observer and positive when it's moving away.

\part When the observer is moving towards/away from the source, the velocity of the wave relative to the observer ($v'$) is:
\begin{align*}
v'=v\pm v_o
\end{align*}
The observed frequency is given by the relative velocity divided by the emitted wavelength:
\begin{align*}
f_o=\frac{v\pm v_o}{\lambda_s}
\end{align*}
Putting the emitted wavelength in terms of the velocity of the wave and the emitted frequency gives:
\begin{align*}
f_o&=(v\pm v_o)\frac{f_s}{v}\\
f_o&=\frac{v\pm v_o}{v}f_s
\end{align*}
The observed velocity will be greater when the observer is moving towards the source, so the sign is positive when the observer is moving towards the source and negative when it's moving away from it.

\end{parts}
\end{solution}

\question Show that the function $D(x,t) = A\sin\left( kx -\omega t + \phi \right)$ satisfies the one-dimensional wave equation:
\begin{align*}
\die{^2D}{x^2}=\frac{1}{v^2}\die{^2D}{t^2}
\end{align*}
\begin{solution}
We start by taking the second partial derivative of $D$ with respect to $x$. Remember that when taking a partial derivative, we treat all other variables as constants. 
\begin{align*}
\die{D}{x}&=\die{}{x}A\sin\left( kx -\omega t + \phi \right)=kA\cos\left( kx -\omega t + \phi \right)\\
\therefore \die{^2D}{x^2}&=\die{}{x}kA\cos\left( kx -\omega t + \phi \right)=-k^2A\sin\left( kx -\omega t + \phi \right)
\end{align*}
where we used the chain rule. Now we take the second partial derivative of $D$ with respect to $t$:
\begin{align*}
\die{D}{t}&=\die{}{t}A\sin\left( kx -\omega t + \phi \right)=-\omega A\cos\left( kx -\omega t + \phi \right)\\
\therefore \die{^2D}{t^2}&=\die{}{t}(-\omega)A\cos\left( kx -\omega t + \phi \right)=-\omega^2A\sin\left( kx -\omega t + \phi \right)
\end{align*}
Putting this into the wave equation:
\begin{align*}
\die{^2D}{x^2}&=\frac{1}{v^2}\die{^2D}{t^2}\\
-k^2A\sin\left( kx -\omega t + \phi \right)&=\frac{1}{v^2}\cdot (-\omega^2)A\sin\left( kx -\omega t + \phi \right)\\
-k^2&=\frac{1}{v^2}(-\omega^2)\\
\end{align*}
Using the fact that $v=\omega/k$, we get:
\begin{align*}
-k^2&=-\frac{1}{v^2}\omega^2\\
-k^2&=-\frac{k^2}{\omega^2}\omega^2\\
-k^2&=-k^2\\
\end{align*}
which, shows that $D(x,t)$ was indeed a solution to the wave equation.
\end{solution}

%written by Jill Bennett -run this through RM. Is the part about it being fixed on both ends correct?
\question You have decided to learn a song a guitar. Unfortunately, you know nothing about guitars, and don't have one. You know that the fundamental frequency of an E note is $f = \SI{123.6}{Hz}$, and that the length of a string on a guitar is \SI{60}{cm}. You have a string of mass $m = \SI{4.2}{g}$. If the tension acting on the string is $F = \SI{500}{N}$, to what length must the string be shortened in order for it to produce an E note?

\begin{finalanswer}
	The string should be shortened to a length of \SI{23}{cm}.
\end{finalanswer}
\begin{solution}
	First, we must find the linear density of the string.
	\begin{align*}
	\mu&= \frac{m}{L}\\
	&= \frac{\SI{0.0042}{kg}}{\SI{0.6}{m}}\\
	&= \SI{7.0e-3}{kg/m}
	\end{align*}
	We can now find the velocity of the wave travelling through the string.
	
\begin{align*}
v& = \sqrt{\frac{F}{\mu}}\\
& = \sqrt{\frac{\SI{500}{N}}{\SI{7.0e-3}{kg/m}}}\\
& = \SI{267.26}{m/s}
\end{align*}
We can now find the fundamental frequency's wavelength if the string is only fixed on one end.
\begin{align*}
\lambda &= \frac{f}{v}\\
&=\frac{\SI{123.6}{Hz}}{\SI{267.26}{m/s}}\\
&= \SI{0.46}{m}
\end{align*}
Which gives a wavelength of $\lambda = \SI{0.46}{m}$. However, to calculate the length of the fundamental frequency, we must take into account that the guitar string is fixed at both ends. This means that we must halve the distance calculated above.
\begin{align*}
L_{two}&= \frac{L_{one}}{2}\\
&=\frac{\SI{0.46}{m}}{2}\\
&=\SI{0.23}{m}
\end{align*}
Therefore, we must shorten the string to $\SI{0.23}{cm}$
\end{solution}




 
 %TODO: A problem based on reflection/transmision between ropes of different densities that shows that the energy of the initial wave is conserved 

%Calculate and compare the energy in a standing wave for different harmonics

% TODO: Calculate the total force on a guitar handle, due to the tension in each string...
