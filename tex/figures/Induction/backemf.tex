\documentclass[12pt, openany]{book}
\usepackage{amssymb}
\usepackage{amsmath}
\usepackage{amstext}
\usepackage[font={small,it}]{caption}
\usepackage[pdftex]{graphicx} %Does not work in pressbooks!!!
\usepackage{color}
\usepackage{caption}
\usepackage[separate-uncertainty = true]{siunitx}
\usepackage{tikz}
\usepackage[american,smartlabels]{circuitikz}%for drawing circuits

\usetikzlibrary{arrows}

\begin{document}

\begin{center}
\begin{circuitikz}[]
\draw (0,0) to [battery1,l=$\Delta V$,*-*, i<=$I$] (4,0)
      to [short,i<=$I$] (4,2)
      to [R,l_=$R_{motor}$,*-] (2,2) 
      to [battery1,l_=Back emf,-*, invert] (0,2)
      to [short,i<=$I$](0,0);  
     \draw (1.65,0.3) node{$+$};
     \draw (2.35,0.3) node{$-$};
     \draw (0.65,1.7) node{$+$};
     \draw (1.35,1.7) node{$-$};
\end{circuitikz}
\captionof{figure}{\label{fig:induction:backemf}A simple circuit illustrating how a motor, with resistance, $R_{motor}$, will generate a ``back emf'', equivalent to a battery that produces a voltage in the direction to oppose the current from the actual battery that is powering the motor, $\Delta V$. } 
\end{center}

\end{document}