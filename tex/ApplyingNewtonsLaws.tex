\section{Applying Newton's Laws}

%%%%%%%%%%%%%%%%%%%%%%%%%%%%%%%%%%%
%%
%% Multiple Choice
%%
%%%%%%%%%%%%%%%%%%%%%%%%%%%%%%%%%%%

\subsection{Multiple Choice}

\question \label{question:applyingnewtonslaws:conicalpendulum} A conical pendulum is made by swinging a mass, $m$, attached to a mass-less string of length, $L$, so that the mass moves in a horizontal circle of radius, $R$, with a constant speed, $v$, as shown in Figure \ref{fig:applyingnewtonslaws:conicalpendulum}. Which of the following describes half of the opening angle of the cone (the angle $\theta$ in Figure \ref{fig:applyingnewtonslaws:conicalpendulum})?
\begin{checkboxes} 
	\choice $\tan\theta=\sqrt{\frac{v^2}{gR}}$
	\CorrectChoice $\tan\theta=\frac{v^2}{gR}$
	\choice $\tan\theta=\frac{gR}{v^2}$
	\choice $\tan\theta=\sqrt{\frac{gR}{v^2}}$
\end{checkboxes}
\capfig{0.30\textwidth}{figures/ApplyingNewtonsLaws/conicalpendulum.png}{\label{fig:applyingnewtonslaws:conicalpendulum}A conical pendulum executing uniform circular motion in a horizontal plane (Question \ref{question:applyingnewtonslaws:conicalpendulum}).}

\question You are sitting on a playground merry-go-round spinning at a constant speed and you remain in the same spot on the platform. Based on Newton's Laws, what statement best describes your situation?
\begin{checkboxes}
\choice You must apply an equal and opposite force to the one trying to fling you off in order to hold yourself on the merry-go-round.
\choice If the merry-go-round doesn't slow down or speed up, you are able to easily remain on the platform as you are moving at constant speed.
\CorrectChoice A force must be applied to keep you on the merry-go-round. \correct
\end{checkboxes}


\question A block of mass \SI{5}{kg} is on a slope held in place by a rope. The angle of the slope incline is $\alpha = \SI{60}{\degree}$ and the acceleration due to gravity is $a_g = \SI{9.8}{m/s^2}$. Assuming no frictional forces, what is the tension of the rope, $T$, keeping the block from sliding down?
\begin{checkboxes}
\choice \SI{40.3}{N}
\choice \SI{37.5}{N}
\choice \SI{78.6}{N}
\choice \SI{53.8}{N}
\CorrectChoice \SI{42.4}{N} \correct
\end{checkboxes}



%Ali Pirhadi
\question Two boxes of mass $m_1=\SI{5}{kg}$ and $m_2=\SI{3}{kg}$ are connected by a rope and rest on a frictionless surface. You pull on the box of mass $m_2=\SI{3}{kg}$ with a force $F=\SI{32}{N}$ so that the rope between the boxes is taught. What is the tension in the rope connecting the boxes?
\begin{checkboxes} 
\CorrectChoice \SI{20}{N} \correct
\choice \SI{23}{N}
\choice \SI{25}{N}
\choice \SI{32}{N}
\end{checkboxes}


%Zifeng Chen
\question A block of mass $M=\SI{10}{kg}$ is held by a force $\vec F$ on a surface inclined at angle of \SI{45}{\degree} with respect to the horizontal. Given that the coefficient of static friction between the block and the surface is 0.5, what is the minimum force $F$ necessary to prevent the box from slipping? 
\begin{checkboxes} 
\CorrectChoice \SI{34.6}{N} \correct
\choice \SI{49.0}{N}
\choice \SI{69.3}{N}
\choice \SI{138.6}{N}
\end{checkboxes}


%Jonathan Abott
\question A car goes around a curve of radius $R$ at a constant speed $v$. It then goes around a second curve of radius $2R$ at speed $2v$. What is the centripetal force on the car as it goes around the second curve, compared to the first?
\begin{checkboxes} 
\choice The centripetal forces in both curves are equal
\choice The centripetal force in the second curve is half as big as in the first curve
\CorrectChoice The centripetal force in the second curve is twice as big as in the first curve \correct
\choice The centripetal force in the second curve is four times as big as in the first curve
\end{checkboxes}

\question You design a banked curve on a highway for trucks to safely navigate a \SI{90}{\degree} turn in the winter. The turn can be approximated by a quarter of a circle of radius $R=\SI{500}{m}$. Which bank angle should you use so that trucks going \SI{110}{km/h} will be able to go around the curve even when the road is covered in ice (and thus frictionless to a good approximation)? \textit{The bank angle is defined as the angle between the plane of the curve and the horizontal.}
\begin{checkboxes}
\choice \SI{68}{\degree}
\choice \SI{18}{\degree}
\CorrectChoice \SI{11}{\degree} \correct
\choice \SI{0.2}{\degree}
\end{checkboxes}
\begin{solution}
\begin{align*}
N\cos\theta &= mg\\
N\sin\theta &= m\frac{v^2}{R}\\
\therefore \tan\theta&=\frac{v^2}{gR}=\frac{(\SI{30.56}{m/s})^2}{(\SI{9.80}{m/s^2})(\SI{500}{m})}=0.19\\
\therefore \theta &= \SI{10.79}{\degree}
\end{align*}
\end{solution}

\question Ozzy is riding in a clown car with a mass of \SI{10000}{kg} at a speed of \SI{100}{km/h} around a curve with a radius of \SI{400}{m}. What is the minimum coefficient of static friction between the car's four tires and the road to prevent the clown car from flying off the road?
\begin{checkboxes}
\choice 0.049
\CorrectChoice 0.197 \correct
\choice 0.420
\choice 0.787
\end{checkboxes}
\begin{solution}
The road pushes up on each tire with a force equal to a quarter of the weight of the car. The force of friction on each tire is thus:
\begin{align*}
f=\mu_s\frac{1}{4}Mg
\end{align*}
The only forces with components along the centre of the circle are the four frictional forces:
\begin{align*}
4f&=M\frac{v^2}{R}\\
\mu_sMg&=M\frac{v^2}{R}\\
\therefore \mu_s&=\frac{v^2}{Rg}=\frac{(\SI{27.78}{m/s})^2}{(\SI{9.8}{m/s^2})(\SI{400}{m})}=0.197
\end{align*}
The number of tires does not matter...
\end{solution}

%Olivia Bouaban
\question What should the speed limit be on a curve with a radius of \SI{100}{m} to prevent cars from slipping if the static coefficient of friction between the road and wheels of cars is 0.70? Assume that the road is flat.
\begin{checkboxes}
\choice \SI{26}{km/h}
\choice \SI{50}{km/h}
\CorrectChoice \SI{94}{km/h} \correct
\choice Not enough information to determine
\end{checkboxes}
\begin{solution}
\begin{align*}
\mu_sMg&=M\frac{v^2}{R}\\
\therefore v&=\sqrt{\mu_sgR}=\sqrt{(0.7)(\SI{9.8}{m/s^2})(\SI{100}{m})}=\SI{26.16}{m/s}
\end{align*}
\end{solution}


%Jesse
\question You are planning your descent onto planet Camelid with your crew by deploying parachutes from the command module where you are all strapped in. Currently, the projected terminal velocity of \SI{80}{km/h} is too high and will destroy your cool physics experiments on-board. You need to cut this velocity in half. 
Assuming all the drag force comes from the parachute, your crew come up with the following in order to cut your speed in half:
\begin{checkboxes}
\CorrectChoice Quadruple the size of the parachute \correct
\choice Halve the size of the parachute
\choice Double the size of the parachute
\choice Increase the size of the parachute by \SI{50}{\%}
\end{checkboxes}



\question Which statement(s) is correct? [Select all that apply]
\begin{checkboxes}
\choice The centripetal force and centrifugal force are the same force, but named differently.
\CorrectChoice Centrifugal force is equal and opposite to the centripetal force. \correct
\CorrectChoice The centrifugal force is actually an apparent force. \correct
\choice The centripetal force is an inertial force.
\end{checkboxes}

%Submitted by Jack Fitzgerald
\question An American man at a gun range is practising his marksmanship. Using his high-powered rifle, he attempts to hit a target \SI{1200}{yards} away, but notices all of his shots roll to the left. After several attempts, he gets frustrated and leaves. What was happening to the bullet at those long ranges?
\begin{checkboxes}
\choice Voodoo
\CorrectChoice Coriolis Effect \correct
\choice Drag Force
\choice Stokes' Law
\end{checkboxes}


\question Why does a banked curve allow a car to go through a turn at a higher speed? [Select all that apply]
\begin{checkboxes}
\choice The centripetal force necessary is lessened by the bank.
\CorrectChoice The bank forces a component of the normal force to make up part of the centripetal force. \correct
\choice The bank is more fun to drive around.
\CorrectChoice The friction force necessary to keep the tires from slipping is lessened. \correct
\end{checkboxes}

%Duncan Gould
\question A mass on a string rotates in a vertical circle around a point.  What can be said about the tension in the string when the object is at the top of the circle ($T_{top}$) and when the mass is at the bottom of the circle ($T_{bottom}$)?
\begin{checkboxes}
\choice $T_{top}=T_{bottom}$
\CorrectChoice $T_{top}<T_{bottom}$
\choice $T_{top}>T_{bottom}$
\choice Not enough information to tell.
\end{checkboxes}

%Based on Neil Rajan
\question A roller coaster cart of mass $m$ is going through a vertical loop with a radius $R$. What is the minimum speed that the cart must have at the top of the loop for the cart to stay on the track?
\begin{checkboxes}
\choice $v_{top}\geq m\sqrt{gR}$
\choice   $v_{top}\geq\sqrt{mgR}$
\CorrectChoice  $v_{top}\geq\sqrt{gR}$
\choice  $v_{top}\geq gR$
\end{checkboxes}

%Harder...
\question What is the ideal bank angle (relative to the horizontal) for a road that goes around a curve of radius $R=\SI{20}{m}$ if vehicles drive at $\SI{60}{km/h}$. In other words, if the road is perfectly frictionless (e.g. covered with ice), at what angle should the curve be banked?
\begin{checkboxes}
\choice    $\SI{5}{\degree}$
\choice  $\SI{17}{\degree}$
\CorrectChoice  $\SI{55}{\degree}$
\choice    $\SI{87}{\degree}$
\end{checkboxes}

%%%%%%%%%%%%%%%%%%%%%%%%%%%%%%%%%%%
%
% Long Answer
%
%%%%%%%%%%%%%%%%%%%%%%%%%%%%%%%%%%%
\subsection{Long answers}

\question At how many revolutions per minute must a \SI{22}{m} diameter Ferris wheel turn for passengers to feel weightless at the top?
\begin{solution}
For the passengers to feel weightless, they must not feel a normal force. Thus, their centripetal acceleration must be given by the force of gravity:
\begin{align*}
m\omega^2R&=mg\\
\therefore \omega&=\sqrt{\frac{g}{R}}=\sqrt{\frac{(\SI{9.8}{m/s^2})}{(\SI{11}{m})}}=\SI{0.94}{rad/s}\left(\frac{1\si{rev}}{2\pi\si{rad}}\right)\left( \SI{60}{s/min} \right)=\SI{9.01}{rpm}
\end{align*}
\end{solution}


\question A funnel is rotating with an angular speed $\omega$ about a vertical axis. A block of mass $m$ sits inside of the funnel, at a distance $r$ from the axis of rotation (Figure \ref{fig:applyingnewtonslaws:funnel}). The opening angle of the funnel is $\theta$. The coefficient of static friction between the funnel and the block is $\mu$. What are the minimum and maximum values of $\omega$ such that the block will not slide?
\capfig{0.25\textwidth}{figures/ApplyingNewtonsLaws/funnel.png}{\label{fig:applyingnewtonslaws:funnel}A block in a rotating funnel.}

\begin{finalanswer}
$\omega_{min} = \sqrt{\frac{g}{r}\left(\frac{\cos\theta-\mu\sin\theta}{\sin\theta+\mu\cos\theta}\right)}$,$\omega_{max} = \sqrt{\frac{g}{r}\left(\frac{\cos\theta+\mu\sin\theta}{\sin\theta-\mu\cos\theta}\right)}$
\end{finalanswer}
\begin{solution}
The minimum angular velocity could be 0, if the force of static friction is high enough to keep the block from sliding.  For the minimum angular velocity, the force of friction will have an upwards component, as shown in Figure \ref{fig:applyingnewtonslaws:funnel_FBDmin}
\capfig{0.15\textwidth}{figures/ApplyingNewtonsLaws/funnel_FBDmin.png}{\label{fig:applyingnewtonslaws:funnel_FBDmin}Forces on the block when the angular speed is minimal (possibly 0).}
For the block not to move, the sum of the forces in the vertical direction (choose positive $y$ upwards) must be zero, and the sum of the horizontal forces (choose positive $x$ to the left) must equal the force required for the centripetal acceleration:
\begin{align*}
\sum F_y &= N\sin\theta-mg+f_s\cos\theta = 0 \\
\sum F_x &= N\cos\theta-f_s\sin\theta = m\omega_{min}^2 r \\
\end{align*}
The magnitude of the force of static friction is given by $f_s=\mu N$, so we have:
\begin{align*}
 N\sin\theta-mg+\mu N\cos\theta &= 0 \\
 N(\cos\theta-\mu\sin\theta) &= m\omega_{min}^2 r \\
\end{align*}
Isolating $N$ and solving for $\omega$:
\begin{align*}
N &= \frac{mg}{\sin\theta+\mu\cos\theta}\\
m\omega_{min}^2 r &= mg\frac{\cos\theta-\mu\sin\theta}{\sin\theta+\mu\cos\theta}\\
\therefore \omega_{min} &= \sqrt{\frac{g}{r}\left(\frac{\cos\theta-\mu\sin\theta}{\sin\theta+\mu\cos\theta}\right)}
\end{align*}
For the maximum value of $\omega$, the force of static friction will be in the opposite direction, preventing the block from flying out of the funnel, as in Figure \ref{fig:applyingnewtonslaws:funnel_FBDmax}.
\capfig{0.15\textwidth}{figures/ApplyingNewtonsLaws/funnel_FBDmax.png}{\label{fig:applyingnewtonslaws:funnel_FBDmax}Forces on the block when the angular speed is maximal.}
We still require that the sum of the vertical forces be 0 and the sum of the horizontal forces give the force required for the centripetal acceleration. Using the same coordinates as before, we have:
\begin{align*}
\sum F_y &= N\sin\theta-mg-f_s\cos\theta = 0 \\
\sum F_x &= N\cos\theta+f_s\sin\theta = m\omega_{max}^2 r \\
\end{align*}
The magnitude of the force of static friction is given by $f_s=\mu N$, so we have:
\begin{align*}
 N\sin\theta-mg-\mu N\cos\theta &= 0 \\
 N(\cos\theta+\mu\sin\theta) &= m\omega_{max}^2 r \\
\end{align*}
Isolating $N$ and solving for $\omega$:
\begin{align*}
N &= \frac{mg}{\sin\theta-\mu\cos\theta}\\
m\omega_{max}^2 r &= mg\frac{\cos\theta+\mu\sin\theta}{\sin\theta-\mu\cos\theta}\\
\therefore \omega_{max} &= \sqrt{\frac{g}{r}\left(\frac{\cos\theta+\mu\sin\theta}{\sin\theta-\mu\cos\theta}\right)}
\end{align*}
\end{solution}

%Giancolli 5-33 -fixed
\question A child is playing with wooden blocks on a frictionless horizontal surface. The child places a wedge-shaped block of mass $M$ on the ground such that the hypotenuse is at an angle $\theta$ to the horizontal. The child then places a smaller cube of mass $m$ on top of the hypotenuse of the wedge-shaped block. The coefficient of friction between the two blocks is $\mu$. What is the minimum force that the child must apply to the wedge-shaped block for the small cube resting on top of it to accelerate up the hypotenuse?
\capfig{0.3\textwidth}{figures/ApplyingNewtonsLaws/WedgeBlock.png}{\label{fig:applyingnewtonslaws:WedgeBlock}Block $m$ sitting on wedge-shaped block of mass $M$.}

\begin{finalanswer}
$F =(M+m)g\left(\frac{\sin\theta+\mu\cos\theta}{\cos\theta-\mu\sin\theta}\right)$
\end{finalanswer}
\begin{solution}
We start by drawing the free body diagrams for both masses, noting that the masses are accelerating towards the right (positive $x$) with the same acceleration and that they have no acceleration in the vertical direction ($y$ positive upwards). We also note that just before the small block starts sliding upwards, the force of static friction between it and the large block must have a downwards component, and its magnitude is given by $\mu N_m$, where $\vec N_m$ is the normal force exerted by the big block on the small block.

Figure \ref{fig:applyingnewtonslaws:WedgeBlock_FBD} shows the free body diagrams for each block.  $\vec f_s$ is the force of friction between the two blocks, and $\vec N_M$ is the normal force exerted by the ground on the big block. Both $N_m$ an $f_s$ have corresponding reaction forces ($\vec N_m'$, $\vec f_s'$)) exerted on the large block (with equal magnitude, but opposite directions)
\capfig{0.3\textwidth}{figures/ApplyingNewtonsLaws/WedgeBlock_FBD.png}{\label{fig:applyingnewtonslaws:WedgeBlock_FBD}Free body diagram for each block.}

Writing the sum of the forces in each direction for the small block:
\begin{align*}
\sum F_x &= N_m\sin\theta+f_s\cos\theta = ma \\
\sum F_y &= N_m\cos\theta-mg-f_s\sin\theta = 0 \\
\end{align*}
Repeating for the large block:
\begin{align*}
\sum F_x &= F-N_m\sin\theta-f_s\cos\theta = Ma \\
\sum F_y &= N_M-N_m\cos\theta-Mg+f_s\sin\theta = 0 \\
\end{align*}
This gives us 4 equations for the 4 unknowns: ${F, N_m, N_M, a}$ ($f_s$ is not really unknown, since it is related to $N_m$, $f_s=\mu N_m$). Adding the $x$ equations gives:
\begin{align*}
F = (M+m)a
\end{align*}
which makes sense, if we consider the two blocks as a single system with mass $M+m$. We can re-arrange the two equations for the small block of mass $m$ to isolate $N_m$:
\begin{align*}
N_m& = \frac{ma}{\sin\theta+\mu\cos\theta} \\
N_m& = \frac{mg}{\cos\theta-\mu\sin\theta}
\end{align*}
Setting these two equal to each other and isolating for $a$:
\begin{align*}
\frac{ma}{\sin\theta+\mu\cos\theta} &= \frac{mg}{\cos\theta-\mu\sin\theta}\\
\therefore a&=g\left(\frac{\sin\theta+\mu\cos\theta}{\cos\theta-\mu\sin\theta}\right)
\end{align*}
which gives $F$:
\begin{align*}
F &=(M+m)a=(M+m)g\left(\frac{\sin\theta+\mu\cos\theta}{\cos\theta-\mu\sin\theta}\right)
\end{align*}
\end{solution}


%modified Giancolli 5-20
\question Two blocks  made of different materials and with masses $m_A$ and $m_B$, respectively, are connected by a rope and slide down an inclined slope as shown in Figure \ref{fig:applyingnewtonslaws:TwoBlockWedge}. The incline makes an angle $\theta$ with the horizontal. The (different) coefficients of kinetic friction between the blocks and the incline are $\mu_A$ and $\mu_B$ for block A and B, respectively.
\begin{parts}
\part Given the masses of the blocks, the angle $\theta$, and the coefficients of kinetic friction, what is the acceleration of the two blocks?
\part Given the masses of the blocks, the angle $\theta$, and the coefficients of kinetic friction, what is the tension in the rope?
\end{parts} 
\capfig{0.3\textwidth}{figures/ApplyingNewtonsLaws/TwoBlockWedge.png}{\label{fig:applyingnewtonslaws:TwoBlockWedge}Two connected blocks sliding down an incline.}

\begin{finalanswer}
\begin{enumerate}[(a)]
\item $a = g\sin\theta-g\cos\theta\left( \frac{\mu_A m_A+\mu_B m_B}{m_A+m_B} \right)$
\item $T = m_Bg\cos\theta\left(  \mu_B  - \left( \frac{\mu_A m_A+\mu_B m_B}{m_A+m_B} \right) \right)$
\end{enumerate}
\end{finalanswer}
\begin{solution}
We start with a free body diagram for the two blocks, as shown in Figure \ref{fig:applyingnewtonslaws:TwoBlockWedge_FBD}. The tension force has the same magnitude on each block (but different directions). Both blocks experience a force of gravity, a normal force, and a force of kinetic friction. 
\capfig{0.3\textwidth}{figures/ApplyingNewtonsLaws/TwoBlockWedge_FBD.png}{\label{fig:applyingnewtonslaws:TwoBlockWedge_FBD}Free body diagram for the situation depicted in Figure \ref{fig:applyingnewtonslaws:TwoBlockWedge}}
To simplify the analysis, we choose the positive $x$ direction to be along the direction of motion (parallel to the incline), and the $y$ direction to be perpendicular and away from the incline. Both blocks must have the same acceleration.

The sum of the forces for block A is:
\begin{align*}
\sum F_x &= m_Ag\sin\theta-T-\mu_A N_A = m_Aa \\
\sum F_y &= N_A-m_Ag\cos\theta = 0 \\
\end{align*}
and for block B, using the same coordinate system:
\begin{align*}
\sum F_x &= m_Bg\sin\theta+T-\mu_B N_B = m_Ba \\
\sum F_y &= N_B-m_Bg\cos\theta = 0 \\
\end{align*}
In both cases, the $y$ equations allow us to solve the normal forces:
\begin{align*}
N_A&=m_Ag\cos\theta\\
N_B&=m_Bg\cos\theta\\
\end{align*}
Using the $x$ component of the equation for block B, we can isolate for the tension:
\begin{align*}
m_Ba&=m_Bg\sin\theta+T-\mu_B m_Bg\cos\theta \\
\therefore T&=m_Ba-m_Bg\sin\theta+\mu_B m_Bg\cos\theta
\end{align*}
We can then substitute the tension into the $x$ component of the equation for Block A, and solve for the acceleration:
\begin{align*}
m_Aa &= m_Ag\sin\theta-(m_Ba-m_Bg\sin\theta+\mu_B m_Bg\cos\theta)-\mu_A m_Ag\cos\theta \\
&=m_Ag\sin\theta-m_Ba+m_Bg\sin\theta-\mu_B m_Bg\cos\theta-\mu_A m_Ag\cos\theta \\
\therefore a &= \frac{1}{m_A+m_B}(m_Ag\sin\theta+m_Bg\sin\theta-\mu_B m_Bg\cos\theta-\mu_A m_Ag\cos\theta)\\
&=\frac{g\sin\theta(m_A+m_B)-g\cos\theta(\mu_B m_B+\mu_A m_A)}{m_A+m_B}\\
&=g\sin\theta-g\cos\theta\left( \frac{\mu_A m_A+\mu_B m_B}{m_A+m_B} \right)
\end{align*}
This makes sense, as the acceleration reduces to $g\sin\theta$ in the absence of friction.

We can easily find the tension by using the acceleration that we just found:
\begin{align*}
 T&=m_Ba-m_Bg\sin\theta+\mu_B m_Bg\cos\theta\\
 &=m_B\left( g\sin\theta-g\cos\theta\left( \frac{\mu_A m_A+\mu_B m_B}{m_A+m_B} \right) \right)-m_Bg\sin\theta+\mu_B m_Bg\cos\theta\\
 &=m_Bg\cos\theta\left(  \mu_B  - \left( \frac{\mu_A m_A+\mu_B m_B}{m_A+m_B} \right) \right)
\end{align*}
Again, this makes qualitative sense. If the mass $m_B$ is zero, then there is no tension. If the coefficient $\mu_B$ is too small, then there is no tension (a negative tension means no tension); this would correspond to Block B accelerating more than Block A and catching up.
\end{solution}

%Giancolli 5-31 -fixed
\question A square block, $m_A$, is placed on top of a rectangular block, $m_B$. A wire is connected to both blocks through a pulley which is attached to a nearby wall, as shown in Figure \ref{fig:applyingnewtonslaws:TwoBlockPulley}. Suppose the coefficient of static friction between block A and block B and the coefficient of static friction between block B and the ground are both equal to $\mu$. A force $F$ pulls block B away from the pulley. What is the maximum force that can be applied such that block A and block B remain at rest?

\capfig{0.3\textwidth}{figures/ApplyingNewtonsLaws/TwoBlockPulley.png}{\label{fig:applyingnewtonslaws:TwoBlockPulley}Two blocks connected by a pulley.}

\begin{finalanswer}
$F=\mu g(3m_A+m_B)$
\end{finalanswer}
\begin{solution}
We start by drawing a free body diagram for both blocks, as shown in Figure \ref{fig:applyingnewtonslaws:TwoBlockPulley_FBD}, where:
\begin{itemize}
\item $\vec f_A$ is the force of friction between block A and block B (with magnitude $\mu N_A$)
\item $\vec N_A$ is the normal force on block A
\item $\vec T$ is the tension in the rope
\item $\vec f_B$ is the force of friction between block B and the ground (with magnitude $\mu N_B$)
\item $\vec N_B$ is the normal force on block B from the ground
\item everything with a (') is the corresponding reaction force.
\end{itemize}
\capfig{0.3\textwidth}{figures/ApplyingNewtonsLaws/TwoBlockPulley_FBD.png}{\label{fig:applyingnewtonslaws:TwoBlockPulley_FBD}Two blocks connected by a pulley.}
We can now write the sum of the forces in each direction, for each block and set them equal to zero. We can define a $xy$ coordinate system where $x$ is positive to the right and $y$ is positive upwards. For block A, we have:
\begin{align*}
\sum F_x &=\mu N_A - T = 0 \\
\sum F_y &= N_A-m_Ag = 0 \\
\end{align*}
For block B:
\begin{align*}
\sum F_x &= F-\mu N_B-\mu N_A-T= 0 \\
\sum F_y &= N_B-m_Bg-N_A = 0 \\
\end{align*}
From block A, we can easily solve for the normal force $N_A$ (using the $y$ component) and then determine the tension $T$:
\begin{align*}
N_A&=m_Ag\\
\therefore T&=\mu m_Ag
\end{align*}
From the $y$ component of the block B equation, we can easily find $N_B$ (using the result above for $N_A$), and then substitute everything into the $x$ equation for block B:
\begin{align*}
N_B&=m_Bg+N_A=g(m_A+m_B)\\
0&=F-\mu g(m_A+m_B)-\mu m_Ag-\mu m_Ag\\
\therefore F&=\mu g(m_A+m_B)+\mu m_Ag+\mu m_Ag\\
&=\mu g(3m_A+m_B)
\end{align*}
\end{solution}

\question An air plane is flying at \SI{200}{m/s} in a holding pattern above the Cusco airport. The airplane flies around in horizontal circles, such that each circle takes \SI{5}{min} to complete.
\begin{parts}
\part What bank angle must the pilot fly so that each circle takes \SI{5}{min} to complete?
\part What is the percentage increase in the perceived weight of the passengers? (This is given by the new perceived weight minus the old perceived weight divided by the old perceived weight)
\end{parts}

\begin{finalanswer}
\begin{enumerate}[(a)]
\item \SI{23.14}{\degree}
\item 8.75\%
\end{enumerate}
\end{finalanswer}
\begin{solution}
\textbf{a)} We start by drawing a free body diagram for the airplane; the two only forces are the lift from the wings ($\vec L$) and the airplane's weight, as shown in Figure \ref{fig:applyingnewtonslaws:Plane_FBD}.
\capfig{0.1\textwidth}{figures/ApplyingNewtonsLaws/Plane_FBD.png}{\label{fig:applyingnewtonslaws:Plane_FBD}Free body diagram for plane/passenger.}
The sum of the forces must be towards the centre of the circle and equal to centripetal force required to keep the airplane flying in a circle:
\begin{align*}
L\cos\theta-mg&=0\\
L\sin\theta=m\frac{v^2}{R}\\
\therefore \tan\theta = \frac{v^2}{Rg}
\end{align*}
We are told that the plane take \SI{5}{min} to complete a circle, which allows to calculate the radius of the circle:
\begin{align*}
vt&=2\pi R\\
\therefore R&=\frac{vt}{2\pi}=\frac{(\SI{200}{m/s})(\SI{5}{min})}{2\pi}=\SI{9549.30}{m}
\end{align*}
giving the bank angle to be:
\begin{align*}
\theta = \tan^{-1}\left(\frac{v^2}{Rg}\right)=\tan^{-1}\left(\frac{(\SI{200}{m/s})^2}{(\SI{9549.30}{m})(\SI{9.8}{m/s^2})}\right)=\SI{23.14}{\degree}
\end{align*}
\textbf{b} The perceived weight of the passengers is equal to the normal force that their seat exerts on them. The free body diagram is identical to that in Figure \ref{fig:applyingnewtonslaws:Plane_FBD}, except with the normal force instead of lift.  With no bank, they would feel a normal force equal to $mg$. In the bank, the normal force is:
\begin{align*}
N^{bank}&=\frac{mg}{\cos\theta}
\end{align*}
The percent increase in the perceived weight is:
\begin{align*}
\Delta &= \frac{N^{bank}-N}{N}\\
&=\frac{\frac{mg}{\cos\theta}-mg}{mg}\\
&=\frac{1}{\cos\theta} - 1 \\
&=0.0875=8.75\%
\end{align*}
\end{solution}

\question A rocket of mass $m$ burns fuel producing an upwards force (thrust) on the rocket, $F(t)$, that varies with time:
\begin{align*}
F(t) = A+Bt
\end{align*}
The rocket starts at rest and travels in the vertical direction. Assume that the only two forces on the rocket are its weight and the thrust (no drag).
\begin{parts}
\part Assuming that the mass of the rocket is constant with time, and that the acceleration due Earth's gravity does not change as the rocket's altitude increases, what is the rocket's altitude as a function of time (in terms of $A$, $B$, $m$, $g$, and time)?
\part If the rocket loses mass as it burns fuel, would it gain altitude at a higher or at a lower rate than in part a)? Justify your answer!
\end{parts}

\begin{finalanswer}
\begin{enumerate}[(a)]
\item $y(t) = \frac{A}{2m}t^2 + \frac{B}{6m}t^3 - \frac{1}{2}gt^2$
\item If the mass is decreasing as a function of time, then the acceleration will be higher than in part a), so the rocket will cover more distance and its altitude will thus increase at a higher rate.
\end{enumerate}
\end{finalanswer}
\begin{solution}
\textbf{a)} The rocket only has two forces on it, the thrust, and its weight, acting in opposite directions. If we define the positive $y$ directions as upwards, we have:
\begin{align*}
\sum \vec F = \sum F_y = F(t)-mg=ma
\end{align*}
The acceleration is thus given by:
\begin{align*}
a(t) &= \frac{F(t)}{m} - mg\\
&=\frac{A+Bt}{m}-g
\end{align*}
and it changes with time. To find position as a function of time, we need to integrate this twice, first to get velocity, then position:
\begin{align*}
v(t) &= \int a(t) dt +C\\
&=\int (\frac{F(t)}{m}-g) dt + C\\
&=\int (\frac{A+Bt}{m}-g) dt + C\\
&=  \frac{A}{m}t + \frac{B}{2m}t^2 - gt + C
\end{align*}
Since the rocket starts at rest at $t=0$, we can set the constant $C$ to zero. We now integrate to get altitude as a function of time:
\begin{align*}
y(t) &= \int v(t) dt + C'\\
&= \int \left( \frac{A}{m}t - \frac{B}{2m}t^2  \right) dt + C'\\
&= \frac{A}{2m}t^2 + \frac{B}{6m}t^3 - \frac{1}{2}gt^2+ C'\\  
\end{align*}
where again, if we define the origin to be where the rocket starts, then we can set $C'=0$:
\begin{align*}
y(t) = \frac{A}{2m}t^2 + \frac{B}{6m}t^3 - \frac{1}{2}gt^2
\end{align*} 
\textbf{b)} If the mass is decreasing as a function of time, then the acceleration will be higher than in part a), so the rocket will cover more distance and its altitude will thus increase at a higher rate.
\end{solution}

\question Three blocks with masses $m_1$, $m_2$, and $m_3$ are stacked upon each other as shown in Figure \ref{fig:applyingnewtonslaws:ThreeBlocks}. The coefficients of static friction between blocks is $\mu_s$, and the coefficient of kinetic friction between the bottom block and the ground is $\mu_k$. The blocks are moving in unison towards the right. Show that the maximum magnitude of the force that can be applied to the bottom block before the blocks slide with respect to each other is given by:
\begin{align*}
F=(\mu_s+\mu_k)(m_1+m_2+m_3)g
\end{align*}

\capfig{0.4\textwidth}{figures/ApplyingNewtonsLaws/ThreeBlocks.png}{\label{fig:applyingnewtonslaws:ThreeBlocks} Three blocks sliding with friction.}

\begin{solution}
We first draw a free body diagram for each of the three masses, noting that they all have the same acceleration to the right, as shown in Figure \ref{fig:applyingnewtonslaws:ThreeBlocks_FBD}.
\capfig{0.9\textwidth}{figures/ApplyingNewtonsLaws/ThreeBlocks_FBD.png}{\label{fig:applyingnewtonslaws:ThreeBlocks_FBD} Free body diagram for the blocks in Figure \ref{fig:applyingnewtonslaws:ThreeBlocks}.}

We can use the coordinate system shown in Figure \ref{fig:applyingnewtonslaws:ThreeBlocks_FBD} ($y$ vertical and $x$ to the right) for all three blocks). Noting that in each case the friction forces are related to the corresponding normal force, the sum of the forces in each direction for each block can be written as:
\begin{align*}
\sum F_y^{(3)} &= N_3-m_3g=0 \\
\sum F_x^{(3)} &= f_3 =m_3 a\\
\sum F_y^{(2)} &= N_2-N_3-m_2g=0 \\\
\sum F_x^{(2)} &= f_2-f_3 = \mu_sN_2-\mu_sN_3=m_2 a\\
\sum F_y^{(1)} &= N_1-N_2-m_1g=0 \\\
\sum F_x^{(1)} &= F-f_2-f_1 = F-\mu_sN_2-\mu_kN_1=m_1 a\\
\end{align*}
The $x$ and $y$ equations are independent, and we can easily use the $y$ equations to solve for all three normal forces:
\begin{align*}
N_3 &= m_3g\\
N_2 &= (m_2+m_3)g\\
N_1 &= (m_1+m_2+m_3)g
\end{align*}
Substituting these into the $x$ equations:
\begin{align*}
m_3a &= \mu_sm_3g \to a=\mu_sg\\
m_2a &= \mu_s(m_2+m_3)g-\mu_sm_3g=\mu_sm_2g\\
m_1a &= m_1\mu_sg=F-\mu_s(m_2+m_3)g-\mu_k(m_1+m_2+m_3)g\\
\therefore F &= \mu_s(m_1+m_2+m_3)g+\mu_k(m_1+m_2+m_3)g\\
&=(\mu_s+\mu_k)(m_1+m_2+m_3)g
\end{align*}
as required.
\end{solution}

%From Midyear Makeup Exam F17
%Harrison/Zaremba 1999 December exam
\question Block $A$ and $B$ are connected by a string and massless frictionless pulleys as shown in Figure \ref{fig:applyingnewtonslaws:BlockPulleys}. The blocks have masses of $m_A=\SI{2.0}{kg}$ and $m_B=\SI{4.0}{kg}$. If the blocks are in motion ($A$ is falling), and the coefficient of kinetic friction between block $B$ and the surface of the table-top is $\mu=0.5$, what are the accelerations of blocks $A$ and $B$?
\capfig{0.4\textwidth}{figures/ApplyingNewtonsLaws/BlockPulleys.png}{\label{fig:applyingnewtonslaws:BlockPulleys}A system of blocks and pulleys.}
\begin{finalanswer}
$a_A=\SI{3.27}{m/s^2}, a_B=\SI{1.63}{m/s^2}$
\end{finalanswer}

\begin{solution}
Figure \ref{fig:applyingnewtonslaws:BlockPulleys_FBD} shows the free body diagrams for each mass. 
\capfig{0.2\textwidth}{figures/ApplyingNewtonsLaws/BlockPulleys_FBD.png}{\label{fig:applyingnewtonslaws:BlockPulleys_FBD}Free body diagram for each mass.}
Because of the pulley on mass $B$, it will have half of the acceleration of mass $A$, as it covers half of distance as mass $A$ in any given amount of time. Since the tension on either side of the pulley attached to mass $B$ is the same, the force on $B$ from the rope is twice the tension in the rope. 

For block $B$, the sum of the forces in the vertical direction is zero, so the normal force is given by:
\begin{align*}
N=m_Bg
\end{align*}
In the horizontal direction (positive to the right), Newton's second Law gives:
\begin{align*}
2T-f&=m_Ba\\
2T-\mu m_Bg&=m_Ba
\end{align*}
where we used $f=\mu N$ for the force of kinetic friction, and $a$ is the acceleration of block $B$. 
 
For block $A$, with twice the acceleration, we can write Newton's second law in the vertical direction (positive downwards):
\begin{align*}
m_Ag-T=2m_Aa
\end{align*}
We can solve for the tension in the above equation:
\begin{align*}
T=m_Ag-2m_Aa
\end{align*}
which we then place into the equation for the horizontal acceleration of block $B$:
\begin{align*}
2T-\mu m_Bg&=m_Ba \\
2(m_Ag-2m_Aa)-\mu m_Bg&=m_Ba \\
2m_Ag-4m_Aa-\mu m_Bg&=m_Ba\\
(2m_A-\mu m_B)g &= (m_B+4m_A)a\\
\therefore a &=g \frac{2m_A-\mu m_B}{m_B+4m_A}=(\SI{9.8}{m/s^2}) \frac{2(\SI{2.0}{kg})-0.5(\SI{4.0}{kg})}{(\SI{4.0}{kg})+4(\SI{2.0}{kg})}\\
&=\SI{1.63}{m/s^2}
\end{align*}
which corresponds to the acceleration of block $B$. The acceleration of block $A$ is just twice this value, \SI{3.27}{m/s^2}

\end{solution}

\question 
You push with an unknown horizontal force, $\vec F$, against a \SI{4}{kg} crate that is located on an inclined plane that makes a \SI{30}{\degree} angle with respect to the horizontal, as shown in Figure \ref{fig:applyingnewtonslaws:inclineconstant}. The coefficient of kinetic friction between the crate and the incline is 0.2.
\capfig{0.3\textwidth}{figures/ApplyingNewtonsLaws/inclineconstant.png}{\label{fig:applyingnewtonslaws:inclineconstant} A crate being pushed up an incline.}
\begin{parts} 
\part What is the magnitude of the force $\vec F$ that must be applied in order for the box to move at a constant speed?
\part What is the value of the normal force if the box instead accelerates at $\SI{2}{m/s^2}$ up the incline? How does this compare to the magnitude of the normal force when the box moves at a constant speed?
\end{parts}

\begin{finalanswer}
\begin{enumerate}[(a)]
\item \SI{34}{N}
\item \SI{56}{N}
\end{enumerate}
\end{finalanswer}
\begin{solution}
\begin{parts}
\part We start by drawing a free body diagram, shown in Figure \ref{fig:applyingnewtonslaws:inclineconstantfbd}.
\capfig{0.3\textwidth}{figures/ApplyingNewtonsLaws/inclineconstantfbd.png}{\label{fig:applyingnewtonslaws:inclineconstantfbd} Free-body diagram for the crate on the incline.}
 In order to find the magnitude of $\vec F$, we first need to determine the magnitude of the normal force. Since we want the crate to move at a constant velocity, its acceleration must be zero, so the sum of the forces must be zero. Writing out the $y$ component of Newton's Second Law allows us to find the magnitude of the normal force:
\begin{align*}
\sum F_y &= N\cos\theta -F_g - f_k\sin\theta = 0\\
\therefore mg &= N\cos\theta-\mu_kN\sin\theta = N(\cos\theta-\mu_k\sin\theta)\\
\therefore N &= \frac{mg}{\cos\theta-\mu_k\sin\theta}
\end{align*}
Writing out the $x$ component of Newton's Second Law allows us to find the magnitude of the unknown force $F$:
\begin{align*}
\sum F_x &= F - N\sin\theta - f_k\cos\theta = 0\\
\therefore F &= N\sin\theta+\mu_kN\cos\theta = N(\sin\theta+\mu_k\cos\theta)\\
&=mg\frac{\sin\theta+\mu_k\cos\theta}{\cos\theta-\mu_k\sin\theta}
\end{align*}
Now we just need to substitute in values for $m$, $g$, $\mu_k$ and $\theta$ to find the magnitude of $F$:
\begin{align*}
F&=mg\frac{\sin\theta+\mu_k\cos\theta}{\cos\theta-\mu_k\sin\theta}\\
&=(\SI{4}{kg})(\SI{9.8}{m/s^2})\frac{\sin 30 + 0.2\cos 30}{\cos 30 - \mu_k \sin 30}\\
&=\SI{34}{N}
\end{align*}
\part We want to compare the magnitude of the normal force when the acceleration of the crate is $\SI{2}{m/s^2}$ to the normal force when the crate moves at a constant velocity. We already found an equation for the magnitude of the normal force when the acceleration is zero:
\begin{align*}
N = \frac{mg}{\cos\theta-\mu_k\sin\theta}
\end{align*}
Plugging in our values, we find that the magnitude of the normal force is:
\begin{align*}
N&=\frac{(\SI{4}{kg})(\SI{9.8}{m/s^2})}{\cos 30 - \mu_k \sin 30}\\
\therefore N&=\SI{51}{N}
\end{align*} 
To find the magnitude of $N$ when the box is accelerating, we first need to find the $y$ component of the acceleration: 
\begin{align*}
a_y=a\sin\theta&=(\SI{2}{m/s^2})\sin 30\\
a_y&=\SI{1}{m/s^2}
\end{align*} 
Writing out the $y$ component of the Newton's Second Law, we get:
\begin{align*}
\sum F_y &= N\cos\theta -F_g - f_k\sin\theta\\
ma_y&= N\cos\theta -mg - \mu_kN\sin\theta\\
ma_y+mg&=N(\cos\theta - \mu_k\sin\theta)\\
N&=\frac{m(a_y+g)}{\cos\theta-\mu_k\sin\theta}\\
N&=\frac{(\SI{4}{kg})(\SI{1}{m/s^2}+\SI{9.8}{m/s^2})}{\cos 30 - 0.2\sin 30}\\
\therefore N&=\SI{56}{N}
\end{align*}
The normal force is greater when the crate accelerates up the ramp. This makes sense because, if the crate is accelerating up the ramp, the applied force must be larger than when it is moving at a constant speed. Since you are pushing the crate ``into'' to the ramp more, the normal force must be greater in response. 
\end{parts}
\end{solution}


%original
\question An object is released from an airplane that is flying in the horizontal direction with speed $v_A$, as depicted in Figure \ref{fig:applyingnewtonslaws:plane}. The force of drag on the object is given by:
\begin{align*}
\vec F_d = -b\vec v
\end{align*}
where $b$ is a positive coefficient, and $\vec v$ is the velocity of the object. 
\begin{parts}
\item Find expressions for the $x$ and $y$ components of the velocity of the object as a function of time. \textbf{Hint:} You should first show that you can obtain the two following equations:
\begin{align*}
\frac{dv_x}{dt} &=-\frac{b}{m}v_x\\
\frac{dv_y}{dt} &= g-\frac{b}{m}v_y
\end{align*}
by applying Newton's Second Law at some point on the trajectory of the object, when the velocity vector of the object makes some angle $\theta$ with the horizontal. Note that the velocity vector is anti-parallel to the drag force, and use that to write out the two components of Newton's Second Law to obtain the equations above. 
\item Plot the $x$ and $y$ components of the velocity as a function of time. 
\item Show that the terminal velocity of the object is vertical, and determine its magnitude.
\end{parts}
\capfig{0.35\textwidth}{figures/ApplyingNewtonsLaws/plane.png}{\label{fig:applyingnewtonslaws:plane} An object dropped from a moving airplane.}

\begin{solution}
\begin{parts}
\item The force of drag points in the opposite direction of the velocity, as shown in the free-body diagram in Figure \ref{fig:applyingnewtonslaws:plane2_fbd}. The free-body diagram is drawn at some instant in time after the object was released and also shows the velocity vector of the object in order for the drag force to be drawn in the correct direction. 
\capfig{0.2\textwidth}{figures/ApplyingNewtonsLaws/plane2_fbd.png}{\label{fig:applyingnewtonslaws:plane2_fbd} Free-body diagram for an object falling with a force of drag, when the initial velocity of the object was horizontal. The velocity of the object is shown so that the direction of the force of drag can be determined.}
Since the velocity depends on the acceleration, the acceleration depends on the force of drag, and the force of drag depends on the velocity, the acceleration vector will continuously change in both magnitude and direction. At some time, $t$, (as shown in Figure \ref{fig:applyingnewtonslaws:plane2_fbd}), we can write the $x$ and $y$ components of Newton's Second Law to obtain the $x$ and $y$ components of the acceleration vector:
\begin{align*}
\sum F_x &= -F_{dx} = ma_x \\
\therefore a_x &= -\frac{1}{m}F_{dx}\\
\sum F_y &= F_g-F_{dy} = ma_y \\
\therefore a_y &= g-\frac{1}{m}F_{dy}
\end{align*}
We can write the velocity vector as $\vec v(t) = (v_x(t), v_y(t))$, and the force of drag is given by:
\begin{align*}
\vec F_d&=-b\vec v
\end{align*}
so it has $x$ and $y$ components given by:
\begin{align*}
F_{dx}&=-bv_x\\
F_{dy}&=-bv_y
\end{align*}

We can then write the $x$ and $y$ components of the acceleration as:
\begin{align*}
a_x&=\frac{dv_x}{dt}=-\frac{1}{m}F_{dx}=-\frac{b}{m}v_x\\
a_y&=\frac{dv_y}{dt}=g -\frac{1}{m}F_{dy}=g-\frac{b}{m}v_y\\
\end{align*}
which gives us two independent\footnote{The differential equations are ``independent'' because the equation for, say, the $x$ component does not refer to any of the $y$ components. If the drag force were proportional to the speed squared (instead of the speed), the equations would not be independent.} separable differential equations for the $x$ and $y$ components of the speed as a function of time:
\begin{align*}
\frac{dv_x}{dt} &=-\frac{b}{m}v_x\\
\frac{dv_y}{dt} &= g-\frac{b}{m}v_y
\end{align*}
We can solve each equation independently by separation of variables. Starting with the $x$ equation, where the initial $x$ component of the velocity is $v_A$ at time $t=0$ (that of the airplane), and the velocity at some time $t$ is $v_x(t)$:
\begin{align*}
\frac{dv_x}{dt} &=-\frac{b}{m}v_x \\
\int_{v_A}^{v_x(t)}\frac{1}{v_x} dv_x&=-\frac{b}{m}\int_0^t dt\\
\ln\left( \frac{v_x(t)}{v_A} \right)&=-\frac{b}{m}t\\
\therefore v_x(t)&=v_A e^{-\frac{b}{m}t}
\end{align*}
The $y$ equation is the same as one obtains for an object falling vertically in the presence of drag, with an initial $y$ component of the speed equal to zero at time $t=0$ and equal to $v_y(t)$ at some time $t$.
\begin{align*}
\frac{dv_y}{dt} &= g-\frac{b}{m}v_y\\
\int_0^{v_y(t)} \frac{dv_y}{v_y-\frac{mg}{b}}&=-\int_0^t\frac{b}{m}dt\\
\ln\left( v_y-\frac{mg}{b} \right)\Bigr|_0^{v_y(t)}&=-\frac{b}{m}t\\
\ln\left( \frac{v_y(t)-\frac{mg}{b}}{-\frac{mg}{b}} \right)&=-\frac{b}{m}t\\
v_y(t)-\frac{mg}{b} &=-\frac{mg}{b}e^{-\frac{b}{m}t}\\
v_y(t)&=\frac{mg}{b}\left(1-e^{-\frac{b}{m}t}\right)
\end{align*}
\item We can easily plot these in python (for example), choosing values for $m$, $b$, and $v_A$:
\begin{verbatim}
import pylab as pl
import numpy as np

#Define some constants:
b = 1 #drag coefficient
m = 1 #object mass
g = 9.8 #acceleration from gravity
vA = 50 #speed of plane

#Define functions for the x and y components of velocity
def vx(t):
    return vA*np.exp(-b/m*t)

def vy(t):
    return m*g/b*(1-np.exp(-b/m*t))

#Define some values of t
tvals = np.linspace(0,10,1000)

#Calculate the corresponding values for the velocity components
vxvals = vx(tvals)
vyvals = vy(tvals)

#Plot them
pl.figure(figsize=(10,5))
pl.subplot(1,2,1)
pl.plot(tvals,vxvals)
pl.xlabel("time [s]")
pl.ylabel("x component of velocity [m/s]")

pl.subplot(1,2,2)
pl.xlabel("time [s]")
pl.ylabel("y component of velocity [m/s]")
pl.plot(tvals,vyvals)

pl.tight_layout()
pl.show()
\end{verbatim}

\capfig{0.7\textwidth}{figures/ApplyingNewtonsLaws/plane_velocities.png}{\label{fig:applyingnewtonslaws:plane_velocities} $x$ and $y$ components of the velocity as a function of time for an object falling from a plane.}
\item One can easily see from the plots, and mathematically, that:
\begin{align*}
\lim_{t\to\infty}v_x(t)&=0\\
\lim_{t\to\infty}v_y(t)&=\frac{mg}{b}
\end{align*}
The final velocity of the object will have a magnitude of $\frac{mg}{b}$ and be in the $y$ direction.
\end{parts}
\end{solution}

\question Two blocks, with masses $m_a$ and $m_b$, are attached to a mass-less string and move in uniform circular motion on a horizontal friction-less table, as shown in Figure \ref{fig:applyingnewtonslaws:TwoRotatingBlocks}. The blocks describe circles of radius $R_a$ and $R_b$, respectively. Find an expression for the ratio of the tension in the two section of the string in terms of $m_a$, $m_b$, $R_a$ and $R_b$. 
\capfig{0.3\textwidth}{figures/ApplyingNewtonsLaws/TwoRotatingBlocks.png}{\label{fig:applyingnewtonslaws:TwoRotatingBlocks} Two blocks attached to a string, rotating on a friction-less horizontal table, as seen from above.}
\begin{solution}
Both blocks will have an acceleration towards the centre of the circle. The forces on each block are illustrated in the free-body diagrams in Figure \ref{fig:applyingnewtonslaws:TwoRotatingBlocks_FBD}.
\capfig{0.3\textwidth}{figures/ApplyingNewtonsLaws/TwoRotatingBlocks_FBD.png}{\label{fig:applyingnewtonslaws:TwoRotatingBlocks_FBD} Free-body diagrams for the blocks in figure \ref{fig:applyingnewtonslaws:TwoRotatingBlocks}, as seen from the side.}
$T_a$ is the tension in the section of the string between the centre of the circle and block $a$, whereas $T_b$, is  the tension in the section of string between the two blocks. We can choose a coordinate system with the $x$ axis towards the centre of the circle, and the $y$ axis upwards. Newton's Second Law in the $y$ direction, for both block, simply yields that the normal force is equal to the weight. In the $x$ direction, for block $a$, Newton's Second Law is given by:
\begin{align*}
\sum F_x = T_a-T_b&=m_aa_a\\
T_a-T_b&=m_a\omega^2R_a
\end{align*}
where $a_a=\omega^2R_a$ is the centripetal acceleration of block $a$. Similarly, for block $b$, we have:
\begin{align*}
\sum F_x = T_b&=m_aa_b\\
T_b&=m_b\omega^2R_b
\end{align*}
The value of $T_a$ is then found to be:
\begin{align*}
T_a &=m_a\omega^2R_a+T_b=m_a\omega^2R_a+m_b\omega^2R_b=\omega^2(m_aR_a+m_bR_b)
\end{align*}
The ratio of the tensions is thus:
\begin{align*}
\frac{T_a}{T_b}=\frac{m_aR_a+m_bR_b}{m_bR_b}
\end{align*}
\end{solution}

%Based on Giancolli, 5-93
\question A vertical hoop of radius, $R$, rotates with constant angular velocity, $\omega$, about a vertical axis through its centre, as shown in Figure \ref{fig:applyingnewtonslaws:rotatinghoop}. The hoop has a track in it which allows a marble to roll inside the hoop. When the hoop rotates at a particular angular velocity, the marble is in equilibrium and rotates with the hoop at a position that makes an angle $\theta$ from the vertical, as shown.
\begin{parts}
\part Give an expression for the angle $\theta$ in terms of the radius of the hoop and the angular velocity.
\part Explain or show why one can never have $\theta=\SI{90}{\degree}$.
\end{parts}
\capfig{0.15\textwidth}{figures/ApplyingNewtonsLaws/rotatinghoop.png}{\label{fig:applyingnewtonslaws:rotatinghoop} A vertical hoop with a marble in it rotating about a vertical axis.}

\begin{solution}
\begin{parts}
\part The marble executes uniform circular motion about a circle with a radius $r=R\sin\theta$. The only forces exerted on the marble are the normal force from the hoop (towards the centre of the hoop), and its weight. We can write Newton's Second Law in the vertical direction (positive upwards):
\begin{align*}
\sum F_{vert}=N\cos\theta-mg &=0\\
\therefore N\cos\theta&=mg
\end{align*}
We can also write Newton's Second Law in the radial direction (positive towards the axis of rotation):
\begin{align*}
\sum F_{r}=N\sin\theta&=m\omega^2 R\sin\theta\\
\therefore N&=m\omega^2R
\end{align*}
Dividing one equation by the other, we find:
\begin{align*}
\cos\theta=\frac{g}{\omega^2 R}
\end{align*}
\part If the angle were $\SI{90}{\degree}$, the normal force would have no component to offset the force of gravity, and the marble would immediately fall down. Based on the answer in part a), the angular velocity would need to be infinite for the angle to be $\SI{90}{\degree}$, which is also not possible.
\end{parts}

\end{solution}
