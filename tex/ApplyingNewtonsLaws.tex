
\chapter{Applying Newton's Laws}
\label{chap:ApplyingNewtonsLaws}
In this chapter, we take a closer look at how to use Newton's Laws to build models to describe motion. Whereas the previous chapter was focused on identifying the forces that are acting on an object, this chapter focuses on using those forces to describe the motion of the object.

Newton's Laws are meant to describe ``point particles'', that is, objects that can be thought of as a point and thus have no orientation. A block sliding down a hill, a person on a merry-go-round, a bird flying through the air can all be modelled as point particles, as long as we do not need to model their orientation. In all these cases, we can model the forces on the object using a free-body diagram as the location of where the forces are applied on the object to not matter. In a later chapter, we will introduce the tools required to apply Newton's Second Law to objects that can rotate, where we will see that the location of where a force is exerted is important.

\vspace{1cm}
\begin{learningObjectives}
\item Understand the definitions of linear motion and uniform circular motion.
\item Understand how to use inertial forces.
\end{learningObjectives}

%%%%%%%%%%%%%%%%%%%%%%%%%%%%%%%%%%%%
%% Beginning of chapter content
%%%%%%%%%%%%%%%%%%%%%%%%%%%%%%%%%%%%

\section{Linear motion}
We can broadly describe an object whose \textit{velocity does not change continuously} as undergoing linear motion. For example, an object that moves along a straight line in a particular direction, then abruptly changes direction and continues to move in a straight line can be modelled as undergoing linear motion over two different segments (which we would model individually). An object moving around a circle, with its velocity vector continuously changing direction, would not be considered as undergoing linear motion. Example paths of objects undergoing linear and non-linear motion are illustrated in Figure \ref{chap:ApplyingNewtonsLaws}.
\capfig{0.3\textwidth}{figures/ApplyingNewtonsLaws/linearmotion.png}{\label{fig:applyingnewtonslaws:linearmotion} (Left:) Displacement vector for an object undergoing three segments that can each be modelled as linear motion. (Right:) Path of an object whose velocity vector changes continuously and cannot be considered as linear motion.}

When an object undergoes linear motion, we always model the motion of the object over straight segments separately, if the motion is made of multiple straight segments. Over one such segment, the acceleration vector will be co-linear with the displacement vector of the object (parallel or anti-parallel - note that the acceleration can change direction, but will always be co-linear with the displacement).



\subsection{}

\subsection{Inertial forces}

\subsection{Using the drag force}


\section{Uniform circular motion}
As we saw in Chapter \ref{chap:describingmotioninnd}, uniform circular motion is defined to be motion along a circle with constant speed. 

\section{Non-uniform circular motion}










%%%%%%%%%%%%%%%%%%%%%%%%%%%%%%%%%%%%
%% End of chapter content
%%%%%%%%%%%%%%%%%%%%%%%%%%%%%%%%%%%%

\newpage
\section{Summary}
\vspace{1cm}
\begin{chapterSummary}
\item something you learned
\end{chapterSummary}


\section{Thinking about the material}

\subsection{Finding more context}
\begin{enumerate}
\item what
\end{enumerate}

\subsection{Experiments to try at home}

\subsection{Experiment to try in the lab}
\begin{enumerate}
\item try
\end{enumerate}

\subsection{Problems and Solutions}
 