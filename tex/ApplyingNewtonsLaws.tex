
\chapter{Applying Newton's Laws}
\label{chap:ApplyingNewtonsLaws}
In this chapter, we take a closer look at how to use Newton's Laws to build models to describe motion. Newton's Laws are meant to describe ``point particles'', that is, objects that can be thought of as a point and thus have no orientation. A block sliding down a hill, a person on a merry-go-round, a bird flying through the air can all be modelled as point particles, as long as we do not need to model their orientation. 

 \vspace{1cm}
\begin{learningObjectives}
\item Something to learn
\end{learningObjectives}

%%%%%%%%%%%%%%%%%%%%%%%%%%%%%%%%%%%%
%% Beginning of chapter content
%%%%%%%%%%%%%%%%%%%%%%%%%%%%%%%%%%%%


\section{Linear motion}

\subsection{Inertial forces}

\subsection{Using the drag force}




\section{Uniform circular motion}

\section{Non-uniform circular motion}










%%%%%%%%%%%%%%%%%%%%%%%%%%%%%%%%%%%%
%% End of chapter content
%%%%%%%%%%%%%%%%%%%%%%%%%%%%%%%%%%%%

\newpage
\section{Summary}
\vspace{1cm}
\begin{chapterSummary}
\item something you learned
\end{chapterSummary}


\section{Thinking about the material}

\subsection{Finding more context}
\begin{enumerate}
\item what
\end{enumerate}

\subsection{Experiments to try at home}

\subsection{Experiment to try in the lab}
\begin{enumerate}
\item try
\end{enumerate}

\subsection{Problems and Solutions}
 