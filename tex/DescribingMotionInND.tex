
\chapter{Describing motion in multiple dimensions}
\label{chapter:describingmotioninnd}
In this chapter, we will learn how to extend our description of an object's motion to two and three dimensions by using vectors. We will also consider the specific case of an object moving along the circumference of a circle. 

\begin{learningObjectives}{
 \item Describe motion in a 2D plane.
 \item Describe motion in 3D space.
 \item Describe motion along the circumference of a circle.}
\end{learningObjectives}

\begin{opening}
\begin{MCquestion}{Jake and Madi are riding a carousel that spins at a constant rate. Madi is closer to the centre of the carousel than Jake is. What can you say about their accelerations?}
\item Both of their accelerations are zero.
\item Madi's acceleration is greater than Jake's.
\item Jake's acceleration is greater than Madi's. %correct
\item Madi and Jake have the same non-zero acceleration.
\end{MCquestion}
\end{opening}

\section{Motion in two dimensions}

\subsection{Using vectors to describe motion in two dimensions}
We can specify the location of an object with its coordinates, and we can describe any displacement by a vector. First, consider the case of an object moving with a constant velocity in a particular direction.  We can specify the position of the object at any time, $t$, using its \textbf{position vector}, $\vec r(t)$, which is a function of time. The position vector is a vector that goes from the origin of the coordinate system to the position of the object. We can describe the $x$ and $y$ components of the position vector with independent functions, $x(t)$, and $y(t)$, that correspond to the $x$ and $y$ coordinates of the object at time $t$, respectively:
\begin{align*}
\vec r(t) = \begin{pmatrix}
           x(t) \\
           y(t) \\
         \end{pmatrix}= x(t) \hat x + y(t) \hat y
\end{align*}
Suppose that in a period of time $\Delta t$, the object goes from a position described by the position vector $\vec r_1$ to a position described by the position vector $\vec r_2$, as illustrated in Figure \ref{fig:describingmotioninnd:xydrvec}.
\capfig{0.3\textwidth}{figures/DescribingMotionInND/xydrvec.png}{\label{fig:describingmotioninnd:xydrvec}Illustration of a displacement vector, $\Delta \vec r = \vec r_2 -\vec r_1$, for an object that was located at position $\vec r_1$ at time $t_1$ and at position $\vec r_2$ at time $t_2=t_1+\Delta t$.}
We can define a \textbf{displacement vector}, $\Delta \vec r=\vec r_2-\vec r_1$, and by analogy to the one dimensional case, we can define an \textbf{average} velocity vector, $\vec v$ as:
\begin{align}
\vec v = \frac{\Delta \vec r}{\Delta t}
\end{align}
The average velocity vector will have the same direction as $\Delta \vec r$, since it is the displacement vector divided by a scalar ($\Delta t$). The magnitude of the velocity vector, which we call ``speed'', will be proportional to the length of the displacement vector. If the object moves a large distance in a small amount of time, it will thus have a large velocity vector. This definition of the velocity vector thus has the correct intuitive properties (points in the direction of motion, is larger for faster objects).

For example, if the object went from position $(x_1,y_1)$ to position $(x_2,y_2)$ in an amount of time $\Delta t$, the average velocity vector is given by:
\begin{align*}
\vec v &= \frac{\Delta \vec r}{\Delta t}\\
&=\frac{1}{\Delta t}\begin{pmatrix}
           x_2-x_1 \\
           y_2-y_1 \\
         \end{pmatrix}\\
 &=\frac{1}{\Delta t}\begin{pmatrix}
           \Delta x \\
           \Delta y \\
         \end{pmatrix}\\     
\end{align*}
\begin{align*}
 &=\begin{pmatrix}
           \frac{\Delta x}{\Delta t} \\
           \frac{\Delta y}{\Delta t}\\
         \end{pmatrix}\\       
 &=\begin{pmatrix}
           v_x \\
           v_y \\
         \end{pmatrix}\\    
\therefore \vec v &= v_x\hat x+v_y\hat y                     
\end{align*}
That is, the $x$ and $y$ components of the average velocity vector can be found by separately determining the average velocity in each direction. For example, $v_x=\frac{\Delta x}{\Delta t}$ corresponds to the average velocity in the $x$ direction, and can be considered independent from the velocity in the $y$ direction, $v_y$. The magnitude of the average velocity vector (i.e. the average speed), is given by:
\begin{align*}
||\vec v||&=\sqrt{v_x^2+v_y^2}=\frac{1}{\Delta t}\sqrt{\Delta x^2+\Delta y^2}=\frac{\Delta r}{\Delta t}
\end{align*}
where $\Delta r$ is the magnitude of the displacement vector. Thus, the average speed is given by the distance covered divided by the time taken to cover that distance, in analogy to the one dimensional case.

\begin{checkpoint}\label{cp:describingmotioninnd:llama}
\begin{MCquestion}{A llama runs in a field from a position $(x_1,y_1)=(\SI{2}{m},\SI{5}{m})$ to a position $(x_2,y_2)=(\SI{6}{m},\SI{8}{m})$ in a time $\Delta t=\SI{0.5}{s}$, as measured by Marcel, a llama farmer standing at the origin of the Cartesian coordinate system. What is the average speed of the llama?}
\item \SI{1}{m/s}
\item \SI{5}{m/s}
\item \SI{10}{m/s} \correct
\item \SI{15}{m/s}
\end{MCquestion}
\end{checkpoint}

If the velocity of the object is not constant, then we define the \textbf{instantaneous velocity vector} by taking the limit $\Delta t\to 0$:
\begin{align}
\vec v(t) &= \lim_{\Delta t \to 0}\frac{\Delta \vec r}{\Delta t}=\frac{d\vec r}{dt}
\end{align}
which gives us the time derivative of the position vector (in one dimension, it was the time derivative of position). Writing the components of the position vector as functions $x(t)$ and $y(t)$, the instantaneous velocity becomes:
\begin{align}
\label{eqn:describingmotioninnd:vvecdef}
\Aboxed{\vec v(t) &=\frac{d}{dt}\vec r(t) }\\
&=\frac{d}{dt} \begin{pmatrix}
           x(t) \\
           y(t) \\
         \end{pmatrix}\nonumber\\ 
&=\begin{pmatrix}
           \frac{dx}{dt}  \\
          \frac{dy}{dt}  \\
         \end{pmatrix}\nonumber\\ 
 &=\begin{pmatrix}
           v_x(t) \\
           v_y(t) \\
         \end{pmatrix}\nonumber\\   
\therefore \vec v(t) &= v_x(t)\hat x+v_y(t)\hat y  \nonumber     
\end{align}
where, again, we find that the components of the velocity vector are simply the velocities in the $x$ and $y$ direction. This means that we can treat motion in two dimensions as two times one-dimensional motion: a motion along $x$ and a separate motion along $y$. This highlights the usefulness of the vector notation for allowing us to use one vector equation ($\vec v=\frac{d}{dt}\Delta \vec r$) to represent two equations (one for $x$ and one for $y$). 

Similarly the acceleration vector is given by:
\begin{align}
\label{eqn:describingmotioninnd:avecdef}
\Aboxed{\vec a(t) &= \frac{d}{dt}\vec v(t)} \\
&=\begin{pmatrix}
           \frac{dv_x}{dt}  \\
          \frac{dv_y}{dt}  \\
         \end{pmatrix}\nonumber\\
&=\begin{pmatrix}
           a_x(t) \\
           a_y(t) \\
         \end{pmatrix}\nonumber\\
\therefore \vec a(t) &= a_x(t)\hat x+a_y(t)\hat y      \nonumber        
\end{align}

If an object is at position $\vec r_0=(x_0,y_0)$ with a velocity vector $\vec v_0=v_{0x}\hat x + v_{0y}\hat y$ at time $t=0$, and has a \textbf{constant acceleration vector}\footnote{Where a constant vector means that both the magnitude and direction are constant in time.}, $\vec a = a_x\hat x+a_y\hat y$, then the velocity vector at some later time $t$, $\vec v(t)$, is given by:
\begin{align*}
\vec v(t) = \vec v_0 + \vec a t
\end{align*}
Or, if we write out the components explicitly:
\begin{align*}
\begin{pmatrix}
           v_x(t) \\
           v_y(t) \\
         \end{pmatrix} = \begin{pmatrix}
           v_{0x} \\
           v_{0y} \\
         \end{pmatrix} + \begin{pmatrix}
           a_xt \\
           a_yt \\
         \end{pmatrix}
\end{align*}
these be considered as two independent equations for the components of the velocity vector:
\begin{align*}
v_x(t)&=v_{0x}+a_xt \\
v_y(t)&=v_{0y}+a_yt \\
\end{align*}
which is the same equation that we had for one dimensional kinematics, but once for each coordinate. The position vector is given by:
\begin{align*}
\vec r(t) = \vec r_0 + \vec v_0 t + \frac{1}{2} \vec at^2
\end{align*}
with components:
\begin{align*}
x(t) &= x_0+v_{0x}t+\frac{1}{2}a_xt^2\\
y(t) &= y_0+v_{0y}t+\frac{1}{2}a_yt^2\\
\end{align*}
which again shows that two dimensional motion can be considered as separate and independent motions in each direction.

\begin{example}{An object starts at the origin of a coordinate system at time $t=\SI{0}{s}$, with an initial velocity vector $\vec v_0=(\SI{10}{m/s})\hat x+(\SI{15}{m/s})\hat y$. The acceleration in the $x$ direction is \SI{0}{m/s^2} and the acceleration in the $y$ direction is \SI{-10}{m/s^2}.
\begin{enumerate}[label=(\alph*)]
\item Write an equation for the position vector as a function of time.
\item Determine the position of the object at $t=\SI{10}{s}$.
\item Plot the trajectory of the object for the first \SI{5}{s} of motion.
\end{enumerate}
\ }
\label{ex:describingmotioninnd:parabola}
\textbf{a)}We can consider the motion in the $x$ and $y$ direction separately. In the $x$ direction, the acceleration is 0, and the position is thus given by:
\begin{align*}
x(t)&=x_0+v_{0x}t\\
&=(\SI{0}{m})+(\SI{10}{m/s})t\\
&=(\SI{10}{m/s})t
\end{align*}
In the $y$ direction, we have a constant acceleration, so the position is given by:
\begin{align*}
y(t) &= y_0+v_{0y}t+\frac{1}{2}a_yt^2\\
&=(\SI{0}{m})+(\SI{15}{m/s})t+\frac{1}{2}(\SI{-10}{m/s^2})t^2\\
&=(\SI{15}{m/s})t-\frac{1}{2}(\SI{10}{m/s^2})t^2\\
\end{align*}
The position vector as a function of time can thus be written as:
\begin{align*}
\vec r(t) &= \begin{pmatrix}
           x(t) \\
           y(t) \\
          \end{pmatrix}\\
          &= \begin{pmatrix}
           (\SI{10}{m/s})t \\
           (\SI{15}{m/s})t-\frac{1}{2}(\SI{10}{m/s^2})t^2 \\
         \end{pmatrix}
\end{align*}
\textbf{b)} Using $t=\SI{10}{s}$ in the above equation gives:
\begin{align*}
\vec r(t=\SI{10}{s})&= \begin{pmatrix}
           (\SI{10}{m/s})(\SI{10}{s}) \\
           (\SI{15}{m/s})(\SI{10}{s})-\frac{1}{2}(\SI{10}{m/s^2})(\SI{10}{s})^2 \\
         \end{pmatrix}\\
         &= \begin{pmatrix}
           (\SI{100}{m}) \\
           (\SI{-350}{m})\\
         \end{pmatrix}
\end{align*}
\textbf{c)} We can plot the trajectory using python, as in Figure \ref{fig:describingmotioninnd:parabola}.
\capfig{0.5\textwidth}{figures/DescribingMotionInND/parabola.png}{\label{fig:describingmotioninnd:parabola}Parabolic trajectory of an object with no acceleration in the $x$ direction and a negative acceleration in the $y$ direction.}

As you can see, the trajectory is a parabola, and corresponds to what you would get when throwing an object with an initial velocity with upwards (positive $y$) and horizontal (positive $x$) components. If you look at only the $y$ axis, you will see that the object first goes up, then turns around and goes back down. This is exactly what happens when you throw a ball upwards, independently of whether the object is moving in the $x$ direction. In the $x$ direction, the object just moves with a constant velocity. The points on the graph are drawn for constant time intervals (the time between each point, $\Delta t$ is constant). If you look at the distance between points projected onto the $x$ axis, you will see that they are all equidistant and that along $x$, the motion corresponds to that of an object with constant velocity. 
\end{example}
\newpage
\begin{checkpoint}{\begin{MCquestion}{In example \ref{ex:describingmotioninnd:parabola}, what is the velocity vector exactly at the top of the parabola in Figure \ref{fig:describingmotioninnd:parabola}?}
\item $\vec v=(\SI{10}{m/s})\hat x+(\SI{15}{m/s})\hat y$
\item $\vec v=(\SI{15}{m/s})\hat y$
\item $\vec v=(\SI{10}{m/s})\hat x$ \correct
\item None of the above.
\end{MCquestion}}
\end{checkpoint}
%\vspace{-1cm}
\begin{example}{
A monkey is hanging from a tree branch and you want to feed the monkey by throwing it a banana (Figure \ref{fig:describingmotioninnd:monkeytreeexample}). You know that the monkey is easily frightened and will let go of the tree branch the instant you throw the banana. The monkey is a horizontal distance $d$ away and a height $h$ above the point from which you release the banana when you throw it. At what angle with respect to the horizontal should you throw the banana so that the banana reaches the monkey? 
\capfig{0.55\textwidth}{figures/DescribingMotionInND/MonkeyTreeExample.png}{\label{fig:describingmotioninnd:monkeytreeexample} Feeding a monkey in a tree.}}
This question is asking us to find the angle, $\theta$, between the banana's initial velocity vector, $\vec v_{0B}$, and the horizontal for the banana to hit the monkey. This angle is given by the horizontal ($v_{B0x}$) and vertical ($v_{B0y}$) components of the initial velocity vector of the banana:
\begin{align*}
\tan\theta&=\frac{v_{B0y}}{v_{B0x}}
\end{align*}
In order for the banana to hit the monkey, the banana and the monkey must be \textbf{in the same place at the same time} at some time, $t$. Our approach will be as follows: we will start by finding equations that describe the $x$ and $y$ position of the monkey and of the banana. Then, we will use our conditions for a successful ``hit'' to find the ratio ($\tan\theta=v_{B0y}/v_{B0x}$) that we want for our initial throw, and use that to find $\theta$.

First, we define a coordinate system. We choose the origin to be where the banana is released. We let $y$ be in the vertical direction (positive upwards) and let $x$ be in the horizontal direction (positive towards the monkey), as shown in Figure \ref{fig:describingmotioninnd:monkeytreeexample}. 

We treat the $x$ and $y$ components of the banana and monkey's velocity and position vectors as independent. The monkey's motion has only a vertical component. The $y$ component of the monkey's acceleration is the acceleration due to gravity, $a_y=-\SI{9.8}{m/s^2}= -g$, which is negative, since gravity produces an acceleration in the negative $y$ direction. The $y$ component of the monkey's initial position is $y_{M0}=h$ and the $y$ component of its initial velocity is $v_{M0y}=0$. The $y$ component of the monkey's position as a function of time, $y_M(t)$, is given by:
\begin{align*}
y_M(t)&=y_{M0}+v_{My0}t+\frac{1}{2}a_yt^2\\
&=h+(0)-\frac{1}{2}gt^2
\end{align*}
The horizontal position of the monkey is constant, and is equal to $x_M(t)=d$.

The banana's motion has both $x$ and $y$ components. There is no acceleration in the $x$ direction, so the $x$ component of the banana's velocity is $v_{B0x}$ and constant. We defined the banana's initial $x$ coordinate to be $x_{B0}=0$, so the $x$ position of the banana as a function of time, $x_B(t)$ is given by:
\begin{align*}
x_B(t)&=x_{B0}+v_{B0x}\\
&=(0)+v_{B0x}t
\end{align*}
We defined the initial $y$ position of the banana to be $y_{B0}=0$. The $y$ position of the banana as a function of time, $y_B(t)$, can thus be described by:
\begin{align*}
y_B(t)&=y_{B0}+v_{B0y}t+\frac{1}{2}a_yt^2\\
&=(0)+v_{B0y}t-\frac{1}{2}gt^2
\end{align*}
where $v_{B0y}$ is the $y$ component of the banana's initial velocity and $a_y=-g$ is the $y$ component of the banana's acceleration (due to gravity). Now that we have equations that describe the position of both the banana and the monkey, we can use our conditions for the banana and monkey to be at the same position at the same time. For the monkey and the banana to be in the same position, we need $y_M(t)=y_B(t)$ and $x_B(t)=x_M(t) =d$ at some time $t$.

Setting our equations for $y_M(t)$ and $y_B(t)$ equal to one another gives:
\begin{align*}
h-\frac{1}{2} gt^2&=v_{0yB}t-\frac{1}{2}gt^2\\
\therefore h&=v_{0yB}t
\end{align*}
And setting $x_M(t)=d$ equal to $x_B(t)$ gives:
\begin{align*}
\therefore d&=v_{xB}
\end{align*} 
We can just divide one equation by the other to find:
\begin{align*}
\frac{h}{d}&=\frac{v_{0yB}t}{v_{xB}t}\\
\frac{h}{d}&=\frac{v_{0yB}}{v_{xB}}
\end{align*}
This gives us the ratio we are looking for, so we now know that
\begin{align*}
\tan\theta&=\frac{h}{d}\\
\therefore \theta&=\tan^{-1}\left(\frac{h}{d}\right)
\end{align*}
This is a somewhat surprising result, as it means that you only need to thrown the banana in the direction of the monkey (that is, aim at the monkey, and throw!). Thus, it will not matter how fast you throw the banana, and you will always hit the monkey if you aimed correctly. When you throw the banana faster, you will hit the monkey higher in its trajectory. If there is no ground for the monkey to hit, you can throw the banana as slowly as you like, and it will eventually catch up with the monkey when the banana reaches $x=d$.
\end{example} 

\subsection{Relative motion}
\label{sec:desribingmotioninnd:relativemotion}
In the previous chapter, we examined how to convert the description of motion from one reference frame to another. Recall the one dimensional situation where we described the position of an object, $A$, using an axis $x$ as $x^A(t)$. Suppose that the reference frame, $x$, is moving with a constant speed, $v'^B$, relative to a second reference frame, $x'$. We found that the position of the object is described in the $x'$ reference frame as:
\begin{align*}
x'^A(t)=v'^Bt+x^A(t)
\end{align*}
if the origins of the two systems coincided at $t=0$. The equation above simply states that the distance of the object to the $x'$ origin is the sum of the distance from the $x'$ origin to the $x$ origin \textbf{and} the distance from the $x$ origin to the object.

In two dimensions, we proceed in exactly the same way, but use vectors instead:
\begin{align*}
\pvec r'^A(t) = \pvec v'^Bt+\vec r^A(t)
\end{align*}
where $r^A(t)$ is the position of the object as described in the $xy$ reference frame, $\pvec v'^B$, is the velocity vector describing the motion of the origin of the $xy$ coordinate system relative to an $x'y'$ coordinate system and $\pvec r'^A(t)$ is the position of the object in the $x'y'$ coordinate system. We have assumed that the origins of the two coordinate systems coincided at $t=0$ and that the axes of the coordinate systems are parallel ($x$ parallel to $x'$ and $y$ parallel to $y'$).

Note that the velocity of the object in the $x'y'$ system is found by adding the velocity of $xy$ relative to $x'y'$ and the velocity of the object in the $xy$ frame ($\vec v^A(t)$):
\begin{align*}
\frac{d}{dt}\pvec r'^A(t) &=\frac{d}{dt}(\pvec v'^Bt+\vec r^A(t))\\
&=\pvec v'^B+\vec v^A(t)
\end{align*}

As an example, consider the situation depicted in Figure \ref{fig:describingmotioninnd:2drel}. Brice is on a boat off the shore of Nice, with a coordinate system $xy$, and is describing the position of a boat carrying Alice. He describes Alice's position as $\vec r^A(t)$ in the $xy$ coordinate system. Igor is on the shore and also wishes to describe Alice's position using the work done by Brice. Igor sees Brice's boat move with a velocity $\pvec v'^B$ as measured in his $x'y'$ coordinate system. In order to find the vector pointing to Alice's position $\pvec r'^A(t)$, he adds the vector from his origin to Brice's origin ($\pvec v'^B t$) and the vector from Brice's origin to Alice $\vec r^A(t)$.

\capfig{0.7\textwidth}{figures/DescribingMotionInND/2drel.png}{\label{fig:describingmotioninnd:2drel} Example of converting from one reference frame to another in two dimensions using vector addition.}

Writing this out by coordinate, we have:
\begin{align*}
x'^A(t)&=v'^B_xt+x^A(t)\\
y'^A(t)&=v'^B_yt+y^A(t)
\end{align*}
and for the velocities:
\begin{align*}
v_x'^A(t)&=v'^B_x+v_x^A(t)\\
v_y'^A(t)&=v'^B_y+v_y^A(t)
\end{align*}


\begin{checkpoint}{\begin{MCquestion}{You are on a boat and crossing a North-flowing river, from the East bank to the West bank. You point your boat in the West direction and cross the river. \chloe is watching your boat cross the river from the shore, in which direction does she measure your velocity vector to be?}
\item In the North direction.
\item In the West direction.
\item A combination of North and West directions.\correct
\end{MCquestion}}
\end{checkpoint}


\section{Motion in three dimensions}
The big challenge was to expand our description of motion from one dimension to two. Adding a third dimension ends up being trivial now that we know how to use vectors. In three dimensions, we describe the position of a point using three coordinates, so all of the vectors simply have three independent components, but are treated in exactly the same way as in the two dimensional case. The position of an object is now described by three independent functions, $x(t)$, $y(t)$, $z(t)$, that make up the three components of a position vector $\vec r(t)$:
\begin{align*}
\vec r(t) &= \begin{pmatrix}
           x(t) \\
           y(t) \\
           z(t)  \\
         \end{pmatrix}\\
\therefore \vec r(t)  &= x(t) \hat x + y(t) \hat y + z(t) \hat z
\end{align*}
The velocity vector now has three components and is defined analogously to the 2D case:
\begin{align*}
\vec v(t) &=\frac{d\vec r}{dt}
 =\begin{pmatrix}
           \frac{dx}{dt}  \\
          \frac{dy}{dt}  \\
          \frac{dz}{dt}  \\
         \end{pmatrix}
 =\begin{pmatrix}
           v_x(t) \\
           v_y(t) \\
           v_z(t) \\
         \end{pmatrix}\\   
\therefore \vec v(t) &= v_x(t)\hat x+v_y(t)\hat y+v_z(t)\hat z  \nonumber 
\end{align*}
and the acceleration is defined in a similar way:
\begin{align*}
\vec a(t)  &=\frac{d\vec v}{dt}
 =\begin{pmatrix}
           \frac{dv_x}{dt}  \\
          \frac{dv_y}{dt}  \\
          \frac{dv_z}{dt}  \\
         \end{pmatrix}
 =\begin{pmatrix}
           a_x(t) \\
           a_y(t) \\
           a_z(t) \\
         \end{pmatrix}\\   
\therefore \vec a(t) &= a_x(t)\hat x+a_y(t)\hat y+a_z(t)\hat z  \nonumber 
\end{align*}

In particular, if an object has a constant acceleration, $\vec a=a_x\hat x+a_y\hat y+a_z\hat z$, and started at $t=0$ with a position $\vec r_0$ and velocity $\vec v_0$, then its velocity vector is given by:
\begin{align*}
\vec v(t)  &= \vec v_0+\vec at=\begin{pmatrix}
           v_{0x}+ a_xt \\
           v_{0y}+ a_yt \\
           v_{0z}+ a_zt \\
         \end{pmatrix}\\
\end{align*}
and the position vector is given by:
\begin{align*}
\vec r(t)= \vec r_0+\vec v_0 t+\frac{1}{2}\vec a t^2=\begin{pmatrix}
           x_0+v_{0x}t+\frac{1}{2} a_xt^2 \\
           y_0+v_{0y}t+\frac{1}{2} a_yt^2 \\
           z_0+v_{0z}t+\frac{1}{2} a_zt^2 \\
         \end{pmatrix}\\
\end{align*}
where again, we see how writing a single vector equation (e.g. $\vec v(t) = \vec v_0+\vec at$) is really just a way to write the three independent equations that are true for each component.

\section{Accelerated motion when the velocity vector changes direction}
\label{sec:describingmotioninnd:accvconst}
One key difference with one dimensional motion is that, in two dimensions, it is possible to have an acceleration even when the speed is constant. Recall, the acceleration \textbf{vector} is defined as the time derivative of the velocity \textbf{vector} (equation \ref{eqn:describingmotioninnd:avecdef}). This means that if the velocity vector changes with time, then the acceleration vector is non-zero. If the length of the velocity vector (speed) is constant, it is still possible that the \textbf{direction} of the velocity vector changes with time, and thus, that the acceleration vector is non-zero. This is, for example, what happens when an object goes around in a circle with a constant speed (the direction of the velocity vector changes). 
\capfig{0.35\textwidth}{figures/DescribingMotionInND/deltav.png}{\label{fig:describingmotioninnd:deltav} Illustration of how the direction of the velocity vector can change when speed is constant.}

Figure \ref{fig:describingmotioninnd:deltav} shows an illustration of a velocity vector, $\vec v(t)$, at two different times, $\vec v_1$ and $\vec v_2$, as well as the vector difference, $\Delta \vec v=\vec v_2 - \vec v_1$, between the two. In this case, the length of the velocity vector did not change with time ($||\vec v_1||=||\vec v_2||$). The acceleration vector is given by:
\begin{align*}
\vec a = \lim_{\Delta t\to 0}\frac{\Delta \vec v}{\Delta t}
\end{align*}
and will have a direction parallel to $\Delta \vec v$, and a magnitude that is proportional to $\Delta v$. Thus, even if the velocity vector does not change amplitude (speed is constant), the acceleration vector can be non-zero if the velocity vector changes \textit{direction}.

Let us write the velocity vector, $\vec v$, in terms of its magnitude, $v$, and a unit vector, $\hat v$, in the direction of $\vec v$:
\begin{align*}
\vec v &=v_x\hat x+v_y\hat y= v \hat v\\
v&=||\vec v||=\sqrt{v_x^2+v_y^2}\\
\hat v &= \frac{v_x}{v}\hat x+\frac{v_y}{v}\hat y\\
\end{align*}
In the most general case, both the magnitude of the velocity and its direction can change with time. That is, both the direction and the magnitude of the velocity vector are functions of time:
\begin{align*}
\vec v(t)&=v(t)\hat v(t)
\end{align*}
When we take the time derivative of $\vec v(t)$ to obtain the acceleration vector, we need to take the derivative of a product of two functions of time, $v(t)$ and $\hat v(t)$. Using the rules for taking the derivative of a product, the acceleration vector is given by:
\begin{align}
\label{eqn:describingmotioninnd:avecdef2}
\vec a &= \frac{d}{dt}\vec v(t)= \frac{d}{dt}v(t)\hat v(t)\nonumber\\
\Aboxed{\vec a&=\frac{dv}{dt}\hat v(t)+v(t)\frac{d\hat v}{dt}}
\end{align}
and has two terms. The first term, $\frac{dv}{dt}\hat v(t)$, is zero if the speed is constant ($\frac{dv}{dt}=0$). The second term, $v(t)\frac{d\hat v}{dt}$, is zero if the direction of the velocity vector is constant ($\frac{d\hat v}{dt}=0$). In general though, the acceleration vector has two terms corresponding to the change in speed, and to the change in the direction of the velocity, respectively.

The specific functional form of the acceleration vector will depend on the path being taken by the object. If we consider the case where speed is constant, then we have:
\begin{align*}
v(t) &= v \\
\frac{dv}{dt}&=0\\
v_x^2(t)+v_y^2(t) &=v^2 \\
\therefore v_y(t)&=\sqrt{v^2-v_x(t)^2}
\end{align*}
In other words, if the magnitude of the velocity is constant, then the $x$ and $y$ components are no longer independent (if the $x$ component gets larger, then the $y$ component must get smaller so that the total magnitude remains unchanged). If the speed is constant, then the acceleration vector is given by:
\begin{align}
\label{eqn:describingmotioninnd:vecaconstv}
\vec a&=\frac{dv}{dt}\hat v(t)+v\frac{d\hat v}{dt}\nonumber\\
&=0 + v\frac{d}{dt}\hat v(t)\nonumber\\
&=v\frac{d}{dt}\left(\frac{v_x(t)}{v}\hat x+\frac{v_y(t)}{v}\hat y  \right)\nonumber\\
&=\frac{dv_x}{dt}\hat x + \frac{d}{dt}\sqrt{v^2-v_x(t)^2}\hat y\nonumber\\
&=\frac{dv_x}{dt}\hat x + \frac{1}{2\sqrt{v^2-v_x(t)^2}}(-2v_x(t))\frac{dv_x}{dt}\hat y\nonumber\\
&=\frac{dv_x}{dt}\hat x - \frac{v_x(t)}{\sqrt{v^2-v_x(t)^2}}\frac{dv_x}{dt}\hat y\nonumber\\
&=\frac{dv_x}{dt}\hat x - \frac{v_x(t)}{v_y(t)}\frac{dv_x}{dt}\hat y\nonumber\\
\therefore\quad\Aboxed{\vec a&=\frac{dv_x}{dt} \left(\hat x - \frac{v_x(t)}{v_y(t)}\hat y\right)}
\end{align}
where most of the algebra that we did was to separate the $x$ and $y$ components of the acceleration vector, and we used the Chain Rule to take the derivative of the square root. The resulting acceleration vector is illustrated in Figure \ref{fig:describingmotioninnd:aperpv} along with the velocity vector\footnote{Rather, it is a vector parallel to the acceleration vector that is illustrated, as the factor of $\frac{dv_x}{dt}$ was omitted (as you recall, multiplying by a scalar only changes the length, not the direction)}.
\capfig{0.35\textwidth}{figures/DescribingMotionInND/aperpv.png}{\label{fig:describingmotioninnd:aperpv} Illustration that the acceleration vector is perpendicular to the velocity vector if speed is constant.}
The velocity vector has components $v_x$ and $v_y$, which allows us to calculate the angle, $\theta$ that it makes with the $x$ axis:
\begin{align*}
\tan(\theta)=\frac{v_y}{v_x}
\end{align*}
Similarly, the vector that is parallel to the acceleration has components of $1$ and $-\frac{v_x}{v_y}$, allowing us to determine the angle, $\phi$, that it makes with the $x$ axis:
\begin{align*}
\tan(\phi)=\frac{v_x}{v_y}
\end{align*}
Note that $\tan(\theta)$ is the inverse of $\tan(\phi)$, or in other words, $\tan(\theta)=\cot(\phi)$, meaning that $\theta$ and $\phi$ are complementary and thus must sum to $\frac{\pi}{2}$ (\SI{90}{\degree}). This means that \textbf{the acceleration vector is perpendicular to the velocity vector if the speed is constant and the direction of the velocity changes}. 

In other words, when we write the acceleration vector, we can identify two components, $\vec a_{\parallel}(t)$ and $\vec a_{\perp}(t)$:
\begin{align*}
\vec a&=\frac{dv}{dt}\hat v(t)+v(t)\frac{d\hat v}{dt}\\
&=\vec a_{\parallel}(t) + \vec a_{\perp}(t)\\
\therefore \vec a_{\parallel}(t)&=\frac{dv}{dt}\hat v(t)\\
\therefore \vec a_{\perp}(t)&=v\frac{d\hat v}{dt}=\frac{dv_x}{dt} \left(\hat x - \frac{v_x(t)}{v_y(t)}\hat y\right)
\end{align*}
where $\vec a_{\parallel}(t)$ is the component of the acceleration that is parallel to the velocity vector, and is responsible for changing its magnitude, and $\vec a_{\perp}(t)$, is the component that is perpendicular to the velocity vector and is responsible for changing the direction of the motion.

\begin{checkpoint}{\begin{MCquestion}{A satellite moves in a circular orbit around the Earth with a constant speed. What can you say about its acceleration vector?}
\item It has a magnitude of zero.
\item It is perpendicular to the velocity vector. \correct
\item It is parallel to the velocity vector.
\item It is in a direction other than parallel or perpendicular to the velocity vector.
\end{MCquestion}}
\end{checkpoint}

\section{Circular motion}
\label{sec:describingmotioninnd:circularmotion}
We often consider the motion of an object around a circle of fixed radius, $R$. In principle, this is motion in two dimensions, as a circle is necessarily in a two dimensional plane. However, since the object is constrained to move along the circumference of the circle, it can be thought of (and treated as) motion along a one dimensional axis that is curved. 
\capfig{0.35\textwidth}{figures/DescribingMotionInND/circle.png}{\label{fig:describingmotioninnd:circle} Describing the motion of an object around a circle of radius $R$.}

Figure \ref{fig:describingmotioninnd:circle} shows how we can describe motion along a circle of radius, $R$. We could use $x(t)$ and $y(t)$ to describe the position on the circle, however, $x(t)$ and $y(t)$ are no longer independent since they have to correspond to the coordinates of points on a circle:
\begin{align*}
x^2(t)+y^2(t)=R^2
\end{align*}
Instead of using $x$ and $y$, we could think of an axis that is bent around the circle (as shown by the curved arrow in Figure \ref{fig:describingmotioninnd:circle}, the $s$ axis). The $s$ axis is such that $s=0$ where the circle intersects the $x$ axis, and the value of $s$ increases as we move counter-clockwise along the circle. Distance along the $s$ axis thus corresponds to the distance along the circumference of the circle.

Another variable that could be used for position instead of $s$ is the angle, $\theta$, between the position vector of the object and the $x$ axis, as illustrated in Figure \ref{fig:describingmotioninnd:circle}. If we express the angle $\theta$ in radians, then it is easy to convert between $s$ and $\theta$. Recall, an angle in radians is defined as the length of an arc subtended by that angle divided by the radius of the circle. We thus have:
\begin{align}
\label{eqn:describingmotioninnd:raddef}
\Aboxed{\theta(t)=\frac{s(t)}{R}}
\end{align}
In particular, if the object has gone around the whole circle, then $s=2\pi R$ (the circumference of a circle), and the corresponding angle is, $\theta=\frac{2\pi R}{R}=2\pi$, namely \SI{360}{\degree}. 

By using the angle, $\theta$, instead of $x$ and $y$, we are effectively using polar coordinates, with a fixed radius. As we already saw, the $x$ and $y$ positions are related to $\theta$ by:
\begin{align*}
x(t) &= R\cos(\theta(t))\\
y(t) &= R\sin(\theta(t))\\
\end{align*}
where $R$ is a constant. For an object moving along the circle, we can write its position vector, $\vec r(t)$, as:
\begin{align*}
\vec r(t)&= \begin{pmatrix}
           x(t) \\
           y(t) \\
         \end{pmatrix}
         =R \begin{pmatrix}
           \cos(\theta(t)) \\
           \sin(\theta(t)) \\
         \end{pmatrix}
\end{align*}
and the velocity vector is thus given by:
\begin{align*}
\vec v(t) &=\frac{d}{dt}\vec r(t) 
=\frac{d}{dt} R \begin{pmatrix}
           \cos(\theta(t)) \\
           \sin(\theta(t)) \\
         \end{pmatrix} \\
&= R \begin{pmatrix}
           \frac{d}{dt}\cos(\theta(t)) \\
           \frac{d}{dt}\sin(\theta(t)) \\
         \end{pmatrix} \\
 &= R \begin{pmatrix}
           -\sin(\theta(t))\frac{d\theta}{dt} \\
           \cos(\theta(t))\frac{d\theta}{dt} \\
         \end{pmatrix}     \\  
\end{align*}         
where we used the Chain Rule to calculate the time derivatives of the trigonometric functions (since $\theta(t)$ is function of time). We can write this in component form:
\begin{align}
\label{eqn:describingmotioninnd:vcircle}
v_x &= -R\sin(\theta(t))\frac{d\theta}{dt}\nonumber\\
v_y &= R\cos(\theta(t))\frac{d\theta}{dt}
\end{align}
The magnitude of the velocity vector is given by:
\begin{align*}
||\vec v|| &=\sqrt{ v_x^2+v_y^2}\\
&=\sqrt{ \left(-R\sin(\theta(t))\frac{d\theta}{dt}\right)^2+\left(R\cos(\theta(t))\frac{d\theta}{dt}\right)^2}\\
&=\sqrt{ R^2\left( \frac{d\theta}{dt}\right)^2[\sin^2(\theta(t))+\cos^2(\theta(t)]}\\
&=R\left |\frac{d\theta}{dt}\right|
\end{align*}
The position and velocity vectors are illustrated in Figure \ref{fig:describingmotioninnd:vcircle} for an angle $\theta$ in the first quadrant ($0<\theta<\frac{\pi}{2}$).
\capfig{0.35\textwidth}{figures/DescribingMotionInND/vcircle.png}{\label{fig:describingmotioninnd:vcircle} The position vector, $\vec r(t)$ is always perpendicular to the velocity vector, $\vec v(t)$, for motion on a circle.}
In this case, you can note that the $x$ component of the velocity is negative (from the diagram and from Equation \ref{eqn:describingmotioninnd:vcircle}). From Equation \ref{eqn:describingmotioninnd:vcircle}, you can also see that $\frac{|v_x|}{|v_y|}=\tan(\theta)$, which is illustrated in Figure \ref{fig:describingmotioninnd:vcircle}, showing that \textbf{the velocity vector is tangent to the circle} and perpendicular to the position vector. This is always the case for motion along a circle.

We can simplify our description of motion along the circle by using either $s(t)$ or $\theta(t)$ instead of the vectors for position and velocity. If we use $s(t)$ to represent position along the circumference ($s=0$ where the circle intersects the $x$ axis), then the velocity along the $s$ axis is:
\begin{align*}
v_s(t)&=\frac{d}{dt}s(t)\\
&=\frac{d}{dt}R\theta(t)\\
&=R\frac{d\theta}{dt}
\end{align*}
where we used the fact that $\theta=s/R$ to convert from $s$ to $\theta$. The velocity along the $s$ axis is thus precisely equal to the magnitude of the two-dimensional velocity vector (derived above), which makes sense since the velocity vector is tangent to the circle (and thus in the $s$ ``direction'').

If the object has a \textbf{constant speed}, $v_s$, along the circle and started at a position along the circumference $s=s_0$, then its position along the $s$ axis can be described using 1D kinematics:
\begin{align*}
s(t)=s_0+v_st
\end{align*}
or, in terms of $\theta$:
\begin{align*}
\theta(t)&=\frac{s(t)}{R}=\frac{s_0}{R}+\frac{v_s}{R}t\\
&=\theta_0 + \frac{d\theta}{dt}t\\
&=\theta_0 + \omega t\\
\Aboxed{\therefore \omega &= \frac{d\theta}{dt}}
\end{align*}
where we introduced $\theta_0$ as the angle corresponding to the position $s_0$, and we introduced $\omega=\frac{d\theta}{dt}$, which is analogous to velocity, but for an angle. $\omega$ is called the \textbf{angular velocity} and is a measure of the rate of change of the angle $\theta$ (as it is the time derivative of the angle). The relation between the ``linear'' velocity $v_s$ (the magnitude of the velocity vector, which corresponds to the velocity in the direction tangent to the circle) and $\omega$ is:
\begin{align*}
\Aboxed{v_s=R\frac{d\theta}{dt}=R\omega }
\end{align*}

Similarly, if the object is accelerating, we can define an \textbf{angular acceleration}, $\alpha(t)$, as the rate of change of the angular velocity:
\begin{align*}
\alpha(t)=\frac{d\omega}{dt}
\end{align*}
which can directly be related to the acceleration in the $s$ direction, $a_s(t)$:
\begin{align*}
a_s(t) &= \frac{d}{dt}v_s\\
&=\frac{d}{dt}\omega R=R\frac{d\omega}{dt}\\
\Aboxed{a_s(t)&=R\alpha }
\end{align*}
Thus, the linear quantities (those along the $s$ axis) can be related to the angular quantities by multiplying the angular quantities by $R$:
\begin{align}
s&=R\theta\\
v_s&=R\omega\\
a_s&=R\alpha
\end{align}
If the object started at $t=0$ with a position $s=s_0$ ($\theta=\theta_0$), and an initial linear velocity $v_{0s}$ (angular velocity $\omega_0$), and has a \textbf{constant linear acceleration} around the circle, $a_s$ (angular acceleration, $\alpha$), then the position of the object can be described using either the linear or the angular quantities:
\begin{align*}
s(t) &= s_0+v_{s0}t+\frac{1}{2}a_s t^2\\
\theta(t) &= \theta_0+\omega_0t+\frac{1}{2}\alpha t^2
\end{align*}

As you recall from section \ref{sec:describingmotioninnd:accvconst}, we can compute the acceleration \textbf{vector} and identify components that are parallel and perpendicular to the velocity vector:
\begin{align*}
\vec a&=\vec a_{\parallel}(t) + \vec a_{\bot}(t)\\
&=\frac{dv}{dt}\hat v(t)+v\frac{d\hat v}{dt}\\
\end{align*}
The first term, $\vec a_{\parallel}(t)=\frac{dv}{dt}\hat v(t)$, is parallel to the velocity vector $\hat v$, and has a magnitude given by:
\begin{align*}
||\vec a_{\parallel}(t)||&=\frac{dv}{dt}=\ddt v(t)=\ddt R\omega=R\alpha
\end{align*}
That is, the component of the acceleration vector that is parallel to the velocity is precisely the acceleration in the $s$ direction (the linear acceleration). This component of the acceleration is responsible for increasing (or decreasing) the speed of the object and is zero if the object goes around the circle with a constant speed (linear or angular). 

As we saw earlier, the perpendicular component of the acceleration, $\vec a_{\bot}(t)$, is responsible for changing the direction of the velocity vector (as the object continuously changes direction when going in a circle). When the motion is around a circle, this component of the acceleration vector is called ``centripetal'' acceleration (i.e. acceleration pointing towards the centre of the circle, as we will see). We can calculate the centripetal acceleration in terms of our angular variables, noting that the unit vector in the direction of the velocity is $\hat v=-\sin(\theta)\hat x+\cos(\theta)\hat y$:
\begin{align}
\vec a_{\bot}(t)&=v\frac{d\hat v}{dt}\nonumber\\
&=(\omega R)\ddt \left[-\sin(\theta)\hat x+\cos(\theta)\hat y\right]\nonumber\\
&=\omega R \left[-\ddt\sin(\theta)\hat x+\ddt\cos(\theta)\hat y\right]\nonumber\\
&=\omega R \left[-\cos(\theta)\frac{d\theta}{dt}\hat x-\sin(\theta)\frac{d\theta}{dt}\hat y\right]\nonumber\\
&=\omega R [-\cos(\theta)\omega\hat x-\sin(\theta)\omega\hat y]\nonumber\\
\Aboxed{\vec a_{\bot}(t)&=\omega^2 R[-\cos(\theta)\hat x-\sin(\theta)\hat y]}
\end{align}
where you can easily verify that the vector $[-\cos(\theta)\hat x-\sin(\theta)\hat y]$ has unit length and points towards the centre of the circle (when the tail is placed on a point on the circle at angle $\theta$). The centripetal acceleration thus points towards the centre of the circle and has magnitude:
\begin{align}
a_c(t) = ||\vec a_{\bot}(t)||=\omega^2(t) R = \frac{v^2(t)}{R}
\end{align}
where in the last equal sign, we wrote the centripetal acceleration in terms of the speed around the circle ($v=||\vec v||=v_s$).

If an object goes around a circle, it will always have a centripetal acceleration (since its velocity vector must change direction). In addition, if the object's speed is changing, it will also have a linear acceleration, which points in the same direction as the velocity vector (it changes the velocity vector's length but not its direction).

\begin{checkpoint}{\begin{MCquestion}{A vicu\~na is going clockwise around a circle that is centred at the origin of an $xy$ coordinate system that is in the plane of the circle. The vicu\~na runs faster and faster around the circle. In which direction does its acceleration vector point just as the vicu\~na is at the point where the circle intersects the positive $y$ axis?}
\item In the negative $y$ direction.
\item In the positive $y$ direction.
\item A combination of the positive $y$ and positive $x$ directions.
\item A combination of the negative $y$ and positive $x$ directions. \correct
\item A combination of the negative $y$ and negative $x$ directions.
\end{MCquestion}}
\end{checkpoint}

\subsection{Period and frequency}
When an object is moving around in a circle, it will typically complete more than one revolution. If the object is going around the circle with a constant speed, we call the motion ``uniform circular motion'', and we can define the \textbf{period and frequency} of the motion. 

The period, $T$, is defined to be the time that it takes to complete one revolution around the circle. If the object has constant angular speed $\omega$, we can find the time, $T$, that it takes to complete one full revolution, from $\theta=0$ to $\theta=2\pi$:
\begin{align}
\omega&=\frac{\Delta \theta}{T}=\frac{2\pi}{T}\nonumber\\
\Aboxed{\therefore T&=\frac{2\pi}{\omega}}
\end{align}
We would obtain the same result using the linear quantities; in one revolution, the object covers a distance of $2\pi R$ at a speed of $v$:
\begin{align*}
v&=\frac{2\pi R}{T}\\
T&=\frac{2\pi R}{v}=\frac{2\pi R}{\omega R}=\frac{2\pi}{\omega}
\end{align*}

The frequency, $f$, is defined to be the inverse of the period:
\begin{align*}
f&=\frac{1}{T}=\frac{\omega}{2\pi}
\end{align*}
and has SI units of $\si{Hz}=\si{s^{-1}}$. Think of frequency as the number of revolutions completed per second. Thus, if the frequency is $f=\SI{1}{Hz}$, the object goes around the circle once per second. Given the frequency, we can of course obtain the angular velocity:
\begin{align*}
\omega = 2\pi f
\end{align*}
which is sometimes called the ``angular frequency'' instead of the angular velocity. The angular velocity can really be thought of as a frequency, as it represents the ``amount of angle'' per second that an object covers when going around a circle. The angular velocity does not tell us anything about the actual speed of the object, which depends on the radius $v=\omega R$. This is illustrated in Figure \ref{fig:describingmotioninnd:twocircles}, where two objects can be travelling around two circles of radius $R_1$ and $R_2$ with the same angular velocity $\omega$. If they have the same angular velocity, then it will take them the same amount of time to complete a revolution. However, the outer object has to cover a much larger distance (the circumference is larger), and thus has to move with a larger linear speed.
\capfig{0.35\textwidth}{figures/DescribingMotionInND/twocircles.png}{\label{fig:describingmotioninnd:twocircles} For a given angular velocity, the linear velocity will be larger on a larger circle ($v=\omega R$).} 
\begin{checkpoint}{\begin{MCquestion}{A motor is rotating at \SI{3000}{rpm}, what is the corresponding frequency in \si{Hz}?}
\item \SI{5}{Hz}
\item \SI{50}{Hz}\correct
\item \SI{500}{Hz}
\end{MCquestion}}
\end{checkpoint}
\vspace{-0.5cm}
\begin{studentOpinion}{Olivia}There's a trick I like to use to remember how linear and angular velocities work. Figure \ref{fig:describingmotioninnd:handcircularmotion} shows your hand in two positions, which we call (1) and (2).
\capfig{0.7\textwidth}{figures/DescribingMotionInND/HandPolarCoordinates.png}{\label{fig:describingmotioninnd:handcircularmotion} How to use your hand to better understand circular motion} 
Let's say you want to describe the location of your fingers in (2). Start by putting your hand in position (1). This is the position where $\theta=0$ and $s=0$. Imagine that your wrist (or your thumb, whichever you prefer) is fixed at the origin. If you keep your fingers perpendicular to your hand, they will always point in the positive $s$ direction. 

Imagine that you have a blue glob of paint on the back of your pinky. Rotate your hand until it is in position (2). The length of the curve that the paint makes is the value of $s$. The angle between the back of your hand and the positive $x$-axis is $\theta$. Now, imagine that there is a red glob of paint at your palm. It takes the same amount of time for your palm to get from position (1) to position (2) as it does for your fingers. Since they both go through the same angle $\theta$ in the same amount of time, the \textbf{angular velocity}, $\omega$ must be the same for both. However, the blue line left by your fingers will be much longer than the red line left by your palm. Your fingers travelled a greater distance than your palm in the same amount of time, so they must have a greater \textbf{linear velocity}, $v_s$. The further you are from your thumb, the greater the linear velocity will be, which we know from the formula $v_s=R\omega$.

If you kept rotating your hand around the circle, you would see that your fingers always point in the same  direction as your linear velocity. This means that if you are using cartesian coordinates, the direction of your linear velocity is always changing.

There are a couple of limitations to this trick. Remember that this only works for circular motion (the radius $R$ must be constant) and that if you are moving in the negative $s$ direction, your fingers will point antiparallel to the linear velocity.
\end{studentOpinion}



\newpage
\section{Summary}

\begin{chapterSummary}When the motion of an object is in more than one dimension, we describe the position of the object using a vector, $\vec{r}$. 
\begin{align*}
\vec r(t) = \begin{pmatrix}
          x(t) \\
          y(t) \\
          z(t) \\
        \end{pmatrix}= x(t) \hat x + y(t) \hat y + z(t) \hat z
\end{align*}
where $x(t)$, $y(t)$, and $z(t)$, are the position coordinates of the object. We treat the motion in each dimension as independent.

The instantaneous velocity vector and the acceleration vector are given by:
\begin{align*}
\vec v(t) &=\frac{d}{dt}\vec r(t)\\
\vec a(t) &= \frac{d}{dt}\vec v(t)
\end{align*}

If the acceleration vector is constant (in magnitude and direction), then the position and velocity of the object are described by:
\begin{align*}
\vec r(t) &= \vec r_0 + \vec v_0 t + \frac{1}{2} \vec at^2 \\
\vec v(t) &= \vec v_0 + \vec a t
\end{align*}
where each of these vector equations represents 3 independent equations, one for each of the $x$, $y$, and $z$ component of the vectors.

If an object has position $\vec{r}^A$ as measured in a frame of reference $xy$ that is moving at constant speed $\pvec{v}'^B$ as measured in a second frame of reference $x'y'$, then in the $x'y'$ reference frame:
\begin{align*}
\pvec r'^A(t) &= \pvec v'^Bt+\vec r^A(t)\\
\pvec v'^A(t) &=\pvec v'^B+\vec v^A(t)\\
\pvec a'^A(t)&=\vec a^A(t)
\end{align*}
An acceleration can change the magnitude and/or the direction of the velocity vector.
\begin{enumerate}
\item The component of the acceleration vector that is parallel to the velocity vector changes the magnitude of the velocity.
\item The component of the acceleration vector that is perpendicular to the velocity vector changes the direction of the velocity.
\end{enumerate}
The acceleration vector for motion in two dimensions can be written as the sum of vectors that are parallel ($\vec a_{\parallel}$) and perpendicular ($\vec a_{\perp}$) to the velocity vector:
\begin{align*}
\vec a&=\frac{dv}{dt}\hat v(t)+v(t)\frac{d\hat v}{dt} = \vec a_{\parallel} + \vec a_{\perp}
\end{align*}

If the position of an object moving in a circle of radius $R$ is described by its position along the curved axis $s$, then its position along the circle can be described using an angle, $\theta$, in radians:
\begin{align*}
\theta(t)=\frac{s(t)}{R}
\end{align*}
For an object moving along a circle, we can write its position vector, $\vec r(t)$, as:
\begin{align*}
\vec r(t)&= \begin{pmatrix}
          x(t) \\
          y(t) \\
        \end{pmatrix}
        =R \begin{pmatrix}
          \cos(\theta(t)) \\
          \sin(\theta(t)) \\
        \end{pmatrix}
\end{align*}
The angular velocity, $\omega$, is the rate of change of the angle. The angular acceleration, $\alpha$, is the rate of change of the angular velocity:
\begin{align*}
\omega &= \frac{d\theta}{dt}\\
\alpha &= \frac{d\omega}{dt}
\end{align*}
The linear kinematic quantities can be found from the angular quantities:
\begin{align*}
s=R\theta\\
v_s=R\omega\\
a_s=R\alpha
\end{align*}
For circular motion, the velocity vector is tangent to the circle and the perpendicular component of the acceleration is called the centripetal acceleration. The centripetal acceleration points towards the centre of the circle and has a magnitude of:
\begin{align*}
a_c(t) = \omega^2(t)R = \frac{v^2(t)}{R}
\end{align*}
The centripetal acceleration vector can be written as:
\begin{align*}
\vec a_{\bot}(t)&=\omega^2 R[-\cos(\theta)\hat x-\sin(\theta)\hat y]
\end{align*}
Uniform circular is the motion of an object around a circle with a constant speed. The period, $T$, is the time that it takes for the object to complete one revolution. The frequency, $f$, is the inverse of the period, and can be thought of as the number of revolutions completed per second:
\begin{align*}
T&=\frac{2\pi}{\omega}\\
f=\frac{1}{T}&=\frac{\omega}{2\pi}
\end{align*}
\end{chapterSummary}


\begin{importantEquations}
\begin{multicols}{2}
\textbf{Motion in 2D:}
\begin{align*}
\vec r(t) = \begin{pmatrix}
          x(t) \\
          y(t) \\
        \end{pmatrix}&= x(t) \hat x + y(t) \hat y\\
\vec v(t) &=\frac{d}{dt}\vec r(t)\\
\vec a(t) &= \frac{d}{dt}\vec v(t)
\end{align*}
\textbf{Relative Motion 2D:}
\begin{align*}
\pvec r'^A(t) &= \pvec v'^Bt+\vec r^A(t)\\
\pvec v'^A(t) &=\pvec v'^B+\vec v^A(t)\\
\pvec a'^A(t)&=\vec a^A(t)
\end{align*}
\textbf{Acceleration Vector 2D:}
\begin{align*}
\vec a&=\frac{dv}{dt}\hat v(t)+v(t)\frac{d\hat v}{dt}\\
\textrm{(constant speed:)} \quad \vec a&=\frac{dv_x}{dt} \left(\hat x - \frac{v_x(t)}{v_y(t)}\hat y\right) 
\end{align*}
\columnbreak

\textbf{Circular Motion:}
\begin{align*}
\vec r(t)&= \begin{pmatrix}
          x(t) \\
          y(t) \\
        \end{pmatrix}
        =R \begin{pmatrix}
          \cos(\theta(t)) \\
          \sin(\theta(t)) \\
        \end{pmatrix}\\
\omega &= \frac{d\theta}{dt}\\
\alpha &= \frac{d\omega}{dt}\\
s&=R\theta\\
v_s&=R\omega\\
a_s&=R\alpha\\
a_c(t) &= \omega^2(t)R = \frac{v^2(t)}{R}\\
\vec a_{\bot}(t)&=\omega^2 R[-\cos(\theta)\hat x-\sin(\theta)\hat y]\\
T&=\frac{2\pi}{\omega}\\
f=\frac{1}{T}&=\frac{\omega}{2\pi}
\end{align*}
\end{multicols}
\end{importantEquations}

\begin{definitions}
    \textbf{Position vector:} A vector, usually labelled, $\vec r$, to describe the position of an object relative to the origin of a coordinate system. In Cartesian coordinates, the position vector is simply given by the $x$, $y$, and $z$ coordinates of the object, $\vec r = x\hat x + y \hat y+ z\hat z$.

    \item \textbf{Velocity vector:} A vector, usually labelled, $\vec v$, which corresponds to the time-rate of change (the derivative with respect to time) of the position vector.

    \item \textbf{Acceleration vector:} A vector, usually labelled, $\vec a$, which corresponds to the time-rate of change (the derivative with respect to time) of the velocity vector.

	\item \textbf{Angular position:} The angle that the position vector makes with either the $x$ or $z$ axis. SI units: none. Common variable(s): $\theta$ (angle with the $z$ axis), $\phi$ (angle with the $x$ axis).

	\item \textbf{Angular velocity:} The rate at which an angle changes with respect to time. SI units: [\SI{}{s^{-1}}]. Common variable(s): $\vec \omega$. The angular velocity can be represented by a vector, using the right-hand rule for axial vectors.

	\item \textbf{Angular acceleration:} The rate at which angular velocity changes with respect to time. SI units: [\SI{}{s^{-2}}]. Common variable(s): $\vec \alpha$. The angular acceleration can be represented by a vector, using the right-hand rule for axial vectors. 

	\item \textbf{Uniform circular motion}: The motion of an object with constant speed around a circle.
\end{definitions}

\newpage
\section{Thinking about the material}

\begin{chapteractivity}{Reflect and research}
{
\item It was once believed that there was an absolute reference frame called the ``luminiferous aether''. What was the name of the experiment that disproved the existence of this frame of reference?
\item Find the centripetal acceleration of the Earth around the Sun.
}
\end{chapteractivity}


\begin{chapteractivity}{To try at home}
{
\item Describe and carry out a small experiment to confirm that the amount of time that it takes for a projectile to fall a certain distance does not depend on the horizontal component of its velocity.
}
\end{chapteractivity}


\begin{chapteractivity}{To try in the lab}
{
\item Develop a proposal for measuring how fast you can throw a ball, and carry out the experiment.
\item Develop a proposal for measuring how far you can jump with a running start (e.g. a long jump).
\item Propose an experiment to determine the period of the sun's rotational motion.
}
\end{chapteractivity}

\newpage
\section{Sample problems and solutions}
\subsection{Problems}
\begin{problemParts}{soln:describingmotioninnd:hurdle}{\label{prob:describingmotioninnd:hurdle}Ethan is jumping hurdles. He gets a running start, moving with a speed of $\SI{3}{m/s}$. The hurdle is $\SI{0.5}{m}$ high and the maximum speed that he can have when he leaves the ground is $\SI{5}{m/s}$. (You can assume Ethan is a point particle, and ignore air resistance).}
\item What is the closest distance from the hurdle at which Ethan can jump and still clear the hurdle?
\item What maximum height does he reach?
\end{problemParts}

\begin{problemParts}{soln:describingmotioninnd:cowboy}{\label{prob:describingmotioninnd:cowboy}A cowboy swings a lasso above his head. The lasso moves at a constant speed in a circle of radius $\SI{1.5}{m}$ in the horizontal plane. A hawk flies toward the lasso at $\SI{50}{km/h}$. The hawk sees the end of the lasso moving at $\SI{60}{km/h}$ when the lasso is directly in front of it (see Figure \ref{fig:describingmotioninnd:cowboyquestion}). In the reference frame of the cowboy ...}
\item How long does it take for the lasso to complete one revolution? (Hint: From the point of view of the hawk, the lasso is moving towards him in addition to moving in a circle. You will have to use your knowledge of relative motion to solve this problem!)
\item What is the centripetal acceleration of the end of the lasso? 
\item What is the angular acceleration of the lasso?
\capfig{0.3\textwidth}{figures/DescribingMotionInND/CowboyQuestionGiven.png}{\label{fig:describingmotioninnd:cowboyquestion} The problem as viewed from above. This diagram depicts the moment that the end of the lasso passes in front of the hawk.}
\end{problemParts} 

\newpage
\subsection{Solutions}
\begin{solution}{prob:describingmotioninnd:hurdle}\label{soln:describingmotioninnd:hurdle}Our approach will be to consider the $x$ and $y$ components of the motion separately. We start by drawing a diagram and choosing our coordinate system. We will choose $y$ to be vertical and positive upwards and $x$ to be in the direction that Ethan is running. We choose the origin to be the location where Ethan leaves the ground for the jump, as illustrated in Figure  \ref{fig:describingmotioninnd:hurdle}.

\capfig{0.7\textwidth}{figures/DescribingMotionInND/HurdleQuestion.png}{\label{fig:describingmotioninnd:hurdle} Ethan wants to clear a \SI{0.5}{m} hurdle and has an initial velocity $\vec v_0$ with $x$ and $y$ components.}

\begin{enumerate}[label=\alph*)]
\item Ethan's speed at the beginning of the jump is $v_0=\SI{5}{m/s}$ and the horizontal ($x$) component of his velocity is $v_x=\SI{3}{m/s}$.  The $y$ component of his initial velocity, $v_{0y}$, is given by: 
\begin{align*}
v_x^2+v_{0y}^2&=v_0^2\\
v_{0y}&=\sqrt{v_0^2-v_x^2}\\
v_{0y}&=\sqrt{(\SI{5}{m/s})^2-(\SI{3}{m/s})^2}=\SI{4}{m/s}
\end{align*}
We chose the origin at the beginning of the jump, so that Ethan's $x$ and $y$ coordinates at time $t=0$ are $x_0=0$ and $y_0=0$, respectively. Once Ethan is in the air, there will be no acceleration in the $x$ direction, and the only acceleration is in the $y$ direction and will be that due to gravity.
Ethan's position at any time $t$ can be described by the following equations: 
\begin{align*}
x(t)&=v_{x}t\\
y(t)&=v_{0y}t-\frac{1}{2}gt^2
\end{align*}
where $g$ is the acceleration due to gravity, $g=\SI{9.8}{m/s^2}$.

We want to determine the value of $x(t)$ when the vertical displacement, $y(t)$, is equal to the height of the hurdle, $h$. We thus find the value of $t$ when $y=\SI{0.5}{m}$ and then find the value of $x$ at that time.

We can re-arrange the equation for $y(t)$ and solve the resulting quadratic for $t$ (we get two solutions):
\begin{align*}
0&=-\frac{1}{2}gt^2+v_{0y}t-h\\
0&=\frac{1}{2}(\SI{-9.8}{m/s^2})t^2+(\SI{4}{m/s})t-\SI{0.5}{m}\\
t&=\SI{0.15}{s},\quad \SI{0.66}{s}
\end{align*}
The jump will be a parabola, and Ethan will cross a height of $\SI{0.5}{m}$ twice, once on the way up, and once on the way down. We want to know when Ethan reaches $\SI{0.5}{m}$ for the first time (on the way up), so we choose $t=\SI{0.15}{s}$. The horizontal displacement at this time is:
\begin{align*}
x&=v_xt\\
&=(\SI{3}{m/s})(\SI{0.15}{s})\\
&=\SI{0.45}{m}
\end{align*}
Therefore, he can get as close as $\SI{0.45}{m}$ from the hurdle before he has to jump, if his initial horizontal velocity is $\SI{3}{m/s}$.
\item Ethan's motion follows a parabolic shape. At the maximum height, Ethan's vertical velocity is equal to zero. We can model only the vertical part of the motion to solve for the value of $y$ when $v_y=0$. We know the following quantities:
\begin{align*}
v_{0y}&=\SI{4}{m/s}\\
v_y&=\SI{0}{m/s}\\
g&=\SI{9.8}{m/s^2}
\end{align*}
The easiest way to determine $y$ is to use the formula, 
\begin{align*}
v_y^2&=v_{0y}^2-2g(y-y_0)\\
\therefore y&=\frac{v_y^2-v_{0y}^2}{(-2g)}
\end{align*}
Substituting our values for $v_y$, $v_{0y}$, and $g$, we get:
\begin{align*}
y_{max}&=\frac{(\SI{-4}{m/s})^2}{(2)(\SI{-9.8}{m/s^2})}\\
y_{max}&=\SI{0.82}{m}
\end{align*}
Ethan reaches a maximum height of $\SI{0.82}{m}$. 

\end{enumerate}
\end{solution}

\newpage
\begin{solution}{prob:describingmotioninnd:cowboy}\label{soln:describingmotioninnd:cowboy}
\begin{enumerate}[label=\alph*)]
\item We need to determine the speed of the end of the lasso in the cowboy's frame of reference, knowing its speed in the hawk's frame of reference and knowing the velocity of the hawk. Once we know the speed of the lasso in the cowboy's frame of reference we can easily determine how long it takes to complete one revolution (its period). \\


\capfig{0.3\textwidth}{figures/DescribingMotionInND/CowboySolution.png}{\label{fig:describingmotioninnd:cowboysolution}The two coordinate systems are aligned so that positive $y'$ and positive $y$ are in the same direction. The velocity vectors of the hawk and the lasso in the reference frame of the cowboy are shown.}


We start by introducing coordinate systems for the hawk  ($xy$) and the cowboy ($x'y'$), and choose for the $x$ ($y$) and $x'$ ($y'$) axes to be parallel. We choose the axes such that $x$ is to the right (when seen from above, as in Figure \ref{fig:describingmotioninnd:cowboysolution}) and $y$ is in the direction of motion of the hawk as seen in the cowboy's reference frame. The velocity vector of the hawk in the cowboy's frame of reference is:
\begin{align*}
\pvec v'_H =  v'_H \hat y = (\SI{50}{km/h})\hat y
\end{align*}
In the hawk's frame of reference, the lasso will have a $y$ component of velocity in the negative $y$ direction with the same magnitude as the speed of the hawk, and an unknown component, $v_{Lx}$, in the $x$ direction. The velocity of the lasso in the hawk's frame of reference is:
\begin{align*}
\vec v_L=v_{Lx}\hat x - v'_H \hat y
\end{align*}
However, we know the speed of the lasso in the hawk's frame of reference ($v_L=\SI{60}{km/h}$), so we can easily find $v_{Lx}$:
\begin{align*}
v_{Lx}=\sqrt{v_L^2-v'^2_H}=\sqrt{(\SI{60}{km/h})^2-(\SI{50}{km/h})^2}=\SI{33.17}{km/h}
\end{align*}
\capfig{0.25\textwidth}{figures/DescribingMotionInND/CowboyVectorAddition.png}{\label{fig:describingmotioninnd:cowboyvector} Vector addition to determine the velocity of the lasso in the cowboy's reference frame.}
In the cowboy's frame of reference, the lasso will have a velocity vector (Figure \ref{fig:describingmotioninnd:cowboyvector}), $\pvec v'_L$, given by:
\begin{align*}
\pvec v'_L &= \pvec v'_H + \vec v_L\\
&= v'_H \hat y + v_{Lx}\hat x - v'_H \hat y\\
&=v_{Lx}\hat x = (\SI{33.17}{km/h})\hat x
\end{align*}
That is, in the cowboy's frame of reference, the lasso has a velocity that is in the $x$ direction. This corresponds to the speed, $v_s$, of the end of the lasso in uniform circular motion about a circle of radius $R=\SI{1.5}{m}$. We can thus find the time required for one revolution to be:
\begin{align*}
v_s &= \frac{2\pi R}{T}\\
\therefore T &= \frac{2\pi R}{v_s} =\frac{2\pi (\SI{1.5}{m})}{(\SI{33.17}{km/h})} = \frac{2\pi (\SI{1.5}{m})}{(\SI{9.2}{m/s})}=\SI{1.02}{s}
\end{align*}
where we converted the speed into $\si{m/s}$ before determining the time.


\item The motion is uniform circular motion, so it has a centripetal acceleration given by
\begin{align*}
a_c(t)&=\frac{v_s^2(t)}{R}
\end{align*}
To find the centripetal acceleration of the end of the lasso, we just use our values for $v_s$ and $R$.
\begin{align*}
a_c(t)&=\frac{(\SI{9.2}{m/s})^2}{\SI{1.5}{m}}=\SI{56}{m/s^2}
\end{align*}
 
\item The angular acceleration of the lasso is zero. The angular acceleration refers to the rate of change of the angular velocity (the rate at which the lasso rotates), which is constant for uniform circular motion.
\end{enumerate}
\end{solution}