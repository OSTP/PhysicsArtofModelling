\section{Rotational energy and momentum}

%%%%%%%%%%%%%%%%%%%%%%%%%%%%%%%%%%%
%%
%% Multiple Choice
%%
%%%%%%%%%%%%%%%%%%%%%%%%%%%%%%%%%%%
\subsection{Multiple Choice}

\question A solid disk of mass $M$ and radius $R$ rolls without slipping along a horizontal surface, so that its centre of mass has a speed $v$ relative to the ground. What is the total kinetic energy of the disk?
\begin{checkboxes} 
	\choice $K_{tot}=\frac{1}{2}mv^2$ 
	\choice  $K_{tot}=\frac{1}{4}mv^2$ 
	\choice $K_{tot}=\frac{3}{2}mv^2$ 
	\CorrectChoice $K_{tot}=\frac{3}{4}mv^2$ 
\end{checkboxes}

\question A figure skater is spinning with her arms extended. As she brings her arms in close to her body, she begins spinning faster because
\begin{checkboxes}
\choice Her arms do work on her body, and thus increases her kinetic energy.
\choice Her moment of inertia decreases, which increases her angular momentum.
\CorrectChoice Her moment of inertia decreases, which increases her angular velocity. \correct
\choice It only looks like she's spinning faster because she is more compact.
\end{checkboxes}

\question If a figure skater starts to spin twice as fast when he brings his arms in during a spin, he 
\begin{choices} 
	\choice increased his moment of inertia by a factor of 2.
	\CorrectChoice  decreased his moment of inertial by a factor of 2.
	\choice  increased his moment of inertia by a factor of 4.
	\choice decreased his moment of inertial by a factor of 4.
\end{choices}

\question Two identical solid spheres roll down two different inclined planes A and B. Both A and B have the same height, but different angles of inclination. If you release the balls at the same time, then they will reach the bottom:
\begin{checkboxes}
\choice With different speeds at different times
\choice With different speeds at the same time
\CorrectChoice With the same speed at different times \correct
\choice With the same speed at the same time
\end{checkboxes}

\question A particle is travelling along a straight line. If you measured the angular momentum of the particle about each of the following points, for which would the result be zero?
\begin{checkboxes}
\CorrectChoice Any point along its axis of travel \correct
\choice Any point, because it is travelling in a straight line
\choice There are no points where the angular momentum will measured to be zero
\end{checkboxes}

%Based on a question by Wei Zhuolin
\question  A pinball ($m=\SI{10}{g}$) is shot at with a speed $v=\SI{100}{m/s}$ into the end of a small rod of length $L=\SI{2}{m}$, and mass $M=\SI{2}{kg}$, in the direction perpendicular to the length of the rod.  The rod has an axis of rotation at the centre.  The rod is initially at rest.  What is the angular velocity of the rod after impact?
\begin{checkboxes}
\choice 2.91 rad/s
\CorrectChoice 1.47 rad/s \correct
\choice 2.01 rad/s
\choice 1.78 rad/s
\end{checkboxes}

%From Midyear Exam F17
\question[1] For your next demonstration, you wish to demonstrate how figure skaters can increase their angular velocity when doing spins. You place one of your friends on your desk chair (which can spin with virtually no friction) and ask them to cross their arms tightly against their chest. You give the chair a good spin and then ask your friend to spread their arms out. 
\begin{checkboxes} 
\choice The chair starts to spin faster because the moment of inertia of your friend about the rotation axis increased
\choice The chair starts to spin faster because the angular momentum of your friend about the rotation axis increased
\CorrectChoice The chair starts to spin slower because the moment of inertia of your friend about the rotation axis increased \correct
\choice The chair starts to spin slower because the angular momentum of your friend about the rotation axis increased
\end{checkboxes}


%%%%%%%%%%%%%%%%%%%%%%%%%%%%%%%%%%%
%
% long answer
%
%%%%%%%%%%%%%%%%%%%%%%%%%%%%%%%%%%%
\subsection{Long answers}

%Thomas
\question A large meteorite of mass $m$ strikes the Earth. It hits with a velocity perpendicular to the ground. It causes the Earth to start spinning slightly slower such that each day is slightly longer. Assume that the Earth is a solid sphere of radius $R_\oplus$ and mass $M_\oplus$, with uniform density.
\begin{parts}
\part If each day before impact lasts exactly $d=\SI{24}{hr}$, find the Earth's rotational (not orbital) angular momentum before the meteorite hits. Express your answer in terms of the variables above.
\part Express the angular momentum of the Earth after the meteorite hits in terms of the variables above, and in terms of $\Delta t$, the extra time in seconds that it takes the Earth to complete one day.
\part Find an expression for $\Delta t$, in terms of the mass of the meteorite and the mass of the Earth, expressed as a fraction of a day.
\part Hoba, the largest single meteorite ever found, has a mass of about \SI{60000}{kg}. Based on our assumptions here, how much longer is each day since it hit? The mass of the Earth is \SI{5.972e24}{kg}.
\end{parts}
\begin{finalanswer}
\begin{enumerate}[(a)]
\item $\frac{2}{5} M_\oplus R_\oplus^2\frac{2 \pi}{d}$, where $d=\SI{86400}{s}$ is the length of one day.
\item  $\left(\frac{2}{5}M_\oplus R_\oplus^2+ m R_\oplus^2\right)\frac{2\pi}{d + \Delta t}$
\item $\frac{\Delta t}{d} = \frac{5}{2}\frac{m}{M_\oplus}$
\item $\SI{2.17e-15}{s}$
\end{enumerate}
\end{finalanswer}
\begin{solution}
\begin{parts}
\part The angular momentum is given by the Earth's initial moment of inertia times its angular velocity:
\begin{align*}
L_0 &= I_0 \omega_0\\
I_0 &= \frac{2}{5} M_\oplus R_\oplus^2\\
\omega_0 & = \frac{2 \pi}{d}\\
\therefore L_0 &= \frac{2}{5} M_\oplus R_\oplus^2\frac{2 \pi}{d}
\end{align*}
where $d=\SI{86400}{s}$ is the length of one day.
 
\part The meteorite changes the moment of inertia of the Earth from $I_0$ to $I_f$. We assume that all of the mass is deposited at the outer radius as a point. The angular velocity is now expressed in terms of $\Delta t$:
\begin{align*}
I_{f} &= \frac{2}{5} M_\oplus R_\oplus^2+ m R_\oplus^2\\
\omega_f&=\frac{2\pi}{d+\Delta t}\\
\therefore L_f &= I_f\omega_f= \left(\frac{2}{5}M_\oplus R_\oplus^2+ m R_\oplus^2\right)\frac{2\pi}{d + \Delta t} 
\end{align*}

\part Since no external torques are applied to the Earth+meteorite system, the total angular momentum is conserved. Since the meteorite hits perpendicularly to the surface of the Earth, it carries no angular momentum, and the initial angular momentum is what we calculated in part (a).
\begin{align*}
L_0&=L_f\\
I_0\omega_0 &=I_f\omega_f\\
\therefore \omega_f &= \frac{I_0}{I_f}\omega_0\\
\frac{2\pi}{d+\Delta t} &= \frac{\frac{2}{5} M_\oplus R_\oplus^2}{\frac{2}{5} M_\oplus R_\oplus^2+ m R_\oplus^2}\frac{2 \pi}{d}\\
\frac{1}{d+\Delta t} &= \frac{\frac{2}{5} M_\oplus R_\oplus^2}{\frac{2}{5} M_\oplus R_\oplus^2+ m R_\oplus^2}\frac{1}{d}\\
\frac{1}{d+\Delta t} &= \frac{1}{1 +\frac{5}{2}\frac{m}{M_\oplus} }\frac{1}{d}\\
d+\Delta t &=\left(1 +\frac{5}{2}\frac{m}{M_\oplus}\right)d\\
\frac{\Delta t}{d} &= \frac{5}{2}\frac{m}{M_\oplus}
\end{align*}

\part Plug the numbers into part (c) and obtain:
\begin{align*}
\frac{\Delta t}{d} &= \frac{5}{2}\frac{m}{M_\oplus}=\frac{5}{2}\frac{\SI{6e4}{kg}}{\SI{5.972e24}{kg}}=\num{2.5e-20}
\end{align*}
which corresponds to \SI{2.17e-15}{s}.
\end{parts}
\end{solution}

%Giancolli 11-51
\question A child is attempting to create a lollipop by rolling a small hard candy along a frictionless surface and onto a plastic stick. The plastic stick has a mass $M$ and length $L$ and the hard candy has a mass $m$ and is moving at a speed $v$. The hard candy strikes the plastic stick at the point $\frac{L}{4}$ above its centre of mass. The hard candy is sticky, so when it collides with the plastic stick, the two objects remain in contact and move together. 
\begin{enumerate}[(a)]
\item What is the velocity of the centre of mass of the lollipop? 
\item Show that the angular speed of the lollipop is given by:
\begin{align*}
\omega = \frac{12mv}{L(4M+7m)}
\end{align*}
\end{enumerate}

\capfig{0.2\textwidth}{figures/AngularMomentumRolling/RodClay.png}{\label{fig:angularmomentumrolling:RodClay}A piece of hard candy strikes the plastic stick at a distance $\frac{L}{4}$ above the stick's CM.}

\begin{finalanswer}
\begin{enumerate}[(a)]
\item $\vec v_{CM}=\frac{m}{M+m}\vec v$
\item N/A
\end{enumerate}

\end{finalanswer}
\begin{solution}
The collision is inelastic, but both linear and angular momentum are conserved, as there are no external torques or forces on the system. Conservation of linear momentum trivially gives the velocity of the centre of mass after the collision:
\begin{align*}
m\vec v &= \vec v_{CM}(M+m)\\
\therefore \vec v_{CM} &=\frac{m}{M+m}\vec v
\end{align*}
Angular momentum about the centre of mass of the system is also conserved. First, we need to determine the position of the centre of mass so that we can calculate the angular momentum. Referring to the figure, the $x$ component of the centre of mass moves with the velocity $\vec v_{CM}$, and the $y$ position is fixed. Right at the instant of the collision, the CM is on the stick, with $y$ coordinate given by:
\begin{align*}
y_{CM}=\frac{mL}{4(M+m)}
\end{align*}
where we have chosen the origin such that the $y$ position of the centre of mass of the stick is zero. The initial angular momentum comes from the hard candy (it strikes the stick with linear momentum perpendicular to the stick, so $L=rp$):
\begin{align*}
L_i&=mvr=mv\left(\frac{L}{4}-y_{CM}\right)\\
&=mv\left(\frac{L}{4}-\frac{mL}{4(M+m)}\right)\\
&=mv\left(\frac{L(M+m)}{4}-\frac{Lm}{4(M+m)}\right)\\
&=mv\left(\frac{ML}{4(M+m)}\right)\\
\end{align*}
where we have taken into account the fact that the hard candy is moving perpendicular to the stick when it strikes. After the collision, the stick and hard candy form a system that rotates about their common centre of mass. The moment of inertia of the system is given by:
\begin{align*}
I &= I_m+I_M=m\left(\frac{L}{4}-y_{CM}\right)^2+\left(\frac{1}{12}ML^2+M y_{CM}^2\right)\\
&=m\left(\frac{ML}{4(M+m)}\right)^2+\frac{1}{12}ML^2+M\left( \frac{mL}{4(M+m)}\right)^2\\
&=\frac{mM^2L^2}{16(M+m)^2}+\frac{1}{12}ML^2+\frac{Mm^2L^2}{16(M+m)^2}\\
&=\frac{ML^2}{(16)(12)(M+m)^2} (12Mm+16(M+m)^2)+12m^2)\\ 
&=\frac{ML^2}{(16)(12)(M+m)^2} (12Mm+16M^2+32Mm+16m^2+12m^2)\\ 
&=\frac{ML^2}{(16)(12)(M+m)^2} (16M^2+44Mm+28m^2)\\ 
&=\frac{ML^2}{48(M+m)^2} (4M^2+11Mm+7m^2)\\ 
&=\frac{ML^2}{48(M+m)} (4M+7m)\\ 
\end{align*}
By conservation of angular momentum, we have:
\begin{align*}
L_i &= I\omega\\
mv\left(\frac{ML}{4(M+m)}\right) &=\frac{ML^2}{48(M+m)} (4M+7m)\omega\\
mv &=\frac{L}{12}(4M+7m)\omega\\
\therefore \omega &=\frac{12mv}{L(4M+7m)}
\end{align*}
\end{solution}

%%%%%%%%%%%%%%%%%%%%%%%%%%%%%%%%%%%
%
% This question taken from http://oyc.yale.edu/sites/default/files/midterm_solutions_0.pdf
%
%%%%%%%%%%%%%%%%%%%%%%%%%%%%%%%%%%%
%%USe for quiz
\question
A bullet of mass $m$ is moving horizontally with velocity $v$ when it strikes the edge of a uniform horizontal disk with mass $M$ and radius $R$. The disk is anchored at its center but free to rotate about a vertical axis. Assume that the bullet struck the disk tangentially at its edge.
\begin{parts}
\part Which of energy, momentum, and angular momentum are conserved for the bullet+disk system? Explain.
\part Find $\omega$, the angular velocity of the disk with respect to its centre, in terms of the given variables. Assume that the disk started at rest.
\end{parts}
\begin{finalanswer}
\begin{enumerate}[(a)]
\item Only angular momentum is conserved.
\item $\omega =\frac{mv}{\left(\frac{1}{2}M+m\right)R}$
\end{enumerate}
\end{finalanswer}
\begin{solution}  
\begin{parts}
\part Only angular momentum is conserved. The collision is an inelastic collision, so energy is not conserved.
Momentum is not conserved because there is an external force acting at the centre of the disk that keeps
the disk-bullet system from moving forward after the collision. Angular momentum is conserved because
there are no external torques acting on the system. (The force acting at the center of the disk does not
provide a torque because it is acting at the pivot point.)

\part Since the disk started at rest, the initial angular momentum is that of the bullet:
\begin{align*}
L_i=\vec r\times\vec p
\end{align*} 
where $\vec r$ is the location that the bullet strikes with respect to the centre of the disk, and $\vec p$ is the momentum of the bullet. Since the bullet strikes the disk tangentially, $\vec r$ and $\vec p$ are perpendicular, and we can write:
\begin{align*}
L_i=rp=Rmv
\end{align*}
After the collision, the bullet and disk form a single object with moment of inertia:
\begin{align*}
I=\frac{1}{2}MR^2+mr^2=\left(\frac{1}{2}M+m\right)R^2
\end{align*}
and angular momentum:
\begin{align*}
L_f=I\omega
\end{align*}
Since angular momentum is conserved, we can find the angular velocity of the disk and bullet after the collision:
\begin{align*}
L_i &= L_f\\
Rmv &= I\omega\\
\therefore \omega &= \frac{Rmv}{I}=\frac{Rmv}{\left(\frac{1}{2}M+m\right)R^2}\\
&=\frac{mv}{\left(\frac{1}{2}M+m\right)R}
\end{align*}
\end{parts}
\end{solution}

%%%%%%%%%%%%%%%%
%Rolling
%%%%%%%%%%%%%%

\question A solid cylinder of mass $M$ and radius $R$ is placed on an incline that makes and angle $\theta=\SI{45}{\degree}$ with the horizontal, as shown at the top of Figure \ref{fig:angularmomentumrolling:CylinderIncline}.
\begin{parts}
\part What is the minimum coefficient of static friction between the cylinder and the incline that will allow the cylinder to roll down the incline without slipping?
\part What is the acceleration vector of the centre of mass of the cylinder as it rolls down the incline?
\part If the cylinder was released from a height $h=\SI{3.0}{m}$ above the ground (as shown), and it rolls down the incline without slipping, what is the velocity vector of point $A$ on the cylinder as shown in the bottom of Figure \ref{fig:angularmomentumrolling:CylinderIncline}?
\part If the cylinder was released from a height $h=\SI{3.0}{m}$ above the ground (as shown), and it rolls down the incline without slipping, what is the velocity vector of point $B$ on the cylinder as shown in the bottom of Figure \ref{fig:angularmomentumrolling:CylinderIncline}?
\end{parts}
\capfig{0.4\textwidth}{figures/AngularMomentumRolling/CylinderIncline.png}{\label{fig:angularmomentumrolling:CylinderIncline} A cylinder rolling without slipping down an incline, before being released, and at the bottom.}
\begin{finalanswer}
\begin{enumerate}[(a)]
\item 1/3
\item $\SI{4.62}{m/s^2}$; points downwards and parallel to the plane
\item $\SI{12.52}{m/s}$, horizontal and to the right.
\item $\vec v_B=(\SI{6.26}{m/s})\hat i-(\SI{6.26}{m/s})\hat j$
\end{enumerate}
\end{finalanswer}
\begin{solution}
\begin{parts}
\part Figure \ref{fig:angularmomentumrolling:CylinderIncline_FBD} shows the forces and their point of application on the cylinder. The only three forces are the normal force ($\vec N$), the weight ($M\vec g$), and the force of static friction ($\vec f_s$).
\capfig{0.2\textwidth}{figures/AngularMomentumRolling/CylinderIncline_FBD.png}{\label{fig:angularmomentumrolling:CylinderIncline_FBD} }

If we use the centre of mass of the cylinder as the rotation point, then only static friction provides a torque:
\begin{align*}
\sum \tau &= f_sR=I\alpha=\frac{1}{2}MR^2\alpha\\
\therefore f_s=\frac{1}{2}MR\alpha
\end{align*}
We choose a coordinate system such that $x$ is parallel to the acceleration of the centre of mass (positive downwards parallel to the incline) and $y$ is perpendicular to the plane and positive upwards. The sum of the forces in the $x$ and $y$ directions are:
\begin{align*}
\sum F_x&=Mg\sin\theta-f_s=Ma_{CM}\\
\sum F_y&=N-Mg\cos\theta = 0\\
 N&=Mg\cos\theta\
\end{align*}
From the $y$ equation, we can solve for the normal force in terms of the weight, and us $f_s=\mu_sN$ in the $x$ equation:
\begin{align*}
Mg\sin\theta-\mu_sMg\cos\theta&=Ma_{CM}\\
\therefore g(\sin\theta-\mu_s\cos\theta) &=a_{cm}
\end{align*}
Since this is rolling without slipping, we have $a_{CM}=\alpha R$:
\begin{align*}
g(\sin\theta-\mu_s\cos\theta) &=\alpha R\\
\therefore \alpha &=\frac{g}{R}(\sin\theta-\mu_s\cos\theta)
\end{align*}
Re-writing the equation that we had from the torque:
\begin{align*}
f_s&=\mu_sMg\cos\theta=\frac{1}{2}MR\alpha\\
\therefore \alpha&=2\mu_sg\cos\theta\frac{1}{R}
\end{align*}
Combining the two expressions for $\alpha$ gives:
\begin{align*}
\sin\theta-\mu_s\cos\theta &= 2\mu_s\cos\theta\\
\sin\theta &=  3\mu_s\cos\theta\\
\therefore \mu_s &= \frac{1}{3}\tan\theta=\frac{1}{3}\tan(\SI{45}{\degree})=\frac{1}{3}
\end{align*}
\part The acceleration vector points downwards and parallel to the plane. Its magnitude is given by:
\begin{align*}
a_{CM}&=\alpha R=\frac{2}{3}g\tan\theta\cos\theta\\
&=\frac{2}{3}(\SI{9.8}{m/s^2})\tan(\SI{45}{\degree})\cos(\SI{45}{\degree})=\SI{4.62}{m/s^2}
\end{align*}
\part We can use conservation of energy and determine the speed of the centre of mass of the cylinder at the bottom of the incline. The centre of mass potential energy at the top is $Mgh$ and zero at the bottom. The kinetic energy is zero at the top and has linear and rotational components at the bottom:
\begin{align*}
Mgh &= \frac{1}{2}Mv_{CM}^2+\frac{1}{2}I\omega^2
\end{align*}
Since the cylinder rolls without slipping, $v_{cm}=\omega R$:
\begin{align*}
Mgh &= \frac{1}{2}Mv_{CM}^2+\frac{1}{2}I\frac{v^2_{CM}}{R^2}\\
Mgh &= \frac{1}{2}Mv_{CM}^2+\frac{1}{2}\frac{1}{2}MR^2\frac{v^2_{CM}}{R^2}\\
gh &= \frac{1}{2}v_{CM}^2+\frac{1}{4}Mv^2_{CM}\\
gh &= \frac{3}{4}v_{CM}^2\\
\therefore v_{CM}&=\sqrt{\frac{4}{3}gh}=\sqrt{\frac{4}{3}(\SI{9.8}{m/s^2})\SI{3.0}{m}}=\SI{6.26}{m/s}
\end{align*}
For rolling without slipping, point A has a velocity that is horizontal and to the right, with magnitude:
\begin{align*}
v_A=2v_{CM}=2\sqrt{\frac{4}{3}gh}= 2(\SI{6.26}{m/s}) =\SI{12.52}{m/s}
\end{align*}
\part For point B, we need to vectorially add the velocity of the centre of mass to the tangential velocity from rolling. We choose a coordinate system where $x$ is horizontal and positive to the right, and $y$ is positive upwards. Thus the velocity of point B is:
\begin{align*}
\vec v_B&=v_{CM}\hat i- \omega R\hat j=v_{CM}\hat i- v_{CM}\hat j\\
&=(\SI{6.26}{m/s})\hat i-(\SI{6.26}{m/s})\hat j
\end{align*}
\end{parts}
\end{solution}

%Zaremba 2003 Final exam
\question A flat circular disk of radius $R$ and mass $M$ has a string wrapped around its rim and is laid flat on a horizontal frictionless table, as in Figure \ref{fig:angularmomentumrolling:DiskAndMass}. The string passes over a frictionless massless pulley and a mass $m$ is suspended from the end. 
\capfig{0.5\textwidth}{figures/AngularMomentumRolling/DiskAndMass.png}{\label{fig:angularmomentumrolling:DiskAndMass}A disk on frictionless table being pulled on by a falling mass}
\begin{parts}
\part What are the linear accelerations of the centre of mass of the disk and of the mass $m$, and the angular acceleration of the disk?
\part What is the angular speed of the disk after it has completed one revolution (assuming the system was released from rest)?
\end{parts}
\begin{finalanswer}
\begin{enumerate}[(a)]
\item $a_M=\frac{m}{M+(3/2)m}g$; $a_m=\frac{3m}{2M+3m}g$
\item $\omega =\sqrt{4\pi\frac{m}{M+\frac{3}{2}m}\frac{g}{2R}}$
\end{enumerate}
\end{finalanswer}
\begin{solution}
\begin{parts}
\part On the disk, the vertical forces (gravity and normal) just cancel, and since the table is frictionless, we do not need to use the normal force. The only other force on the disk is the tension from the string un-winding. The forces on the disk and mass are shown in Figure \ref{fig:angularmomentumrolling:DiskAndMass_FBD}.
\capfig{0.2\textwidth}{figures/AngularMomentumRolling/DiskAndMass_FBD.png}{\label{fig:angularmomentumrolling:DiskAndMass_FBD}Forces on the disk (seen from above) and the falling mass.}
Newton's Second Law, in the horizontal plane for the disk gives:
\begin{align*}
T=Ma_{M}
\end{align*}
where $a_M$ is the acceleration of the centre of mass of the disk. If we now write the sum of torques on the disk, we obtain:
\begin{align*}
\sum \tau &= TR = I\alpha = \frac{1}{2}MR^2\alpha\\
\therefore T &= \frac{1}{2}MR\alpha
\end{align*}
Equating the values of $T$ with that obtained from the forces, we can find the relation between the angular and linear acceleration of the disk:
\begin{align*}
Ma_{M} &= \frac{1}{2}MR\alpha\\
\therefore \alpha &=\frac{1}{2}\frac{a_M}{R}
\end{align*}
The acceleration of the falling mass is the same as that of the string, which itself has the same acceleration as the point where it meets the disk. The acceleration of the point on the disk where the string is attached is given by:
\begin{align*}
a_m=\alpha R+a_M=\frac{3}{2}a_M
\end{align*}
The sum of the forces on the falling mass is given by:
\begin{align*}
mg-T&=ma_m=\frac{3}{2}a_M\\
\therefore T&=mg-\frac{3}{2}a_Mma_M
\end{align*}
We can equate this for the expression for $T$ that we obtained from the sum of the forces on the disk to obtain the linear acceleration of the centre of mass of the disk:
\begin{align*}
Ma_M &= mg-\frac{3}{2}a_Mma_M\\
\left(M+\frac{3}{2}m\right) a_M &= mg\\
\therefore a_M&=\frac{m}{M+\frac{3}{2}m}g\
\end{align*}
The acceleration of the falling block and the angular acceleration of the disk are given by:
\begin{align*}
\alpha&=\frac{1}{2}\frac{a_M}{R} = \frac{m}{M+\frac{3}{2}m}\frac{g}{2R}\\
a_m&=\frac{3}{2}a_M=\frac{3m}{2M+3m}g
\end{align*}
\part Knowing the angular acceleration, $\alpha$, the total displacement, $\Delta \theta = 2\pi$, we can use one of the kinematic equations to get the angular velocity:
\begin{align*}
\omega^2-0 &= 2\alpha\Delta\theta\\
\therefore \omega^2 &= 4\pi\alpha\\
\therefore \omega &=\sqrt{4\pi\frac{m}{M+\frac{3}{2}m}\frac{g}{2R}}
\end{align*}
\end{parts}
\end{solution}


%Zaremba 2007 Final exam
\question A uniform disk of mass $M$ and radius $R$ has an angular velocity $\omega_0$ with its axis of rotation in the horizontal plane through its centre of mass. The disk is gently placed on a horizontal surface. The coefficient of kinetic friction between the surface and the disk is $\mu_k$. 
\capfig{0.4\textwidth}{figures/AngularMomentumRolling/DiskSlipping.png}{\label{fig:angularmomentumrolling:DiskSlipping}A vertical rotating disk gently lowered onto a surface.}
\begin{parts}
\part What is the linear acceleration of the centre of mass of the disk, and what is its angular acceleration about its centre of mass?
\part How far along the surface does the disk travel before it stops sliding?
\end{parts}
\begin{finalanswer}
\begin{enumerate}[(a)]
\item $a_{CM}=\mu_kg$, $\alpha=-2\frac{\mu_kg}{R}$
\item $\frac{1}{18}\frac{R^2\omega_0^2}{\mu_kg}$
\end{enumerate}
\end{finalanswer}
\begin{solution}
\begin{parts}
\part When the disk touches the surface, there will be a force of kinetic friction between the surface and the disk that will cause an acceleration of the centre of mass and an angular deceleration, as shown in Figure \ref{fig:angularmomentumrolling:DiskSlipping_FBD}
\capfig{0.1\textwidth}{figures/AngularMomentumRolling/DiskSlipping_FBD.png}{\label{fig:angularmomentumrolling:DiskSlipping_FBD}Forces on the disk as it touches the surface.}
 The angular velocity will decrease until the centre of mass has accelerated enough such that $\omega = v_{CM}$. Since the only force acting on the disk is friction, it will cause both the angular deceleration and the acceleration of the CM. Since the disk is slipping, the angular acceleration, $\alpha$ is not related the acceleration of the CM, $a_{CM}$. The force of kinetic friction is given by:
\begin{align*}
f = \mu_kN=\mu_kMg
\end{align*}
The sum of the forces gives:
\begin{align*}
f &= Ma_{CM}\\
\therefore a_{CM}&=\mu_k g
\end{align*}
And similarly, the sum of torques gives:
\begin{align*}
-fR&=I\alpha=\frac{1}{2}MR^2\alpha\\
-\mu_kMg&=\frac{1}{2}MR\alpha\\
\therefore \alpha&=-2\frac{\mu_kg}{R}
\end{align*}
\part The disk will stop slipping once $\omega(t)R=v_{CM}(t)$, so we need to find out at which time that happens. The angular velocity and the speed of the CM as a function of time are given by:
\begin{align*}
\omega(t)&=\omega_0+\alpha t = \omega_0-2\frac{\mu_kg}{R}t\\
v_{CM}(t) &= a_{CM}t=\mu_k gt
\end{align*}
The time is then given by:
\begin{align*}
\omega(t)R&=v_{CM}(t)\\
R\omega_0-2\mu_kgt&=\mu_k gt\\
\therefore t&=\frac{R\omega_0}{3\mu_kg}\\
\end{align*}
The distance travelled by the disk is then:
\begin{align*}
d &= \frac{1}{2}a_{CM}t^2\\
&=\frac{1}{2}\mu_k g \left( \frac{R\omega_0}{3\mu_kg}\right)^2\\
&=\frac{1}{18}\frac{R^2\omega_0^2}{\mu_kg}
\end{align*}
\end{parts}
\end{solution}


\question You are in a spaceship of mass $m=\SI{4e3}{kg}$ in a circular orbit around planet Camelid. The planet is spherical with mass $M=\SI{3e24}{kg}$ and radius $R=\SI{6e6}{m}$. You are in a circular orbit of radius $2R$. You wish to take a closer look at the inhabitants of the planet which are able to live with zero atmosphere. You briefly fire the rockets on your ship in the direction opposite to your motion so that you slow down and place yourself on an elliptical orbit that will skim the planet (the perigee will be a distance $R$ from the planet's centre), as shown in Figure \ref{fig:angularmomentumrolling:EarthSkim}. How much work do the rockets need to do in order to place your ship in this new orbit?
\capfig{0.4\textwidth}{figures/AngularMomentumRolling/EarthSkim.png}{\label{fig:angularmomentumrolling:EarthSkim}An elliptical orbit skimming the planet.}
\begin{solution}
The mechanical energy of your spaceship in the circular orbit of radius $2R$ is given by:
\begin{align*}
E_{circ}=K+U=\frac{1}{2}mv^2-G\frac{Mm}{2R}
\end{align*}
where we have defined gravitational potential energy so that it is zero infinitely far from the planet. The speed in the circular orbit is given by the condition that gravity provide the required centripetal acceleration:
\begin{align*}
m\frac{v^2}{2R}&=G\frac{Mm}{4R^2}\\
\therefore \frac{1}{2}mv^2&=G\frac{M}{4R}
\end{align*}
So that the total energy in the circular orbit of radius $2R$ is given by:
\begin{align*}
E_{circ}=K+U=G\frac{M}{4R}-G\frac{Mm}{2R}=-G\frac{Mm}{4R}
\end{align*}
The work done by the rocket will be the difference in energy between the circular orbit and the elliptical orbit.

After decelerating, you spaceship will be at the apogee of the orbit, with speed $v_A$, and energy:
\begin{align*}
E_A=K+U=\frac{1}{2}mv_A^2-G\frac{Mm}{2R}
\end{align*}
where we do not know $v_A$. The energy at the perigee of the orbit will be the same (mechanical energy is conserved through the orbit):
\begin{align*}
E_P=\frac{1}{2}mv_P^2-G\frac{Mm}{R}
\end{align*}
where $v_P$ is the speed of your spaceship at the perigee, which we also do not know.

However, the angular momentum of the spaceship about the planet is also conserved throughout the orbit. The magnitude of the angular momentum of the spaceship relative to the centre of the planet at the apogee is given by:
\begin{align*}
L_A=||\vec r_A \times \vec p_A||=2Rmv_A
\end{align*}
Similarly, the angular momentum at the perigee is given by:
\begin{align*}
L_P=Rmv_P
\end{align*}
Conservation of energy and angular momentum thus give us two equations which we can solve for the speeds at the perigee and apogee:
\begin{align*}
\frac{1}{2}mv_A^2-G\frac{Mm}{2R} &= \frac{1}{2}mv_P^2-G\frac{Mm}{R} \\
2Rmv_A &= Rmv_P
\end{align*}
The second equation simply yields that the speed at the perigee is twice that at the apogee. We can substitute this into the the energy conservation equation to solve for the speed at the apogee (or more usefully, the kinetic energy at the apogee):
\begin{align*}
\frac{1}{2}mv_A^2-G\frac{Mm}{2R} &= \frac{1}{2}mv_P^2-G\frac{Mm}{R}\\
\frac{1}{2}mv_A^2-G\frac{Mm}{2R} &= 2mv_A^2-G\frac{Mm}{R}\\
\frac{3}{2}mv_A^2 &= -G\frac{Mm}{2R}+G\frac{Mm}{R}\\
\therefore \frac{1}{2}mv_A^2&=G\frac{Mm}{6R}
\end{align*}
The mechanical energy of the elliptical orbit is thus given by:
\begin{align*}
E_A=\frac{1}{2}mv_A^2-G\frac{Mm}{2R}=G\frac{Mm}{6R}-G\frac{Mm}{2R}=-G\frac{Mm}{3R}
\end{align*}
The net work done by the rockets is thus the change in energy between the elliptical and circular orbits:
\begin{align*}
W &= E_A-E_{circ}=-G\frac{Mm}{3R}+G\frac{Mm}{4R}=-G\frac{Mm}{12R}\\
&=-(\SI{6.67e-11}{N\cdot m^2/kg^2})\frac{(\SI{3e24}{kg})(\SI{4e3}{kg})}{12(\SI{6e6}{m})}\\
&=\SI{-1.11e10}{J}
\end{align*}
and is negative, as expected.

\end{solution}


%Joshua Rinaldo
\question A cube of mass $m = \SI{8}{kg}$ is fixed at a position along a frictionless surface where it compresses a spring of $k = 60+15x+5x^2$ a distance of $x = \SI{0.8}{m}$, as shown in Figure \ref{fig:angularmomentumrolling:cube_spring}.
\begin{parts}
\part If the cube is released from its fixed position, what will its final velocity be?
\part Suppose that instead of a cube, a sphere is fixed at the position, as shown in Figure \ref{fig:angularmomentumrolling:sphere_spring}. Now, assume that the surface is not frictionless, but instead has a $\mu_s$ sufficient for rolling without slipping. If the sphere is released from its fixed position, what will its final velocity be?
\capfig{0.4\textwidth}{figures/AngularMomentumRolling/cube_spring}{\label{fig:angularmomentumrolling:cube_spring} A cube compressing a spring.}
\capfig{0.4\textwidth}{figures/AngularMomentumRolling/sphere_spring}{\label{fig:angularmomentumrolling:sphere_spring} A sphere compressing a spring.}
In parts a) and b), you will notice a difference in velocity for the cube and sphere. Suppose you are tasked with moving an object which can either be spherical or cubic across a surface. Determine which shape of the two would move across the following surfaces most quickly, and justify your answer:
\part The surface is a distance $d$, with $\mu_s = 0$ and $\mu_k = 0$.
\part The surface is a very large distance $d$, with $\mu_s \neq 0$ and $\mu_k \neq 0$.
\part The surface is a distance $d$, with a very large $\mu_s$ and $\mu_k$.
\part The surface is a distance $d$, with a very large $\mu_s$ and $\mu_k=0$
\end{parts}
\begin{finalanswer}
	\begin{parts}
	\part The cube will travel at a velocity of \SI{2.36}{m/s} after being released from its fixed position.
	\part The sphere will travel at \SI{1.99}{m/s} after being released from its fixed position.
	\part The cube and the sphere will be reach the distance $d$ at the same time.
	\part The sphere will reach the distance $d$ most quickly.
	\part The sphere will reach the distance $d$ most quickly.
	\part The cube will reach the distance $d$ most quickly.
	\end{parts}
\end{finalanswer}
\begin{solution}
	\begin{parts}
	\part We must first find the energy stored in the compressed spring.
	\begin{align*}
	W&=\int_{0}^{x}Fdx\\
	W&=\int_{0}^{0.8}(60x+15x^2+5x^3)dx\\
	W&=(30x^2+5x^3+\frac{5}{4}x^4)|_0^{0.8}\\
	W&=\SI{22.272}{J}
	\end{align*}
	In the case of the cube, all of the potential energy will be converted to kinetic energy.
	\begin{align*}
	W&=\frac{1}{2}mv^2\\
	\SI{22.272}{J}&=\frac{1}{2}(\SI{8}{kg})v^2\\
	v&= \sqrt{2(\SI{22.272}{J})\frac{1}{(\SI{8}{kg})}}\\
	v&= \SI{2.36}{m/s}
	\end{align*}
	Therefore, the cube will travel at \SI{2.36}{m/s} after being released from the spring.
	\part We have already calculated the potential energy stored in the spring, so we begin with the same value of $W = \SI{22.272}{J}$, however, we must take rotational energy into consideration when calculating the final velocity of the sphere.
	\begin{align*}
	W&= \frac{1}{2}(mv^2+I_s\omega^2)\\
	W&= \frac{1}{2}(mv^2+\frac{2}{5}mr^2(\frac{v}{r})^2)\\
	W&= \frac{1}{2}(mv^2+\frac{2}{5}mv^2)\\
	\SI{22.272}{J}&=\frac{7}{10}(\SI{8}{kg})v^2\\
	v&=\sqrt{\frac{10}{7}(\SI{22.272}{J})(\frac{1}{\SI{8}{kg}})}\\
	v&=\SI{1.99}{m/s}
	\end{align*}
	Therefore the sphere will travel at \SI{1.99}{m/s} after being released from the spring.
	\part If $\mu_s=0$ and $\mu_k=0$, the sphere will not roll at all, and will instead slide along the surface at the same velocity as the cube. Therefore, the cube and the sphere are equally valid choices.
	\part If $\mu_s \neq 0$ and $\mu_k \neq 0$, the cube will decelerate at a constant rate, eventually coming to a stop before reaching the very large distance $d$. The sphere may slide, causing it to decelerate, but if $\mu_s \neq 0$, then it will eventually stop slipping and will begin to roll at a constant velocity. Therefore, the sphere is the best choice for these conditions.
	\part If both $\mu_s$ and $\mu_k$ are very large, then the cube will decelerate quickly then come to a stop. The sphere will roll without slipping until reaching the distance $d$. Therefore, the sphere is the best choice for these condition.
	\part This is the scenario we modelled in parts a) and b) of this question, so simply citing the mathematical outcome is a sufficient explaination. We can also cite the conservation of energy, knowning that there is movement along a rolling sphere which is not translational, making the cube slightly faster than the sphere. Therefore, the cube is the best choice for these conditions.
	\end{parts}
\end{solution}

%TODO (from Alex Wright) A truck with muddy tires is driving in front of you, and you notice that your car is getting covered with mud because the mud flaps are in the wrong position. How long should the mud flaps be if the truck is driving at 80 km/h and you follow 2m behind? [Calculate the parabolic trajectory of mud flying off the wheel, and intercept those that could hit the car. Need to specify radius of truck tires, position of mud flap, etc]