\chapter{Vectors}
\label{app:vectors}
This appendix gives a very brief introduction to coordinate systems and vectors.
 \vspace{1cm}
\begin{learningObjectives}
\item Understand the definition of a coordinate system
\item Understand the definition of a vector and of a scalar
\item Be able to perform algebra with vectors (addition, scalar products, vector products)
\end{learningObjectives}

\section{Coordinate systems}
Coordinate systems are used in order to be able to describe the position of an object in space. A coordinate system is an artificial mathematical tool that we construct in order to describe the position of a real object. 

\subsection{1D Coordinate systems} 
The easiest coordinate system to construct is one where we need to describe the location of objects in one dimensional space. For example, we may wish to describe the location of a train along a straight section of track that runs in the East-West direction. In order to do so, we must first define an ``origin'', which is the reference point of our coordinate system. For example, the origin for our train track may be the Kingston train station. We can describe the position of the train by specifying how far it is from the train station (the origin), using a single real number, say $x_T$. If the train is at position $x_T=0$, then we know that it is at the Kingston station. If the object is not at the origin, then we need to be able to specify on which side (East or West in our train example) of the origin the object is located. We do this by choosing a direction for our one dimensional coordinate $x$. For example, we may choose that the East side of the track corresponds to positive values of $x_T$ and that the West side of the track correspond to the negative values of $x_T$. Thus, in order to fully specify a coordinate system we need to choose:
\begin{itemize}
\item the location of the origin
\item the direction in which the coordinate, $x$, increases
\item the units in which we wish to express $x$
\end{itemize} 

TODO: make figure to illustrate 1D x-axis

In one dimension, it is common to use the variable $x$ to define the position along the ``$x$-axis''. The $x$-axis \textit{is} our coordinate system in one dimension, and we represent it by drawing a line with an arrow in the direction of increasing $x$ and indicate where the origin is located.
 
\subsection{2D Coordinate systems}
\rwcapfig[14]{0.35\textwidth}{figures/AppendixB/xyp.png}{\label{fig:appB:xyp}Example of Cartesian coordinate system and a point $P$ with coordinates $(x_p,y_p)$.}
To describe the position of an object in two dimensions (e.g. a marble rolling on a table), we need to specify two numbers. The easiest way to do this is to define two axes, $x$ and $y$, whose origin and direction we must define. Figure \ref{fig:appB:xyp} shows an example of such a coordinate system. Although it is not necessary to do so, we chose $x$ and $y$ axes that are perpendicular to each other. The origin of the coordinate system is where the two axes intersect. One is free to choose any two directions for the axes (as long as they are not parallel). However, choosing axes that are perpendicular (a ``Cartesian'' coordinate system) is usually the most convenient.

To fully describe the position of an object, we must specify both its position along the $x$ and $y$ axes. For example, point $P$ in Figure \ref{fig:appB:xyp} has two \textbf{coordinates}, $x_p$ and $y_p$ that define its position. The $x$ coordinate is found by drawing a line through $P$ that is parallel to the $y$ axis and is given by the intersection of that line with the $x$ axis. The $y$ coordinate is found by drawing a line through point $P$ that is parallel to the $x$ axis and is given by the intersection of that line with the $y$ axis.


\begin{checkpointMC}{Figure \ref{fig:appB:xyslant} shows a coordinate system that is not orthogonal (where the $x$ and $y$ axes are not perpendicular). Which value on the figure correctly indicates the $y$ coordinate of point $P$?
\capfig{0.35\textwidth}{figures/AppendixB/xyslant.png}{\label{fig:appB:xyslant}A non-orthogonal coordinate system (the $x$ and $y$ axes are not perpendicular).}}
\item $y_1$ %correct
\item $y_2$
\item $y_3$
\end{checkpointMC}
\capfig{0.3\textwidth}{figures/AppendixB/polarp.png}{\label{fig:appB:polarp}Example of a polar coordinate system and a point $P$ with coordinates $(r,\theta)$.}

The most common choice of coordinate system in two dimensions is the Cartesian coordinate system that we just described, where the $x$ and $y$ axes are perpendicular and share a common origin, as shown in Figure \ref{fig:chap4:xyp}. When applicable, by convention, we usually choose the $y$ axis to correspond to the vertical direction.

Another common choice is a ``polar'' coordinate system where the position of an object is specified by a distance to the origin, $r$, and an angle, $\theta$, relative to a specified direction, as shown in Figure \ref{fig:chap4:polarp}. Often, a polar coordinate system is defined alongside a Cartesian system, so that $r$ is the distance to the origin of the Cartesian system and $\theta$ is the angle with respect to the $x$ axis.

One can easily convert between the two Cartesian coordinates, $x,y$, and the two corresponding polar coordinates, $r,\theta$:
\begin{align*}
x&=r\cos(\theta)\\
y&=r\sin(\theta)\\
r&=\sqrt(x^2+y^2)\\
\tan(\theta) &= \frac{y}{x}
\end{align*}
Polar coordinates are often used to describe the motion of an object moving around a circle, as this means that only one of the coordinates ($\theta$) changes with time (if the origin of the coordinate system is chosen to coincide with the centre of the circle).

\subsection{3D Coordinate systems}
In three dimensions, we need to specify three numbers to describe the position of an object (e.g. a bird flying in the air). In a three dimensional Cartesian coordinate system, we simply add a third axis, $z$, that is mutually perpendicular to both $x$ and $y$. The position of an object can then specified using the three coordinates, $x$, $y$, and $z$. 

Two additional coordinate systems are common in three dimensions: ``cylindrical'' and ``spherical coordinates''. All three systems are illustrated in Figure \ref{fig:appB:3dcoords} superimposed onto the Cartesian system.
\capfig{0.85\textwidth}{figures/AppendixB/3dcoords.png}{\label{fig:appB:3dcoords} Cartesian (left), cylindrical (centre) and spherical (right) coordinate systems used in three dimensions. The $y$ and $z$ axes are in the plane of the page, whereas the $x$ axis comes out of the page.}

By convention, we use the $z$ axis to be the vertical direction in three dimensions. In cylindrical coordinates, we keep the same Cartesian coordinate $z$ to indicate the height above the $xy$ plane. However, we use the \textit{azimuthal angle}, $\phi$, and the radius, $\rho$, to describe the position of the projection of a point onto the $xy$ plane. $\phi$ is the angle that the projected point makes with the $x$ axis and $\rho$ is the distance of that projected point to the origin. Thus, cylindrical coordinates are very similar to the polar coordinate system introduced in two dimensions, except with the addition of the $z$ coordinate. Cylindrical coordinates are useful for describing situations with azimuthal symmetry, such as motion along the surface of a cylinder. The cylindrical coordinates are related to the Cartesian coordinates by:
\begin{align*}
\rho &= \sqrt{x^2+y^2}\\
\tan(\phi) &= \frac{y}{x}\\
z&=z
\end{align*}
In spherical coordinates, a point $P$ is described by the radius, $r$, the \textit{polar angle} $\theta$, and the \textit{azimuthal angle}, $\phi$. The radius is the distance between the point and the origin. The polar angle is the angle with the $z$ axis that is made by the line from the origin to the point. The azimuthal angle is defined in the same way as in polar coordinates. Spherical coordinates are useful for describing situations that have spherical symmetry, such as a person walking on the surface of the Earth. The spherical coordinates are related to the Cartesian coordinates by:
\begin{align*}
r &= \sqrt{x^2+y^2+z^2}\\
\tan(\theta) &= \frac{z}{r}=\frac{z}{\sqrt{x^2+y^2+z^2}}\\
\tan(\phi) &= \frac{y}{x}\\
\end{align*}

\section{Vectors}
So far, we have seen how to use a coordinate system to describe the position of a single point in space relative to an origin. In this section, we introduce the notion of a ``vector'', which allows us to describe quantities that have a \textbf{magnitude} and a \textbf{direction}. For example, you can use a vector to describe the fact that you walked \SI{5}{km} in the North direction. A vector can be visualized by an arrow. The length of the arrow is the magnitude that we wish to describe, and the direction of the arrow corresponds to the direction that we would like to describe. 

Unlike a point in space, vectors \textbf{have no location}. That is, vectors are simply an arrow, and you can choose to draw that arrow anywhere you like. In two dimensional space, one requires two numbers to completely define a vector. In three dimensional space, one requires three numbers to completely define a vector. Figure \ref{fig:appB:dvec} shows a two dimensional vector, $\vec d$, twice. Because both arrows in the figure have the same magnitude and direction, they represent the \textit{same} vector. When we refer to quantities that are vectors, we usually draw an arrow on top of the quantity ($\vec d$) to indicate that they are vectors. We use the word ``scalar'' to refer to numbers that are not vectors (a regular number is thus also called a scalar to distinguish it from a quantity that is a vector).

\capfig{0.35\textwidth}{figures/AppendixB/dvec.png}{\label{fig:appB:dvec}A vector $\vec d$ shown twice, once with its Cartesian components ($d_x$, $d_y$) and once with its magnitude and direction ($d$, $\phi$).}

In analogy with coordinate systems, we have multiple ways to choose the numbers that we use to define the vector. The most convenient choice is usually to use the ``Cartesian components'' of the vector which correspond to the length of the vector when projected onto a Cartesian coordinate system. For example, in Figure \ref{fig:appB:dvec}, the Cartesian components of the vector $\vec d$ are labelled as ($d_x$, $d_y$) indicating that the vector has a length of $d_x$ in the $x$ direction and $d_y$ in the $y$ direction. Furthermore, the number $d_x$ is negative, since the vector points in the negative $x$ direction. Another common choice is to use the length of the vector, which we label $d$ (the name of the vector without the arrow on top), and the angle, $\phi$ that the vector makes with the $x$-axis, as illustrated in Figure \ref{fig:appB:dvec}. In terms of the Cartesian components, the magnitude of the vector is given by:
\begin{align*}
d&= ||\vec d||= \sqrt{d_x^2+d_y^2}
\end{align*}
where we also introduced the notation that placing two vertical bars around a vector ($||\vec d||$) is used to indicated its magnitude.


\subsection{Unit vectors} 
A special category of vectors is ``unit vectors'', which are simply vectors that have a length (magnitude) of 1 (in whichever units the coordinate system is defined). Unit vectors are particularly useful for indicating direction. For example, in Figure \ref{fig:appB:dvec}, we may be interested in indicating the direction of the vector $\vec d$. Unit vectors are denoted by using a ``hat'' instead of an arrow. Thus, the vector $\hat d$, is the vector of length 1 that points in the same direction as $\vec d$. The (Cartesian) components of $\hat d$ are easily found by dividing the corresponding components of $\vec d$ by $d$ (the magnitude):
\begin{align*}
(\hat d)_x &= \frac{d_x}{d}=\frac{d_x}{\sqrt{d_x^2+d_y^2}}\\
(\hat d)_y &= \frac{d_y}{d}=\frac{d_y}{\sqrt{d_x^2+d_y^2}}\\
\therefore ||\hat d||&=\sqrt{(\hat d)_x^2+(\hat d)_y^2}=\sqrt{\frac{d_x^2}{d_x^2+d_y^2}+\frac{d_y^2}{d_x^2+d_y^2}}=1
\end{align*}

A specific type of unit vectors are those units vectors that are parallel to the axes of the coordinate system. Those vectors are denoted $\hat x$, $\hat y$, $\hat z$ (and sometimes $\hat i$, $\hat j$, $\hat k$ or $\hat e_x$, $\hat e_y$, $\hat e_z$) for the $x$, $y$, and $z$ axes, respectively. 

\subsection{Notations for vectors}
There are multiple notations for describing a vector using its components. The following, are all equivalent ways to write down the vector $\vec d$ in terms of its components $d_x$ and $d_y$:
\begin{align*}
\vec d &= (d_x,d_y)\\
       &=\begin{pmatrix}
           d_x \\
           d_y \\
         \end{pmatrix}\\
         &= d_x\hat x +d_y \hat y\\
         &=d_x\hat i +d_y \hat j 
\end{align*}
For example, the unit vector $\hat y$ can be written down as (0,1,0) in three dimensions. 

\begin{checkpointMC}{What is the magnitude (the length) of the vector $5\hat x-2\hat y$?}
\item 3.0
\item 5.4% correct
\item 7.0
\item 10.0
\end{checkpointMC}

\section{Vector algebra}
In this section, we describe the various algebraic operations that can be performed using vectors. 
\subsection{Multiplication/division of a vector by a scalar}
One can multiply (or divide) a vector by a scalar (a number). Suppose that we are given a vector $\vec v=(v_c, v_y, v_z)$ and a scalar $a$. The multiplication $a\vec v$ is defined to be a new vector, say $\vec w$, whose components are the components of $\vec v$ multiplied by $a$:
\begin{align*}
\vec w = a\vec v = (av_x, a v_y)
\end{align*}
Similarly, the division of a vector by a scalar is defined analogously:
\begin{align*}
\vec w = \frac{\vec v}{a} = \left(\frac{v_x}{a}, \frac{v_y}{a}\right)
\end{align*}
\begin{checkpointMC}{What happens to the length of a vector if the vector is multiplied by 2?}
\item The length doubles% correct
\item The length is halved
\item The length is quadrupled
\item It depends on the direction of the vector
\end{checkpointMC}

In particular, this makes it easy to determine the unit vector, $\hat v$, that points in the same direction as $\vec v$:
\begin{align*}
\hat v = \frac{\vec v}{v}
\end{align*}
where $v$ is the magnitude of $\vec v$. 

\subsection{Addition/subtraction of two vectors}
The addition (subtraction) of two vectors, $\vec a$ and $\vec b$, is found by adding (subtracting) the components of the two vectors. For example, if $\vec c=\vec a+\vec b$, the components of $\vec c$ are given by:
\begin{align*}
\vec c &= \vec a + \vec b = \begin{pmatrix}
           a_x \\
           a_y \\
         \end{pmatrix} + \begin{pmatrix}
           b_x \\
           b_y \\
         \end{pmatrix}\\
         &=\begin{pmatrix}
           a_x+b_x \\
           a_y+b_y \\
         \end{pmatrix}
\end{align*}
where we chose to use the ``column vector'' notation. The column vector notation highlights the fact that the algebra (addition, subtraction) is performed independently on the $x$ and $y$ components. 
\begin{example}{Given two vectors, $\vec a=2\hat x+3\hat y$, and $\vec b=5\hat x-2\hat y$, calculate the vector $\vec c= 2\vec a- 3\vec b$.}
This can easily be solved algebraically:
\begin{align*}
\vec c &= 2\vec a- 3\vec b\\
&=2 (2\hat x+3\hat y) - 3 (5\hat x-2\hat y) \\
&=(4\hat x+6\hat y)-(15\hat x-6\hat y) \\
&=(4-15)\hat x + (6+6) \hat y\\
&= -11 \hat x + 12 \hat y
\end{align*}
We can think of these operations as being performed independently on the components:
\begin{align*}
c_x&=2a_x-3b_x=-11\\
c_y&=2a_y-3b_y=12
\end{align*} 
\end{example}

Geometrically, one can easily visualize the addition and subtraction of vectors. This is illustrated in Figure \ref{fig:appB:aplusbvec} for the case of adding vectors $\vec a$ and $\vec b$ to get the vector $\vec c$. Geometrically, the sum of the vectors $\vec a$ and $\vec b$ (sometimes also called the ``resultant'') can be found by:
\begin{enumerate}
\item Placing the ``tail'' of vector $\vec b$ at the ``head'' of $\vec a$ (think of an arrow, the pointy part is the head and the feathery part is the tail)
\item Drawing the vector goes from the tail of vector $\vec a$ to the head of vector $\vec b$.
\end{enumerate}

\capfig{0.55\textwidth}{figures/AppendixB/aplusbvec.png}{\label{fig:appB:aplusbvec}Geometric addition of the vectors $\vec a$ and $\vec b$ by placing them ``head to tail''.}

Subtracting two vectors geometrically is done in the same way as addition. For example, the vector $vec c$, given by $\vec c=\vec a -\vec b$ can also be expressed as $\vec c = \vec a + (-1) \vec b$. That is, first multiply the vector $\vec b$ by minus 1 (which just reverses its direction), then add that vector, ``head to tail'', to the vector $\vec a$. 

Now that we know how to add vectors, we can better understand the notation $\vec a = a_x \hat x+ a_y\hat y$. This is not simply a notation, but is in fact algebraically correct. It means: ``multiply the vector $\hat x$ by $a_x$ (thus giving it a length of $a_x$) and then add $a_y$ times the vector $\hat y$''. This is illustrated in Figure \ref{fig:appB:acomponents}.

\capfig{0.35\textwidth}{figures/AppendixB/acomponents.png}{\label{fig:appB:acomponents}Illustration that the notation $\vec a = a_x \hat x+ a_y\hat y$ is in fact the vector addition of $a_x \hat x$ and $a_y \hat y$.}


\subsection{The scalar product}
There are two ways to ``multiply'' vectors: the ``scalar product'' and the ``vector product''. The scalar product (or ``dot product'') takes two vectors and results in a scalar (a number). The vector product (or ``cros product'') takes two vectors and results in a third vector. 

The scalar product, $\vec a \cdot \vec b$, of two vectors $\vec a$ and $\vec b$, is defined as the following:
\begin{align*}
\vec a \cdot \vec b=a_xb_x +a_yb_y
\end{align*}
That is, one multiplies the individual components of the two vectors and then adds those products for each component. This is easily extended to the three dimensional case by adding a term $a_zb_z$ to the sum. One can easily show that the scalar product is also related to the angle between the two vectors when these are placed ``tail to tail'', as in Figure \ref{fig:appB:scalarproduct}
\begin{align*}
\vec a \cdot \vec b= ab\cos\theta
\end{align*}

\capfig{0.3\textwidth}{figures/AppendixB/scalarproduct.png}{\label{fig:appB:scalarproduct}Illustration of the angle between vectors $\vec a$ and $\vec b$ when these are placed tail to tail.}

The scalar product between two vectors of a fixed length will be maximal when the two vectors are parallel ($\cos\theta=1$) and zero when the vectors are perpendicular ($\cos\theta =0$). The scalar product is thus useful when we want to calculate quantities that are maximal when two vectors are parallel. 


\subsection{The vector product}
The vector (or cross) product takes two vectors to produce a third vector that is \textbf{mutually perpendicular} to both vectors. The vector product only has meaning in three dimensions. Two vectors that are not co-linear can always be used to define a plane in three dimensions. The cross product of those two vectors will give a third vector that is thus perpendicular to the plane (thus making it perpendicular to both vectors). 

Algebraically, the three components of the vector product, $\vec a\times \vec b$, of vectors $\vec a$ and $\vec b$ are found as follows:
\begin{align}
\label{eqn:appB:crossproduct}
\vec a \times \vec b =\begin{pmatrix}
           a_yb_z - a_z b_y\\
           a_zb_x - a_x b_z\\
           a_xb_y - a_y b_x\\
         \end{pmatrix}
\end{align}

One important property to note is that $\vec a \times \vec b = -\vec b \times \vec a$; that is, the cross product is not commutative (the order matters). The magnitude of the vector obtained by a cross product is given by:
\begin{align}
\label{eqn:appB:crossproductmag}
||\vec a \times \vec b ||=ab\sin\theta
\end{align}
where $\theta$ is the angle between the vectors $\vec a$ and $\vec b$ when these are placed tail to tail (Figure \ref{fig:appB:scalarproduct}). The vector resulting from a cross product will be null (have a zero length) if the vectors $\vec a$ and $\vec b$ are parallel, and will have a maximal length when these are perpendicular. The cross product is thus useful to determine quantities that are maximal when two vectors are perpendicular (the opposite use case from the scalar product). 

Geometrically, one can determine the direction of the cross product of two vectors by using the ``right hand rule''. This is done by using your right hand, aligning your thumb with the first vector, your index with the second vector, and the cross product will point in the direction of your middle finger (when you hold your middle finger perpendicular to the other two fingers). This is illustrated in Figure TODO. Thus, you can often avoid using equation \ref{eqn:appB:crossproduct} and instead use the right hand rule and equation \ref{eqn:appB:crossproductmag} to find the vector resulting from a cross product.

TODO: make a figure for the right hand rule

The unit vectors that define a coordinate system have the following properties relative to the cross product:
\begin{align*}
\vec x \times \vec y &= \vec z\\
\vec y \times \vec z &= \vec x\\
\vec z \times \vec x &= \vec y\\
\end{align*}
For these properties to be correct, it should be noted that the direction of the $z$ axis in three dimensions is specified by the choice of $x$ and $y$ axes. That is, one can freely choose the direction of the $x$ and $y$ axes, which then define a plane to which the $z$ axis will be perpendicular. The direction of the $z$ axis must be chosen so that $\vec x \times \vec y = \vec z$ (this guarantees that the coordinate system is ``right handed''). 








%%%%%%%%%%%%%%%%%%%%%%%% Old - cut exept part to relate to physics.
\section{Position and displacement vectors}
\rwcapfig[11]{0.35\textwidth}{figures/AppendixB/xydvec.png}{\label{fig:appB:xydvec}Example of a displacement vector, $\vec d$, from point $P_1$ to point $P_2$.}
When we describe the motion of an object in two dimensions, we must specify two coordinates that can change with time. In Cartesian coordinates, this means that we must specify two functions, $x(t)$ and $y(t)$, to fully describe the motion of an object. When an object moves from one position to another, we need a way to describe that \textit{displacement}. To do this, we use \textbf{vectors}, which can be represented as arrows.

Figure \ref{fig:appB:xydvec} shows an example of a displacement vector, $\vec d$, to represent the motion from $P_1$ to $P_2$. A vector has two important features:
\begin{enumerate}
\item a direction (represented by the direction of the arrow)
\item a magnitude (represented by the length of the arrow)
\end{enumerate}
We specify that a variable is a vector by drawing a little arrow on top of the variable. An important feature is that a vector is not fixed anywhere in space. In Figure \ref{fig:appB:xydvec}, we drew the vector between points $P_1$ and $P_2$ for illustration. The vector $\vec d$ only represents a change in position in a certain direction and of a certain distance and is not tied to the specific points $P_1$ and $P_2$. One could have the \textit{same} displacement vector, $\vec d$, between two other points, $P_3$ and $P_4$, if the displacement is in the same direction and over the same distance. 

All we need to specify a vector in two dimension are two \textbf{components}. In Cartesian coordinates, we can specify the length of the vector projected onto the $x$ and $y$ axes ($\Delta x$ and $\Delta y$ in Figure \ref{fig:appB:xydvec}), its $x$ and $y$ components. In polar coordinates, we could specify the total magnitude of the vector (its length, $r$) and the angle that it makes with the $x$ axis, $\theta$.

Several, equivalent, notations can be used for specifying a vector in Cartesian coordinates:
\begin{align*}
\vec d &= \begin{pmatrix}
           d_x \\
           d_y \\
         \end{pmatrix}\\
         &= d_x\hat x +d_y \hat y\\
         &=d_x\hat i +d_y \hat j
\end{align*}
where $d_x$ and $d_y$ are the $x$ and $y$ components of the vector $\vec d$, respectively. In Figure \ref{fig:chap4:xydvec}, we have $d_x=\Delta x$ and $d_y=\Delta y$. $\hat x$ and $\hat y$ are called ``unit vectors'' for the $x$ and $y$ axes, respectively. $\hat i$ and $\hat j$ are equivalent to $\hat x$ and $\hat y$.  Unit vectors are vectors that have a length of 1, and are usually denoted with a hat\footnote{It's actually called a ``caret'' not a hat!} (\^{}) instead of an arrow. The $\hat x$ and $\hat y$ are special unit vectors as they are parallel to the $x$ and $y$ axes, respectively. In vector notation, they are written:
\begin{align*}
\hat x &= \begin{pmatrix}
           1\\
           0 \\
         \end{pmatrix}\\
\hat y &= \begin{pmatrix}
           0\\
           1 \\
         \end{pmatrix}\\       
\end{align*}
Any vector can be made to have a unit length, if you divide its components by its magnitude. For example, given a vector with components $(d_x,d_y)$, the corresponding unit vector is:
\begin{align*}
\hat d &= \frac{\vec d}{||\vec d||}\\
       &= \frac{\vec d}{\sqrt{d_x^2+d_y^2}}\\
       &= \frac{d_x}{\sqrt{d_x^2+d_y^2}}\hat x+\frac{d_y}{\sqrt{d_x^2+d_y^2}}\hat y
\end{align*}
where $||\vec d||$ is the magnitude of $\vec d$ and is given by Pythagoras' theorem. If it is clear that $\vec d$ is a vector, then one can also denote its magnitude by $d$ (without the arrow), although this notation should only be used if it is clear that you are referring to the magnitude of a vector, and not some other variable.

Note that in the first line above, we implied that we \textbf{can divide a vector by a number}. Dividing (or multiplying) a vector by a number means individually dividing (or multiplying) the components of the vector by that number and gives a new vector. If you multiply a vector by (-1), then you get a new vector of the same magnitude but that points in the opposite direction. When working with vectors, we use the term ``scalar'' to refer to a quantity that is not a vector (i.e. a regular number is called a scalar). Thus, we have just examined how we can multiply a vector by a scalar to obtain a vector \footnote{Note that this is \textbf{not} called a scalar product.}. Also note that multiplying a vector by a scalar ultimately multiplies the length of the vector by that scalar. Thus, multiplying a vector by 2 makes it twice as long and in the same direction.

Referring to Figure \ref{fig:appB:xydvec}, we have introduced the displacement vector, $\vec d$, from the point $P_1$ to the point $P_2$. We can also define ``position vectors'' corresponding to the points $P_1$ and $P_2$. The position vector for a point is the displacement vector corresponding to going from the origin to that point (and can be drawn as an arrow that goes from the origin to the point). The components of the position vector for a point are the same as the coordinates of that point. Often, the letter $r$ is used for position vectors. For $P_1$ and $P_2$, the position vectors, $\vec r_1$ and $\vec r_2$, respectively, are given by:
\begin{align*}
\vec r_1&=\begin{pmatrix}
           x_1 \\
           y_1 \\
         \end{pmatrix}\\
\vec r_2&=\begin{pmatrix}
           x_2 \\
           y_2 \\
         \end{pmatrix}\\
\end{align*}
\rwcapfig[10]{0.35\textwidth}{figures/AppendixB/vecadd.png}{\label{fig:appB:vecadd}Example of adding the displacement vector $\vec d$ to the position vector $\vec r_1$ to obtain the position vector $\vec r_2=\vec r_1+\vec d$.}
The displacement vector, $\vec d$, from $P_1$ to $P_2$, has the following components:
\begin{align*}
\vec{d}&=\begin{pmatrix}
           x_2-x_1 \\
           y_2-y_1 \\
         \end{pmatrix}\\
\end{align*}
This suggests that \textbf{we can define addition (and subtraction) of two vectors}:
\begin{align*}
\vec{d}&=\begin{pmatrix}
           x_2-x_1 \\
           y_2-y_1 \\
         \end{pmatrix}=\begin{pmatrix}
           x_2 \\
           y_2 \\
         \end{pmatrix}-\begin{pmatrix}
           x_1 \\
           y_1 \\
         \end{pmatrix}\\
         \vec d&=\vec r_2-\vec r_1\\
         \therefore \vec r_2&=\vec r_1+\vec d
\end{align*}
where adding (or subtracting) two vectors means to add (or subtract) each component to obtain the corresponding components of the new vector. In the last line, we re-arranged the equation to show that the position vector $\vec r_2$ is the sum of $\vec r_1$ and $\vec d$. This highlights a geometrical aspect of vector addition, shown in Figure \ref{fig:appB:vecadd}. When you want to add vectors, simply draw the two vectors ``head to tail'' to get the sum vector (from the tail of the first vector to the head of the last vector).

\begin{example}{Given two vectors, $\vec a=2\hat x+3\hat y$, and $\vec b=5\hat x-2\hat y$, calculate the vector $\vec c= 2\vec a- 3\vec b$.}
This can easily be solved algebraically:
\begin{align*}
\vec c &= 2\vec a- 3\vec b\\
&=2 (2\hat x+3\hat y) - 3 (5\hat x-2\hat y) \\
&=(4\hat x+6\hat y)-(15\hat x-6\hat y) \\
&=(4-15)\hat x + (6+6) \hat y\\
&= -9 \hat x + 12 \hat y
\end{align*}
\end{example}

