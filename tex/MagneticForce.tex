\section{Magnetic fields and forces}

%%%%%%%%%%%%%%%%%%%%%%%%%%%%%%%%%%%
%%
%% Multiple Choice
%%
%%%%%%%%%%%%%%%%%%%%%%%%%%%%%%%%%%%
\subsection{Multiple Choice}

%Question submitted by Madison Facchini
\question Which of the following is FALSE concerning the properties of magnetic field lines?
\begin{checkboxes}
\choice The field lines cannot cross
\choice The field strength is proportional to the line density
\choice The field is tangent to the magnetic field line
\CorrectChoice The field lines can form open loops \correct
\end{checkboxes}

\question Two vertical infinitely long wires are separated by a distance of \SI{1}{m}. One wire carries an upwards current of \SI{2}{A}, whereas the other wire caries a downward current of \SI{3}{A}. What is the strength of the magnetic field midway between the wires?
\begin{checkboxes}
	\choice \SI{9.6e-13}{T}
	\choice \SI{4.0e-7}{T}
	\CorrectChoice \SI{2.0e-6}{T}
	\choice \SI{0.7}{T}
\end{checkboxes}

\question What is the minimum magnitude of the magnetic field required to maintain electrons (mass, $m_e=\SI{9.11e-31}{kg}$, and charge $q=\SI{1.6e-19}{C}$) with a speed of $\SI{1.5e8}{m/s}$ in a circular orbit of radius $\SI{1.0}{m}$?
\begin{checkboxes}
	\choice \SI{1.7e-3}{T}
	\CorrectChoice \SI{8.5e-4}{T}
	\choice \SI{5.9e3}{T}
	\choice \SI{1.3e5}{T}
\end{checkboxes}

%Neil Rajan
\question An electron is moving West through a magnetic field that points North. The direction of the force on the electron due to that field is: 
\begin{checkboxes}
\CorrectChoice Up 
\choice Down
\choice East
\choice South
\end{checkboxes}

%Ashley Braun
\question A charged particle in a magnetic field will undergo helical motion if
\begin{checkboxes}
\choice it has component of velocity parallel to the field 
\choice it has component of velocity perpendicular to the field 
\CorrectChoice All of the above
\choice None of the above
\end{checkboxes}

%Question submitted by Quentin Sanders - Check that not plagiarized!
\question If a moving charged particle enters a region with a magnetic field perpendicular to the direction of motion and begins to follow a curved path, which of the following sets of conditions will result in the largest radius?
\begin{checkboxes}
\CorrectChoice Mass $m$, speed $v$, charge $q$, magnetic field $B$ \correct
\choice Mass $2m$, speed $v$, charge $q$, magnetic field $4B$
\choice Mass $2m$, speed $v$, charge $4q$, magnetic field $B$
\choice Mass $m$, speed $2v$, charge $2q$, magnetic field $2B$
\end{checkboxes}

%Question submitted by Joanna Fu
\question Which of the three following statements about magnetic fields is false?
\begin{checkboxes}
\choice Magnetic field lines can never cross one another
\choice Magnetic field lines, unlike electric field lines, never end
\choice A charged particle travelling anti-parallel to a magnetic field does not experience a force from that field.
\CorrectChoice None of the three statements are false \correct
\choice All three statements are false
\end{checkboxes}

\question A segment of wire carries current, $I=0.7$\,A. One end of the segment is at the origin of the coordinate system and the other end is at a point $P=(20,30,10)$\,cm. The current is flowing from the origin towards point $P$. A uniform magnetic field, $\vec B=(0,0.5,0.6)$\,T, is present. What is the force vector on the wire? 
\begin{checkboxes}
\choice $\vec F=(0.084, -0.091, 0.07)\,N$
\CorrectChoice $\vec F=(0.091, -0.084, 0.07)\,N$ \correct
\choice $\vec F=(-0.084, 0.091, 0.07)\,N$
\choice $\vec F=(0.091, -0.084, -0.07)\,N$
\end{checkboxes}
\begin{solution}
\begin{align*}
\vec F&= I \vec l \times \vec B=(0.7A) (0.2\hat x+0.3\hat y+0.1\hat z)\times (0.5 \hat y+0.6 \hat z)\\
      &= (0.7A) [ (0.2m)(0.5T)(\hat x \times \hat y)+ (0.2m)(0.6T) (\hat x \times \hat z)\\
      &+ (0.3m)(0.6T) (\hat y\times \hat z) +(0.1m)(0.5T)(\hat z \times \hat y) ]\\
      &=(0.7A) [ (0.1mT)(\hat z)+ (0.12mT) (-\hat y)+ (0.18mT) (\hat x) +(0.05mT)(-\hat x) ]\\
      &=(0.7A) [ (0.13mT) (\hat x) - (0.12mT) (\hat y) +(0.1mT)(\hat z) ]   \\
      &=(0.091, -0.084, 0.07)\,N 
\end{align*}
\end{solution}

\question What velocity vector must an electron ($m=9.11\times 10^{-31}$\,kg, $q=-1.6\times 10^{-19}$\,C) have in order to perform uniform circular motion with a diameter $D=0.4$\,m after entering a region where the magnetic field is uniform and given by $B=1.0\times 10^{-4}$\,T in the positive z-direction?
\begin{checkboxes}
\choice $(0,0,3.51)\times 10^{6}$\,m/s 
\choice $(0,7.03,0)\times 10^{6}$\,m/s 
\CorrectChoice $(3.51,0,0)\times 10^{6}$\,m/s \correct
\choice $(0,0,7.03)\times 10^{6}$\,m/s 
\end{checkboxes}
\begin{solution}
\begin{align*}
\frac{mv^2}{R}&=qvB\\
v&=\frac{qRB}{m}=\frac{(1.6\times 10^{-19}\,C)(0.2\,m)(1.0\times 10^{-4}\,T)}{9.11\times 10^{-31}\,kg}\\
\end{align*}
The velocity has to be in the x-y plane, only one option in the xy-plane has the right magnitude.
\end{solution}

\question An electron is initially moving in the positive x-direction, when suddenly a uniform magnetic field is turned on. The electron is observed to be deflected in the negative y-direction. In which direction is the magnetic field?
\begin{checkboxes}
\choice positive x-direction
\choice negative x-direction
\choice positive y-direction
\choice negative y-direction
\choice positive z-direction
\CorrectChoice negative z-direction \correct
\end{checkboxes}
\newpage
%%%%%%%%%%%%%%%%%%%%%%%%%%%%%%%%%%%
%
% long answer
%
%%%%%%%%%%%%%%%%%%%%%%%%%%%%%%%%%%%
\subsection{Long answers}
%Young and Freedman (modified)
\question A particle with a charge of $q=\SI{-1.24e-8}{C}$ is moving with instantaneous velocity $\vec{v}=(\SI{4.19e4}{m/s})\hat{x}+(\SI{-3.85e4}{m/s})\hat{y}$. What is the force vector exerted on this particle by a magnetic field at that instant, if:
\begin{parts}
\part $\vec{B}=(\SI{1.40}{T})\hat{x}$ 
\part $\vec{B}=(\SI{1.40}{T})\hat{x}+(\SI{1.40}{T})\hat{y}+(\SI{1.40}{T})\hat{z}$?
\end{parts}
\begin{finalanswer}
\begin{enumerate}[(a)]
\item \begin{align*}
\vec{F}=(\SI{-6.69e-4}{N})\hat{z}
\end{align*}
\item  \begin{align*}
\vec{F}=(\SI{6.69e-4}{N})\hat x+ (\SI{7.27e-4}{N})\hat y -(\SI{13.96e-4}{N})\hat z
\end{align*}
\end{enumerate}
\end{finalanswer}
\begin{solution}
\begin{parts}
\part We simply use the Lorentz force equation:
\begin{align*}
\vec{F}&=q\vec{v} \times \vec{B}\\
     &=q [(v_x\hat x+v_y\hat y)\times(B_x\hat x)]\\
     &=q [v_xB_x(\hat x\times \hat x)+v_yB_x(\hat y\times\hat x)]\\
     &=q v_yB_x(\hat y\times\hat x)\\
     &=q v_yB_x\hat z\\
	   &=(\SI{-1.24e-8}{C})(\SI{-3.85e4}{m/s})(\SI{1.40}{T})\hat z\\
	   &=(\SI{-6.69e-4}{N})\hat{z}
\end{align*}

\part Again, using the Lorentz force equation:
\begin{align*}
\vec{F}&=q\vec{v} \times \vec{B}\\
	   &=q[(v_x\hat x+v_y\hat y)\times(B_x\hat x+B_y\hat y+B_z\hat z)]\\
	   &=q[v_xB_x(\hat x \times \hat x)+v_xB_y(\hat x\times \hat y)+v_xB_z(\hat x\times \hat z)]\\
	   &+q[v_yB_x(\hat y \times \hat x)+v_yB_y(\hat y\times \hat y)+v_yB_z(\hat y\times \hat z)]\\
	   &=q[v_xB_y(\hat x\times \hat y)+v_xB_z(\hat x\times \hat z)]\\
	   &+q[v_yB_x(\hat y \times \hat x)+v_yB_z(\hat y\times \hat z)]\\
	   &=q[v_xB_y(\hat z)+v_xB_z(-\hat y)]\\
	   &+q[v_yB_x(-\hat z)+v_yB_z(\hat x)]\\
	   &=q\left[v_yB_z\hat x-v_xB_z\hat y+(v_xB_y-v_yB_x)\hat z\right]\\
	   &=qB_0\left[v_y\hat x-v_x\hat y+(v_x-v_y)\hat z\right]\\
	   &=(\SI{6.69e-4}{N})\hat x+ (\SI{7.27e-4}{N})\hat y -(\SI{13.96e-4}{N})\hat z\\
\end{align*}
\end{parts}
\end{solution}


%Young and Freedman (modified)
\question The magnetic flux through one face of a cube is \SI{0.120}{Wb}. 
\begin{parts}
\part What must be the total magnetic flux through the other five faces of the cube?
\part Would this change if it was not a cube, but a rectangular box?
\part What is the equivalent of Gauss' Law for the magnetic field?
\end{parts}
\begin{finalanswer}
\begin{enumerate}[(a)]
\item The flux through the other 5 faces of the cube is \SI{-0.120}{Wb}.
\item No, this would not make a difference.
\item $\oint \vec B\cdot d\vec A=0$
\end{enumerate}
\end{finalanswer}
\begin{solution}
\begin{parts}
\part  Magnetic field lines are closed loops, so the net flux out of any closed surface must be zero. Since the total flux is zero, the sum of the flux through all faces of the cube must add up to zero. The flux through the other 5 faces of the cube is \SI{-0.120}{Wb}.
\part No this would not make a difference. The shape does not matter. If you are given the flux through one area of an enclosed surface, then the flux through the rest of the area will be the negative.
\part Since there are no magnetic charges, for the magnetic field, we would write:
\begin{align*}
\oint \vec B\cdot d\vec A=0
\end{align*}
\end{parts}
\end{solution}

%Young and Freedman (heavily modified)
\question An electron moves at \SI{2.50e6}{m/s} through a region in which there is a magnetic field of unspecified direction and magnitude \SI{8.40e-2}{T}.
\begin{parts}
\part What are the largest and smallest possible magnitude of the acceleration of the electron due to the magnetic field?
\part If the actual acceleration of the electron is one-fourth of the largest possible, what is the angle between the direction the electron is moving and the magnetic field?
\part For the situation described in part (b), the electron will travel in a helix. What is the radius of the helix, and what is the pitch of the helix (the distance between the ``coils'' in the helix)?
\end{parts}
\textbf{Note:} The charge and mass of an electron are $q=\SI{1.6e-19}{C}$ and $m_e=\SI{9.11e-31}{kg}$.
\begin{finalanswer}
\begin{enumerate}[(a)]
\item Smallest: \SI{0}{m/s^2}, Largest: \SI{3.688e16}{m/s^2}
\item \SI{14.5}{\degree}
\item Radius: $r=\SI{4.236e-5}{m}$, Pitch: $p=\SI{1.03}{mm}$
\end{enumerate}
\end{finalanswer}
\begin{solution} 
\begin{parts}
\part If the direction of motion is parallel to the magnetic field, the acceleration will be zero. This is the smallest possible magnitude.

The greatest acceleration will occur if the magnetic field is perpendicular to the direction of motion. Thus:
\begin{align*}
F&=qvB\sin \theta\\
m_ea&=qvB\sin 90\\
a&=\frac{qvB}{m_e}\\
a&=\frac{(\SI{1.6e-19}{C})(\SI{2.50e6}{m/s})(\SI{8.40e-2}{T})}{(\SI{9.11e-31}{kg})}
a&=\SI{3.688e16}{m/s^2}.
\end{align*}

\part We need to find the angle $\theta$ such that the $\sin\theta$ term is equal to one quarter:
\begin{align*}
\sin \theta& = \frac{1}{4}\\
\theta &= \sin^{-1}{\frac{1}{4}}\\
\theta &= \SI{14.5}{\degree}
\end{align*}

\part We know from parts (a) and (b) that the acceleration from the magnetic field is:
\begin{align*}
a=\frac{1}{4}(\SI{3.688e16}{m/s^2})=\SI{9.221e15}{m/s^2}
\end{align*}
This corresponds to the centripetal acceleration of the electron:
\begin{align*}
a=\frac{v_\perp^2}{r}
\end{align*}
where $v_\perp=v\sin\theta$ is the component of the velocity that is perpendicular to the magnetic field (and thus the component that is tangent to the circles in the helix). The radius is thus given by:
\begin{align*}
r&=\frac{v_\perp^2}{a}=\frac{v^2\sin^2\theta}{a}=\frac{v^2}{16a}\\
&=\frac{(\SI{2.50e6}{m/s})^2}{16(\SI{9.221e15}{m/s^2})}\\
&=\SI{4.236e-5}{m}
\end{align*}
%%%% Fix from here!!!!
The pitch is given by the distance travelled in the direction parallel to the magnetic field over one period of rotation. The period of rotation is:
\begin{align*}
T&=\frac{2\pi r}{v_\perp}=\frac{8\pi r}{v}
\end{align*}
The pitch, $p$, is thus:
\begin{align*}
p&=v_\parallel T=v\cos(\SI{14.5}{\degree})T=v\cos(\SI{14.5}{\degree})\frac{8\pi r}{v}\\
&=8\pi\cos(\SI{14.5}{\degree}) (\SI{4.236e-5}{m})\\
&=\SI{1.03}{mm}
\end{align*}
Note that in the above equation, although the speed cancelled, the radius depends on the speed, so the pitch is not independent of speed.
\end{parts}

\end{solution}


%Giancolli 27-40 -fixed
\question A magician's trick requires the creation of a magnetic dipole moment. The magician decides to create a magic wand of length $d$ which has a uniformly distributed static charge, $Q$, as shown in Figure \ref{fig:magneticforce:rod}. The magician holds the wand by one of its tips and begins twirling it with an angular velocity of $\omega$. Show that the magnetic dipole moment of the magician's wand is given by:
\begin{align*}
\mu=\frac{1}{6}Q\omega d^2
\end{align*}
\capfig{0.45\textwidth}{figures/MagneticForce/rod.png}{\label{fig:magneticforce:rod}A charged rod rotating about an axis perpendicular to the page.}
\begin{solution}
We break up the rod into small piece of length $dr$ and charge $dq=dr\frac{Q}{d}$. Each little segment of rod completes a circle of radius $r$ where $r$ is the distance between the rotation axis and the small piece of charge. 

It take an amount of time $T=\frac{2\pi}{\omega}$ for any segment to complete a full revolution. Thus, the current from one element of charge is given by:
\begin{align*}
dI=\frac{dq}{T}=\frac{1}{2\pi}\omega dq
\end{align*}
The magnetic moment associated with that small current is:
\begin{align*}
d\mu=A dI &= \pi r^2 \frac{1}{2\pi}\omega dq\\
&=\frac{1}{2}\omega\frac{Q}{d} r^2dr
\end{align*}
The total magnetic moment is then the sum of the magnetic moments of each charge element:
\begin{align*}
\mu=\int d\mu=\frac{1}{2}\omega\frac{Q}{d}\int_0^dr^2dr=\frac{1}{6}Q\omega d^2
\end{align*}
as required. 
\end{solution}


%Giancolli 27-12 -fixed
\question A circular loop of radius $r$ carries a current $I$. The centre of the circular loop is placed a distance $d$ above a point which emits a magnetic field isotropically, as shown in Figure \ref{fig:magneticforce:ring}. Show that the magnitude of the net force on the circular loop is given by:
\begin{align*}
F=2\pi IB\frac{r^2}{\sqrt{r^2+d^2}}
\end{align*}
\capfig{0.35\textwidth}{figures/MagneticForce/ring.png}{\label{fig:magneticforce:ring}A current carrying ring in a diverging magnetic field.}
\begin{finalanswer}
\begin{enumerate}[(a)]
\item
\end{enumerate}
\end{finalanswer}
\begin{solution}
We break up the loop into small sections of length $dl$. At all points on the loop, the magnetic field is perpendicular to the element $d\vec l$, whose direction is parallel to the current. The magnitude of the force $dF$ on the small section of loop is given by:
\begin{align*}
dF=IdlB
\end{align*}
\capfig{0.45\textwidth}{figures/MagneticForce/ring_sol.png}{\label{fig:magneticforce:ring_sol}Force on two small elements of the loop.}
As shown in Figure \ref{fig:magneticforce:ring_sol}, the force on each segment will be radially outwards and downwards. The radial components of the force will cancel between pairs of elements that are across from each other (as illustrated). Thus, the total force will in the downward direction, and only the vertical components will contribute:
\begin{align*}
F=\int dF\sin\theta=\int_0^{2\pi r}IdlB\sin\theta=2\pi r IB\sin\theta=2\pi IB\frac{r^2}{\sqrt{r^2+d^2}}
\end{align*}

\end{solution}

%Giancolli 27-55 -fixed, I can change up the numbers to make it a different particle if needed.
\question You are tasked with identifying a mysterious particle. In order to narrow down your choices, you decide to determine its mass. To do this, you place the particle in a magnetic field of amplitude $B = \SI{0.034}{T}$ and an electric field of magnitude $E = \SI{1.5}{kV/m}$. When the particle is in this region, it moves in a straight line. You shut the electric field off, and find that the particle begins to move in a circle of radius $r = \SI{2.7}{cm}$. Given this information, what possible particles could this mystery particle be?
\begin{finalanswer}
The mass is a multiple of \SI{3.329e-27}{kg}. Note that one atomic mass unit is $u=\SI{1.66e-17}{kg}$, which is half of \SI{3.329e-27}{kg}. Thus, the mass is given by:
\begin{align*}
m=n (2u)
\end{align*}
where $n$ is the charge of the particle in terms of the charge of the electron. Thus, if the particle as a charge of $e$ ($n=1$), then it has a mass of $2u$, which could correspond to a deuterium nucleus ($^2$H, a proton and a neutron). If the particle has a charge of $2e$, it has a mass of $4u$, so it could be a $^4$He nucleus (a so-called alpha particle).
\end{finalanswer}
\begin{solution}
Since the particle moves in a straight line when both the electric and magnetic fields are turned on, the forces from each field must be equal in magnitude and opposite in direction. Furthermore, for the two forces to cancel, the particle must be moving in a direction that is perpendicular to both of the electric and magnetic fields. The magnitudes of the two forces are the same, giving:
\begin{align*}
qE&=qvB\\
\therefore v&=\frac{E}{B}
\end{align*}
Once the electric field is switched off, the particle goes in a circle, where the centripetal acceleration is due to the magnetic force:
\begin{align*}
m\frac{v^2}{r}&=qvB\\
\frac{m}{q}&=\frac{Br}{v}=\frac{Br}{\frac{E}{B}}=\frac{B^2r}{E}\\
&=\frac{(\SI{0.034}{T})^2(\SI{0.027}{m})}{(\SI{1.5e3}{V/m})}=\SI{2.08e-8}{T^2m^2/V}
\end{align*}
This only gives us the ratio of the mass to the charge. Since this is a particle, it will have a charge that is a multiple of the electron charge. Suppose that the charge is given by $ne$:
\begin{align*}
\frac{m}{ne}&=\SI{2.08e-8}{T^2m^2/V}\\
m&=ne(\SI{2.08e-8}{T^2m^2/V})\\
&=n(\SI{1.6e-19}{C})(\SI{2.08e-8}{T^2m^2/V})=n(\SI{3.329e-27}{kg})
\end{align*}
The mass is thus a multiple of \SI{3.329e-27}{kg}. Note that one atomic mass unit is $u=\SI{1.66e-17}{kg}$, which is half of \SI{3.329e-27}{kg}. Thus, the mass is given by:
\begin{align*}
m=n (2u)
\end{align*}
where $n$ is the charge of the particle in terms of the charge of the electron. Thus, if the particle as a charge of $e$ ($n=1$), then it has a mass of $2u$, which could correspond to a deuterium nucleus ($^2$H, a proton and a neutron). If the particle has a charge of $2e$, it has a mass of $4u$, so it could be a $^4$He nucleus (a so-called alpha particle). 
\end{solution}

\question You have been hired by a hair importer to search for trace amounts of arsenic in guanaco hair destined for the Canadian market. The importer has an old mass spectrometer lying around which you propose to modify in order to measure the presence of arsenic. A schematic of the mass spectrometer is shown in Figure \ref{fig:magneticforce:MassSpec}.

A magnet creates a uniform magnetic field of \SI{1}{T} in the region shown by the dashed box. A sensor to detect ions is placed at a location to detect only those ions that have a circular path with radius $R=\SI{10}{cm}$. Ions are initially accelerated by two parallel plates through a potential difference $\Delta V_1$. The ions then enter a velocity selector where two plates separated by a distance $D=\SI{1}{cm}$ are held at a potential difference $\Delta V_2$ in the region of magnetic field. After the velocity selector, the ions are forced along a semi-circular path of radius $R$ where they will hit the sensor directly below the point where they exit the velocity selector.

Arsenic ions have a charge of $-3e$ and a mass of \SI{75}{amu} (\SI{1}{amu}=\SI{1.66e-27}{kg}). Ignore gravity and assume that there are no energy losses due to friction and drag. 
\begin{parts}
\part Using the given coordinate system, in which direction must the magnetic field point in order for the arsenic ions to reach the sensor?
\part Show that the speed of the ions exiting the velocity selector must be \SI{3.9e5}{m/s} in order to be detected at the sensor. Calculate the corresponding relativistic factor, $\gamma$, and comment on whether effects from special relativity are relevant in this situation.
\part What is the potential difference, $\Delta V_2$, required so that arsenic ions pass through the velocity selector un-deflected if they are moving with the speed identified in part (b)?
\part What is the potential difference, $\Delta V_1$, required in order to accelerate the arsenic ions to the speed identified in part (b)?
\end{parts}

\capfig{0.5\textwidth}{figures/MagneticForce/MassSpec.png}{\label{fig:magneticforce:MassSpec}A mass spectrometer}
\begin{finalanswer}
\begin{enumerate}[(a)]
\item Into the page (negative $z$ direction).
\item $\gamma=1.0000008$, which is very close to 1.
\item \SI{3900}{V}
\item \SI{19725.47}{V}
\end{enumerate}
\end{finalanswer}
\begin{solution}
\begin{parts}
\part Since the ions are negative, the magnetic field must be into the page (negative $z$ direction)
\part The ions perform uniform circular motion due to the magnetic force:
\begin{align*}
m\frac{v^2}{R}&=qvB\\
\therefore v&=\frac{qBR}{m}=\frac{(\SI{3}{e})(\SI{1.6e-19}{C/e})(\SI{1}{T})(\SI{0.1}{m})}{(\SI{75}{amu})(\SI{1.66e-27}{kg/amu})}=\SI{3.855e5}{m/s}
\end{align*}
This corresponds to $\approx 0.0013c$, so we don't expect Special Relativity to play any role. The relativistic factor, $\gamma$, is given by:
\begin{align*}
\gamma=\frac{1}{\sqrt{1-\frac{v^2}{c^2}}}=\frac{1}{\sqrt{1-\frac{(\SI{3.855e5}{m/s})^2}{(\SI{3e8}{m/s})^2}}}=1.0000008
\end{align*}
and indeed, $\gamma$ is very close to 1.
\part In the velocity selector, the magnetic and electric forces balance each other for the ions going the correct speed (regardless of their charge):
\begin{align*}
qE &= qvB\\
E&=vB
\end{align*}
Since the electric field is generated between two parallel plates with a potential difference of $\Delta V_2$, separated by a distance $D$, the electric field is given by $E=\frac{\Delta V_2}{D}$, and the potential difference is given by:
\begin{align*}
E&=\frac{\Delta V_2}{D}=vB\\
\therefore \Delta V_2&=vBD=(\SI{3.9e5}{m/s})(\SI{1}{T})(\SI{0.01}{m})=\SI{3900}{V}
\end{align*}
\part The ions are accelerated by a fixed electric potential difference $\Delta V_1$:
\begin{align*}
q\Delta V_1&=\frac{1}{2}mv^2\\
\therefore \Delta V_1&=\frac{mv^2}{2q}=\frac{(\SI{75}{amu})(\SI{1.66e-27}{kg/amu})(\SI{3.9e5}{m/s})^2}{2(\SI{3}{e})(\SI{1.6e-19}{C/e})}\\
&=\SI{19725.47}{V}
\end{align*}
\end{parts}
\end{solution}


