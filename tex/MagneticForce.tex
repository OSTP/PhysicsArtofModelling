
\chapter{The magnetic force}
\label{chapter:magneticforce}
We all have experienced the force from magnets in our common life, and this chapter introduces the tools to model the magnetic force. As we will see, the magnetic force acts on moving (electric) charges, and is thus fundamentally different from the electric force. In later chapters, we will develop the tools that allow us to view the electric and magnetic forces as two aspects of the same phenomenon.

\begin{learningObjectives}{
 \item Understand the key characteristics of a magnetic field and what makes it different from an electric field.
 \item Understand how to model the magnetic force on a moving charge.
 \item Understand how to model the magnetic force on a wire carrying current.
 \item Understand how to model the torque exerted on a current-carrying loop by a magnetic field.
 \item Understand how to model the Hall Effect.
 \item Understand how a velocity selector and a mass spectrometer are designed.
 }
\end{learningObjectives}

\begin{opening}
\begin{MCquestion}{When you go through airport security, they sometimes sample your luggage with sticky tape and place that tape into a machine to detect trace amounts of explosives. How does that machine work?}
\item The machine detects trace amounts by ``sniffing'' the sample using similar chemical reactions as those in our olfactory system.
\item The machine vaporizes the sample and accelerates the resulting charged vapour around a circle to determine its constituents. \correct
\end{MCquestion}
\end{opening}

\section{Magnetic fields}
Just as we can model the electric force on a charge by using the electric field (e.g. from another charge), we can model the force on a magnet by using a magnetic field (e.g. from another magnet). In your experience, every magnet that you have seen always has a ``North'' pole and a ``South'' pole. Most likely, you have noticed that the North pole of a magnet is attracted to the South pole of another magnet, and that the two North (or South) poles of different magnets repel each other. Thus, the magnetic force is attractive between two opposite poles, and repulsive otherwise. The Earth itself can be modeled as a giant bar magnet, with North and South magnetic poles. The poles on a magnet are labeled North and South according to which geographic pole they point to (a magnetic compass needle has a magnetic North pole in the direction of the Earth's North geographic pole).
%TODO Checkpoint question: Is the magnetic North pole of the earth located closer to the Earth's geopgraphic North pole or closer to its geographic South Pole? (correct: South)

Although it may seem that the magnetic force can be described in the same way as the electric force, having two opposite sign ``charges'' (or poles for magnets), it is fundamentally different. As far as we can tell, there are no magnets that have only a North or a South pole. Every magnet must have a North \textit{and} a South pole. This is fundamentally different from the electric force, where an object can have a net positive or negative charge. In the context of magnetism, we say that ``monopoles do not exist'' (an object that has only a North or South pole would be called a monopole). This is illustrated in Figure \ref{fig:magneticforce:magnetcut}, which shows what happens as one tries to cut a bar magnet into two pieces; rather than ending up with a North and a South piece (monopoles), we end up with two bar magnets, each with their own North and South poles.

\capfig{0.4\textwidth}{figures/MagneticForce/magnetcut.png}{\label{fig:magneticforce:magnetcut}When a bar magnet is cut through the middle, one obtains two magnets, each with a North and South pole, rather than a North and a South magnet.}
 
We draw magnetic field lines in much the same way that we draw electric field lines. The magnetic field lines are such that the magnetic field vector, $\vec B$, at some point in space is tangent to the field line at that point. The strength of the magnetic field is determined by the density of field lines at that position in space. The direction of the magnetic field, $\vec B$, indicates the direction of the force that is exerted on the North pole of a magnet. Magnetic field lines thus flow away from North poles and towards South poles. Thus far, the magnetic field description is similar to that of the electric field, with North magnetic poles being similar to positive electric charges, and vice versa.  However, because magnetic monopoles do not exist, magnetic field lines do not end (or start) on the pole of a magnet. Rather, magnetic field lines must always form \textbf{closed loops}, as illustrated in Figure \ref{fig:magneticforce:barfield} (where some of the field lines close outside of the figure). Thus, the magnetic field from a bar magnet should be compared to the electric field from an electric dipole (TODO - REF LINK TO DIPOLE SECTION). In order to make the analogy with the electric field, one should thus think of magnetic dipoles as being similar to electric dipoles, rather than the North side of a magnet being similar to a positive charge. 
%TODO Much better figure, showing the magnetic field from a bar magnet, with closed field lines, etc... Show the B vector at some positions (tangent to the line, magnitude proportional to line density). Show arrows on the field lines. Make the field lines a separate figure/component, as we'll need the lines again for the field from a loop. 
\capfig{0.4\textwidth}{figures/MagneticForce/barfield.png}{\label{fig:magneticforce:barfield}The magnetic field lines for a bar magnet always form closed loops as they do not end at the North or South pole of the magnet.}

We will discuss how to determine the strength of a magnetic field in the next chapter, but it is important to understand magnetic fields are created by moving electric charges. The electrons that orbit atoms in a bar magnet are the moving charges that create the magnetic field. In fact, as we will see, the magnetic field from a charge moving around in a circle (or a circular loop of current), has exactly the same shape as that of a bar magnet, as illustrated in Figure \ref{fig:magneticforce:loopfield}. We can thus think of charge moving in a circle as a small bar magnet, or more precisely, as a magnetic dipole.

\capfig{0.4\textwidth}{figures/MagneticForce/loopfield.png}{\label{fig:magneticforce:loopfield}The magnetic field lines created by electric charges moving in a circular loop are identical to that from a bar magnet.}

In a magnetic material, the electrons in the material are moving in such a way that the magnetic fields that they generate are all in the same direction, which results in a larger magnetic field as these all sum together. This also allows us to understand Figure \ref{fig:magneticforce:magnetcut}, since cutting a bar magnet just results in less material, but it still comprises electrons that are moving in the same orientation, and thus creates a magnet with a North and South pole. Note that it is not the motion of electrons around their nuclei that results in the magnetic field, and one really required quantum mechanics and the notion of ``spin'' to describe this in detail. 
 
Most materials will respond to magnetic fields, but the behaviour is most evident in ``ferromagnetic'' materials, such as iron (Fe). Ferromagnetic materials can be magnetized by an external magnetic field, effectively transforming them into magnets. One can think of a material as containing many little loops of electric current, which themselves are like bar magnets. If that material is ferromagnetic,  an external magnetic field can act on the ``little bar magnets'', orienting them all in the same way, so that the material as a whole becomes magnetic. For some ferromagnetic materials, that common orientation will remain when the external magnetic field is removed, creating a ``permanent magnets''. For other ferromagnetic materials, the common orientation disappears when the external field is removed; those materials are thus attracted to a magnet, but they cannot form a magnet. 
  
\section{The magnetic force on a charge}
%TODO Review box linking to appendix section on cross products
When an electric charge, $q$, has a velocity, $\vec v$, relative to a magnetic field, $\vec B$, a magnetic force is exerted on the particle:
\begin{align*}
\Aboxed{\vec F_B = q \vec v \times \vec B}
\end{align*}
We can make a few remarks about the magnetic force:
\begin{itemize}
\item The magnetic force is always perpendicular to the velocity and to the magnetic field (since it is given by a cross-product).
\item The magnetic force can do no work, since it is always perpendicular to the velocity (and thus displacement).
\item There is no force if the particle's velocity is in the same direction as the magnetic field vector. 
\item The force increases with charge, speed, and strength of the magnetic field.
\end{itemize}
You should be somewhat bothered by the fact that the force depends on the velocity of the charge, since velocity depends on the frame of reference. The above equation has a strange implication: if we observe an electron moving in a magnetic field, we will see its motion be deflected by the magnetic field. If we move along with the electron, so that it has a velocity of zero in our frame of reference, we should not see the electron being deflected, since the magnetic force would be zero. Clearly, the motion of the electron cannot depend on the frame of reference from which we observe it. Thus, the only way that this equation can make sense is if the magnetic field also depends on our frame of reference. We will revisit this in a subsequent chapter, but for now, remember that this equation only makes sense if the velocity is measured in the same reference frame as that in which the magnetic field is defined.

Another bothersome issue with the magnetic force is that it appears to depend on the fact that most humans are right-handed. Indeed, the direction of the force requires one to use the right-hand rule, which appears arbitrary. This is a common occurrence in physics, as many quantities are defined using a cross-product. However, no physical quantity can ever depend on our choice of right or left hand for determining cross-products. It turns out that any physical quantity (such as the force on a particle, which will deflect the particle in a clearly identifiable direction that does not depends on human's choice of right and left), always depends on two successive applications of the right-hand rule. In this case, the direction of the magnetic field is also given by a right-hand rule applied to the moving charges that create the field. The successive use of the right hand twice ``cancel'', and one finds that a charge is deflected in the same direction if one uses the left hand to define the magnetic field, and then the left-hand again for the cross-product!

Consider the motion of a charged particle in a region where the magnetic field is uniform (constant in magnitude and direction). If the magnetic field is perpendicular to the velocity vector of the particle, the particle will undergo uniform circular motion. Indeed, the force is always perpendicular to the velocity, and the force is constant in magnitude since both the speed and magnetic field remain constant. These are the only conditions required for uniform circular motion. We can easily determine the radius, $R$, of the circle, since the magnetic force is responsible for the centripetal acceleration:
\begin{align*}
F_B &= m\frac{v^2}{R}\\
qvB &= m\frac{v^2}{R}\\
\therefore R &= \frac{qB}{mv}
\end{align*}

%Example Mass spec

%Describe helical motion.


%describe helical path, 

\section{The magnetic force on a current-carrying wire}

\section{Magnetic dipole moment}
%torque on a loop

\section{The Hall Effect}

\section{Applications}
%Velocity selector
%Mass spec



\newpage
\section{Summary}

\begin{chapterSummary}
 Something that was learned
\end{chapterSummary}

\newpage
\begin{importantEquations}
\medskip
\begin{multicols}{2}
\textbf{Momentum of a point particle:}
\begin{align*}
\vec p = m\vec v \\
\frac{d}{dt}\vec p = \sum \vec F = \vec F^{net}
\end{align*}
\columnbreak
\\
\textbf{Position of the Centre of Mass \\ of a system:}
\begin{align*}
\vec r_{CM} &=\frac{1}{M}\sum_i m_i\vec r_i 
\end{align*}
\medskip
\end{multicols}
\end{importantEquations}

\newpage
\section{Thinking about the material}

\begin{chapteractivity}{Reflect and research}
{
\item When was magnetism first discovered?
\item What is the origin of the word ``magnetism''?
\item What experimental supports that magnetic monopoles do not exist?
}
\end{chapteractivity}

\begin{chapteractivity}{To try at home}
{
\item Try
}
\end{chapteractivity}

\begin{chapteractivity}{To try in the lab}
{
\item Propose an experiment
}
\end{chapteractivity}

\newpage
\section{Sample problems and solutions}
\subsection{Problems}
\begin{problem}{soln:template:ballistic}{\label{prob:template:ballistic} 

}
\end{problem}

\newpage
\subsection{Solutions}
\begin{solution}{prob:template:ballistic}\label{soln:template:ballistic}

\end{solution}

