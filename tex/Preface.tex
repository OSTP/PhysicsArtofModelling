\section*{License}
This textbook is shared under the CC-BY-SA 3.0 (Creative Commons) license. You are free to copy and redistribute the material in any medium or format, remix, transform, and build upon the material for any purpose, even commercially. You must give appropriate credit, provide a link to the license, and indicate if changes were made. You may do so in any reasonable manner, but not in any way that suggests the licensor endorses you or your use. If you remix, transform, or build upon the material, you must distribute your contributions under the same license as the original.
\vspace{\fill}
\begin{center}
\includegraphics[width=0.4\textwidth]{figures/Preface/license.png}
\end{center}
\newpage

\chapter*{Preface}
\label{chap:ipreface}
\section*{About this textbook}
This textbook is written to fill several needs that we believe were not already met by the many existing introductory physics textbooks. First, we wanted to ensure that the textbook is free to use for students and professors. Second, we wanted to design a textbook that is mindful of the new pedagogies being used in introductory physics, by writing it in a way that is adapted to a flipped-classroom approach where students complete readings, think about the readings, and then discuss the material in class. Third, we wanted to create a textbook that also addresses the experimental aspect of physics, by proposing experiments to be conducted at home or in the lab, as well as providing guidelines for designing experiments and reporting on experimental results. Finally, we wanted to create a textbook that is a sort of ``living document'', that professors can edit and re-mix for their own needs, and to which students can contribute material as well. The textbook is hosted on \href{https://github.com/OSTP/PhysicsArtofModelling}{GitHub}, which allows anyone to make suggestions, point out issues and mistakes, and contribute material.

This textbook is meant to be paired with the accompanying ``Question Library'', which contains many practice problems, many of which were contributed by students.

This textbook would not have been possible without the support of Queen's University and the Department of Physics, Engineering Physics \& Astronomy at Queen's University, as well as the many helpful discussions with the students, technicians and professors at Queen's University.

\section*{Hello from the authors}
\lwfig[9]{0.2\textwidth}{figures/Preface/Ryan.png}
\textbf{Ryan Martin} I am a professor of physics at Queen's University. My main research is in the field of particle astrophysics, particularly in studying the properties of neutrinos. I grew up in Switzerland, obtained my Bachelor's, Master's and Ph.D. at Queen's University. I was then a postdoctoral fellow at Lawrence Berkeley National Laboratory, a faculty at the University of South Dakota, before returning to Queen's. I am particularly passionate about education, and I am always seeking opportunities to involve students in helping to make education more accessible. I also like to cook and to play volleyball.

\lwfig[9]{0.2\textwidth}{figures/Preface/Emma.jpg}
\textbf{Emma Neary} I am currently a second year physics major and QuARMS (Queen's University Accelerated Route to Medical School) student, as well as a native of St. John's, Newfoundland. Uniting the perspectives of students and professors in an accessible way is important to me. I strongly believe in the importance of building physical models; whether it be in physics, medicine, sciences or the arts. It has been my goal to infuse the textbook with the theme of modelling in a creative and engaging way. Aside from doing physics, I enjoy hiking, dancing, reading and doing research in gastroenterology and neuropsychiatry.

\lwfig[9]{0.2\textwidth}{figures/Preface/josh.png}
\textbf{Joshua Rinaldo} I am a third year physics major and concurrent education student. I was first introduced to the flipped classroom approach in Ryan Martin's first year physics class, and have found that the experience shaped the way I approach education. I intend on continuing to make use of the flipped classroom approach as I move forward in my career. Being able to co-author this textbook has been an amazing opportunity for me to grow as an educator, and I look forward to applying the skills I learned while working on the textbook. Outside of physics, I enjoy making jewelry and practicing mixed martial arts.

\lwfig[9]{0.2\textwidth}{figures/Preface/Olivia.png}
\textbf{Olivia Woodman} I am a currently a third year undergraduate student at Queen's Univeristy, majoring in physics. The flipped classroom approach has been beneficial to my own learning, and I think that we have created a textbook that really complements this learning style. Throughout this book, I have shared my thoughts on various topics in physics, as well as some useful tips and tricks. I hope that students enjoy using this book and continue to contribute to it in the future. Working on this textbook has also allowed me to combine my love of physics with my love of doodling, so I hope you enjoy the drawings!

\section*{How to use this textbook}
This textbook is designed to be used in a flipped-classroom approach, where students complete readings at home, and the material is then discussed in class. The material is thus presented fairly succinctly, and contains \textbf{Checkpoint Questions} throughout that are meant to be answered as the students complete the reading. We suggest including these Checkpoint Questions as part of a quiz in a reading assignment (marked based on completion, not correctness), and then using these questions as a starting point for discussions in class.

For topics that are particularly difficult, we have included \textbf{Thought Boxes} written by students that try to present the material in a different light. We are always happy if students (or professors) wish to contribute additional thought boxes.

Chapters start with a set of \textbf{Learning outcomes} and an \textbf{Opening question} to help students have a sense of the chapter contents. The chapters have \textbf{Examples} throughout, as well as additional practice problems at the end. The \textbf{Question Library} should be consulted for additional practice problems. At the end of the chapter, a \textbf{Summary} presents the key points from the chapter. We suggest that students carefully read the summaries to make sure that they understand the contents of the chapter (and potentially identify, before reading the chapter, if the content is review to them). At the end of the chapters, we also present a section to \textbf{Think about the material}. This includes questions that can be assigned in reading assignments to research applications of the material or historical context. The thinking about the material section also includes experiments that can be done at home (as part of the reading assignment) or in the lab.

Appendices cover the main background in mathematics (Calculus and Vectors), as well as present an introduction to programming in python, which we feel is a useful skill to have in science. There is also an Appendix that is intended to guide work in the lab, by providing examples of how to write experimental proposals and reports, as well as guidelines for reviewing proposals and reports. We believe that introductory laboratories should not be be ``recipe-based'', but rather that students should take an approach similar to that of a researcher in designing (proposing) an experiment, conducting it, and reviewing the proposals and results of their peers.

\section*{Credits}
This textbook, and especially the many questions in the Question Library would not have been possible without the many contributions from students, teaching assistants and other professors. Below is a list of the students that have contributed material that have made this textbook and the Question Library possible.

\begin{multicols}{3}
\begin{center}
Adam McCaw\\
Ali Pirhadi\\
Alex Hughes\\
Alexis Brossard\\
Allyson Smith\\
Amy Van Nest\\
Camren Oakes\\
Ceaira Hiemstra\\
Damara Gagnier\\
Daniel Barake \\
Daniel Tazbaz\\
David Cutler\\
Emily Darling\\
Emily Mendelson\\
Emily Wener\\
Emma Lanciault\\
Erin Parson\\
Genevieve Fawcett\\
Gregory Love\\
Haoyuan Wang\\
Ian McClean\\
Jack Fitzgerald\\
James Godfrey\\
\columnbreak
Jenna Vanker\\
Jesse Fu\\
Jesse Simmons\\
Jessica Grennan\\
Joanna Fu\\
Jonathan Abott\\
Kate Fenwick\\
Lily Dodd\\
Madison Facchini\\
Marie Vidal\\
Matt Routliffe\\
Maya Gibb\\
Natalie Dubas\\
Nathan Wilson\\
Neil Rajan\\
Nicholas Everton\\
Nick Brown\\
Nicole Gaul\\
Noah Rowe\\
Olivia Bouaban\\
Patrick Singal\\
Qiqi Zhang\\
Quentin Sanders\\
\columnbreak
Robin Joshi\\
Ryan Underwood\\
Sam Connolly\\
Sara Stephens\\
Sarmund Mahmood\\
Shaundra Buelow\\
Shona Birkett\\
Stephanie Ciccone\\
Tai Withers\\
Talia Castillo\\
Tamy Puniani\\
Thomas Faour\\
Troy Allen\\
Tashifa Imtiaz\\
Wei Zhuolin\\
Yannick Bisson\\
Yumian Chen\\
Zifeng Chen\\
Zoe Macmillan
\end{center}
\end{multicols}
