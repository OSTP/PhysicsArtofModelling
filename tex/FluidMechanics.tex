\section{Fluid Mechanics}

%%%%%%%%%%%%%%%%%%%%%%%%%%%%%%%%%%%
%%
%% Multiple Choice
%%
%%%%%%%%%%%%%%%%%%%%%%%%%%%%%%%%%%%
\subsection{Multiple Choice}

% Matt Routliffe
\question A reservoir's surface is \SI{500}{m} above an output pipe. Given this information (and neglecting pipe friction), what is the velocity of the water as it exits the pipe?
\begin{checkboxes}
	\choice \SI{49}{m/s}
	\choice \SI{25}{m/s}
	\CorrectChoice \SI{99}{m/s} \correct
	\choice \SI{200}{m/s}
\end{checkboxes}

\question A cork floats such that three quarters of its volume is below water. If the density of water is taken to be 1.0, what is the density of the cork?
\begin{checkboxes}
	\choice 0.25
	\choice 0.50
	\CorrectChoice 0.75
	\choice 0.83
	\choice None of the above
\end{checkboxes}

\question The wings of an Airbus 320 airplane have a total useful area of \SI{123}{m^2}, while the average mass of the airplane at take-off when carrying passengers is \SI{6e4}{kg}. If the difference in air pressure above and below the wings is solely responsible for the airplane's lift, what is that minimum pressure difference in order for the plane to have enough lift to take-off?
\begin{checkboxes} 
	\choice \SI{488}{Pa}
	\CorrectChoice \SI{4780}{Pa}
	\choice  \SI{7.38e6}{Pa}
	\choice \SI{72.3e6}{Pa}
\end{checkboxes}

\question You would like to double the water flow rate in your shower, so you replace the horizontal pipe from the main water supply to your shower with a wider pipe that has a lower resistance to water flow. The original pipe has radius of \SI{1}{cm}, what radius should the new pipe have? (Assume a horizontal pipe and incompressible laminar flow of water with non-negligible viscosity, so that the flow is described by the Poiseuille equation).
\begin{checkboxes}
	\CorrectChoice \SI{1.19}{cm}
	\choice \SI{1.41}{cm}
	\choice \SI{2.00}{cm}
	\choice \SI{4.00}{cm}
\end{checkboxes}

\question An object made of an unknown material is observed to float in water. Does the object still float in water on the moon, where the force of gravity is about six times less?
\begin{checkboxes}
	\CorrectChoice Yes.
	\choice No.
	\choice It depends on the density of the object. 
\end{checkboxes}

%Maaike de Lint (based upon)
\question When is the pressure inside an incompressible fluid the highest, if the fluid is undergoing laminar flow?
\begin{checkboxes}
	\CorrectChoice When the fluid is moving slow.
	\choice When the fluid is moving fast.
	\choice The pressure does not depend on the speed of the fluid.
\end{checkboxes} the speed has no effect on the pressure inside the fluid.

\question A \SI{60}{kg} woman stands on a square piston of side \SI{2}{m}, which pushes down on a hydraulic fluid.  How much force does a second piston (a circle of radius \SI{2}{m}, which is at the same height) feel?
\begin{checkboxes}
	\choice \SI{60}{N}
	\choice \SI{188.5}{N}
	\choice \SI{588}{N}
	\CorrectChoice \SI{1847}{N} \correct
\end{checkboxes}

%From Giancolli -fixed
\question Suppose there are three glasses, all filled with the same amount of water. Glass A in shaped such that the top has a larger radius than the bottom, glass B is cylindrical, and glass C is shaed such that the bottom has a larger radius than the top. Which of the three glasses have the greatest liquid pressure at the bottom?
\begin{checkboxes}
	\choice Glass A
	\choice Glass B
	\choice Glass C
	\CorrectChoice All three have equal non-zero pressure at the bottom \correct
	\choice All three have zero pressure at the bottom.  
\end{checkboxes}

% Question submitted by Robin Joshi
\question A 50 kg pig floats in fresh water with 27\% of its volume submerged. What is its density?
\begin{checkboxes}
	\choice \SI{420}{kg/m^3}
	\choice \SI{380}{kg/m^3}
	\CorrectChoice \SI{270}{kg/m^3} \correct
	\choice \SI{150}{kg/m^3}
\end{checkboxes}

% Question submitted by Patrick Singal
\question Which principle describes the lift produced by an airplanes wing?
\begin{checkboxes}
	\choice Pascal's principle
	\CorrectChoice Bernouilli's principle \correct
	\choice Archimedes' principle
	\choice Poiseuille's principle
\end{checkboxes}

\question An incompressible fluid flows with negligible resistance through a pipe that gets progressively narrower in the direction of the fluid flow as shown in Figure \ref{fig:FluidMechanics:NarrowPipe}. Which one of the following is correct about the pressure and speed of the fluid?
\capfig{0.2\textwidth}{figures/FluidMechanics/NarrowPipe.png}{\label{fig:FluidMechanics:NarrowPipe} Fluid flowing in the direction of the arrow through a narrowing pipe.}
\begin{choices} 
	\choice The pressure and speed are highest at point 1
	\choice The pressure and speed are highest at point 2
	\CorrectChoice The pressure is highest at point 1 and the speed highest at point 2 \correct
	\choice The pressure is highest at point 2 and the speed highest at point 1
\end{choices}

%Matt Routliffe
\question The water level in a reservoir is \SI{500}{m} above the output of a pipe.  If one neglects pipe friction and the difference in atmospheric pressure, what is the velocity of the water as it exits the pipe?
\begin{choices} 
	\choice \SI{50}{m/s}
	\choice \SI{70}{m/s}
	\CorrectChoice \SI{99}{m/s} \correct
	\choice \SI{9800}{m/s}
\end{choices}

%%%%%%%%%%%%%%%%%%%%%%%%%%%%%%%%%%%
%
% long answer
%
%%%%%%%%%%%%%%%%%%%%%%%%%%%%%%%%%%%
\subsection{Long answers}
\question Your friend decides that they want to go for the Guinness World Record for the longest drinking straw used in a vertical setting. That is, they wish to place a very long vertical drinking straw into a glass of water on the ground (at sea level) and drink from the top of the straw by leaning out of the balcony of a nearby building. You tell your friend that there is a physical limit to how long the straw can be and for them to still be able to drink the water. What is that length? How would your answer differ if your friend used a glass of olive oil instead?
\begin{finalanswer}
	10.31; the max height would be larger if they used olive oil.
\end{finalanswer}
\begin{solution}
	When you drink from a straw, you create a vacuum above the liquid. The liquid is drawn up by the difference in pressure between your mouth at the top of the straw and the atmospheric pressure on the water, at the bottom of the straw. Thus, if you create a perfect vacuum (zero pressure) at the top of the straw, the highest that the water can go corresponds to a change in pressure of the water column of 1 atmosphere.
	\begin{align*}
	\Delta p &= \rho g h = \SI{1}{atm} = \SI{1.01e5}{Pa}\\\
	\therefore h &= \frac{\Delta p}{\rho g}=\frac{(\SI{1.01e5}{Pa})}{(\SI{1e3}{kg/m^3})(\SI{9.8}{m/s^2})}=\SI{10.31}{m}
	\end{align*}
	which corresponds to the highest distance from which one can pump water from above. Since olive oil has a lower density, the maximum achievable height would be larger.
\end{solution}


%Modified from Giancolli 13-17
\question A vertical U-shaped tube with open ends is filled with two immiscible fluids (e.g. water and oil), as shown in Figure \ref{fig:FluidMechanics:UTube}. Because the two fluids have different densities, they do not reach the same height on either side of the tube. If the fluid on the right is a height $h_1$ below the the fluid on the left, and the fluid on the left has a height of $h_2$, what is the density, $\rho_2$ of the fluid on the left in terms of $\rho_1$, the density of the fluid on the right?
\capfig{0.2\textwidth}{figures/FluidMechanics/UTube.png}{\label{fig:FluidMechanics:UTube}A U-shaped tube filled with two different fluids.}
\begin{finalanswer}
	$\rho_2=\rho_1\frac{h_2-h_1}{h_2}$
\end{finalanswer}
\begin{solution}
	The pressure at the bottom of the fluid on the left, must equal the pressure of the fluid on the right at the same height. Thus:
	\begin{align*}
	\rho_2gh_2&=\rho_1g(h_2-h_1)\\
	\therefore \rho_2&=\rho_1\frac{h_2-h_1}{h_2}
	\end{align*}
\end{solution}


\question A tank of diameter $D$ is filled with water up to a height $h$ above the bottom of the tank (Figure \ref{fig:FluidMechanics:TankDrain}). At the bottom of the tank is a hole of diameter $d$. Assume that the water flows out of the hole with a laminar flow and that the difference in atmospheric pressure between the top and the bottom of the tank is negligible.
\capfig{0.2\textwidth}{figures/FluidMechanics/TankDrain.png}{\label{fig:FluidMechanics:TankDrain}A tank draining.}
\begin{parts}
	\part What speed will the water have just as it exits the bottom of the tank? Assume that the hole is small enough so that the water level in the tank is constant.
	\part As the stream of water gets further from the bottom of the tank, it will narrow to a smaller diameter than $d$. You can observe the same process for water flowing out of a tap. What will be the diameter of the (laminar) water stream a distance $x$ below the bottom of the tank?
	\part If you no longer assume that the level in the tank is constant, but rather that the top of the water has a speed downwards as the tank drains, show that the speed of the water exiting the bottom of the tank is given by:
	\begin{align*}
	v = \sqrt{\frac{2gh}{(1-\frac{d^4}{D^4})}}
	\end{align*}
	at an instant when the water level in the tank is $h$. 
\end{parts}
\begin{finalanswer}
	\begin{enumerate}[(a)]
		\item $\sqrt{2gh}$
		\item $2r = d\left(\frac{h}{h+x}\right)^{\frac{1}{4}}$
		\item N/A
	\end{enumerate}
\end{finalanswer}

\begin{solution}
	\begin{parts}
		\part We can use Bernoulli's equation, which is equivalent to conservation of energy, and compare the flow at the top of the tank and at the bottom. In both locations, there is no external pressure applied. Furthermore, we define a $y$ axis that is positive upwards with the origin at the bottom of the tank. The speed of the water is zero at the top of the tank and $v$ at the bottom. Bernoulli's equation gives:
		\begin{align*}
		\rho g h &= \frac{1}{2}\rho v^2\\
		\therefore v = \sqrt{2gh}
		\end{align*}
		which is the same speed that one would obtain from conservation of energy on a mass element of water.
		\part Since the flow of water out of the tank is laminar and water is incompressible, the flow rate ($Av$) of the water in the stream must remain constant. Since the water speeds up as it falls, its cross-sectional area must decrease. 
		
		A distance $h$ below the bottom of the tank, the water will have ``fallen'' a distance of $h+x$, so its speed will be:
		\begin{align*}
		v'=\sqrt{2g(h+x)}
		\end{align*}
		The flow at the bottom of the tank is:
		\begin{align*}
		Q = Av = \pi\frac{d^2}{4}\sqrt{2gh}
		\end{align*}
		This must equal the flow a distance $h$ below the tank, where the water stream has a diameter of $2r$:
		\begin{align*}
		Q &= A'v' = \pi r^2 \sqrt{2g(h+x)}\\
		\therefore \pi r^2 \sqrt{2g(h+x)}&=\pi\frac{d^2}{4}\sqrt{2gh}\\
		\therefore 2r &= d\left(\frac{h}{h+x}\right)^{\frac{1}{4}}
		\end{align*}
		\part In this case, when using Bernoulli's equation, we must take into account the speed of the water at the top of the tank, $v_h$:
		\begin{align*}
		\rho g h + \frac{1}{2}\rho v_h^2&= \frac{1}{2}\rho v^2\\
		g h + \frac{1}{2} v_h^2&= \frac{1}{2} v^2\\
		\end{align*}
		We can use the continuity equation (the fact that the flow out of the hole must be the same as the flow at the top of the tank) to relate $v$ and $v_h$:
		\begin{align*}
		v(\pi\frac{d^2}{4}) &= v_h(\pi\frac{D^2}{4})\\
		\therefore v_h &= v \frac{d^2}{D^2}
		\end{align*}
		which we can put into Bernoulli's equation and re-arrange:
		\begin{align*}
		g h + \frac{1}{2} v_h^2&= \frac{1}{2} v^2\\
		g h + \frac{1}{2} v^2 \frac{d^4}{D^4}&= \frac{1}{2} v^2\\
		\therefore v &= \sqrt{\frac{2gh}{(1-\frac{d^4}{D^4})}}
		\end{align*}
		as required.
	\end{parts}
\end{solution}


\question A cylindrical cork with density $\rho_c=\SI{0.2e3}{kg/m^3}$ and length $L$ is held submerged under water ($\rho_w=\SI{1.0e3}{kg/m^3}$) such that the cork is vertical and its top is just below the surface of the water (Figure \ref{fig:FluidMechanics:Cork}). When the cork is released, how far above the surface of the water will the top of the cork reach before it falls back down? Express your answer in terms of $L$ and assume that there are no drag forces on the cork as it moves through the water or air.
\capfig{0.2\textwidth}{figures/FluidMechanics/Cork.png}{\label{fig:FluidMechanics:Cork}A cork held under water.}
\begin{finalanswer}
	$\frac{5}{2}L$
\end{finalanswer}
\begin{solution}
	When the cork is submerged, the forces acting on it are gravity and the buoyancy force. As the cork moves up, the buoyancy force will decrease (as the submerged volume decreases), so we will have to integrate the work done by the buoyancy and gravitational forces to determine the cork's kinetic energy as it rises.
	
	If the cork has a cross sectional area, $A$, the force of gravity is constant and given by:
	\begin{align*}
	F_g = mg = \rho_c ALg
	\end{align*}
	The force of buoyancy on the cork, when submerged a distance $x$ (where $x=L$ when the cork is fully submerged) is given by:
	\begin{align*}
	F_b = \rho_w Axg
	\end{align*}
	where $Ax$ is the volume of water displaced by the cork, and $F_b=0$ if $x\leq 	0$. 
	
	It is not immediately clear if the cork will fully exit the water; this can only happen if the (positive) work done by the buoyancy force over the length of the cork is greater than the (negative) work done by gravity.
	
	If the cork were to fully exit the water (travels a distance $L$), then the work done by the buoyancy force would be:
	\begin{align*}
	W_b = \int_L^0 -F_b dx = \int_0^L F_b dx=\int_0^L \rho_w Axg dx =\frac{1}{2} \rho_w AgL^2
	\end{align*}
	The work done by gravity over the same distance is $W_g=-F_gL=-\rho_c AL^2g$. The net work done on the cork is then:
	\begin{align*}
	W^{net}&=W_b+W_g\\
	&=\frac{1}{2} \rho_w AgL^2-\rho_c AL^2g
	\end{align*}
	which is greater than zero if:
	\begin{align*}
	\frac{1}{2} \rho_w AgL^2&\geq\rho_c AL^2g \\
	\therefore \rho_c&\leq\frac{1}{2} \rho_w 
	\end{align*}
	which is the case, since $\rho_c=\SI{0.2e3}{kg/m^3}$ is less than half of the density of water. We can use conservation of energy to find the total height, $h$, that the top of the cork will reach above the water. The work that we just calculated corresponds to the kinetic energy of the cork as it exits the water, which will ``convert'' to gravitational potential energy as the cork is propelled above the water:
	\begin{align*}
	W^{net}&=mgh \\
	\frac{1}{2} \rho_w AgL^2-\rho_c AL^2g &=\rho_c ALgh\\
	\frac{1}{2} \rho_w L-\rho_c L &=\rho_c h\\
	\therefore h &= L\frac{\frac{1}{2} \rho_w - \rho_c}{\rho_c}\\
	&=\frac{3}{2} L
	\end{align*} 
	This is the distance that the cork rises after it has fully exited the water. When it has fully exited the water, the top of the cork is already at a height of $L$ above the water. Thus, the maximum height that the top of the cork will reach is $\frac{5}{2}L$.
	
\end{solution}

%Giancolli 13-52 -fixed
\question An airplane has a wing with an area of $\SI{35}{m^2}$ which has air flowing through the top at \SI{215}{m/s} and air flowing through the bottom at \SI{180}{m/s}. If the density of air is $\SI{1.29}{kg/m^3}$, what is the lift on the wing in newtons?
\begin{finalanswer}
	$\SI{3.121e5}{N}$
\end{finalanswer}
\begin{solution}
	Since the air is flowing at different speeds on either side of the wing, there will be a pressure difference:
	\begin{align*}
	p_{top} + \frac{1}{2}\rho v_{top}^2 &= p_{bot} + \frac{1}{2}\rho v_{bot}^2\\
	\therefore \Delta p = p_{bot}-p_{top} &=\frac{1}{2}\rho (v_{top}^2-v_{bot}^2)\\
	&=\frac{1}{2}(\SI{1.29}{kg/m^3}) ((\SI{215}{m/s})^2-(\SI{180}{m/s})^2) = \SI{8.917e3}{Pa}
	\end{align*}
	The corresponding lift is given by multiplying the change in pressure by the area of the wing:
	\begin{align*}
	F=\Delta p A = (\SI{8.917e3}{Pa})(\SI{35}{m^2})=\SI{3.121e5}{N}
	\end{align*}
\end{solution}

%Based on Giancolli 13-23
\question A vertical dam is constructed between two mountains that are a distance $b$ apart. The dam can be approximated to be uniform and rectangular in cross-section. 
\begin{parts}
	\part If water fills the dam to a height $h$ above the bottom of the dam, what is the net force exerted on the side of the dam?
	\part If water fills the dam to a height $h$ above the bottom of the dam, what is the net torque exerted on the dam about the bottom of the dam?
\end{parts}
\begin{finalanswer}
	\begin{enumerate}[(a)]
		\item $\frac{1}{2}\rho_wgbh^2$
		\item $\frac{1}{6}\rho_wgbh^3$
	\end{enumerate}
\end{finalanswer}
\begin{solution}
	\begin{parts}
		\part Since the pressure in the water changes with height, we need to integrate the force exerted on the dam over its height. We introduce a $y$ axis that is vertical, with the origin at the top of the water level and positive downwards ($y=h$ at the bottom of the dam). The total force on the dam will come from the difference in pressure between the water on one side and the air on the other. 
		
		We can consider a thin horizontal strip of the dam, at a height $y$, of area $dA=bdy$, and calculate the force exerted on it by the water pressure:
		\begin{align*}
		dF_w = p_w(y)dA = p_w(y) bdy
		\end{align*}
		The water pressure as a function of height $y$ is:
		\begin{align*}
		p_w(y) = p_0+\rho_w gy
		\end{align*}
		where $p_0$ is the atmospheric pressure at the top of the water level, and $\rho_w$ is the density of water. Similarly, the air will exert an opposite force on the same area element with a magnitude:
		\begin{align*}
		dF_a=p_a(y) bdy= (p_0+\rho_a gy)bdy
		\end{align*}
		The net force on that area element will then be:
		\begin{align*}
		dF^{net}&=dF_w-dF_a\\
		&=(p_0+\rho_w gy)bdy-(p_0+\rho_a gy)bdy\\
		&=(\rho_w-\rho_a)gybdy\\
		&\sim\rho_wgybdy
		\end{align*}
		where in the last line, we assumed that the density of air compared to that of water is negligible. The total force on the dam from is thus
		\begin{align*}
		F^{net} &=\int dF^{net}= \int_0^h \rho_wgybdy = \frac{1}{2}\rho_wgbh^2
		\end{align*}
		
		\part We proceed in a similar way to calculate the torque. The net force on an area element $dA=bdy$ at some height $y$ is given by:
		\begin{align*}
		dF^{net}=pdA= \rho_w gybdy
		\end{align*}
		The force from the pressure is always perpendicular to the surface, so that the torque about the bottom of the dam on that area element is given by:
		\begin{align*}
		d\tau = dF^{net} (h-y)=\rho_w gybdy(h-y)
		\end{align*}
		where $h-y$ is the distance to the lever arm (since we defined $y=0$ at the top of the dam). The total torque is thus:
		\begin{align*}
		\tau &= \int d\tau = \rho_w gb\int_0^h (hy-y^2)dy=\rho_w gb (\frac{1}{2}h^3-\frac{1}{3}h^3)\\
		&=\frac{1}{6}\rho_w gb h^3
		\end{align*}
	\end{parts}
	\begin{finalanswer}
		41.7 litres/min
	\end{finalanswer}
\end{solution}
%original
\question You measure that water comes out of your bathtub faucet at a rate \SI{10}{litres/min}. The diameter of the bathtub faucet is \SI{2}{cm} and the faucet is \SI{2.5}{m} above ground level. At what rate does water come out of the \SI{1}{cm} diameter faucet in your basement bathroom, located \SI{1.5}{m} below ground level, in \si{litres/min}?

Assume that the atmospheric pressure does not change appreciably between your basement and first floor, that water flows without turbulence or viscosity in your plumbing, and that opening or closing one faucet does not affect the other.

\begin{finalanswer}
	$\SI{41.7}{litres/min}$
\end{finalanswer}

\begin{solution}
	We use Bernoulli's principle to relate the flow of water out of the basement faucet to that out the bathtub faucet on the first floor:
	\begin{align*}
	p_b+\rho gh_b +\frac{1}{2}\rho v_b^2 &= p_1+\rho gh_1 +\frac{1}{2}\rho v_1^2 \\
	\end{align*}
	where subscripts $b$ and $1$ refer to the basement and first floor, respectively. We will take ground level as $h=0$, so that $h_b=\SI{-1.5}{m}$ and $h_1=\SI{2.5}{m}$. Since the pressure can be taken to be atmospheric in the basement and the first floor, $p_1\sim p_b$, these cancel:
	\begin{align*}
	\rho gh_b +\frac{1}{2}\rho v_b^2 &= \rho gh_1 +\frac{1}{2}\rho v_1^2 \\
	2gh_b +v_b^2 &= 2gh_1 + v_1^2 \\
	\therefore v_b^2 &= 2g(h_1-h_b) +v_1^2
	\end{align*}
	Instead of the speed of the fluid, we are interested in the flow rate, $Q$:
	\begin{align*}
	Q&=Av\\
	\therefore v&=\frac{Q}{A}
	\end{align*}
	where $A$ is the area of the faucet, and we are given $Q_1=\SI{10}{litres/min}=\SI{1.67e-4}{m^3/s}$, $A_1=\pi (\SI{0.01}{m})^2=\SI{3.14e-4}{m^2}$, and $A_b=\pi (\SI{0.005}{m})^2=\SI{7.85e-5}{m^2}$. We can re-arrange the equation for speed to be in terms of flow rate:
	\begin{align*}
	v_b^2 &= 2g(h_1-h_b) +v_1^2\\
	\left(\frac{Q_b}{A_b}\right)^2 &= 2g(h_1-h_b) +\left(\frac{Q_1}{A_1}\right)^2\\
	\therefore Q_b^2 &= A_b^2\left(2 g(h_1-h_b)+ \left(\frac{Q_1}{A_1}\right)^2\right)\\
	\therefore Q_b &= A_b\sqrt{2g(h_1-h_b)+ \left(\frac{Q_1}{A_1}\right)^2}\\
	&= (\SI{7.85e-5}{m^2})\sqrt{2(\SI{9.8}{m/s^2})((\SI{2.5}{m})-(\SI{-1.5}{m}))+ \left(\frac{(\SI{1.67e-4}{m^3/s})}{(\SI{3.14e-4}{m^2})}\right)^2}\\
	&=(\SI{7.85e-5}{m^2})\sqrt{2(\SI{9.8}{m/s^2})(\SI{4.0}{m})+ \left(\frac{(\SI{1.67e-4}{m^3/s})}{(\SI{3.14e-4}{m^2})}\right)^2}\\
	&=\SI{6.96e-4}{m^3/s}\\
	&=\SI{41.7}{litres/min}
	\end{align*}
	In reality, there would not be such a dramatic increase in the flow rate, because of flow impedance in your pipe. In reality, there is a pressure gradient even in a horizontal pipe.
\end{solution}

%original
\question You measure that water comes out of your kitchen faucet at a rate of \SI{6}{litres/min}. The faucet has a diameter of \SI{1}{cm}. If you try to plug the running faucet by pressing your thumb against the opening to close it off, how much force must you exert?

The density of water is \SI{1000}{kg/m^3} and you should assume that water flows without turbulence or viscosity in your pipes.
\begin{finalanswer}
	\SI{0.06}{N}
\end{finalanswer}
\begin{solution}
	We can pretend that the water coming out of the faucet comes from a reservoir that is at a height $h$ above the faucet. If the water in the reservoir is not moving and at atmospheric pressure, and the water coming out of the faucet is also at atmospheric pressure, then Bernoulli's principle reduces to:
	\begin{align*}
	\frac{1}{2}\rho v^2 = \rho g h 
	\end{align*}
	Now, since there is not necessarily a reservoir (we don't know its height), the term on the right is just a measure of the pressure of the water in the pipe before it comes out. If you press your thumb against the faucet, you have to exert that amount of pressure, $p$:
	\begin{align*}
	p &= \frac{1}{2}\rho v^2 = \frac{F}{A}\\
	\therefore F &= \frac{1}{2}\rho v^2A
	\end{align*}
	From the flow rate, $Q=\SI{6}{litres/min}=\SI{1e-4}{m^3/s}$ and the area of the faucet, $A=\pi (\SI{0.005}{m})^2=\SI{7.85e-5}{m^2}$ , we can determine the velocity of the water coming out of the faucet:
	\begin{align*}
	v&=\frac{Q}{A}
	\end{align*}
	The required force is thus:
	\begin{align*}
	F &= \frac{1}{2}\rho v^2A\\
	&= \frac{1}{2}\rho \frac{Q^2}{A^2}A\\
	&= \frac{1}{2}\rho \frac{Q^2}{A}\\
	&= \frac{1}{2}(\SI{1000}{kg/m^3})\frac{(\SI{1e-4}{m^3/s})^2}{(\SI{7.85e-5}{m^2})}\\
	&= \SI{0.06}{N}
	\end{align*}
	which is much smaller than it would be in reality (try it!). One cannot not ignore the effect of resistance to flow; without any resistance, the pressure that we found above is what is required to have a flow rate of \SI{6}{liters/min}. When you plug the faucet, the fluid stops flowing and the pressure that you need to push against is the pressure that is required in the pipe to provide a flow rate of \SI{6}{liters/min} in the presence of resistance, which is a much higher pressure, typically $\sim\SI{3e5}{Pa}=\SI{50}{psi}$ in  a household plumbing system.
\end{solution}

%Giancolli 13-20 -fixed
\question A hydraulic press is designed such that a force applied to a lever of length $2l$ is carried through hydraulic fluid and puts pressure on a target sample. The hydraulic press has a small cylinder of diameter $d = \SI{1.5}{cm}$, which carries fluid that is pressed down by a device attached to the lever. The pressure on the small cylinder is carried through the hydraulic fluid, which causes a larger cylinder of diameter $D = \SI{15}{cm}$ to press up on the sample, as shown in Figure \ref{figures/FluidMechanics/Press.png}. If the sample has an area $A = \SI{2.0}{cm^2}$ and $F = \SI{280}{N}$ newtons are applied to the lever, what pressure is applied to the sample?
\capfig{0.5\textwidth}{figures/FluidMechanics/Press.png}{\label{fig:FluidMechanics:Press}A schematic of a press.}
\begin{finalanswer}
	$\SI{2.80e8}{Pa}$
\end{finalanswer}
\begin{solution}
	While force is exerted on the lever, we can assume that the lever is in static equilibrium. The force, $F_d$ exerted upwards by the small cylinder on the lever counters the torque from the applied force:
	\begin{align*}
	F_dl&=F2l\\
	F_d&=2F
	\end{align*}
	The pressure in the fluid from the force on the small cylinder is thus:
	\begin{align*}
	p=\frac{F_d}{A_d}=\frac{2F}{A_d}
	\end{align*}
	where $A_d$ is the area of the small cylinder. The upwards force on the big cylinder, $F_D$ is thus:
	\begin{align*}
	F_D=pA_D=2F\frac{A_D}{A_d}
	\end{align*}
	where $A_D$ is the area of the big cylinder. Finally, the pressure on the sample is the upwards of the big cylinder divided by the area of the sample, $A_s$:
	\begin{align*}
	p_s&=\frac{F_D}{A_s}=2F\frac{A_D}{A_dA_s}=2F\frac{4\pi D^2}{4\pi d^2A_s}\\
	&=2(\SI{280}{N})\frac{(\SI{15}{cm})^2}{(\SI{1.5}{cm})^2(\SI{2.0e-4}{m^2})}\\
	&=\SI{2.80e8}{Pa}
	\end{align*}
\end{solution}

\question A cylindrical cork of radius $R$ and length $L$ is floating in a liquid such that its axis of symmetry is vertical and half of the length of the cork is below the fluid (the density of the fluid is $\rho_f$), as in Figure \ref{fig:FluidMechanics:Cork2}. For this problem, ignore any effects from friction, drag and the viscosity of the fluid.
\begin{parts}
	\part Show that the density of the cork, $\rho_c$, is half that of the fluid ($\rho_c= \frac{1}{2}\rho_f$)
	\part If the cork is displaced vertically by a distance $x$ ($x<\frac{L}{2})$ from its equilibrium (either pushed down or pulled up slightly compared to the depiction in the figure), show that the net force on the cork is given by:
	\begin{align*}
	F=-\pi R^2 \rho_fg x
	\end{align*}
	where the net force is in the opposite direction of the displacement, $x$.
	\part If you press down on the cork slightly (so that part of the cork is still above the fluid) and release it, the cork will oscillate up and down. What should the length of the cork be for it to oscillate with a frequency of \SI{5}{Hz}? 
\end{parts}
\capfig{0.2\textwidth}{figures/FluidMechanics/Cork2.png}{\label{fig:FluidMechanics:Cork2}A cork floating in a liquid.}
\begin{finalanswer}
	\begin{enumerate}[(a)]
		\item N/A
		\item N/A
		\item \SI{1.99}{cm}
	\end{enumerate}
\end{finalanswer}
\begin{solution}
	\begin{parts}
		\part The cork is in equilibrium when half of its volume is submerged. The buoyant force in this case is:
		\begin{align*}
		F_B=\pi R^2\frac{1}{2}L\rho_fg
		\end{align*}
		The weight of the cork, in terms of its density, is given by:
		\begin{align*}
		mg = (\pi R^2L\rho_c)g
		\end{align*}
		Since $F_B=mg$ in equilibrium, we have:
		\begin{align*}
		\pi R^2\frac{1}{2}L\rho_fg &=(\pi R^2L\rho_c)g\\
		\therefore \rho_c&=\frac{1}{2}\rho_f
		\end{align*}
		\part Let us choose that positive $x$ is downwards. The net force on the cork is given by the sum of the weight (positive) and buoyancy (negative) forces:
		\begin{align*}
		\sum F &= mg - F_B=(\pi R^2L\rho_c)g-\pi R^2\left(\frac{1}{2}L+x\right)\rho_fg\\
		&=(\pi R^2L\rho_c)g-\pi R^2\frac{1}{2}L\rho_fg -\pi R^2 \rho_fg x\\
		&=-\pi R^2 \rho_fg x
		\end{align*}
		where the first two terms in the second line cancelled since $\rho_c= \frac{1}{2}\rho_f$ (part (a)).
		\part The result from part (b) shows that the cork will undergo simple harmonic motion with frequency given by:
		\begin{align*}
		f&= \frac{\omega}{2\pi}=\frac{1}{2\pi} \sqrt{\frac{\pi R^2 \rho_fg }{m}}=\frac{1}{2\pi}\sqrt{\frac{\pi R^2 \rho_fg }{\pi R^2L\frac{1}{2}\rho_f}}=\frac{1}{2\pi}\sqrt{\frac{2g}{L}}\\
		\therefore L&=\frac{2g}{4\pi^2 f^2}=\frac{g}{2\pi^2 f^2}=\frac{(\SI{9.8}{m/s^2})}{2\pi^2 (\SI{5}{Hz})^2}=\SI{1.99}{cm}
		\end{align*}
	\end{parts}
\end{solution}

%Olivia W
\question  Divers can travel deep below the surface of the Earth. One of the challenges that divers face is great changes in pressure. Show that, if you take into account that $g$ changes as you get closer to the centre of the Earth, the pressure difference in going from $y_1$ to $y_2$ is:
\begin{align*}
P(y_2)-P(y_1)=-\frac{\rho GM}{2R^3}(y_2^2-y_1^2)
\end{align*}
where $y$ is the distance from the centre of the Earth, $\rho$ is the density of the water, $M$ is the mass of the Earth, $R$ is the radius of the Earth, and $G$ is the gravitational constant. 
Assume the Earth has a uniform density, and that the water is incompressible. 
\item What is the pressure difference between the Challenger Deep and a point just below the surface of the water?
\begin{solution}
	The first thing we will do is find an expression for $g$ as a function of $y$, where $y=0$ corresponds to the centre of the Earth.  To do this, we can use Gauss's law for gravity:
	\begin{align*}
	\int \vec g(\vec y)\cdot dA=4\pi GM^{enc}
	\end{align*}
	The enclosed mass is $\rho\cdot (4/3)\pi y^3$ (the density of the Earth multiplied by the enclosed volume). The mass of the Earth is $M=\rho\cdot (4/3)\pi R^3$, so we can write the enclosed mass as:
	\begin{align*}
	M^{enc}=M\frac{y^3}{R^3}
	\end{align*}
	The gravitational field is constant over the Gaussian surface (which is a sphere with a surface area of $4\pi y^2$), so our expression becomes:
	\begin{align*}
	-g(y)(4\pi y^2)=4\pi GM\frac{r^3}{R^3}
	\end{align*}
	where the left side is negative because the field points towards the centre of the Earth. Rearranging for $g(y)$ gives:
	\begin{align*}
	-\frac{GM}{R^3}y
	\end{align*}
	Now that we have an expression for $g(y)$, we consider an element of fluid with volume $V=Ady$, where $dy$ is the infinitesimal height. The fluid has a constant density, so its mass is $dm=\rho Ady$.
	
	We can model the pressure exerted by the fluid above the fluid element as $P+dP$ and the pressure exerted by the fluid below the fluid element as $P$, where $dP$ is a small negative change in pressure. The $y$ component of Newton's Second Law written for the fluid element is:
	\begin{align*}
	\sum F_y=PA-(P+dP)-dmg=0\\
	-dP-\rho g(y)dy=0\\
	\end{align*}
	Using our expression for $d_y$, this becomes:
	\begin{align*}
	-dP-\rho\left( \frac{GM}{R^3}y\right)dy=0\\
	\int_{y_1}^{y_2}dP=-\frac{\rho GM}{R^3}\int_{y_1}^{y_2}ydy\\
	P(y_2)-P(y_1)=-\frac{\rho GM}{R^3}\left(\frac{y_2^2}{2}-\frac{y_1^2}{2}\right)\\
	P(y_2)-P(y_1)=-\frac{\rho GM}{2R^3}(y_2^2-y_1^2)\\
	\therefore QED
	\end{align*}
\end{solution}



\question A water-based heating system is shown in Figure \ref{fig:FluidMechanics:WaterCircuit}. A pump exerts a constant pressure difference between its inlet and outlet, $\Delta P=\SI{12}{psi}$, to circulate hot water across 2 distinct paths. One path contains a single radiator that can be modelled as a pipe with an effective resistance to water flow of $R_1=\SI{4e7}{kg\cdot m^{-4}\cdot s^{-1}}$. The other path has two radiators with effective resistances to water flow given by $R_2=\SI{5e7}{kg\cdot m^{-4}\cdot s^{-1}}$ and $R_3=\SI{3e7}{kg\cdot m^{-4}\cdot s^{-1}}$. 

All of the pipes are at the same height so that Figure \ref{fig:FluidMechanics:WaterCircuit} is a top view. Assume that the pipes and the pump have effectively no resistance to the water flow (as their diameters are much larger than that of the radiators which are the only components that have a resistance to the flow of water).

\begin{parts}
	\part What is the pressure difference across each radiator?
	\part How many kilograms of water per second go through the pump and through each radiator?
\end{parts}

\capfig{0.3\textwidth}{figures/FluidMechanics/WaterCircuit.png}{\label{fig:FluidMechanics:WaterCircuit}A constant pressure pump feeding a small network of radiators in a house, as seen from above.}


\begin{solution}
	\begin{parts}
		\part Figure \ref{fig:FluidMechanics:WaterCircuit_sol} shows a labelled version of the water system.
		\capfig{0.3\textwidth}{figures/FluidMechanics/WaterCircuit_sol.png}{\label{fig:FluidMechanics:WaterCircuit_sol}Water flow through the different branches of the system.}
		
		We can model the circuit as follows:
		\begin{itemize}
			\item Water with a flow rate $Q$ flows into the pump and then out of the pump (from $D$ to $A$).
			\item Water arrives at the junction at point $A$ and splits into the two branches, with flow rates $Q_1$ (bottom branch) and $Q_2$ (top branch). Since water cannot disappear at the junction, the flow rate of water coming into the junction, $Q$, must be equal the flow rate of water coming out the junction (the sum of the flow rates through the top and bottom sections):
			\begin{align*}
			Q&=Q_1+Q_2
			\end{align*}
			\item The flow rate of water, $Q_1$, coming into radiator $R_1$ must equal the flow rate of water coming back out (by continuity), so one has the same flow rate through the path labelled $AFED$. 
			\item Similarly, the flow rate, $Q_2$, into radiator $R_2$ must equal the flow rate out of radiator $R_2$ (and $R_3$, so that the flow rate in the section $ABCD$ is everywhere the same. 
			\item The water in each branch then join again at point $D$ and sum to equal the flow rate through the pump. 
			\item The water pressure can only change if there is a resistance to the flow and thus the pressure only drops when the water goes through a resistor. As a consequence, the pressure drop across $R_1$ must be the same as across the pump:
			\begin{align*}
			\Delta P_1 = \Delta P
			\end{align*}
			and the sum of the pressure drops across $R_2$ and $R_3$ must also equal the pressure drop across the pump:
			\begin{align*}
			\Delta P_2 + \Delta P_3 = \Delta P 
			\end{align*}
		\end{itemize}
		Putting all this together, we can first easily find the pressure difference across $R_1$, since it is equal to the know pressure difference across the pump:
		\begin{align*}
		\Delta P_1 = \Delta P=\SI{12}{psi}=\SI{82737.1}{Pa}
		\end{align*}
		We are left with the two following equations:
		\begin{align*}
		Q&=Q_1+Q_2 &\text{(Flow in/out of junction A)}\\
		\Delta P_2 + \Delta P_3 &= \Delta P &\text{(Pressure change is that of pump)}
		\end{align*}
		We also know that the flow rate through a resistance is related to the pressure drop across that resistance:
		\begin{align*}
		\Delta P = QR
		\end{align*}
		so that we can write the pressure equation as:
		\begin{align*}
		\Delta P_2 + \Delta P_3 &= \Delta P\\
		Q_2R_2+Q_2R_3&= \Delta P\\
		\therefore Q_2 &=\frac{\Delta P}{R_2+R_3}\\
		&=\frac{(\SI{82737.1}{Pa})}{(\SI{5e7}{kg\cdot m^{-4}\cdot s^{-1}})+(\SI{3e7}{kg\cdot m^{-4}\cdot s^{-1}})}=\SI{1.03e-3}{m^3/s}
		\end{align*}
		Given the flow rate through $R_2$ and $R_3$, we can find the corresponding pressure drops across each radiator:
		\begin{align*}
		\Delta P_2&= R_2Q_2=(\SI{5e7}{kg\cdot m^{-4}\cdot s^{-1}})(\SI{1.03e-3}{m^3/s})=\SI{51710.69}{Pa}\\
		\Delta P_3&= R_3Q_2=(\SI{3e7}{kg\cdot m^{-4}\cdot s^{-1}})(\SI{1.03e-3}{m^3/s})=\SI{31026.41}{Pa}\\
		\end{align*}
		\part To determine the mass flow rate through each component, we multiply the flow rate by the density of water, $\rho$. The mass flow rate through $R_1$ is given by:
		\begin{align*}
		\rho Q_1 = \rho \frac{\Delta P_1}{R_1}=(\SI{1000}{kg/m^3})\frac{(\SI{82737.1}{Pa})}{(\SI{4e7}{kg\cdot m^{-4}\cdot s^{-1}})}=\SI{2.07}{kg/s}\\
		\end{align*}
		The mass flow rate through $R_2$ and $R_3$ is the same:
		\begin{align*}
		\rho Q_2 = \rho \frac{\Delta P_2}{R_2}=(\SI{1000}{kg/m^3})\frac{(\SI{51710.69}{Pa})}{(\SI{5e7}{kg\cdot m^{-4}\cdot s^{-1}})}=\SI{1.03}{kg/s}\\
		\end{align*}
		And the mass flow rate through the pump is given by:
		\begin{align*}
		\rho Q=\rho(Q_1+Q_2)=(\SI{2.07}{kg/s})+(\SI{1.03}{kg/s})=\SI{3.1}{kg/s}
		\end{align*}
	\end{parts}
\end{solution}



\question A cork in the shape of a cylinder of radius $r=\SI{1}{cm}$ and length $L=\SI{5}{cm}$ is floating upright on water (density $\rho_W=\SI{1}{g/cm^3}$) such that one quarter of the cork's height is above water (Figure \ref{fig:FluidMechanics:Cork3}). What is the minimum amount of that work must be done by a person pressing down on the cork in order to fully submerge the cork?

Assume that there is no friction or drag, and that the cork always remains upright. 
\capfig{0.3\textwidth}{figures/FluidMechanics/Cork3.png}{\label{fig:FluidMechanics:Cork3} Cork floating in water.}
\begin{finalanswer}
	$\SI{2.41e-4}{J}$
\end{finalanswer}
\begin{solution}
	We do not know the density of the cork, however we can relate it to the density of water, since the buoyancy force is equal to the weight of the cork when a quarter of the cork is above water:
	\begin{align*}
	F_B &= F_g\\
	\frac{3}{4}\rho_WL\pi r^2 g &= \rho_c L \pi r^2 g\\
	\therefore \frac{3}{4}\rho_W&= \rho_c
	\end{align*}
	In order to submerge the cork, one must do work over a distance $L/4$ that is equal to the sum of the work done by gravity and the work done by the force of buoyancy. Gravity will do positive work given by:
	\begin{align*}
	W_g = F_g \frac{L}{4}=\frac{1}{4}mgL =\frac{1}{4} \rho_c L^2 \pi r^2 g=\frac{3}{16} \rho_W L^2 \pi r^2 g
	\end{align*}
	where we wrote the mass of the cork in terms of its density, $\rho_c$, and its volume $\pi r^2 L$, where $r$ is its radius, and then substituted the density of the cork in terms of the density of water.
	
	The force of buoyancy depends on the submerged volume. If we let $x$ be the height of the cork above water (initially $x=\frac{1}{4}L$), then the magnitude of the force of buoyancy can be written as:
	\begin{align*}
	F_B=\rho_W(L-x)\pi r^2 g
	\end{align*}
	The work done by the force of buoyancy as the cork is moved down is negative and given by:
	\begin{align*}
	W_B&=-\int_0^{\frac{L}{4}}F_Bdx=-\rho_W\pi r^2g\int_0^{\frac{L}{4}}(L-x)dx\\
	&=-\rho_W\pi r^2g\left[Lx-\frac{1}{2}x^2 \right]_0^{\frac{L}{4}}\\
	&=-\rho_W\pi r^2g\left(\frac{1}{4}L^2-\frac{1}{2}\frac{L^2}{16}   \right)\\
	&=-\frac{7}{32}\rho_W\pi r^2gL^2
	\end{align*}
	In order to submerge the cork, the net work done on the cork is zero, so that the work done by the person, $W_P$, can be found:
	\begin{align*}
	W^{net}&=W_g+W_B+W_P=0\\
	\therefore W_P&=-(W_g+W_B)=\frac{7}{32}\rho_W\pi r^2gL^2-\frac{3}{16} \rho_W L^2 \pi r^2 g\\
	&=\frac{1}{32} \rho_W L^2 \pi r^2 g = \frac{1}{32} (\SI{1000}{kg/m^3}) (\SI{0.05}{m})^2 \pi (\SI{0.01}{m})^2 (\SI{9.8}{m/s^2})\\
	&=\SI{2.41E-4}{J}
	\end{align*}
	
\end{solution}


\question The density of the ice in an iceberg is about $\rho_I=\SI{0.92}{g/ml}$. The density of salt water is about $\rho_W=\SI{1.03}{g/ml}$. 
\begin{parts}
	\part If such an iceberg is floating on salt water, what fraction of the iceberg's volume is below the surface of the water (assume that the iceberg has a uniform density)?
	\part If a polar guanaco swims up to an iceberg shaped as a cube of side $L=\SI{10}{m}$ (same density as above), climbs to the top, and then jumps off into the salt water, with what frequency will the iceberg oscillate once the guanaco has jumped off? Assume that there is no friction or drag on the iceberg, and that it oscillates up and down in the vertical direction. 
\end{parts}
\begin{finalanswer}
	\begin{parts}
		\part $\SI{0.89}{}$
		\part $\SI{0.167}{Hz}$
	\end{parts}
\end{finalanswer}
\begin{solution}
	\begin{parts}
		\part The fraction of iceberg below water is given by the requirement that the resulting force of buoyancy be equal to the weight of the iceberg:
		\begin{align*}
		F_B = mg
		\end{align*}
		The mass of the iceberg is given by:
		\begin{align*}
		m = \rho_I V
		\end{align*}
		where $V$ is the total volume of the iceberg. The force of buoyancy is given by:
		\begin{align*}
		F_B = \rho_W g V_{sub}
		\end{align*}
		where $V_{sub}$ is the part of the iceberg that is submerged. Putting this together:
		\begin{align*}
		F_B &= mg\\
		\rho_W g V_{sub} &= \rho_I V g\\
		\therefore \frac{V_{sub}}{V}&=\frac{\rho_I}{\rho_W}=\frac{(\SI{0.92}{g/ml})}{(\SI{1.03}{g/ml})}=0.89
		\end{align*}
		Thus, 89\% of the volume of the iceberg is below the surface of the water.
		\part In equilibrium, $0.89L$ of the vertical height of the iceberg in under water (part a), and the buoyancy force on the iceberg is equal to its weight:
		\begin{align*}
		F_B &= mg \\
		\rho_W g 0.89L^3 &= \rho_I L^3 g
		\end{align*}
		If the iceberg is displaced by a distance $x$ from equilibrium (let positive $x$ mean that the iceberg moved upwards by $x$ relative to its equilibrium), the force of buoyancy is given by:
		\begin{align*}
		F_B = \rho_W g (0.89L-x)L^2 
		\end{align*}
		The net vertical force on the iceberg (positive upwards), when displaced by a distance $x$ from equilibrium, is thus given by:
		\begin{align*}
		\sum F &= F_B - mg = \rho_W g (0.89L-x)L^2 - \rho_I L^3 g\\
		&=\rho_W g 0.89L^3 -\rho_W g xL^2 - \rho_I L^3 g\\
		&=-\rho_W g xL^2
		\end{align*}
		where we used the equilibrium condition to cancel the terms that don't have $x$ in them. The net force is equal to the mass of the iceberg, $m$, times its acceleration:
		\begin{align*}
		ma &= -\rho_W g xL^2\\
		\therefore a &= \frac{d^2x}{dt^2}=-\frac{\rho_W g L^2}{m}x
		\end{align*}
		which corresponds to simple harmonic motion with angular frequency:
		\begin{align*}
		\omega = \sqrt{\frac{\rho_W g L^2}{m}}=\sqrt{\frac{\rho_W g L^2}{\rho_I L^3}}=\sqrt{\frac{\rho_W g }{\rho_I L}}
		\end{align*}
		The frequency is thus given by:
		\begin{align*}
		f &= \frac{\omega}{2\pi}=\frac{1}{2\pi}\sqrt{\frac{\rho_W g }{\rho_I L}}\\
		&=\frac{1}{2\pi}\sqrt{\frac{(\SI{1030}{kg/m^3})(\SI{9.8}{m/s^2}) }{(\SI{920}{kg/m^3}) (\SI{10}{m})}} = \SI{0.167}{Hz}
		\end{align*}
	\end{parts}
	
\end{solution}



\question You connect a straight horizontal water hose (radius $r=\SI{1}{cm}$) to the bottom of a water tank, where the water is at a height of $H=\SI{5}{m}$ from the location where the hose is connected (Figure \ref{fig:FluidMechanics:TankPoiseuille}). The hose has a length of $L=\SI{10}{m}$ and water has a viscosity of $\eta = \SI{0.001}{kg/(m\cdot s)}$. Assume that the flow through the hose is laminar and incompressible, that the level in the tank is constant, and that atmospheric pressure does not change appreciably between the top of the tank and the horizontal hose. The density of water is \SI{1}{kg/litre}.
\begin{parts}
	\part What is the speed of the water as it exits the hose?
	\part What is the speed of the water exiting the hose if you ignore its viscosity?
\end{parts}

\textbf{Hint:} Bernoulli's equation holds in the tank which allows you to determine the pressure in the water as it enters the hose (through which it does not change speed, by continuity). In particular, the pressure difference across the hose is \textbf{not} $\rho g H$.

\capfig{0.3\textwidth}{figures/FluidMechanics/TankPoiseuille.png}{\label{fig:FluidMechanics:TankPoiseuille} Water flowing from a tank through a horizontal hose.}

\begin{finalanswer}
	\begin{parts}
		\part $\SI{9.13}{m/s}$
		\part $\SI{9.90}{m/s}$
	\end{parts}
\end{finalanswer}

\begin{solution}
	\begin{parts}
		\part 
		First, we use Bernoulli's equation to compare the water at the top of the tank (height $y_1=H$, pressure, $P_1=P_0$ (atmospheric), and speed $v_1=0$) to the water at the bottom of the tank (height $y_2=0$, pressure $P_2$ and speed $v_2$), whose speed and pressure we do not know:
		\begin{align*}
		P_1+\frac{1}{2}\rho v_1^2+\rho gy_1 &= P_2 \frac{1}{2}\rho v_2^2 +\rho g y_2\\
		P_0+\rho g H &= P_2 + \frac{1}{2}\rho v_2^2 \\
		\therefore P_2 &= P_0+\rho g H -\frac{1}{2}\rho v_2^2
		\end{align*}
		The flow rate across the hose is given by Poiseuille's equation:
		\begin{align*}
		Q=\frac{\pi r^4}{8\eta L}\Delta P
		\end{align*}
		where $\Delta P$ is the change in pressure across the hose (from $P_2$ to atmospheric $P_0$):
		\begin{align*}
		\Delta P = P_2 - P_0=\rho g H -\frac{1}{2}\rho v_2^2
		\end{align*}
		The speed of the water is constant through the hose and is given by the flow rate and the area of the cross-section of the hose:
		\begin{align*}
		Q &= Av_2=\pi r^2 v_2\\
		\end{align*}
		Putting everything together:
		\begin{align*}
		Q&=\frac{\pi r^4}{8\eta L}\Delta P=\frac{\pi r^4}{8\eta L}(\rho g H -\frac{1}{2}\rho v_2^2)\\
		Q &= \pi r^2 v_2\\
		\therefore \pi r^2 v_2 &= \frac{\pi r^4}{8\eta L}(\rho g H -\frac{1}{2}\rho v_2^2)\\
		\frac{1}{2}\rho v_2^2 + \frac{8\eta L}{r^2} v_2 -\rho gH &=0\\
		\end{align*}
		leads to a quadratic equation for $v_2$:
		\begin{align*}
		\frac{1}{2}(\SI{1000}{kg/m^3}) v_2^2 + \frac{8(\SI{0.001}{kg/(m\cdot s)}) (\SI{10}{m})}{(\SI{0.01}{m})^2} v_2 -(\SI{1000}{kg/m^3}) (\SI{9.8}{m/s^2})(\SI{5}{m}) &=0\\
		(\SI{500}{kg/m^3})v_2^2+(\SI{800}{kg/(m^2\cdot s)})v_2-\SI{49000}{kg/(m \cdot s^2)}&=0
		\end{align*}
		There is only one positive root:
		\begin{align*}
		v_2=\SI{9.13}{m/s}
		\end{align*} 
		\part If we can ignore the viscosity of the water, then there is no pressure drop across the hose, and we can model this as if the water flowed out of a hole at the bottom of the tank directly to atmospheric pressure:
		\begin{align*}
		P_1+\frac{1}{2}\rho v_1^2+\rho gy_1 &= P_2 \frac{1}{2}\rho v_2^2 +\rho g y_2\\
		\rho g H &=  \frac{1}{2}\rho v_2^2 \\
		\therefore v_2&=\sqrt{2gH}=\SI{9.90}{m/s}
		\end{align*}
		which results in a higher speed, as expected.
	\end{parts}
\end{solution}

\question A tank with a total height $h$ is filled to the top with water. A hole is placed a distance $s$ from the top of the tank, as shown in Figure \ref{fig:FluidMechanics:LeakyTank} so that water leaks out of the hole and lands a distance $l$ from the bottom of the tank. At what distance $s$ from the top of the tank should the hole be placed in order to maximize the distance $l$ at which the water lands?

Assume incompressible flow with no viscosity, neglect air friction, and assume that atmospheric pressure does not change appreciably over the height $h$. Furthermore, assume that the level in the tank is constant.

\capfig{0.3\textwidth}{figures/FluidMechanics/LeakyTank.png}{\label{fig:FluidMechanics:LeakyTank}A tank leaking water.}

\begin{finalanswer}
	The hole should be placed halfway down the tank
\end{finalanswer}

\begin{solution}
	
	First, we determine the speed of the water exiting the tank a distance $s$ from the top. We can use Bernoulli's principle; the water is at rest at the top of the tank, where the pressure is atmospheric, $P_0$. When the water exits the tank, it is at atmospheric pressure and has a speed of $v$. For the purpose of Bernoulli's principle, we choose the origin of the vertical axis to be located where the hole is placed:
	\begin{align*}
	P_1 + \frac{1}{2}\rho v_1^2 + \rho gy_1 &=P_2 + \frac{1}{2}\rho v_2^2 + \rho gy_2\\
	P_0 + (0) +\rho g s&=P_0+\frac{1}{2}\rho v^2 + (0)\\
	\therefore v&=\sqrt{2gs}
	\end{align*}
	as expected. We can now find the range of a projectile with an initial horizontal velocity of $v$ that covers a vertical distance $h-s$. The time that it takes to fall that distance is given by:
	\begin{align*}
	y_f &= y_0+v_0t+\frac{1}{2}a_yt^2\\
	0 &= (h-s)-\frac{1}{2}gt^2\\
	\therefore t &= \sqrt{\frac{2(h-s)}{g}}
	\end{align*}
	The horizontal distance that is covered corresponds to:
	\begin{align*}
	l = vt = \sqrt{2gs}\sqrt{\frac{2(h-s)}{g}}=\sqrt{4s(h-s)}
	\end{align*}
	We now need to maximize $l$ with respect to $s$, which we can do by setting the derivative of $l$ with respect to $s$ to zero. However, if the quantity $4gd(h-s)$ is maximized, so will the square root, so we only need to differentiate this last quantity:
	\begin{align*}
	\frac{d}{ds}4s(h-s)&=\frac{d}{ds}(4hs-4s^2)=4h-8s
	\end{align*}
	Setting this to zero and solving for $s$:
	\begin{align*}
	\therefore s&=\frac{h}{2}
	\end{align*}
	The hole should be placed half way down the tank.
	
\end{solution}

%TODO: Calculate the fraction of the volume of an iceberg that is under water.
%TODO: Angle of a balloon in accelerating car, take into account mass of balloon (so you need the inertial force on the balloon, or a net horizontal acceleration to add to g)