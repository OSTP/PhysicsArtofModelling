
\chapter{Electric circuits}
\label{chapter:circuits}
In this chapter, we develop the tools to model electric circuits. This will allow us to determine the current and voltages across different elements, resistors and capacitors, within a circuit. We will also discuss how a battery can provide a current at a fixed potential difference, and how one can construct devices to measure current and voltages.

\begin{learningObjectives}{
 \item Understand how a battery works.
 \item Understand Kirchhoff rules and how to apply them.
 \item Understand how to model a circuit with resistors and/or capacitors.
 \item Understand how an ammeter and voltmeter function, and how to model them.
 }
\end{learningObjectives}

\begin{opening}
\begin{MCquestion}{If two outlets in your house are connected to the same circuit, are the outlets connected in series or in parallel?}
\item series
\item parallel \correct
\end{MCquestion}
\end{opening}

\section{Batteries and simple circuits}
%How a battery works, ideal vs real battery, internal resistance
Energy must be provided in order for charges to move through a circuit, since those charges will expend energy in the form of heat as they move through a resistor. A simple way to provide that energy is to use a battery, which is a device that provides a constant potential difference (a fixed voltage) between its terminals. Luigi Galvani was the first to realize that certain combination of metals placed into contact with each other can lead to an electric potential difference (or rather, they can make the legs of a dead frog twitch, which we now understand to be from the potential difference). Effectively, Galvani created the first ``electrochemical cell''. Alessandro Volta then combined several of these cells together to form the ``voltaic pile'', which is what we would now call a battery (a battery, technically, is a combination of several cells, although one often uses the term battery even if only one cell is involved). 
\subsection{The electrochemical cell}
A simple electric cell can be constructed from metals that have different affinities to be dissolved in acid. A simple cell, similar to that originally made by Volta, can be made using zinc and carbon as the ``electrodes'' (Volta used silver instead of carbon) and a solution of dilute sulfuric acid (the liquid is called the ``electrolyte''), as illustrated in Figure \ref{fig:circuits:electriccell}. Before the cell is constructed, the electrodes and the electrolyte are all electrically neutral.
\capfig{0.5\textwidth}{figures/Circuits/electriccell.png}{\label{fig:circuits:electriccell} A simple electric cell, where zinc ions dissolve in sulfuric acid leaving electrons on the metal.}
%TODO Is it incorrect to show the electrons entering the solution?
Once the zinc is immersed in the electrolyte, the zinc atoms tend to dissolve into the electrolyte in the form of zinc ions (doubly charged, Zn$^{2+}$). This leaves an excess of electrons the zinc electrode, resulting in a net negative electric charge. Similarly, the positively charged zinc ions attract electrons from the carbon electrode into the solution, leaving the carbon electrode positively charged. Very quickly, an equilibrium is reached, since at some point, the negative charge of the zinc electrode will electrically attract positive zinc ions, preventing any more zinc ions from dissolving into the solution. Similarly, as the carbon electrode builds a positive charge, that charge will eventually prevent electrons from ``jumping'' into the solution. At this point, there will be a fixed electric potential difference between the two electrodes. 

If the two electrodes are connected with a resistor, the electrons will leave the zinc electrode, cross the resistor, and end up on the positive carbon electrode. This will leave space for more electrons on the zinc electrode, so more zinc ions will dissolve into the solution. Thus, a circuit is formed, where electron travel up the zinc electrode, through the resistor and back down the carbon electrode. At the same time, more and more zinc ions dissolve into the electrolyte, until the zinc electrode is completely dissolved. In practice, the zinc ions travel through the solution and plate onto the carbon electrode (the electrons do not quite ``jump'' into the electrolyte, rather, it is the zinc ions that move in the electrolyte). Since the charge on the electrodes is continuously replenished, the potential difference between the electrodes remains constant.

The electric cell will stop working once the zinc electrode has completely dissolved. Note that there is a maximum current that the cell can supply, which depends on the rate at which the zinc can dissolve into the electrolyte and plate onto the carbon electrode. If the electrodes of the cell are connected with a very low resistance resistor, the resulting current will be too large for the potential difference to be maintained. 

\subsection{The ideal battery in a circuit}
As we proceed, we will use the term ``battery'' loosely to refer to a device (such as an electric cell or collection of cells) that can provide a fixed potential difference between two terminals (or electrodes). Figure \ref{fig:circuits:batterysymbol} shows the circuit diagram for a battery, consisting in two vertical bars, with the larger bar indicating the positive terminal of the battery.
\begin{center}
\begin{circuitikz}[]
\draw (2,0) to [battery1, l=$\Delta V$] (0,0);
     \draw (0.65,0.3) node{$-$};
     \draw (1.35,0.3) node{$+$};
\end{circuitikz}
\captionof{figure}{\label{fig:circuits:batterysymbol}Circuit diagram symbol for a battery.} 
\end{center}
Figure \ref{fig:circuits:resistorsymbol} shows the circuit diagram symbols that are used for a resistor (different symbols are used in North American and in Europe).
\begin{center}
\begin{circuitikz}[]
\draw (2,0) to [R=$R$] (0,0);
\end{circuitikz}
\begin{circuitikz}[european]
\draw (2,0) to [R=$R$] (0,0);
\end{circuitikz}
\captionof{figure}{\label{fig:circuits:resistorsymbol}Circuit diagram symbols for a resistor, using the North American convention (left), and the European convention (right).} 
\end{center}
Figure \ref{fig:circuits:batteryresistor} shows a circuit diagram for a very simple circuit consisting of a single $\SI{9}{V}$ battery connected to a $\SI{2}{\Omega}$ resistor. When drawing a circuit diagram (or making a real circuit), one connects the various components together (e.g. batteries and resistors) with \textbf{segments of wire that have zero resistance}, even if, in practice, wires always have some resistance. However, since the wires are connected in series with resistors (or other components that have a resistance), one can always include the resistance of the wires by adding it to the resistance of the other components. For example, in Figure \ref{fig:circuits:batteryresistor}, if the wires have a total resistance of $\SI{1}{\Omega}$, we could simply model the circuit as if the resistor had a resistance of $\SI{3}{\Omega}$ instead of $\SI{2}{\Omega}$. In practice, this is usually accounted for when a circuit diagram is made (i.e. any resistors include the resistance of the wires connected to it). 
\begin{center}
\begin{circuitikz}[]
\draw (4,0) node[anchor=north]{b}
      to [battery1,l=\SI{9}{V},*-*, i<=$I$] (0,0) node[anchor=north]{a}
      to [short,i<=$I$] (0,2) node[anchor=south]{d} 
      to [R,l_=\SI{2}{\ohm},i<=$I$,*-*] (4,2) node[anchor=south]{c}
      to [short,i<=$I$](4,0);  
     \draw  [->,>=stealth, line width=1mm] (1.65,0.6) -- (2.35,0.6);
     \draw (1.65,0.3) node{$-$};
     \draw (2.35,0.3) node{$+$};
\end{circuitikz}
\captionof{figure}{\label{fig:circuits:batteryresistor}A simple circuit, showing a \SI{9}{V} battery and a \SI{2}{\Omega} resistor. For ease in analyzing circuits, we suggest drawing a ``battery arrow'' above batteries that goes from the negative to the positive terminal.} 
\end{center}
The circuit in Figure \ref{fig:circuits:batteryresistor} is simple to analyze. In this case, whichever charges exit one terminal of the battery, must pass through the resistor and then enter the other terminal of the battery. We \textbf{always use conventional current} to analyze a circuit. Thus, we model the circuit as if positive charges exit the positive terminal of the battery, go through the resistor, and then enter the negative terminal of the battery.

We recommend that you always draw a ``battery arrow'' arrow for each battery in a circuit diagram that indicates the direction in which conventional current would exit the battery if a simple resistor were connected across the battery. In complex circuits, the current may not necessarily flow in the same direction as the battery arrow, and the battery arrow makes it easier to analyze those circuits. We also indicate the current that is flowing in any wire of the circuit by drawing an arrow direction on that wire (labeled $I$ in Figure \ref{fig:circuits:batteryresistor}).

Since the wires have no resistance, the electric potential cannot change through a section of wire. In other words, because the wire has no resistance, the charges/current cannot dissipate any power in the wire ($P=I^2R$), and the charges do not ``loose'' any potential energy (and the potential thus cannot change). The only place where the charges can dissipate energy is inside the resistor. Once the charges have crossed the resistor, the electric potential in the wire is again constant until they reach the other terminal of the battery. Thus, in this simple circuit, the electric potential difference across the resistor is the same as the potential difference across the terminals of the battery. This allows us to apply Ohm's Law (the macroscopic version) to the resistor and determine the current in the circuit:
\begin{align*}
\Delta V&=RI\\
\therefore I&=\frac{\Delta V}{R}=\frac{(\SI{9}{V})}{(\SI{2}{\Omega})}=\SI{4.5}{A}
\end{align*}

It is helpful to think of circuits in terms of conservation of energy. Charges can only dissipate energy if there is resistance. Thus, charges dissipate no energy in the wires, and the electric potential is always constant along a wire. Batteries provide the energy to ``push'' the charges through a resistor; they convert chemical potential energy into the electrical potential energy of the charges, which then gain kinetic energy and loose that kinetic energy in the form of thermal energy by heating up the resistor.

It is also useful to make the analogy with fluid dynamics; one can think of the battery as a pump that is continuously pushing a viscous incompressible fluid through a pipe with a narrow section, as illustrated in Figure \ref{fig:circuits:watercircuit}. The wide section of the pipe is akin to the wires with no resistance, and the narrow section is akin to the resistor. The pressure difference generated by the pump is analogous to the voltage produced by the battery, and the flow rate of the liquid is analogous to the electric current. The pressure in the pipe does not drop in the wide section, if there is no resistance. The entire pressure drop of the fluid is across the narrow section.
\capfig{0.4\textwidth}{figures/Circuits/watercircuit.png}{\label{fig:circuits:watercircuit} A fluid dynamics analogue of the circuit in Figure \ref{fig:circuits:batteryresistor}, where a pump plays the role of the battery, and a narrow pipe that of a resistor.}
\begin{example}{\label{ex:circuits:tworesistors}Two resistors, of $\SI{2}{\Omega}$ and $\SI{4}{\Omega}$, respectively, are connected in series to a $\SI{12}{V}$ battery. What is the current through each of the resistors, and what is the voltage across each resistor?}
We start by making a circuit diagram, as in Figure \ref{fig:circuits:tworesistors}, showing the resistors, the current, $I$, the battery and the battery arrow. Note that since this is a closed circuit with only one path, the current through the battery, $I$, is the same as the current through the two resistors.
\begin{center}
\begin{circuitikz}[]
\draw (4,0) node[anchor=north]{b}
      to [battery1,l=\SI{12}{V},*-*, i<=$I$] (0,0) node[anchor=north]{a}
      to [short,i<=$I$] (0,2) node[anchor=east]{e} 
      to [R,l_=\SI{4}{\ohm},i<=$I$,*-]node[anchor=south]{d} (2,2) 
      to [R,l_=\SI{2}{\ohm},*-*] (4,2) node[anchor=south]{c}
      to [short,i<=$I$](4,0);  
     \draw  [->,>=stealth, line width=1mm] (1.65,0.6) -- (2.35,0.6);
     \draw (1.65,0.3) node{$-$};
     \draw (2.35,0.3) node{$+$};
\end{circuitikz}
\captionof{figure}{\label{fig:circuits:tworesistors}Two resistors connected in series with a battery} 
\end{center}
If we choose the potential on the negative side of the battery to be $\SI{0}{V}$, then points $a$ and $e$ on the diagram are at a potential of $\SI{0}{V}$, since potential cannot change in a wire with no resistance. Similarly, the points at $b$ and $c$ are at a potential of $\SI{12}{V}$ (relative to points $a$ and $e$). At point $d$, between the two resistors, the potential will be between $\SI{0}{V}$ and $\SI{12}{V}$, since the potential will ``drop'' as the current goes through the $\SI{2}{\Omega}$ resistor.

The easiest way to determine the current through this simple circuit is to combine the two resistor into a single effective resistor with resistance:
\begin{align*}
R_{eff}=(\SI{2}{\Omega})+(\SI{4}{\Omega})=\SI{6}{\Omega}
\end{align*}
so that the circuit can be simplified to that show in Figure \ref{fig:circuits:batteryresistor2}:
\begin{center}
\begin{circuitikz}[]
\draw (4,0) node[anchor=north]{b}
      to [battery1,l=\SI{12}{V},*-*, i<=$I$] (0,0) node[anchor=north]{a}
      to [short,i<=$I$] (0,2) node[anchor=south]{e} 
      to [R,l_={$R_{eff}{=}\SI{6}{\ohm}$},i<=$I$,*-*] (4,2) node[anchor=south]{c}
      to [short,i<=$I$](4,0);  
     \draw  [->,>=stealth, line width=1mm] (1.65,0.6) -- (2.35,0.6);
     \draw (1.65,0.3) node{$-$};
     \draw (2.35,0.3) node{$+$};
\end{circuitikz}
\captionof{figure}{\label{fig:circuits:batteryresistor2}The resistors from the circuit in Figure \ref{fig:circuits:tworesistors} have been combined in series to simplify the circuit.} 
\end{center}
The potential difference across the effective resistor is the same as that across the battery (between points $e$ and $c$), so that Ohm's Law can be applied to the effective resistor to determine the current that traverses it:
\begin{align*}
\Delta V &= R_{eff}I\\
\therefore I&=\frac{\Delta V}{R_{eff}}=\frac{(\SI{12}{V})}{(\SI{6}{\ohm})}=\SI{2}{A}
\end{align*}
Of course, this current is the same that traverses each individual resistor. Referring back to the full circuit (Figure \ref{fig:circuits:tworesistors}), we can now use Ohm's Law to calculate the voltage drop across each resistor, since we know the current through each resistor. The voltage across the $\SI{2}{\Omega}$ resistor is given by:
\begin{align*}
\Delta V_{2\Omega}=RI=(\SI{2}{\Omega})(\SI{2}{A})=\SI{4}{V}
\end{align*}
and the voltage across the $\SI{4}{\Omega}$ resistor is given by:
\begin{align*}
\Delta V_{4\Omega}=RI=(\SI{4}{\Omega})(\SI{2}{A})=\SI{8}{V}
\end{align*}
Note that the sum of these two voltages is equal to the voltage increase across the battery, by conservation of energy. Consider the electric potential at different points in Figure \ref{fig:circuits:tworesistors} as you move clockwise around the loop starting at point $a$. If the electric potential is defined to be $\SI{0}{V}$ at the negative end of the battery (points $a$ and $e$), the potential at point $d$ (between the resistors) is the potential at point $e$ plus the potential difference across the $\SI{4}{\Omega}$ resistor:
\begin{align*}
V_d = V_e+\Delta V_{4\Omega}=(\SI{0}{V})+(\Delta V_{4\Omega})=\SI{8}{V}
\end{align*}
If we then add the potential difference across the $\SI{2}{\Omega}$ resistor to the potential at point $d$, we find  that the potential at point $c$ is $V_c=V_d+\Delta V_{2\Omega}=\SI{12}{V}$, as expected, since this corresponds to the potential at the positive terminal of the battery.

\textbf{Discussion: }In this example, we showed how one can model a circuit by combining resistors together into effective resistors to simplify the circuit. We also showed how the potential differences across different components in a circuit must add up to zero (the voltage drops across the resistors must sum to the voltage increase across the battery). 
\end{example}

%TODO Checkpoint question, what is the voltage across a 3V battery connected in series with a 6V battery, if the negative terminal of the 6V battery faces the positive terminal of the 3V battery? (correct: 9V)


\subsection{The real battery in a circuit}
So far, we have modelled batteries as ``ideal'' devices that provide a fixed potential difference. In reality, this neglects the fact that the components that make the battery will themselves have a resistance. For example, if electrons want to leave the zinc rod in the electric cell illustrated in Figure \ref{fig:circuits:electriccell}, they will loose some energy as they pass through the zinc. Thus, when modelling a battery, it is important to include their ``internal resistance''. This is illustrated in Figure \ref{fig:circuits:realbattery}, which shows the two terminals of a real battery, an ideal battery (with a fixed potential difference, $\Delta V_{ideal}$), and internal resistance, $R$ (which can be drawn on either side of the battery). 
\begin{center}
\begin{circuitikz}[]
\draw (4,0) to [R=internal $R$,*-] (2,0)
     to [battery1, l=$\Delta V_{ideal}$, -*] (0,0);
     \draw (0.65,0.3) node{$-$};
     \draw (1.35,0.3) node{$+$};
     \draw (0,0) node[anchor=east]{Negative terminal};
     \draw (4,0) node[anchor=west]{Positive terminal};
\end{circuitikz}
\captionof{figure}{\label{fig:circuits:realbattery}Circuit diagram symbol for a battery.} 
\end{center}
It is important to note that the potential difference across the terminals of the real battery is only equal to the potential difference across the ideal battery \textbf{if there is no current flowing through the battery}. If there is a current, $I$, flowing through the internal resistance, the electric potential will decrease by an amount $RI$ across the internal resistance, so that the voltage across the real terminals is no longer the same as the voltage across the terminals of the ideal battery. 

\begin{example}{When no resistance is connected across a real battery, the potential difference across its terminals is measured to be $\SI{6}{V}$. When a $R=\SI{2}{\Omega}$ resistor is connected across the battery, a current of $\SI{2}{A}$ is measured through the resistor. What is the internal resistance, $r$, of the battery, and what is the voltage across its terminals when the resistor is connected?}
The real battery can be modelled as an ideal battery with potential difference, $\Delta V_{ideal}$, in series with an internal resistance, $r$. While we do not know the value of the internal resistance, we are told that the potential difference across the terminals of real battery is $\SI{6}{V}$ \textbf{when no current flows through it}. Since no current flows through the internal resistance, the voltage does not drop across the internal resistance, and the voltage across the terminals of the real battery (e.g. Figure \ref{fig:circuits:realbattery}) must thus be equal to the voltage across the terminals of the ideal battery, so that $\Delta V_{ideal}=\SI{6}{V}$.

With this information, we can make a circuit diagram for the case when the $\SI{2}{\Omega}$ resistor is connected across the terminals of the real battery, as in Figure \ref{fig:circuits:realbatterycircuit}.
\begin{center}
\begin{circuitikz}
\draw (4,0) node[anchor=north]{c} to [R=$r$,*-] (2,0) node[anchor=north]{b}
      to [battery1, l=$\SI{6}{V}$, *-*,i<=$I$] (0,0) node[anchor=north]{a}
      to [short,i<=$I$] (0,2) node[anchor=south]{e} 
      to [R,l_={$R{=}\SI{2}{\Omega}$},i<=$I$,*-*] (4,2) node[anchor=south]{d}
      to [short,i<=$I{=}\SI{2}{A}$] (4,0);  
     \draw [->,>=stealth, line width=1mm] (0.65,0.6) -- (1.35,0.6);
     \draw (0.65,0.3) node{$-$};
     \draw (1.35,0.3) node{$+$};
\end{circuitikz}
\captionof{figure}{\label{fig:circuits:realbatterycircuit}A circuit showing a real battery (with internal resistance $r$) in series with a resistor.} 
\end{center}

The terminals of the real battery are located at points $a$ and $c$ of the diagram, whereas the terminals of the ideal battery corresponds to points $a$ and $b$. When no current flows through the internal resistor, there is no voltage drop across that resistor and the potential at $b$ will be equal to the potential at $c$, as we argued above.

The circuit in Figure \ref{fig:circuits:realbatterycircuit} is now identical to that analyzed in Example \ref{ex:circuits:tworesistors}, and can be treated the same way. We can combine the $\SI{2}{\Omega}$ resistor with the internal resistance, $r$, in series to obtain an effective resistor, $R_{eff}=r+R$. The voltage drop across that resistor will the same as the potential difference across the ideal battery ($\Delta V_{ideal}=\SI{6}{V}$), and we can make use of Ohm's Law to find the internal resistance, $r$:
\begin{align*}
\Delta V_{ideal}&= R_{eff}I=(r+R)I\\
\therefore r = \frac{\Delta V_{ideal}}{I}-R=\frac{(\SI{6}{V})}{(\SI{2}{A})}-(\SI{2}{\Omega})=\SI{1}{\Omega}
\end{align*}
Now that we know the internal resistance, we can determine the voltage drop across the internal resistor, using Ohm's Law:
\begin{align*}
\Delta V_r = rI=(\SI{1}{\Omega})(\SI{2}{A})=\SI{2}{V}
\end{align*}
The voltage drop across the real terminal of the battery (between points $a$ and $c$), is thus given by:
\begin{align*}
\Delta V_{real}=\Delta V_{ideal}-\Delta V_r=(\SI{6}{V})-(\SI{2}{V})=\SI{4}{V}
\end{align*}
Again, you can verify that the voltage drops across the two resistors will sum to the total voltage drop across the terminals of the ideal battery. 


\textbf{Discussion: } Modelling real batteries is not so different from modelling ideal batteries, since one only needs to include an internal resistance into the circuit. The key difference with a real battery is that the voltage at its real terminals depends on what is connected to the battery. In the example above, the battery has a real voltage of $\SI{6}{V}$ when nothing is connected, but the voltage drops to $\SI{4}{V}$ when a $\SI{2}{\Omega}$ resistor is connected.
\end{example}
%TODO Checkpoint question: The actual voltage across the terminals of a battery depends on what is connected to a battery. If you connect a resistor across a battery, and want to ensure that the voltage across the terminals of the battery is as close as possible to the value with nothing connected, should you use a large or a small resistor? 


\section{Kirchhoff's rules}
Kirchhoff's rules correspond to concepts that we have already covered, but allow us to easily model more complex circuits, for instance, those where there is more than one path for the current to take. Kirchhoff's rules refer to ``junctions'' and ``loops''. Junctions and loops depend only on the shape of the circuit, and not on the components in the circuit. Figure \ref{fig:circuits:3loopempty} shows a circuit with no components in order to illustrate what is meant by a junction and a loop.
\begin{center}
\begin{circuitikz}
\draw (0,0) node[anchor=east]{a} [short,*-*] to (4,0) node[anchor=west]{b}
      to [short,-*] (4,2) node[anchor=west]{c}
      to [short,-*] (0,2) node[anchor=east]{d} 
      to [short,-*] (0,0);
\draw (4,2)  [short,-*] to (4,4) node[anchor=west]{e}
	  to [short,-*] (0,4) node[anchor=east]{f}
	  to [short] (0,2);
\end{circuitikz}
\captionof{figure}{\label{fig:circuits:3loopempty}A circuit that has 3 loops and 2 junctions.} 
\end{center}
The locations at points $d$ and $c$ are considered ``junctions'', because there are more than 2 segments of wire connected to that point. The points at locations $a$, $b$, $e$ and $f$ only have two segments of wire connected to them. The circuit in Figure \ref{fig:circuits:3loopempty} thus has 2 junctions. 

A loop is a closed path that one can trace around the circuit without passing over the same segment of wire twice. The circuit in Figure \ref{fig:circuits:3loopempty} has 3 such loops, which we can identify using the letters at the various nodes of the circuit:
\begin{enumerate}
\item $abcda$
\item $abcefda$
\item $dcefd$
\end{enumerate}
Note that it does not matter where one starts on the loop, only that one can identify how many different loops are present in the circuit.

%TODO Checkpoint MC question showing a diagram with 6 loops (rectangle with 2 cross bars), ask how many loops (6) and junctions (4)

\subsection{Junction rule}
The junction rule states that: \textbf{The current entering a junction must be equal to the current exiting a junction.}

This is in fact a simple statement about conservation of charge. If charges are flowing into a junction (from one or more segments of wire in that junction), then the same amount of charges must flow back out of the junction (through one or more different segments of wire).

Consider the junction illustrated in Figure \ref{fig:circuits:junction}, comprised of 5 segments of wire, and thus having 5 currents. As shown, currents $I_1$ and $I_4$ flow into the junction, whereas currents $I_2$, $I_3$ and $I_5$ all flow out of the junction. 
\begin{center}
\begin{circuitikz}
\draw (-2,1.7) [short, -*, i=$I_1$] to (0,0)
	  (-2,-1.7) [short, i<=$I_2$] to (0,0)
	  (1.7,-2) [short, i=$I_3$] to (0,0)
	  (2,0) [short, i_=$I_4$] to (0,0)
	  (1.7,2) [short, i_<=$I_5$] to (0,0);
\end{circuitikz}
\captionof{figure}{\label{fig:circuits:junction}A junction with 5 segments and 5 currents.} 
\end{center}
The junction rule states that the current entering the junction must equal the current coming out of the junction. This allows us to relate the currents to each other:
\begin{align*}
\text{incoming currents}&=\text{outgoing currents}\\
I_1+I_4 &=I_2+I_3+I_4
\end{align*}

\subsection{Loop rule}
The loop rule states that: \textbf{The net voltage drop across a loop must be zero.}

This is a statement about conservation of energy, that we already noted in Example \ref{ex:circuits:tworesistors}. Once you have identified a specific loop, if you trace a closed path around the loop, the electric potential must be the same at the end of the path as at the beginning of the path (since it is literally the same point in space). This means that if there is a voltage drop along the path (e.g. due to one or more resistors), then there must be equivalent voltage increases somewhere else on the path (e.g. due to one or more batteries). If this were not the case, it would be possible to have a path where charges could gain a net amount of energy by going around that path, which they could keep doing indefinitely and create an infinite amount of energy; instead, if charges gain potential energy in a battery, they must then loose exactly the same amount of energy inside one or more resistors along the path.

Figure \ref{fig:circuits:loop} shows a loop (which could be part of a larger circuit) to which we can apply the loop rule. The loop contains two batteries, facing in opposite directions (which would not normally be a good use of batteries), as illustrated by the battery arrows. 
\begin{center}
\begin{circuitikz}
\draw (0,0) node[anchor=east]{a} [battery1,l_=$\Delta V_1$, *-*, i>=$I$] to (3,0) node[anchor=north]{b}
	  to [battery1, l_=$\Delta V_2$, -*, i>=$I$, invert] node[anchor=west]{c} (6,0)
      to [short, label=, i>=$I$, -*] node[anchor=west]{d} (6,2)
 	  to [R=$R_1$,-,i>=$I$] node[anchor=south]{e}(4,2)
 	  to [R=$R_2$,-, i>=$I$]node[anchor=south]{f} (2,2)
 	  to [R=$R_3$,-, i>=$I$] node[anchor=east]{g}(0,2)
 	  to [short, label=, i>=$I$,*-] (0,0); 
 \draw [->,>=stealth, line width=1mm] (1.85,0.6)--(1.15,0.6);
 \draw [->,>=stealth, line width=1mm] (4.15,0.6)--(4.85,0.6);
\end{circuitikz}
\captionof{figure}{\label{fig:circuits:loop}A loop with 2 batteries and 3 resistors.} 
\end{center}
The procedure for applying the loop rule is as follows:
\begin{enumerate}
\item Identify the loop, including starting position and direction.
\item Start at the beginning of the loop, and trace around the loop.
\item Each time a battery is encountered, add the battery voltage if you are tracing the loop in the same direction as the corresponding battery arrow, subtract the voltage otherwise.
\item Each time a resistor is encountered, subtract the voltage across that resistor ($RI$, from Ohm's Law) if tracing the loop in the same direction as the current, add the the voltage otherwise.
\item Once you have traced back to the starting point, the resulting sum must be zero.
\end{enumerate}
To illustrate the procedure, we trace out the loop $abcedfga$ in Figure \ref{fig:circuits:loop}. We thus start at point $a$ and trace the loop in the counter-clockwise direction. 
\begin{enumerate}
\item Between points $a$ and $b$ we encounter a battery, and we are tracing in the opposite direction of that battery's arrow, so we subtract the voltage from that battery ($-\Delta V_1$).
\item Between points $b$ and $c$, we encounter a battery, and we are tracing in the same direction as that battery's arrow, so we add the voltage from that battery ($+\Delta V_2$).
\item Nothing happens to the potential along the wire from $c$ to $d$.
\item Between points $d$ and $e$, we encounter a resistor, and we are tracing in the same direction as the current through that resistor, so subtract the voltage across that resistor ($\Delta V_{R1}=-R_1I$).
\item Similarly, we subtract the voltages across resistors $R_2$ and $R_3$, as we are tracing in the same direction as the current through those resistors, ($\Delta V_{R2}=-IR_2$, $\Delta V_{R3}=-IR_3$)
\end{enumerate}
We can now use the loop rule, which states that the sum of the above voltages must be zero:
\begin{align*}
-\Delta V_1 + \Delta V_2 - R_1I - R_2I - R_3I = 0
\end{align*}
This equation then gives us a relation between the various quantities (current, resistors, battery voltages) in the circuit which can be used to model the circuit. Note that we could have combined the three series resistors, and we would have obtained the same equation. 
%TODO: Checkpoint question: Does the equation above change if we start the loop in a different position or trace the loop in the opposite direction? (Give options like it stays the same if the starting point is different, but not if the direction changes, etc..). Correct: you always get the same equation, regardless of starting point or direction.



\section{Modelling circuits with resistors}

\section{Modelling circuits with capacitors}

\section{Measuring current and voltage}

\newpage
\section{Summary}

\begin{chapterSummary}
 Something that was learned
\end{chapterSummary}

\newpage
\begin{importantEquations}
\medskip
\begin{multicols}{2}
\textbf{Momentum of a point particle:}
\begin{align*}
\vec p = m\vec v \\
\frac{d}{dt}\vec p = \sum \vec F = \vec F^{net}
\end{align*}
\columnbreak
\\
\textbf{Position of the Centre of Mass \\ of a system:}
\begin{align*}
\vec r_{CM} &=\frac{1}{M}\sum_i m_i\vec r_i 
\end{align*}
\medskip
\end{multicols}
\end{importantEquations}

\newpage
\section{Thinking about the material}

\begin{chapteractivity}{Reflect and research}
{
\item When did Galvani and Volta experiment with electric cells?
\item What is the largest voltage that Volta obtained with his voltaic pile?
\item How does one charge a rechargeable battery? What would the circuit look like?
}
\end{chapteractivity}

\begin{chapteractivity}{To try at home}
{
\item Try
}
\end{chapteractivity}

\begin{chapteractivity}{To try in the lab}
{
\item Propose an experiment
}
\end{chapteractivity}

\newpage
\section{Sample problems and solutions}
\subsection{Problems}
\begin{problem}{soln:template:ballistic}{\label{prob:template:ballistic} 

}
\end{problem}

\newpage
\subsection{Solutions}
\begin{solution}{prob:template:ballistic}\label{soln:template:ballistic}

\end{solution}

