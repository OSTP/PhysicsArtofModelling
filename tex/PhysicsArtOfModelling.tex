%Copyright 2017 R.D. Martin
%This book is free software: you can redistribute it and/or modify it under the terms of the GNU General Public License as published by the Free Software Foundation, either version 3 of the License, or (at your option) any later version.
%
%This book is distributed in the hope that it will be useful, but WITHOUT ANY WARRANTY; without even the implied warranty of MERCHANTABILITY or FITNESS FOR A PARTICULAR PURPOSE.  See the GNU General Public License for more details, http://www.gnu.org/licenses/.


\documentclass[12pt]{book}
\usepackage{mathtools} % for \Aboxed
\usepackage{paralist}
\usepackage{calc}
\usepackage{subfig}
\usepackage{setspace}
\usepackage{amssymb}
\usepackage{amsmath}
\usepackage{amstext}
\usepackage[font={small,it}]{caption}
\usepackage[pdftex]{graphicx} 
\usepackage{fancyhdr,lastpage}
\usepackage{url}
\usepackage{longtable}
\usepackage{comment}
\usepackage{ifthen}
\usepackage{color}
\usepackage[colorlinks=true,linkcolor=blue]{hyperref}
\usepackage[explicit]{titlesec}
\usepackage{lmodern}
\usepackage{listings}
\usepackage{parskip}
\usepackage[table]{xcolor}
\usepackage{enumitem}
\usepackage{wrapfig}
\usepackage[framemethod=TikZ]{mdframed}
\usepackage{titlesec} %for spacing around titles
\usepackage{caption}
\usepackage[separate-uncertainty = true]{siunitx}
%\lstset{language=Python,showstringspaces=false,commentstyle=} 


%%TODO:
%Chapter reference are out of whack
%Padding around wrapfigures


%%Some math and other shortcuts
\newcommand{\chloe}{Chlo\"e~}
\newcommand{\die}[2]{\frac{\partial #1}{\partial #2}}
\newcommand{\ddt}{\frac{d}{dt}}
\newcommand{\lagd}{\mathcal{L}}
\newcommand{\code}[1]{\texttt{#1}}


%Stuff for writing code:

\definecolor{mygreen}{rgb}{0.2,0.6,0}
\lstset{ %
  belowskip=0pt,
  aboveskip=0pt,
  caption=\relax,
  backgroundcolor=\color{white},   % choose the background color; you must add \usepackage{color} or \usepackage{xcolor}
  basicstyle=\footnotesize,        % the size of the fonts that are used for the code
  breakatwhitespace=false,         % sets if automatic breaks should only happen at whitespace
  breaklines=true,                 % sets automatic line breaking
  captionpos=t,                    % sets the caption-position to bottom
  commentstyle=\color{mygreen},    % comment style
  deletekeywords={...},            % if you want to delete keywords from the given language
  escapeinside={(*}{*)},          % if you want to add LaTeX within your code
  extendedchars=true,              % lets you use non-ASCII characters; for 8-bits encodings only, does not work with UTF-8
  frame=none,	                   % adds a frame around the code
  keepspaces=true,                 % keeps spaces in text, useful for keeping indentation of code (possibly needs columns=flexible)
  keywordstyle=\color{blue},       % keyword style
  language=Python,                 % the language of the code
  otherkeywords={*,...},           % if you want to add more keywords to the set
  numbers=none,                    % where to put the line-numbers; possible values are (none, left, right)
  numbersep=5pt,                   % how far the line-numbers are from the code
  numberstyle=\tiny\color{black}, % the style that is used for the line-numbers
  rulecolor=\color{black},         % if not set, the frame-color may be changed on line-breaks within not-black text (e.g. comments (green here))
  showspaces=false,                % show spaces everywhere adding particular underscores; it overrides 'showstringspaces'
  showstringspaces=false,          % underline spaces within strings only
  showtabs=false,                  % show tabs within strings adding particular underscores
  stepnumber=1,                    % the step between two line-numbers. If it's 1, each line will be numbered
  stringstyle=\color{red},     % string literal style
  tabsize=2,	                   % sets default tabsize to 2 spaces
  title=\lstname                   % show the filename of files included with \lstinputlisting; also try caption instead of title
}

%Environments for writing code
\DeclareCaptionFont{white}{\color{white}}
\DeclareCaptionFormat{listing}{\colorbox{gray}{\parbox{\textwidth}{#1#2#3}}}

\captionsetup[lstlisting]{format=listing,labelfont=white,textfont=white}
\renewcommand{\lstlistingname}{Python Example}
\renewcommand{\lstlistlistingname}{List of \lstlistingname s}

\lstnewenvironment{python}[1][]{
  \lstset{#1, language=Python}%
  \renewcommand\lstlistingname{Python Code}
}{}

\lstnewenvironment{poutput}{
 \lstset{caption=\mbox{}, language=,aboveskip=-3pt}
 \addtocounter{lstlisting}{-1}
 \renewcommand\lstlistingname{Output}
}{}


%%Pretty chapter headings:
\newlength\chapnumb
\setlength\chapnumb{4cm}

\titleformat{\chapter}[block]
{\normalfont\sffamily}{}{0pt}
{\parbox[b]{\chapnumb}{%
   \fontsize{120}{110}\selectfont\thechapter}%
  \parbox[b]{\dimexpr\textwidth-\chapnumb\relax}{%
    \raggedleft%
    \hfill{\LARGE#1}\\
    \rule{\dimexpr\textwidth-\chapnumb\relax}{0.4pt}}}
\titleformat{name=\chapter,numberless}[block]
{\normalfont\sffamily}{}{0pt}
{\parbox[b]{\chapnumb}{%
   \mbox{}}%
  \parbox[b]{\dimexpr\textwidth-\chapnumb\relax}{%
    \raggedleft%
    \hfill{\LARGE#1}\\
    \rule{\dimexpr\textwidth-\chapnumb\relax}{0.4pt}}}


%%%spacing around titles
\setlength{\parindent}{0pt}
\parskip = \baselineskip

%spacing around captions (e.g. caption after a table)
\captionsetup{belowskip=6pt,aboveskip=4pt}

%\titlespacing*{\chapter}
%{0pt}{0ex}{0ex}
\titlespacing*{\section}
{0pt}{4pt-\parskip}{2pt-\parskip}
\titlespacing*{\subsection}
{0pt}{4pt-\parskip}{1pt-\parskip}
\titlespacing*{\subsubsection}
{0pt}{4pt-\parskip}{1pt-\parskip}

%%% Spacing in lists:
\setlist{nosep}

%%Verticall spacing between table rows
\renewcommand{\arraystretch}{1.5}

\setlength{\intextsep}{12pt}

%space before itemized list:
%\setlength{\topsep}{-10pt} %does nothing?

%%Simplifed figure environment:

\newenvironment{capfig}[3]{\begin{center}\includegraphics[width=#1]{#2}\captionof{figure}{#3}\end{center}}{}


%Wrap figure environments (right or left). Argument #1 (default value 12, specified as optional), is the number of 
%lines that the figure should take.
%space around wrap figures:
%\setlength{\intextsep}{20pt}%
%\setlength{\columnsep}{5pt}%
\newenvironment{Rwcapfig}[4][0]{
\begingroup
%\setlength{\intextsep}{0pt}%
\setlength{\columnsep}{10pt}%
\begin{wrapfigure}[#1]{R}{#2}\centering\includegraphics[width=#2]{#3}\caption{#4}\end{wrapfigure}}{\endgroup}

\newenvironment{rwcapfig}[4][0]{
\begingroup
%\setlength{\intextsep}{0pt}%
\setlength{\columnsep}{10pt}%
\begin{wrapfigure}[#1]{r}{#2}\centering\includegraphics[width=#2]{#3}\caption{#4}\end{wrapfigure}}{\endgroup}

\newenvironment{Lwcapfig}[4][0]{
\begingroup
%\setlength{\intextsep}{0pt}%
\setlength{\columnsep}{10pt}%
\begin{wrapfigure}[#1]{L}{#2}\centering\includegraphics[width=#2]{#3}\caption{#4}\end{wrapfigure}}{\endgroup }

\newenvironment{lwcapfig}[4][0]{
\begingroup
%\setlength{\intextsep}{0pt}%
\setlength{\columnsep}{10pt}%
\begin{wrapfigure}[#1]{l}{#2}\centering\includegraphics[width=#2]{#3}\caption{#4}\end{wrapfigure}}{\endgroup }

%%Checkpoint question in a box, with counter:
\newcounter{ncheckpoint}[chapter]
\def\thecheckpoint{\thechapter-\arabic{ncheckpoint}}

%%MC checkpoint
\newenvironment{checkpointMC}[1]{\refstepcounter{ncheckpoint}%
    \textbf{Checkpoint \thecheckpoint: }#1 %
    \begin{enumerate}[label=\Alph*),topsep=-10pt]}%
   {\end{enumerate}}
\surroundwithmdframed[skipabove=10pt,linewidth=2pt, backgroundcolor=green!10, roundcorner=10pt,nobreak=true]{checkpointMC}


%%Short Answer checkpoint
\newenvironment{checkpointSA}[1]{\refstepcounter{ncheckpoint}%
    \textbf{Checkpoint \thecheckpoint: }#1\\}{}
\surroundwithmdframed[skipabove=10pt,linewidth=2pt, backgroundcolor=green!10, roundcorner=10pt,nobreak=true]{checkpointSA}


%%Learning objectives box:
\newenvironment{learningObjectives}{\textbf{Learning Objectives:} \begin{itemize}[topsep=-10pt]}{\end{itemize}}
\surroundwithmdframed[linewidth=2pt, backgroundcolor=blue!10, roundcorner=10pt,nobreak=true]{learningObjectives}

%%End of chapter summary box:
\newenvironment{chapterSummary}{\begin{itemize}[topsep=-10pt]}{\end{itemize}}
\surroundwithmdframed[linewidth=2pt, backgroundcolor=yellow!10, roundcorner=10pt]{chapterSummary}

%%Worked out example box with a counter
\newcounter{example}[chapter]
\def\theexample{\thechapter-\arabic{example}}

\newenvironment{example}[2]
{\refstepcounter{example} \textbf{Example \theexample:} #1\\ \\ \itshape #2}{}
\surroundwithmdframed[skipabove=10pt,linewidth=2pt, backgroundcolor=red!10, roundcorner=10pt]{example}


\newcounter{problem}[chapter]
\def\theproblem{\thechapter-\arabic{problem}}

\newenvironment{problem}[1]
  {\refstepcounter{problem}\textbf{Problem \theproblem: #1}\\}
  {\vspace{2ex}\\}


%\def\secondpage{\clearpage\null\vfill
%%\pagestyle{empty}
%\begin{minipage}[b]{0.9\textwidth}
%\footnotesize\raggedright
%\setlength{\parskip}{0.5\baselineskip}
%Copyright \copyright \the\year\ R.D. Martin\par
%This book is free software: you can redistribute it and/or modify it under the terms of the GNU General Public License as published by the Free Software Foundation, either version 3 of the License, or (at your option) any later version.
%
%This book is distributed in the hope that it will be useful, but WITHOUT ANY WARRANTY; without even the implied warranty of MERCHANTABILITY or FITNESS FOR A PARTICULAR PURPOSE.  See the GNU General Public License for more details, http://www.gnu.org/licenses/.
%\end{minipage}
%\vspace*{2\baselineskip}
%\cleardoublepage
%\rfoot{\thepage}}

%\makeatletter
%\g@addto@macro{\maketitle}{\secondpage}
%\makeatother
          
\usepackage[paper=letterpaper,
            %includefoot, % Uncomment to put page number above margin
            marginparwidth=.0in,     % Length of section titles
            marginparsep=.05in,       % Space between titles and text
            margin=1in,               % 1 inch margins
            includemp]{geometry}

\setcounter{secnumdepth}{2}
\setcounter{tocdepth}{3}

\begin{document}
\title{The Art of Modelling: Introduction to Physics}
\author{Ryan D. Martin}
\pagenumbering{roman}
\maketitle
\tableofcontents
\pagenumbering{arabic}

%Copyright 2016 R.D. Martin
%This book is free software: you can redistribute it and/or modify it under the terms of the GNU General Public License as published by the Free Software Foundation, either version 3 of the License, or (at your option) any later version.
%
%This book is distributed in the hope that it will be useful, but WITHOUT ANY WARRANTY; without even the implied warranty of MERCHANTABILITY or FITNESS FOR A PARTICULAR PURPOSE.  See the GNU General Public License for more details, http://www.gnu.org/licenses/.
\chapter{Introduction to Physics}
\label{chap:Intro}

\begin{learningObjectives}
\item Understand the Scientific Method
\item Define the scope of Physics
\item Understand the difference between theory and model
\end{learningObjectives}

\section{Science and the Scientific Method}
Science is, very generally, an attempt to \textit{describe} the world around us. It is important to note that describing the world around us is not the same as \textit{explaining} the world around us\footnote{Some might say this is the role of religion}. Science is an attempt to answer the question ``How?'' and not ``Why?''. As we develop our description of the physical world, you should remember this important distinction. The Scientific Method is a prescription for coming up with a description of the physical world that anyone can challenge through performing experiments. We can get some insight into the Scientific Method through a simple example. 

Imagine that we wish to describe how long it takes a tennis ball to reach the ground after being released from a certain height. One way to proceed is to describe how long it takes for a tennis ball to drop 1\,m, and then to describe how long it takes a tennis ball to drop 2\,m, etc. One could generate a giant table showing how long it takes a tennis ball to drop for any given height. Someone would then be able to perform an experiment to measure how long a tennis ball takes to drop 1\,m or 2\,m and see if they agree with those descriptions.

If we collected the descriptions for all possible heights, then we would effectively have a valid ``scientific theory'' that describes how long it takes tennis balls to drop from any height. Suppose that a budding scientist, let's call her Chlo\"e, then came along and noticed that there is a pattern in the theory that can be described much more succinctly and generally than the gian table. In particular, suppose that she notices that, mathematically, the time, $t$, that it takes for a tennis ball to drop a height, $h$, is proportional to the square root of the height:
\begin{equation*}
t \propto \sqrt{h}
\end{equation*}
Chlo\"e's ``Theory of Tennis Ball Drop Times'' is appealing because it is succinct, and it also allows us to make \textbf{verifiable predictions}. That is, using this theory, we can predict that it will take a tennis ball $\sqrt 2$ times longer to drop from 2\,m than it will from 1\,m, and then perform an experiment to verify that prediction. If the experiment agrees with the prediction, then we conclude that Chlo\"e's theory adequately describes the result of that particular experiment. If the experiment does not agree with the prediction, then we conclude that the theory is not an adequate description of that experiment, and we try to find a new theory.

Chlo\"e's theory is also appealing because it can describe not only tennis balls, but the time it takes for other objects to fall as well. Scientists can then set out to continue testing her theory with a wide range of objects and drop heights to see if it describes those experiments as well. Inevitably, they will discover situations where Chlo\"e's theory fails to adequately describe the time that it takes for objects to fall (can you think of an example?).

We would then develop a new ``Theory of Falling Objects'' that would include Chlo\"e's theory that describes most objects falling, and additionally, a set of descriptions for the fall times for cases that are not described by Chlo\"e's theory. Ideally, we would seek a new theory that would also describe the new phenomena not described by Chlo\"e's theory in a succinct manner. There is of course no guarantee, ever, that such a theory would exist; it is just an optimistic hope of scientists to find the most general and succinct description of the physical world. 



 



\section{The scope of Physics}

\section{Theories and models}

\section{Fighting intuition}

\section{Summary}
\begin{chapterSummary}
\item Science attempts to \textit{describe} the physical world (answers the question ``How?'', not ``Why?'').
\item The Scientific Method provides a prescription for arriving at theories that describe the physical world and that can be experimentally verified.
\item An experiment can only disprove a theory, not confirm it in any general sense.
\item Theories are typically valid only in well-defined situations.
\end{chapterSummary}
\include{Chapter2_ModelAndExperiment}
%Copyright 2017 R.D. Martin
%This book is free software: you can redistribute it and/or modify it under the terms of the GNU General Public License as published by the Free Software Foundation, either version 3 of the License, or (at your option) any later version.
%
%This book is distributed in the hope that it will be useful, but WITHOUT ANY WARRANTY; without even the implied warranty of MERCHANTABILITY or FITNESS FOR A PARTICULAR PURPOSE.  See the GNU General Public License for more details, http://www.gnu.org/licenses/.
\chapter{Describing motion in one dimension}
\label{chap:3_Kinematics1D}
In this chapter, we will introduce the tools required to describe motion in one dimension. In later chapters, we will use the theories of physics to model the motion of objects, but first, we need to make sure that we have the tools to describe the motion. We generally use the word ``kinematics'' to label the tools for describing motion (e.g. speed, acceleration, position, etc), whereas we refer to ``dynamics'' when we use the laws of physics to predict motion (e.g. what motion will occur if a force is applied to an object). 
 \vspace{1cm}
\begin{learningObjectives}
\item Describe motion in 1D using functions and defining an axis.
\item Define position, velocity, speed, and acceleration.
\item Use derivatives to determine the rate of change of a quantity.
\item Use integrals to calculate sums.
\item Define the meaning of an inertial frame of reference.
\item Use Galilean and Lorentz transformations to convert the description of an object's position from one inertial frame to another.
\end{learningObjectives}


The most simple type of motion to describe is that of a particle that is constrained to move along a straight line (one-dimensional motion); much like a train along a straight piece of track. When we say that we want to describe the motion of the particle (or train), what we mean is that we want to be able to say where it is at what time. Formally, we want to know the particle's \textbf{position as a function of time}, which we will label as $x(t)$. The function will only be meaningful if:
\begin{itemize}
\item we specify an axis along the direction of motion
\item we specify an origin where $x=0$
\item we specify a direction along the axis of motion corresponding to increasing values of $x$
\item we specify the units for the quantity, $x$.
\end{itemize}
That is, unless all of these are specified, you would have a hard time describing the motion of an object to one of your friends over the phone. 

\capfig{0.4\textwidth}{figures/Chapter3/1daxis.png}{\label{fig:chap3:1daxis.png}In order to describe the motion of the grey ball along a straight line, we introduce the x-axis, represented by an arrow to indicate the direction of increasing $x$, and the location of the origin, where $x=\SI{0}{m}$. Given our choice of origin, the ball is currently at a position of $x=\SI{0.5}{m}$.
}
Consider Figure \ref{fig:chap3:1daxis.png} where we would like to describe the motion of the grey ball as it moves along a straight line. In order to quantify where the ball is, we introduce the ``x-axis'', illustrated by the black arrow. The direction of the arrow corresponds to the direction where $x$ increases (i.e. becomes more positive). We have also chosen a point where $x=0$, and by convention, we choose to express $x$ in units of meters (the S.I. unit for the dimension of length).

Note that we are completely free to choose both the direction of the x-axis and the location of the origin. The x-axis is a mathematical construct that we introduce in order to describe the physical world; we could just as easily have chosen for it to point in the opposite direction with a different origin. Since we are completely free to choose where we define the x-axis, we should try to choose the option that is most convenient to us. 

\section{Motion with constant speed}
Now suppose that the ball in Figure \ref{fig:chap3:1daxis.png} is rolling, and that we recorded its x position every second in a table and obtained the values in Table \ref{tab:chap3:1dmotion} (we will ignore measurement uncertainties and pretend that the values are exact).
\begin{table}[!h]
\centering
\begingroup
\renewcommand{\arraystretch}{1.0}
\begin{tabular}{cc}
\textbf{Time [s]}&\textbf{X position [m]}\\
\hline
\hline
\SI{0.0}{s}& \SI{0.5}{m}\\ \hline
\SI{1.0}{s}& \SI{1.0}{m}\\ \hline
\SI{2.0}{s}& \SI{1.5}{m}\\ \hline
\SI{3.0}{s}& \SI{2.0}{m}\\ \hline
\SI{4.0}{s}& \SI{2.5}{m}\\ \hline
\SI{5.0}{s}& \SI{3.0}{m}\\ \hline
\SI{6.0}{s}& \SI{3.5}{m}\\ \hline
\SI{7.0}{s}& \SI{4.0}{m}\\ \hline
\SI{8.0}{s}& \SI{4.5}{m}\\ \hline
\SI{9.0}{s}& \SI{5.0}{m}\\ \hline
\end{tabular}
\caption{\label{tab:chap3:1dmotion} Position of a ball along the x-axis recorded every second.}
\endgroup
\end{table}
The easiest way to visualize the values in the table is to plot them on a graph. Plotting position as a function of time is one of the most common graphs to make in physics, since it is often a complete description of the motion of an object. We can easily plot these values in Python:
\begin{python}[caption=QExpy to calculate mean and standard deviation] 
#First, we load the QExpy module
import qexpy as q
#We define t as a list of values (note the square brackets):
t = [0.0, 1.0, 2.0, 3.0, 4.0, 5.0, 6.0, 7.0, 8.0, 9.0]
#Similarly, we define the corresponding positions:
position = [0.5, 1.0, 1.5, 2.0, 2.5, 3.0, 3.5, 4.0, 4.5, 5.0]
#Define the plot, and show it:
fig = q.MakePlot(xdata=t, xunits="s", xname="time",
                 ydata=position, yunits="m", yname="position",
                 data_name="position vs time")
fig.show()
\end{python}
\begin{poutput}
(* \capfig{0.7\textwidth}{figures/Chapter3/1dxvst.png}{\label{fig:chap3:1dxvst}Plot of position as a function of time using the values from Table \ref{tab:chap3:1dmotion}.} *)
\end{poutput}

The data plotted in Figure \ref{fig:chap3:1dxvst} show that the $x$ position of the ball increases linearly with time (i.e. it is a straight line). This means that in equal time increments, the ball will cover equal distances. Note that we also had the liberty to choose when we define $t=0$; in this case, we chose that time is zero when the ball is at $x=\SI{0.5}{m}$. 

\begin{checkpointSA}{Using the data from Table \ref{tab:chap3:1dmotion}, at what position along the x-axis will the ball be when time is $t=\SI{9.5}{s}$, if it continues its motion undisturbed?} %5.25m
\end{checkpointSA} 

Since the position as a function of time for the ball plotted in Figure \ref{fig:chap3:1dxvst} is linear, we can summarize our description of the motion using a function, $x(t)$, instead of having to tabulate the values as we did in Table \ref{tab:chap3:1dmotion}. The function will have the functional form:
\begin{align*}
x(t) = a + b\times t
\end{align*}
The constant $a$ is the ``offset'' of the function, the value that the function has at $t=\SI{0}{s}$. The constant $b$ is the slope and gives the rate of change of the position as a function of time. We can determine the values for the constants $a$ and $b$ by choosing any two rows from Table \ref{tab:chap3:1dmotion} (to determine 2 unknown quantities, you need 2 equations), and obtain 2 equations and 2 unknowns. For example, choosing the points where $t=\SI{0}{s}$ and $t=\SI{2.0}{s}$:
\begin{align*}
x(t=\SI{0}{s})&=\SI{0.5}{m}=a + b\times(\SI{0}{s}) \\
x(t=\SI{2.0}{s})&=\SI{1.5}{m}=a + b\times(\SI{2.0}{s}) \\
\end{align*}
The first equation immediately gives $a = \SI{0.5}{m}$, which we can substitute into the second equation to get $b$:
\begin{align*}
\SI{1.5}{m}&=a + b\times(\SI{2.0}{s}) = \SI{0.5}{m} + b\times(\SI{2.0}{s})\\
\therefore b &=\frac{(\SI{1.5}{m})-(\SI{0.5}{m})}{(\SI{2.0}{s})}=\SI{0.5}{m\per s}
\end{align*}
which gives us the functional form for $x(t)$:
\begin{align*}
x(t) = (\SI{0.5}{m}) + (\SI{0.5}{m\per s})\times t
\end{align*}
where you should note that $a$ and $b$ have different dimensions. Since $a$ is added to something that must then give dimensions of length (for position, $x$), $a$ has dimensions of length. $b$ is multiplied by time, and that product must have dimensions of length as well; $b$ thus has dimensions of length over time, or ``speed'' (with S.I. units of \si{m\per s}).

We can generalize the description of an object whose position increases linearly with time as:
\begin{align}
\label{eqn:chap3:1dxvst_noa}
\Aboxed{x(t) = x_0 + v_xt}
\end{align}
where $x_0$ is the position of the object at time $t=\SI{0}{s}$ ($a$ from above), and $v_x$ corresponds to the distance that the object covers per unit time ($b$ from above) along the x-axis. We call $v_x$ the ``speed'' of the object. If $v_x$ is large, then the object covers more distance in a given time, i.e. it moves faster. If $v_x$ is a negative number, then the object moves in the negative $x$ direction.

\capfig{0.7\textwidth}{figures/Chapter3/1dturn.png}{\label{fig:chap3:1dturn}Position as a function of time for an object.}
\begin{checkpointMC}{Referring to Figure \ref{fig:chap3:1dturn}, what can you say about the motion of the object? }
\item The object moved faster and faster between $t=\SI{0}{s}$ and $t=\SI{30}{s}$, then slowed down to a stop at $t=\SI{60}{s}$.
\item The object moved in the positive x-direction between $t=\SI{0}{s}$ and $t=\SI{30}{s}$, and then turned around and moved in the negative x-direction between $t=\SI{30}{s}$ and $t=\SI{60}{s}$. %correct
\item The object moved with a higher speed between $t=\SI{0}{s}$ and $t=\SI{30}{s}$ than it did between $t=\SI{30}{s}$ and $t=\SI{60}{s}$.
\end{checkpointMC}

\capfig{0.7\textwidth}{figures/Chapter3/1d2objects.png}{\label{fig:chap3:1d2objects}Positions as a function of time for two objects.}
\begin{checkpointMC}{Referring to Figure \ref{fig:chap3:1d2objects}, what can you say about the motion of the two objects? }
\item Object 1 is slower than Object 2
\item Object 1 is more than twice as fast as Object 2 %correct
\item Object 1 is less than twice as fast as Object 2
\end{checkpointMC}

\section{Motion with constant acceleration}
Until now, we have considered motion where the speed is a constant (i.e. where speed does not change with time). Suppose that we wish to describe the position of a falling object that we released from rest at time $t=\SI{0}{s}$. The object will start with a speed of \SI{0}{m\per s} and it will \textbf{accelerate} as it falls. We say that an object is ``accelerating'' if its speed is not constant. As we will see in later chapters, objects that fall near the surface of the Earth experience a constant acceleration (their speed changes at a constant rate).

Formally, we define acceleration as the rate of change of speed. Recall that speed is the rate of change of position, so acceleration is to speed what speed is to position. In particular, we saw that if the speed, $v_x$, is constant, then position as a function of time is given by:
\begin{align}
x(t) = x_0 + v_xt \tag{\ref{eqn:chap3:1dxvst_noa}}
\end{align} 
In analogy, if the acceleration is constant, then the speed as a function of time is given by:
\begin{align}
\label{eqn:chap3L1dvvst}
\Aboxed{v_x(t) = v_{0x} + a_xt }
\end{align}
where $a_x$ is the ``acceleration'' and $v_{0x}$ is the speed of the object at $t=0$. We can work out the dimensions of acceleration for this equation to make sense. Since we are adding $v_{0x}$ and $a_xt$, we need the dimensions of $a_xt$ to be speed:
\begin{align*}
[a_xt] &= \frac{L}{T} \\
[a_x][t] &= \frac{L}{T} \\
[a_x]T&= \frac{L}{T} \\
[a_x]&= \frac{L}{T^2} \\
\end{align*}
Acceleration thus has dimensions of length over time squared, with corresponding S.I. units of m/s$^2$ (meters per second squared or meters per second per second). 

Now that we have an understanding of acceleration, how do we describe the position of an object that is accelerating? We cannot use equation \ref{eqn:chap3:1dxvst_noa}, since it is only correct if the speed is constant. 

\capfig{0.1\textwidth}{figures/Chapter3/1daxis_vertical.png}{\label{fig:chap3:1daxis_vertical} X-axis for an object that starts at rest at $x=\SI{0}{m}$ when $t=\SI{0}{s}$ and falls downwards (in the direction of increasing $x$).}

Let us work out the corresponding equation for position as a function of time for accelerated motion using the x-axis depicted in Figure \ref{fig:chap3:1daxis_vertical}. We will determine $x(t)$ for the grey ball that starts at rest ($v_{0x}=\SI{0}{m\per s}$) at the position $x=\SI{0}{m}$ at time $t=\SI{0}{s}$ with a constant positive acceleration $a_x=\SI{10}{m\per s\squared}$. We would like to use equation \ref{eqn:chap3:1dxvst_noa}, but we cannot because it only applies if the speed is constant. To remedy this, we pretend (we ``approximate'') that for a a very small amount of time, the speed is almost constant. Let us take a very small interval in time, say $\Delta t=\SI{0.001}{s}$, and approximate that the speed is constant during that interval. 

At $t=\SI{0}{s}$, we have $x=\SI{0}{m}$, $v_{0x}=\SI{0}{m\per s}$ and $a_x=\SI{10}{m \per s\squared}$. We can use equation \ref{eqn:chap3L1dvvst} to find the speed at $t=\Delta t$ (at the end of the first interval):
\begin{align*}
v_x(t=\Delta t) &= v_{0x} + a_x\Delta t\\
&=(\SI{0}{m/s})+ a_x\Delta t\\&=a_x\Delta t
\end{align*}

The average speed during the first interval, $v_1^{avg}$ is then given by averaging the speeds at the beginning and at the end of the interval:
\begin{align*}
v_1^{avg}(t=\Delta t)&=\frac{1}{2}\left( v(t=0) + v(t=\Delta t)\right)\\
&=\frac{1}{2}\left(v_{0x}+a_x\Delta t\right)\\
&=\frac{1}{2}\left((\SI{0}{m/s})+a_x\Delta t\right)\\
&=\frac{1}{2}(\SI{10}{m/s^2})(\SI{0.001}{s})\\
&=\SI{0.005}{m/s}
\end{align*}
With the average speed during the interval, we can use equation \ref{eqn:chap3:1dxvst_noa} to find the position at $t=\Delta t$: 
\begin{align*}
x(t=\Delta t) &= x_0 +v_1^{avg}\Delta t\\
&=(\SI{0}{m}) + \frac{1}{2}a_x(\Delta t)^2\\
&= \frac{1}{2}(\SI{10}{m/s^2})(\SI{0.001}{s})^2\\
&=\SI{0.000005}{m}
\end{align*}
Thus, at time $t=\SI{0.001}{s}$, the object will have a speed of $v=\SI{0.005}{m/s}$ and will have covered a distance of $\SI{0.000005}{m}$. We can now use these values as the starting speed and position for the next interval in time. Using variables, at the beginning of the second interval, the speed is $v(t=\Delta t)=a_x\Delta t$ and at the end of the second interval, it will be $v(t=2\Delta t)=2a_x\Delta t$. The average speed during the second interval is thus given by:
\begin{align*}
v_2^{avg}(t=2\Delta t)&= \frac{1}{2}\left(v(t=\Delta t)+v(t=2\Delta t) \right)\\
&=\frac{1}{2}(a_x\Delta t+2a_x\Delta t)\\
&=\frac{3}{2}a_x\Delta t\\
&=\frac{3}{2}(\SI{10}{m/s^2})(\SI{0.001}{s})\\
&=\SI{0.015}{m/s}
\end{align*}
To find the position at the end of the second time interval, when $t=2\Delta t$, we use equation \ref{eqn:chap3:1dxvst_noa} again, but with a different starting position and the average speed that we just found:
\begin{align*}
x(t=2\Delta t) &= x(t=\Delta t) +v_2^{avg}\Delta t\\
&= \frac{1}{2}a_x(\Delta t)^2+\frac{3}{2}a(\Delta t)^2\\
&= \frac{1}{2}a_x(2\Delta t)^2\\
&=\frac{1}{2}(\SI{10}{m/s^2})(2\times\SI{0.001}{s})^2=\SI{0.00002}{m}
\end{align*}
where we kept the $\frac{1}{2}$ factored out and brought a factor of 2 inside the parenthesis with the $\Delta t$. You can carry out this exercise to ultimately find the position at any time. However, if you carry it out over a few more intervals, you may notice the following pattern: For the Nth interval when $t=N\Delta t$ at the end of the interval, we have:
\begin{align*}
v(t=(N-1)\Delta t) &= a_x (N-1) \Delta t &\text{(at beginning of interval N)}\\
v(t=N\Delta t) &= a_x N \Delta t &\text{(at end of interval N)}\\
v_N^{avg}&=\frac{1}{2}a_x(2N-1)\Delta t&\text{(average during interval)}\\
x(t=N\Delta t)&=\frac{1}{2}a_x(N\Delta t)^2&\text{(position at end of interval)}
\end{align*}

The last line gives us exactly what we were after, namely the position as a function of time for a constant acceleration, $a_x$, when the object started at rest at a position of $x=\SI{0}{m}$:
\begin{align}
\label{eqn:chap3:1dxoft_novonoxo}
 x(t) = \frac{1}{2} a_x t^2
\end{align}

If at $t=0$, the object had an initial position along the x-axis of $x_0$, then the position $x(t)$ would be shifted by an amount $x_0$:

\begin{align}
\label{eqn:chap3:1dxoft_novo}
 x(t) = x_0+\frac{1}{2} a_x t^2
\end{align}

Finally, if the object had an initial speed $v_{0x}$ at $t=0$, one can easily reproduce the iterations above to find that we need to add an additional term to account for this. We arrive at the general description of the position of an object moving in a straight line with acceleration, $a_x$:
\begin{align}
\label{eqn:chap3:1dxvst}
\Aboxed{ x(t) = x_0+v_{0x}t+ \frac{1}{2}a_xt^2}
\end{align}
Note that equation \ref{eqn:chap3:1dxvst_noa} is just a special case of the above when $a=0$. 

\begin{example}{A ball is thrown upwards with a speed of \SI{10}{m/s}. After which distance will the ball stop before falling back down? Assume that gravity causes a constant downwards acceleration of \SI{9.8}{m/s^2}.}
\label{ex:chap3:ballupandown}
We will solve this problem in the following steps:
\begin{enumerate}[topsep=-10pt]
\item Setup a coordinate system (define the x-axis).
\item Identify the condition that corresponds to the ball stopping its upwards motion and falling back down.
\item Determine the distance at which the ball stopped.
\end{enumerate}
Since we throw the ball upwards with an initial speeed upwards, it makes sense to choose an x-axis that points up and has the origin at the point where we release the ball. With this choice, referring to the variables in equation \ref{eqn:chap3:1dxvst}, we have:
\begin{align*}
x_0&=0\\
v_{0x}&=+\SI{10}{m/s}\\
a_x&=\SI{-9.8}{m/s^2}
\end{align*}
where the initial speed is in the positive x-direction, and the acceleration, $a_x$, is in the negative direction (the speed will be getting smaller and smaller, so its rate of change is negative).

The condition for the ball to stop at the top of the trajectory is that its speed will be zero (that is what it means to stop). We can use equation \ref{eqn:chap3L1dvvst} to find what time that corresponds to:
\begin{align*}
v(t) &= v_{0x}+a_xt\\
0 &= (\SI{10}{m/s}) + (\SI{-9.8}{m/s^2})t\\
\therefore t&=\frac{(\SI{10}{m/s})}{(\SI{9.8}{m/s^2})}=\SI{1.02}{s}
\end{align*}
Now that we know that it took \SI{1.02}{s} to reach the top of the trajectory, we can find how much distance was covered:
\begin{align*}
x(t) &= x_0+v_{0x}t+ \frac{1}{2}a_xt^2\\
x &= (\SI{0}{m})+(\SI{10}{m/s})(\SI{1.02}{s})+\frac{1}{2}(\SI{-9.8}{m/s^2})(\SI{1.02}{s})^2 = \SI{5.10}{m}
\end{align*}
and we find that the ball will rise by \SI{5.10}{m} before falling back down. 
\end{example}

\subsection{Visualizing motion with constant acceleration}

When an object has a constant acceleration, its speed and position as a function of time are described by the two equations:
\begin{align*}
v(t) &= v_{0x} + a_xt\\
x(t) &= x_0+v_{0x}t+ \frac{1}{2}a_xt^2
\end{align*}
where the speed changed linearly with time, and the position changes quadratically with time (it goes as $t^2$). Figure \ref{fig:chap3:1dxvvst_aconst} shows the position and the speed as a function of time for the ball from example \ref{ex:chap3:ballupandown} for the first three seconds of the motion.

\capfig{0.7\textwidth}{figures/Chapter3/1dxvvst_aconst.png}{\label{fig:chap3:1dxvvst_aconst} Position and speed as a function of time for the ball in example \ref{ex:chap3:ballupandown}.}

We can divide the motion into three parts:

\textbf{1) Between $t=\SI{0}{s}$ and $t=\SI{1.02}{s}$}

At time $t=\SI{0}{s}$, the ball starts at a position of $x=\SI{0}{m}$ (left) and a speed of $v_{0x}=\SI{10}{m/s}$ (right). During the first second of motion, the position continues to increase (the ball is moving up), until the position stops increasing at $t=\SI{1.02}{s}$, as found in example \ref{ex:chap3:ballupandown}. During that time, the speed decreases linearly from \SI{10}{m/s} to \SI{0}{m/s} due to the constant negative acceleration from gravity. At $t=\SI{1,02}{s}$, the speed is instantaneously \SI{0}{m/s} and the ball at momentarily at rest (as it reaches the top of the trajectory before falling back down).

\textbf{2) Between $t=\SI{1.02}{s}$ and $t=\SI{2.04}{s}$}

At $t=\SI{1.02}{s}$, the speed continues to decrease linearly (it becomes more and more negative) as the ball start to fall back down faster and faster. The position also starts decreasing just after $t=\SI{1,02}{s}$, as the ball returns back down to the point of release. At $t=\SI{2.04}{s}$, the ball returns to the point from which it was thrown, and the ball is going with the same speed (\SI{10}{m/s}) as when it was released, but in the opposite direction (downwards).

\textbf{3) After $t=\SI{2.04}{s}$}

If nothing is there to stop the ball, it continues to move downwards with ever increasing speed. The position continues to become more negative and the speed continues to become larger in magnitude and more negative.

\begin{checkpointSA}{Make a sketch of the acceleration as a function of time corresponding to the position and velocity shown in Figure \ref{fig:chap3:1dxvvst_aconst}.}
\end{checkpointSA}

\subsection{Speed versus velocity}
In the previous example, our language was not quite as precise as it should be when conducting science. Specifically, we need a way to distinguish the situation when the speed is decreasing (becoming more negative), while the object is actually going faster and faster (after $t=\SI{1.02}{s}$ in Figure \ref{fig:chap3:1dxvvst_aconst}). We will use the term \textbf{speed} to refer to how fast an object is moving (how much distance it covers per unit time), and we will use the term \textbf{velocity} to also indicate the direction of the motion. In other words, the speed is the absolute value of the velocity\footnote{This is true for one-dimensional motion, whereas in two or more dimensions, velocity is a vector and speed is the magnitude of that vector.}. The speed is thus always positive, whereas the velocity can also be negative.

With this vocabulary, the speed of the ball in Figure \ref{fig:chap3:1dxvvst_aconst} decreases between $t=\SI{0}{s}$ and $t=\SI{1.02}{s}$, and increases thereafter. On the other hand, the velocity continuously decreases (it is always becoming more and more negative). Velocity is thus the more general term since it tells us both the speed and the direction of the motion. This is why we used the letter 'v' in all of the equations.

\section{Using calculus to describe motion}
Most objects do not have a constant velocity or acceleration. We thus need to extend our description of the position and velocity of an object to a more general case. This can be done in much the same way as we introduced accelerated motion; namely by pretending that during a very small interval in time, $\Delta t$, the velocity and acceleration are constant, and then considering the motion as the sum over many small intervals. In the limit that $\Delta t$ tends to zero, this will be an accurate description. 

Suppose that an object is moving with a non constant velocity, and covers a distance $\Delta x$ in an amount of time $\Delta t$. We can define an \textbf{average velocity}, $v^{avg}$:
\begin{align*}
v^{avg}= \frac{\Delta x}{\Delta t}
\end{align*}
That is, regardless of our choice of time interval, $\Delta t$, we can always calculate the average velocity, $v^{avg}$, over the time interval. That average velocity will be an average over the interval, between some time $t$ and $t+\Delta t$. If we shrink the time interval, and take the limit $\Delta t\to 0$, we can define the \textbf{instantaneous velocity}:
\begin{align*}
v = \lim_{\Delta t\to 0} \frac{\Delta x}{\Delta t}
\end{align*}
The instantaneous velocity is the velocity only in that small instant in time where we choose $\Delta x$ and $\Delta t$. Another way to read this equation is that the velocity, $v$, is the slope of the graph of $x(t)$. Recall that the slope is the ``rise over run'', in other words the change in $x$ divided by the corresponding change in $t$. Indeed, when we had no acceleration, the position as a function of time, equation \ref{eqn:chap3:1dxvst_noa}, explicitly had the velocity as the slope of a linear function:
 \begin{align*}
 x(t) = v_{0x}+v_xt
 \end{align*}
 If we go back to Figure \ref{fig:chap3:1dxvvst_aconst}, where speed was no longer constant, we can indeed see that the graph of the velocity versus time ($v(t)$) corresponds to the instantaneous slope of the graph of position versus time ($x(t)$). For $t<\SI{1.02}{s}$, the slope of the $x(t)$ graph is positive but decreasing (as is $v(t)$). At $t=\SI{1}.02{s}$, the slope of $x(t)$ is instantaneously \SI{0}{m/s} (as is the velocity). Finally, for $t>\SI{1.02}{s}$, the slope of $x(t)$ is negative and increasing in magnitude, as is $v(t)$.

Leibniz and Newton were the first to develop mathematical tools to deal with calculations that involve quantities that tend to zero, as we have here for our time interval $\Delta t$. Nowadays, we call that field of mathematics ``calculus'', and we will make use of it here. Using the vocabulary of calculus, rather than saying that ``instantaneous velocity is the slope of the graph of position versus time at some point in time'', we say that ``instantaneous velocity is the time derivative of position as a function of time''. We also use a slightly different notation so that we do not have to write the limit $\lim_{\Delta t\to 0}$:
\begin{align}
\label{eqn:chap3:vdef}
\Aboxed{v(t)=\lim_{\Delta t\to 0} \frac{\Delta x}{\Delta t}=\frac{dx}{dt}=\frac{d}{dt} x(t)}
\end{align}
where we can really think of $dt$ as $\lim_{\Delta t\to 0}\Delta t$, and $dx$ as the corresponding change in position over an \textit{infinitesimally} small time interval $dt$.

Similarly, we introduce the \textbf{instantaneous acceleration}, as the time derivative of $v(t)$:
\begin{align}
\Aboxed{a_x(t)=\frac{dv}{dt}=\frac{d}{dt}v(t)}
\end{align}
\newpage
\subsection{Using calculus to obtain acceleration from position}
Suppose that we know the function for position as a function of time, and that it is given by our previous result:
\begin{align*}
x(t)=x_0+v_{0x}t+\frac{1}{2}a_xt^2
\end{align*}
\rwcapfig[16]{0.5\textwidth}{figures/Chapter3/1dDeltaXT.png}{\label{fig:chap3:1dDeltaXT}Obtaining the instantaneous velocity from the graph of position as a function of time. As $\Delta t\to 0$, $\frac{\Delta x}{\Delta t}$ approaches the instantaneous velocity $v(t)$, which is the slope of the curve $x(t)$ at time $t$.}
With our calculus-based definitions above, we should be able to recover that:
\begin{align*}
v(t) = v_{0x}t+at\\
a_x(t) = a_x
\end{align*} 
Let us start by determining $v(t)$:
\begin{align*}
v(t) = \frac{dx}{dt}=\lim_{\Delta t\to 0} \frac{\Delta x}{\Delta t}
\end{align*}
Knowing our function $x(t)$, we can can rewrite this as:
\begin{align}
\Aboxed{ \frac{dx}{dt}=\lim_{\Delta t\to 0} \frac{x(t+\Delta t)-x(t)}{\Delta t} }
\end{align}
which is the formal definition of the derivative $\frac{dx}{dt}$ and gives us a prescription to evaluate it. We will proceed as illustrated in Figure \ref{fig:chap3:1dDeltaXT}:
\begin{enumerate}
\item Use $x(t)$ to determine $\Delta x$ for a small interval $\Delta t$
\item Divide $\Delta x$ by $\Delta t$
\item Take the limit, $\lim_{\Delta t\to 0}$
\end{enumerate}
To obtain $\Delta x$, we introduce a start time, $t_1$, and an end time $t_2$ for our time interval, such that $t_2-t_1=\Delta t$, centred about a time $t$. The change in position is then given by:
\begin{align*}
\Delta x &= x(t_2) - x(t_1)\\
&=\left(x_0+v_{0x}t_2+\frac{1}{2}a_xt_2^2\right )- \left(x_0+v_{0x}t_1+\frac{1}{2}a_xt_1^2\right )\\
&=v_{0x}(t_2-t_1)+\frac{1}{2}a_x(t_2^2-t_1^2)\\
&=v_{0x}\Delta t+\frac{1}{2}a_x(t_2-t_1)(t_2+t_1)\\
&=v_{0x}\Delta t+\frac{1}{2}a_x\Delta t (t_2+t_1)\\
\end{align*}
which we divide by $\Delta t$ to get $v(t)$:
\begin{align*}
v(t) &= \frac{\Delta x}{\Delta t}\\
&=\frac{v_{0x}\Delta t+\frac{1}{2}a_x\Delta t (t_2+t_1)}{\Delta t}\\
&=v_{0x}+\frac{1}{2}a_x(t_2+t_1)
\end{align*}
We now take the limit where $\Delta t\to 0$, namely when $t_2-t_1$ is very small. As we make the interval small, $t_2$ and $t_1$ will both approach the same value of time, say $t$, corresponding to the time at the centre of the interval. In particular, the average of $t_1$ and $t_2$, given by $\frac{1}{2}(t_1+t_2)$, will approach the time at the centre of the interval, $t$. We thus recover the equation for instantaneous velocity:
\begin{align*}
v(t) &= v_{0x}+\frac{1}{2}a_xt
\end{align*}
Of course, once you become more familiar with calculus, you will be able to directly use the formulas for derivatives to recover the answer:
\begin{align*}
v(t) &= \frac{d}{dt}\left( x_0+v_{0x}t+\frac{1}{2}a_xt^2 \right) \\
     &= v_{0x}+at 
\end{align*}
Similarly, we can now confirm that the acceleration is a constant, independent of time:
\begin{align*}
a_x(t) &= \frac{dv}{dt} = \frac{d}{dt}\left(v_{0x}+a_xt \right)\\
     &=a_x
\end{align*}
You can easily verify that you obtain this result by first calculating $v(t)$ at two different times, $t_1$ and $t_2$, taking the difference, $\Delta v = v(t_2)-v(t_1)$, and then taking the limit of $\lim_{\Delta t\to 0}\frac{\Delta v}{\Delta t}$ to get $a_x(t)$.

\begin{checkpointMC}{Chlo\"e has been working on a detailed study of how vicu\~nas\footnote{Never heard of vicu\~nas? Internet!} run, and found that their position as a function of time when they start running is well modelled by the function $x(t)=(\SI{40}{m/s^2})t^2+(\SI{20}{m/s^3})t^3$. What is the acceleration of the vicu\~nas?}
\item $a_x(t)=\SI{40}{m/s^2}$
\item $a_x(t)=\SI{80}{m/s^2}$
\item $a_x(t)=\SI{40}{m/s^2}+(\SI{20}{m/s^3})t$
\item $a_x(t)=\SI{80}{m/s^2}+(\SI{120}{m/s^3})t$ % correct
\end{checkpointMC}

\subsection{Using calculus to obtain position from acceleration}
Now that we saw that we can use derivatives to determine acceleration from position, we will see how to do the reverse and use acceleration to determine position. Let us suppose that we have a constant acceleration, $a_x(t)=a_x$, and that we know that at time $t=\SI{0}{s}$, the object had a speed of $v_{0x}$ and was located at a position $x_0$. 

Since we only know the acceleration as a function of time, we first need to find the velocity as a function of time. We start with:
\begin{align*}
a_x(t)=a_x=\frac{d}{dt} v(t)
\end{align*}
which tells us that we know the slope (derivative) of the function $v(t)$, but not the actual function. In this case, we must do the opposite of taking the derivative, which in calculus is called taking the ``anti-derivative'' with respect to $t$ and has the symbol $\int dt$. In other words, if:
\begin{align*}
\frac{d}{dt} v(t) =a_x(t)
\end{align*}
then:
\begin{align*}
v(t) =\int a_x(t) dt +C
\end{align*}
where as we will see later, the constant $C$ is required in order for the function $v(t)$ to go through the point $v(t=0)=v_{0x}$. All we need now is to determine how to calculate the anti-derivative, $\int a_x(t) dt$. Since in this case, $a_x(t)$ is a constant, $a_x$, we can determine the anti-derivative quite easily. 

For a small interval in time, $\Delta t$, the velocity will change by a small amount, $\Delta v$, such that:
\begin{align*}
a_x &= \frac{\Delta v}{\Delta t}\\
\Delta v &= a_x \Delta t
\end{align*}
If we label the start time of the interval as $t_1$ and the end of the interval as $t_2$, we have:
\begin{align*}
\Delta t &= t_2 - t_1 \\
\Delta v &= v(t_2) - v(t_1) = a_x \Delta t\\
\therefore v(t_2) &= v(t_1)+a_x\Delta t
\end{align*}
If we set $t_1=\SI{0}{s}$ to correspond to the point where $v(t)=v_{0x}$, then we can write the velocity at $t_2$ as:
\begin{align*}
v(t_2) = v(t=\Delta t) = v_{0x}+a_x\Delta t
\end{align*}
and we see that the velocity changed by an amount $a \Delta t$ over a period of time $\Delta t$. Since $a_x$ is the same at all times, this is always true, and after a period of time $t = N\Delta t$, the velocity will have changed by $N a_x \Delta t$, and we recover the original equation for velocity as a function of time when acceleration is constant:
\begin{align*}
v(t=N \Delta t) &= v_{0x}+Na_x\Delta t\\
\therefore v(t) &= v_{0x}+a_xt\\
\end{align*}
We can identify the anti-derivative for the case where $a(t)$ is constant:
\begin{align*}
\frac{d}{dt} v(t) &=a\\
\int a dt &=a_xt +C
\end{align*}
where the constant, $C$, is given by $v_{0x}$.

To now obtain position as a function of time, we proceed in the same manner, namely:
\begin{enumerate}
\item Define a small interval in time $\Delta t$
\item Calculate the corresponding change in position $\Delta x$
\item Add $\Delta x$ to our original position, and repeat.
\end{enumerate}
\rwcapfig[20]{0.5\textwidth}{figures/Chapter3/1dDeltaV.png}{\label{fig:chap3:1dDeltaV} Determining $\Delta x$, given $v(t)$ and $\Delta t$. Three different choices of $v(t)$ are shown, depending on whether $v(t)$ is evaluated at the start of the interval, in the middle, or at the end. As $\Delta t \to 0$, these all become equal.}

Again, we have the time-derivative of the position equal to a function of time:
\begin{align*}
\frac{d}{dt}x(t)=v(t)=v_{0x}+a_xt
\end{align*}
and we need to find the anti-derivative:
\begin{align*}
x(t) = \int \left(  v_{0x}+av_xt \right) dt 
\end{align*}
given that, at time $t=0$, the position was $x=x_0$. After a small interval in time, $\Delta t$, the position will have changed by AN amount $\Delta x$:
\begin{align*}
\Delta x &= v(t) \Delta t\\
\end{align*}
so that the position at time $t=\Delta t$ will be given by:
\begin{align*}
x(t=\Delta t) &= x_0+ \Delta x\\
& = x_0+v(t) \Delta t
\end{align*}
The problem here is to evaluate $v(t)$ since the velocity changes throughout the interval. One possible choice is to evaluate the velocity at $t = \frac{1}{2}\Delta t$, midway in the interval, as we did before. We could also choose to use the velocity at the beginning or at the end of the time interval, as all three choices will converge to the same value when $\Delta t \to 0$, as illustrated in Figure \ref{fig:chap3:1dDeltaV}. For now, we will leave the choice open and simply call the velocity that we use $v_1$ to indicate that it is the velocity in the first interval. We thus write the position, $x(t=\Delta t)$, after a time interval $\Delta t$ as:
\begin{align*}
x(t=\Delta t) = x_0+v_1\Delta t
\end{align*}
The position, $x(t=2\Delta t)$, after another interval in time $\Delta t$ will then be given by:
\begin{align*}
x(t=2\Delta t) &= x(t=\Delta t)+v_2\Delta t\\
&=x_0+v_1\Delta t+v_2\Delta t
\end{align*}
where $v_2$ is the velocity over the second interval in time (different than $v_1$, since velocity changes with time). For the Nth interval, we label the position $x_N=x(t=N\Delta t)$:
\begin{align*}
x_N=x(t=N\Delta t)&=x_0+v_1\Delta t+v_2\Delta t+\dots+v_N\Delta t\\
&=x_0+\sum_{i=1}^Nv_i\Delta t 
\end{align*}

\rwcapfig[12]{0.5\textwidth}{figures/Chapter3/1dvint.png}{\label{fig:chap3:1dvint} Illustration of the anti-derivative $\int v(t) dt$ as a sum.}

where we made use of the summation notation ($\sum$) to avoid writing out every term. The above equation is only correct in the limit of $\Delta t\to 0$, in which case it must be the anti-derivative of $v(t)$:
\begin{align*}
x(t) &= \int v(t) dt\\
     &= x_0+\lim_{\Delta t\to 0}\sum_{i=1}^Nv_i\Delta t 
\end{align*}
so that we can identify:
\begin{align}
\label{eqn:chap3:intsum}
\Aboxed{\int v(t) dt&=\lim_{\Delta t\to 0}\sum_{i=1}^Nv_i\Delta t + C}
\end{align}
where it is understood that $v_i$ is the ``average'' velocity in the ith interval, and the constant $C=x_0$ ensures that $x(t=0)=x_0$. This is now a general definition for the anti-derivative, as we have made no specific assumption about the function $v(t)$. Equation \ref{eqn:chap3:intsum} tells us that the anti-derivative of a function (in this case, $v(t)$) can be obtained from a sum.

The sum is illustrated in Figure \ref{fig:chap3:1dvint}, which shows the function $v(t)$ and several intervals $\Delta t$. Over each interval, $i$, we labeled the average velocity, $v_i$. As the intervals shrink, $\Delta t\to 0$, the average velocity $v_i$ approaches the instantaneous velocity, $v(t)$, at the centre for the interval. Since the function $v(t)$ is linear, the speed at the middle of an interval is exactly equal to the average speed in the interval. Taking $v_i$ as the speed in the middle of the interval, we then see that each term in the sum, $v_i\Delta t$, is equal to the area between the curve $v(t)$ and the t-axis. This is illustrated for the second term in the sum, $v_2\Delta t$, with the grey rectangle in Figure \ref{fig:chap3:1dvint}.

The anti-derivative of a function is thus related to the area between the function and the horizontal axis. If we specify limits on the horizontal axis between which we calculate the area, then the anti-derivative is called an \textbf{integral}. For example, if we wish to calculate the sum of $v_i \Delta t$ for values between $t_a$ and $t_b$, we would write the integral as:
\begin{align*}
\int_{t_a}^{t_b}v(t) dt = \lim_{\Delta t\to 0}\sum_{i=1}^Nv_i\Delta t 
\end{align*}
where the sum is such that for $i=1$, $v_i$ is close to $v(t=t_a)$ and for $i=N$, $v_N$ is close to $v(t=t_b)$. An illustration of taking the integral of $v(t)$ is shown in Figure \ref{fig:chap3:1dvintN} where the sum is shown for two different values of $\Delta t$. It is clear that as $\Delta t \to 0$, the sum becomes equal to the area between the curve and the horizontal axis.

\capfig{0.7\textwidth}{figures/Chapter3/1dvintN.png}{\label{fig:chap3:1dvintN}Integral of $v(t)$ between $t_a=\SI{0.2}{s}$ and $t_b=\SI{0.6}{s}$ illustrated as a sum with of 4 terms when $\Delta t=\SI{0.1}{s}$ or of 8 terms when $\Delta t=\SI{0.05}{s}$.}

Since $v(t)$ is a linear function when acceleration is constant, we can easily calculate the area between the curve and the horizontal axis. In the case of a linear function, $v(t)=v_{0x}+at$, the area is a trapezoid, and we have:
\begin{align*}
\int_{t_a}^{t_b}v(t) dt &= \text{base}\times\text{average height}\\
&=(t_b-t_a)\times\frac{1}{2}\left(v(t_a)+v(t_b)\right)\\
&=(t_b-t_a)\frac{1}{2}(v_{0x}+a_xt_a+v_{0x}+a_xt_b)\\
&=\left( \frac{1}{2}(v_{0x}t_b+a_xt_at_b+v_{0x}t_b+a_xt_b^2)  \right)- \left( \frac{1}{2}(v_{0x}t_a+a_xt_a^2+v_{0x}t_a+a_xt_bt_a) \right)\\
&=\left( v_{0x}t_b+\frac{1}{2}a_xt_b^2 \right)-\left( v_{0x}t_a+\frac{1}{2}a_xt_a^2 \right)
\end{align*}
If we define a new function, $V(t)=v_{0x}t+\frac{1}{2}a_xt^2$, then we have:
\begin{align*}
\int_{t_a}^{t_b}v(t) dt &= V(t_b) -V(t_a)
\end{align*}
In other words, given a function, $v(t)$, the integral of that function between two values $t_a$ and $t_b$ can be found by evaluating a different function, $V(t)$, at the end points $t_a$ and $t_b$\footnote{Note that we only explicitly showed that this true if $v(t)$ is linear, but the result is in fact general.}. As you will see in your calculus course, the function $V(t)$ is precisely what we call the anti-derivative:
\begin{align*}
\int v(t) dt= V(t) + C
\end{align*}
which has derivative:
\begin{align*}
\frac{dV}{dt}=v(t)
\end{align*}
Note that when taking the integral, the constant $C$ always cancels.  
\begin{checkpointMC}{The acceleration of a cricket jumping sideways is observed to increase linearly with time, that is, $a_x(t)=a_0+jt$, where $a_0$ and $j$ are constants. What can you say about the velocity of the cricket as a function of time?}
\item it is constant
\item it increases linearly with time ($v(t)\propto t$)
\item it increases quadratically with time ($v(t)\propto t^2$) %correct
\item it increases with the cube of time ($v(t)\propto t^3$)
\end{checkpointMC}


\section{Relative motion}
In order to describe the motion of an object confined to a straight line, we introduced a an axis ($x$) with a specified direction (in which $x$ increases) and an origin (where $x=0$). Sometimes, it can be more convenient to use an axis that is \textit{moving}. For example, consider a person, Alice, moving inside of a train headed for the French town of Nice. The train is moving with a constant speed, $v'^B$ as measured from the ground. Suppose that another person, Brice, describes Alice's position using the function $x^A(t)$ using an x-axis defined inside of the train car ($x=0$ where Brice is sitting, and positive $x$ is in the direction of the train's motion), as depicted in Figure \ref{fig:chap3:TrainABC} below. As long as any person is in the train with Brice, they will easily be able to describe Alice's motion using the x-axis that is moving with the train. Suppose that the train goes through the French town of Hossegor, where a surfer, Igor, watches the train go by. If Igor wishes to describe Alice's motion, it is easier for him to use a different axis, say $x'$, that is fixed to the ground and not moving with the train. 
\capfig{0.7\textwidth}{figures/Chapter3/TrainABC.png}{\label{fig:chap3:TrainABC}Alice is walking in the train and her position is described by both Brice, who is sitting in the train (using the $x$ axis), and Igor, who is at rest on the ground (using the $x'$ axis).} 

Since Brice already went through the work of determining the function $x^A(t)$ in the \textbf{reference frame} of the train, we wish to determine how to \textit{transform} $x^A(t)$ into the reference frame of the train station, $x'^A(t)$, so that Igor can also describe Alice's motion. In other words, we wish to describe Alice's motion in two different \textit{reference frames}.


A reference frame is simply a choice of coordinates, in this case, a choice of x-axis. Ideally, in physics, we prefer to use \textit{inertial} reference frames, which are reference frames that are either ``at rest'' or that are moving at a constant speed relative to a frame that we consider at rest.
 
 
In principle, if you blocked out all of the windows in the train, it would not be possible for Alice and Brice to determine if the train is moving at constant speed or if it is stopped. Thus, the concept of a ``rest frame'' is itself arbitrary. It is not possible to define a frame of reference that is truly at rest. Even Igor's frame of reference, the train station, is on the planet Earth, which is moving around the Sun with a speed of \SI{108000}{km/h}.


Not only is it impossible to define a frame of reference that is truly at rest, the rules from transforming from one frame to the other depend on the speed between the reference frames. Our common experience is described by what we call ``Galilean Relativity'', but if the speed between trains is very large, close to the speed of light, then we need to use Einstein's Special Theory of Relativity.

\subsection{Galilean Relativity}
Referring to Figure \ref{fig:chap3:TrainABC}, we wish to use Brice's description of Alice's motion, $x^A(t)$, and convert it into a description, $x'^A(t)$ that Igor can use in the train station. Since Brice is at rest in the train, the speed of Brice \textit{relative} to Igor is $v'^B(t)$. The first step is for Igor to describe Brice's position, $x'^B(t)$, (that is, the position of Brice's origin). Assume that we choose $t=0$ to be the point in time where the two origins are aligned. Since the train is moving at a constant speed, $v_B$ (as measured by Brice), then the position of Brice's origin as measured from Igor's origin is given by:
\begin{align*}
x'^B(t)=v'^Bt
\end{align*}
Now that Igor can describe the position of the origin of Brice's coordinate system, he can use Brice's description of Alice's motion. Recall that $x^A(t)$ is Brice's measure of Alice's distance from his origin. Similarly, $x'^B(t)$, is Igor's measure of the distance from his origin to Brice's origin. Thus, to obtain Alice's distance from Igor's origin, we simply add the distance, $x'^B(t)$, from Igor's origin to Brice's origin, and then add, $x^A(t)$, the distance from Brice's origin to Alice. Thus:
\begin{align}
\Aboxed{x'^A(t)=x'^B(t)+x^A(t)=v'^Bt+x^A(t)}
\end{align}
which tells us how to obtain the position of object A in the $x'$ reference frame, when $x^A(t)$ is the description the object's position in the $x$ reference frame which is moving with a velocity $v'^B$ relative to the $x'$ reference frame.

Since we know the position of Alice as measured in Igor's frame of reference, we can now easily find her velocity and her acceleration, as measured by Igor. Her velocity as measured by Igor, $v'^A$, is given by the time-derivative of her position measured in Igor's frame of reference:
\begin{align}
v'^A(t)&=\frac{d}{dt}x'^A(t)\\
&=\frac{d}{dt}(v'^Bt+x^A(t))\\
&=v'^B+\frac{d}{dt}x^A(t)\\
&=v'^B+v^A(t)
\end{align}
where $v^A(t)=\frac{d}{dt}x^A(t)$ is Alice's speed as measured by Brice, in the train. That is, the velocity of Alice as measured by Igor is the sum of the velocity of the train relative to the ground and the velocity of Alice relative to the train, which makes sense. If we now determine Alice's acceleration, $a'^A(t)$, as measured by Igor, we find:
\begin{align}
a'^A(t)&=\frac{d}{dt}v'^A(t)\\
&=\frac{d}{dt}(v'^B+v^A)\\
&=0+\frac{d}{dt}v^A(t)\\
&=a^A
\end{align}
where we have explicitly used the fact that the train is moving at constant velocity ($\frac{d}{dt}v'^B=0$). Here we find that both Brice and Igor will measure the same number when referring to Alice's acceleration (if the train is moving at constant velocity). This is a particularity of ``inertial'' frame of references: accelerations do not depend on the reference frame, as long as the reference frames are moving with a constant velocity relative to each other. As we will see later, forces exerted on a object are directly related to the acceleration experienced by that object. Thus, the forces on a object do not depend on the choice of inertial reference frame. 

\begin{example}{A large boat is sailing North at a speed of $v'^B=\SI{15}{m/s}$ and a restless passenger is walking about on the deck. Chlo\"e, another passenger on the boat, finds that the passenger is walking at a constant speed of $v^A=\SI{3}{m/s}$ towards the South (opposite the direction of the boat's motion). Marcel is watching the boat pass by from shore. What velocity (magnitude and direction) does Marcel measure for the restless passenger?}
First, we must choose coordinate systems in the boat and on the shore. On the boat, let us define an $x$ axis that is positive in the North direction and has an origin such that the position of the restless passenger was $x^A(t=0)=0$ at time $t=0$. In Chlo\"e's reference frame, the passenger is thus described by:
\begin{align*}
x^A(t)=v^At=(\SI{-3}{m/s})t
\end{align*}
where we note that $v^A$ is negative since the passenger is moving the in negative $x$ direction (the passenger is walking towards the South, but we chose positive $x$ to be in the North direction). On shore, we choose an $x'$ axis that also is positive in the North direction. We can choose the origin such that the origin of the boat's coordinate system was $x'=0$. The origin of the boat's coordinate system as measured by Marcel (on shore) is thus:
\begin{align*}
x'^B(t)=v'^Bt=(\SI{15}{m/s})t
\end{align*}
The position of the passenger, $x'^A(t)$, as measured by Marcel, is then given by adding the position of the boat's origin and the position of the passenger as measured from the boat's origin:
\begin{align*}
x'^A(t) &= x'^B(t)+x^A(t)\\
&= v'^Bt + v^At \\
&= (v'^B+v^A)t\\
&= ((\SI{15}{m/s})+(\SI{-3}{m/s}))t\\
&= (\SI{12}{m/s})t
\end{align*}
To find the velocity of the passenger as measured by Marcel, we take the time derivative:
\begin{align*}
v'^A &= \frac{d}{dt}x'^A(t)\\
&= \frac{d}{dt} \left((v'^B+v^A)t\right)\\
&=(v'^B+v^A)\\
&=((\SI{15}{m/s})+(\SI{-3}{m/s}))\\
&=\SI{12}{m/s}
\end{align*}
Since this is a positive number, Marcel still sees the passenger moving in the North direction (the direction of his positive $x'$ axis), but with a speed of \SI{12}{m/s}, which is less than that of the boat. On the boat, the passenger appears to be walking towards the South, but the net motion of the passenger relative to the ground is still in the North direction, as their speed is less than that of the boat.
\end{example}

\subsection{Special Relativity}
In the previous section, we found that the position of an object, $x^A$, in one inertial reference frame can be converted to a position, $x'^A$, in a different inertial reference frame using the equation:
\begin{align*}
x'^A=v'^Bt+x^A
\end{align*}
where $v'^B$ is the velocity of the $x$ frame of reference as measured in the $x'$ frame of reference. We found this by arguing that we could sum the distance from the $x'$ origin to the $x$ origin and the distance from the $x$ origin to the object's position, $x^A$, in the $x$ reference frame. We call this transformation, from $x$ to $x'$, a ``Galilean transformation''. In this section, we briefly introduce the more general way to transform descriptions between reference frames, mostly for completeness.

Although most of our common experience is well described by this transformation, Albert Einstein found that it is no longer accurate if the speeds involved in the situation are close to the speed of light. Einstein's Theory of Special Relativity gives the following transformation instead:
\begin{align}
\label{eqn:chap3:LorentzTr}
x'^A&=\frac{1}{\sqrt{1-\left(\frac{v'^B}{c}\right)^2}}(v'^Bt+x^A)\\
t'&=\frac{1}{\sqrt{1-\left(\frac{v'^B}{c}\right)^2}}\left(  t+\frac{v'^Bx^A}{c^2} \right)
\end{align}
where $c$ is the speed of light (\SI{3e8}{m/s}). The first thing to note is that this a considerably more complicated expression. The most bizarre aspect is that the two reference frames have different measures of time ($t$ and $t'$). The above transformation, from $x,t$ to $x',t'$, is called a ``Lorentz transformation''. The fact that we have to include time in the transformation is an aspect related to the fact that Einstein's Theory of Special Relativity tells us that we must consider Space-Time as a single entity (rather than thinking of space and time as separate things).

Let us examine a few aspects of the Lorentz transformation. First, if the velocity, $v'^B$, between reference frames is small when compared to the speed of light, $v'^B<<c$, we have:
\begin{align*}
\left(\frac{v'^B}{c}\right)^2&<<1\\
\therefore \frac{1}{\sqrt{1-\left(\frac{v'^B}{c}\right)^2}} &\sim 1 \\
\end{align*}
In this case, we recover the original Galilean transformation:
\begin{align*}
x'^A&=\lim_{v_B<<c}\frac{1}{\sqrt{1-\left(\frac{v'^B}{c}\right)^2}}(v'^Bt+x^A)=v'^Bt+x^A\\
t'&=\lim_{v_B<<c}\frac{1}{\sqrt{1-\left(\frac{v'^B}{c}\right)^2}}\left(  t+\frac{v'^Bx^A}{c^2} \right)=t
\end{align*}
So \textit{a priori}, this aspect of Einstein's Theory of Special Relativity is difficult to test, as we only expect it to deviate from common experience when $v_B$ is comparable to the speed of light. However, in particle accelerators, subatomic particles are regularly accelerated to speeds close to the speed of light, and their position has to be described using Special Relativity. 

Suppose that a space ship is carrying Chlo\"e and moving with a speed of $v'^B=0.7c=\SI{2.1}{m/s}$. Suppose that Chlo\"e fires an arrow in the same direction as the space ship's motion, and that she measures the speed of the arrow to be $v^A=0.4c=\SI{0.9e8}{m/s}$. Suppose that the space ship goes by Marcel, who is stationary on Earth, right at the moment that Chlo\"e fires the arrow. Let us define the $x$ axis on the ship so that it is positive in the direction of motion with its origin where  Chlo\"e fired the arrow. Let us define $t=t'=0$ to be the moment when the origin of the space ship x-axis is aligned with the origin of Marcel's $x'$ axis on Earth. The $x'$ axis is also set up to be positive in the direction of the space ship's motion and has the origin such that at $t'=0$

Using Galilean Relativity, the speed of the arrow as measured by Marcel would be found by adding the two velocities, $v'^A=v^A+v'^B=1.1c=\SI{3.3e8}{m/s}$. Using Special Relativity, the situation is a little more difficult. The position as measured in Marcel's reference frame is given by:
\begin{align*}
x'^A&=\frac{1}{\sqrt{1-\left(\frac{v'^B}{c}\right)^2}}(v'^Bt+x^A)
\end{align*}
To obtain the arrow's velocity in Marcel's frame, we need the time derivative of $x'^A$ with respect to $t'$, the time measured in Marcel's frame:
\begin{align*}
v'^A=\frac{d}{dt'}x'^A
\end{align*}
However, $x'^A$ is written in terms of $t$, not $t'$. However, equation \ref{eqn:chap3:LorentzTr}, tells us how $t$ depends on $t'$; we thus need to treat $t$ as a function of $t'$ and use the Chain Rule:
\begin{align*}
v'^A&=\frac{d}{dt'}x'^A(t)\\
&=\frac{d}{dt}x'^A(t)\left(\frac{dt}{dt'}\right)\\
&=\frac{d}{dt}\left(\frac{1}{\sqrt{1-\left(\frac{v'^B}{c}\right)^2}} (v'^Bt+x^A)  \right)\left(\frac{dt}{dt'}\right)\\
&=\frac{1}{\sqrt{1-\left(\frac{v'^B}{c}\right)^2}}\frac{d}{dt}\left( (v'^Bt+x^A)  \right)\left(\frac{dt}{dt'}\right)\\
&=\frac{1}{\sqrt{1-\left(\frac{v'^B}{c}\right)^2}}\left( v'^B+\frac{d}{dt}x^A \right)\left(\frac{dt}{dt'}\right)\\
&=\frac{1}{\sqrt{1-\left(\frac{v'^B}{c}\right)^2}}\left( v'^B+v^A \right)\left(\frac{dt}{dt'}\right)\\
\end{align*}
We can differentiate equation \ref{eqn:chap3:LorentzTr} to get $\frac{dt'}{dt}$:
\begin{align*}
\frac{dt'}{dt}&=\frac{d}{dt'}\left(\frac{1}{\sqrt{1+\left(\frac{v'^B}{c}\right)^2}}\left(t+\frac{v'^Bx^A}{c^2} \right)\right)\\
&=\frac{1}{\sqrt{1+\left(\frac{v'^B}{c}\right)^2}}\left(1+\frac{v'^Bv^A}{c^2} \right)
\end{align*}
where we identified $v^A=\frac{dx^A}{dt}$. We can take the inverse to get $\frac{dt}{dt'}$:
\begin{align*}
\frac{dt}{dt'}&=\frac{1}{\frac{dt'}{dt}}\\
&=\frac{\sqrt{1+\left(\frac{v'^B}{c}\right)^2}}{1+\frac{v'^Bv^A}{c^2}}
\end{align*}
Combining everything, we get the velocity of the arrow as measured in Earth's reference frame by Marcel:
\begin{align}
\label{eqn:chap3:LorentzV}
v'^A&=\frac{1}{\sqrt{1-\left(\frac{v'^B}{c}\right)^2}}\left( v'^B+v^A \right)\left(\frac{dt}{dt'}\right)\nonumber\\
\Aboxed{v'^A&=\frac{v'^B+v^A}{1+\frac{v'^Bv^A}{c^2}}}
\end{align}
The above equation tells us how to convert the velocity of the arrow as measured in the space ship ($v^A$) into the velocity as measured on Earth, $v'^A$, when the space ship moves with velocity $v'^B$ as measured on Earth. When $v^A$ and $v'^B$ are small compared to $c$, then this equation reduces to the Galilean version ($v'^A=v'^B+v^A$). However, when the velocities are large this is no longer the case. For the above example, with $v^A=0.4c$, $v'^B=0.7c$, we get:
\begin{align*}
v'^A&=\frac{v'^B+v^A}{1+\frac{v'^Bv^A}{c^2}}\\
&=\frac{(0.7c)+(0.4c)}{1+\frac{(0.7c)(0.4c)}{c^2}}\\
&=\frac{1.1}{1+0.28}c\\
&=0.86c=\SI{2.57e8}{m/s}
\end{align*}
which is well short of the $1.1c$ that you would expect from Galilean relativity. Suppose instead that $v^A=c$, that is, Chlo\"e fires the arrow at the speed of light (she sees the arrow move away at the speed of light). Marcel will measure the speed of the arrow to be:
\begin{align*}
v'^A&=\frac{v'^B+v^A}{1+\frac{v'^Bv^A}{c^2}}\\
&=\frac{v'^B+(1.0c)}{1+\frac{v'^B(1.0c)}{c^2}}\\
&=\frac{v'^B+c}{1+\frac{v'^B}{c}}\\
&=c \frac{v'^B+c}{c+v'^B}\\
&=c
\end{align*}
regardless of the speed of the space ship! This is a truly strange result from Einstein's theory. If Chlo\"e fires an arrow at the speed of light, Marcel will also measure the arrow moving at the speed of light, regardless of whether or not the space ship is moving. Indeed, all observers in inertial frames of references would agree that the arrow is moving at the same speed, $c$. Although we showed this result as a consequence of the Lorentz transformation, Einstein developed his Theory of Special Relativity by postulating that the speed of light is the same in all inertial reference frames. The Lorentz transformation is then just a consequence of this postulate. This postulate has several strange consequence that have been \textit{experimentally} verified:
\begin{itemize}
\item Moving objects contract. That is, fast moving objects become shorter.
\item Moving clocks slow down. That is, people moving very fast age slower.
\item People in different inertial reference frames cannot agree on the order in which events took place. 
\end{itemize}
We will explore this in more detail when we take a closer look at Einstein's Special Theory of Relativity.

\newpage
\section{Summary}
\vspace{1cm}
\begin{chapterSummary}
\item To describe motion in one dimension, we must define an axis with:
\begin{enumerate}
\item An origin (where $x=0$)
\item A direction (the direction in which $x$ increases)
\end{enumerate}
\item We describe the position of an with a function $x(t)$ that \textit{depends} on time
\item The rate of change of position is called ``velocity'' and is a function given by the time-derivative of position.
\item The rate of change of velocity is called ``acceleration'' and is a function given by the time-derivative of velocity.
\item Given a function for acceleration, $a_x(t)$, one can use its anti-derivative to determine velocity.
\item Given a function for velocity, $v_x(t)$, one can use its anti-derivative to determine position.
\item An inertial frame of reference is one that is moving with a constant velocity.
\item It is impossible to define a frame of reference that is truly ``at rest'', so we consider inertial frames of reference only relative to other frames of reference that we also consider to be inertial.
\item If an object has a position $x^A(t)$ in a given inertial frame of reference, $x$, that is moving with a velocity $v'^B$ compared to a different inertial frame of reference, $x'$, then the position of the object in the $x'$ frame of reference is found by:
\begin{enumerate}
\item Galilean transformation, if all speeds are small
\item Lorentz transformation, if any of the speeds are comparable to the speed of light
\end{enumerate}
\end{chapterSummary}
\subsection{Important equations}
If the position of an object is described by a function $x(t)$, then, its velocity, $v_x(t)$, and acceleration, $a_x(t)$, are given by:
\begin{align*}
v_x(t)&=\lim_{\Delta t\to 0}\frac{\Delta x}{\Delta t}=\frac{dx}{dt}\\
a_x(t)&=\lim_{\Delta t\to 0}\frac{\Delta v}{\Delta t}=\frac{dv_x}{dt}\\
\end{align*}
Conversely, given the acceleration, $a_x(t)$, on can find the velocity and position:
\begin{align*}
v_x(t)=\lim_{\Delta t\to 0}\sum_ia_x(t_i)\Delta t+C=\int a_x(t)dt+C\\
x(t)=\lim_{\Delta t\to 0}\sum_iv_x(t_i)\Delta t+C=\int v_x(t)dt+C\\
\end{align*}
With a constant acceleration, $a_x(t)=a_x$, if the object had velocity $v_{0x}$ and position $x_0$ at $t=0$:
\begin{align*}
v_x(t)&=v_{0x}t+a_xt\\
x(t)&=x_0+v_{0x}t+\frac{1}{2}a_xt^2
\end{align*}
If an object has position $x^A$ as measured in a frame of reference $x$ that is moving at constant speed $v'^B$ (small compared to the speed of light) as measured in a second frame of reference $x'$, then in the $x'$ reference frame, the object is described by the Galilean transformation:
\begin{align*}
x'^A(t) &= v'^Bt + x^A(t)\\
v'^A(t) &=v'^B+v^A(t)\\
a'^A(t) &= a(t)
\end{align*}
If the $x$ frame of reference (or the object) is moving very fast compared to the speed of light, then in the $x'$ reference frame, it is described by the Lorentz transformation:
\begin{align*}
x'^A&=\frac{1}{\sqrt{1-\left(\frac{v'^B}{c}\right)^2}}(v'^Bt+x^A)\\
t'&=\frac{1}{\sqrt{1-\left(\frac{v'^B}{c}\right)^2}}\left(  t+\frac{v'^Bx^A}{c^2} \right)\\
v'^A&=\frac{v'^B+v^A}{1+\frac{v'^Bv^A}{c^2}}
\end{align*}
where you should note that one must also transform the meaning of time, $t$.

\chapter{Describing motion in multiple dimensions}
\label{chap:4_KinematicsND}
In this chapter, we will learn how to extend our description of an object's motion to two and three dimensions by using vectors. We will also consider the specific case of an object moving along the circumference of a circle. 

\vspace{1cm}
\begin{learningObjectives}
\item Describe motion in a 2D plane.
\item Describe motion in 3D space.
\item Describe motion along the circumference of a circle.
\end{learningObjectives}

\section{Motion in two dimensions}

\subsection{Using vectors to describe motion in two dimensions}
We can specify the location of an object with its coordinates, and we can quantify any displacement by a vector. First consider the case of an object moving at a constant velocity in a particular direction.  We can describe the object at any time, $t$, using its position vector, $\vec r(t)$, which is a function of time:
\begin{align*}
\vec r(t=t_0)&=\vec r_1\\
\vec r(t=t_0+\Delta t)&=\vec r_2
\end{align*}
More generally, we can describe the $x$ and $y$ components of the position vector with independent functions, $x(t)$, and $y(t)$, respectively:
\begin{align*}
\vec r(t) = \begin{pmatrix}
           x(t) \\
           y(t) \\
         \end{pmatrix}= x(t) \hat x + y(t) \hat y
\end{align*}
Suppose that in a period of time $\Delta t$, the object goes from a position described by the position vector $\vec r_1$ to a position described by the position vector $\vec r_2$, as illustrated in Figure \ref{fig:chap4:xydrvec}. We can define a displacement vector, $\Delta \vec r=\vec r_2-\vec r_1$, and by analogy to the one dimensional case, we can define an \textbf{average} velocity vector, $\vec v$ as:
\begin{align}
\vec v = \frac{\Delta \vec r}{\Delta t}
\end{align}
\capfig{0.3\textwidth}{figures/Chapter4/xydrvec.png}{\label{fig:chap4:xydrvec}Illustration of a displacement vector, $\Delta \vec r = \vec r_2 -\vec r_1$, for an object that was located at position $\vec r_1$ at time $t_1$ and at position $\vec r_2$ at time $t_2=t_1+\Delta t$.}

The average velocity vector will have the same direction as $\Delta \vec r$, since it is the displacement vector divided by a scalar ($\Delta t$). The magnitude of the velocity vector, which we call ``speed'', will be proportional to the length of the displacement vector. If the object moves a large distance in a small amount of time, it will thus have a large velocity vector. This definition of the velocity vector thus has the correct intuitive properties (points in the direction of motion, is larger for faster objects).

For example, if the object went from position $(x_1,y_1)$ to position $(x_2,y_2)$ in an amount of time $\Delta t$, the average velocity vector is given by:
\begin{align*}
\vec v &= \frac{\Delta \vec r}{\Delta t}\\
&=\frac{1}{\Delta t}\begin{pmatrix}
           x_2-x_1 \\
           y_2-y_1 \\
         \end{pmatrix}\\
 &=\frac{1}{\Delta t}\begin{pmatrix}
           \Delta x \\
           \Delta y \\
         \end{pmatrix}\\     
 &=\begin{pmatrix}
           \frac{\Delta x}{\Delta t} \\
           \frac{\Delta y}{\Delta t}\\
         \end{pmatrix}\\       
 &=\begin{pmatrix}
           v_x \\
           v_y \\
         \end{pmatrix}\\    
\therefore \vec v &= v_x\hat x+v_y\hat y                     
\end{align*}
That is, the $x$ and $y$ components of the average velocity vector can be found by separately determining the average velocity in each direction. For example, $v_x=\frac{\Delta x}{\Delta t}$ corresponds to the average velocity in the $x$ direction, and can be considered independent from the velocity in the $y$ direction, $v_y$. The magnitude of the average velocity vector (i.e. the average speed), is given by:
\begin{align*}
||\vec v||&=\sqrt{v_x^2+v_y^2}=\frac{1}{\Delta t}\sqrt{\Delta x^2+\Delta y^2}=\frac{\Delta r}{\Delta t}
\end{align*}
where $\Delta r$ is the magnitude of the displacement vector. Thus, the average speed is given by the distance covered divided by the time taken to cover that distance, in analogy to the one dimensional case.

\begin{checkpointMC}{A llama runs in a field from a position $(x_1,y_1)=(\SI{2}{m},\SI{5}{m})$ to a position $(x_2,y_2)=(\SI{6}{m},\SI{8}{m})$ in a time $\Delta t=\SI{0.5}{s}$, as measured by Marcel, a llama farmer standing at the origin of the Cartesian coordinate system. What is the average speed of the llama?}
\item \SI{1}{m/s}
\item \SI{5}{m/s}
\item \SI{10}{m/s}%correct
\item \SI{15}{m/s}
\end{checkpointMC}

If the velocity of the object is not constant, then we define the \textbf{instantaneous velocity vector} by taking the limit $\Delta t\to 0$:
\begin{align}
\vec v(t) &= \lim_{\Delta t \to 0}\frac{\Delta \vec r}{\Delta t}=\frac{d\vec r}{dt}
\end{align}
which gives us the time derivative of the position vector (in one dimension, it was the time derivative of position). Writing the components of the position vector as functions $x(t)$ and $y(t)$, the instantaneous velocity becomes:
\begin{align}
\label{eqn:chap4:vvecdef}
\Aboxed{\vec v(t) &=\frac{d}{dt}\vec r(t) }\\
&=\frac{d}{dt} \begin{pmatrix}
           x(t) \\
           y(t) \\
         \end{pmatrix}\nonumber\\ 
&=\begin{pmatrix}
           \frac{dx}{dt}  \\
          \frac{dy}{dt}  \\
         \end{pmatrix}\nonumber\\ 
 &=\begin{pmatrix}
           v_x(t) \\
           v_y(t) \\
         \end{pmatrix}\nonumber\\   
\therefore \vec v(t) &= v_x(t)\hat x+v_y(t)\hat y  \nonumber     
\end{align}
where, again, we find that the components of the velocity vector are simply the velocities in the $x$ and $y$ direction. This means that we can treat motion in two dimensions as having two independent components: a motion along $x$ and a separate motion along $y$. This highlights the usefulness of the vector notation for allowing us to use one vector equation ($\vec v=\frac{d}{dt}\Delta \vec r$) to represent two equations (one for $x$ and one for $y$). 

Similarly the acceleration vector is given by:
\begin{align}
\label{eqn:chap4:avecdef}
\Aboxed{\vec a(t) &= \frac{d}{dt}\vec v(t)} \\
&=\begin{pmatrix}
           \frac{dv_x}{dt}  \\
          \frac{dv_y}{dt}  \\
         \end{pmatrix}\nonumber\\
&=\begin{pmatrix}
           a_x(t) \\
           a_y(t) \\
         \end{pmatrix}\nonumber\\
\therefore \vec a(t) &= a_x(t)\hat x+a_y(t)\hat y      \nonumber        
\end{align}

For example, if an object is at position $\vec r_0=(x_0,y_0)$ with a velocity vector $\vec v_0=v_{0x}\hat x + v_{0y}\hat y$ at time $t=0$, and has a constant acceleration vector, $\vec a = a_x\hat x+a_y\hat y$, then the velocity vector at some later time $t$, $\vec v(t)$, is given by:
\begin{align*}
\vec v(t) = \vec v_0 + \vec a t
\end{align*}
Or, if we write out the components explicitly:
\begin{align*}
\begin{pmatrix}
           v_x(t) \\
           v_y(t) \\
         \end{pmatrix} = \begin{pmatrix}
           v_{0x} \\
           v_{0y} \\
         \end{pmatrix} + \begin{pmatrix}
           a_xt \\
           a_yt \\
         \end{pmatrix}
\end{align*}
which really can be considered as two independent equations for the components of the velocity vector:
\begin{align*}
v_x(t)&=v_{0x}+a_xt \\
v_y(t)&=v_{0y}+a_yt \\
\end{align*}
which is the same equation that we had for one dimensional kinematics, but once for each coordinate. The position vector is given by:
\begin{align*}
\vec r(t) = \vec r_0 + \vec v_0 t + \frac{1}{2} \vec at^2
\end{align*}
with components:
\begin{align*}
x(t) &= x_0+v_{0x}t+\frac{1}{2}a_xt^2\\
y(t) &= y_0+v_{0y}t+\frac{1}{2}a_yt^2\\
\end{align*}
which again shows that two dimensional motion can be considered as separate and independent motions in each direction.

\begin{example}{An object starts at the origin of a coordinate system at time $t=\SI{0}{s}$, with an initial velocity vector $\vec v_0=(\SI{10}{m/s})\hat x+(\SI{15}{m/s})\hat y$. The acceleration in the $x$ direction is \SI{0}{m/s^2} and the acceleration in the $y$ direction is \SI{-10}{m/s^2}.
\begin{enumerate}[label=(\alph*)]
\item Write an equation for the position vector as a function of time.
\item Determine the position of the object at $t=\SI{10}{s}$.
\item Plot the trajectory of the object for the first \SI{5}{s} of motion.
\end{enumerate}
\ }
\label{ex:chap4:parabola}
\textbf{a)}We can consider the motion in the $x$ and $y$ direction separately. In the $x$ direction, the acceleration is 0, and the position is thus given by:
\begin{align*}
x(t)&=x_0+v_{0x}t\\
&=(\SI{0}{m})+(\SI{10}{m/s})t\\
&=(\SI{10}{m/s})t
\end{align*}
In the $y$ direction, we have a constant acceleration, so the position is given by:
\begin{align*}
y(t) &= y_0+v_{0y}t+\frac{1}{2}a_yt^2\\
&=(\SI{0}{m})+(\SI{15}{m/s})t+\frac{1}{2}(\SI{-10}{m/s^2})t^2\\
&=(\SI{15}{m/s})t-\frac{1}{2}(\SI{10}{m/s^2})t^2\\
\end{align*}
The position vector as a function of time can thus be written as:
\begin{align*}
\vec r(t) &= \begin{pmatrix}
           x(t) \\
           y(t) \\
          \end{pmatrix}\\
          &= \begin{pmatrix}
           (\SI{10}{m/s})t \\
           (\SI{15}{m/s})t-\frac{1}{2}(\SI{10}{m/s^2})t^2 \\
         \end{pmatrix}
\end{align*}
\textbf{b)} Using $t=\SI{10}{s}$ in the above equation gives:
\begin{align*}
\vec r(t=\SI{10}{s})&= \begin{pmatrix}
           (\SI{10}{m/s})(\SI{10}{s}) \\
           (\SI{15}{m/s})(\SI{10}{s})-\frac{1}{2}(\SI{10}{m/s^2})(\SI{10}{s})^2 \\
         \end{pmatrix}\\
         &= \begin{pmatrix}
           (\SI{100}{m}) \\
           (\SI{-350}{m})\\
         \end{pmatrix}
\end{align*}
\textbf{c)} We can plot the trajectory using python:

\begin{python}[caption=Trajectory in xy plane]
#import modules that we need
import numpy as np #for arrays of numbers
import pylab as pl #for plotting

#define functions for the x and y positions:
def x(t):
    return 10*t

def y(t):
    return 15*t-0.5*10*t**2

#define 10 values of t from 0 to 5 s:
tvals = np.linspace(0,5,10)

#calculate x and y at those 10 values of t using the functions
#we defined above:
xvals = x(tvals)
yvals = y(tvals)

#plot the result:
pl.plot(xvals,yvals, marker='o')
pl.xlabel("x [m]",fontsize=14)
pl.ylabel("y [m]",fontsize=14)
pl.title("Trajectory in the xy plane",fontsize=14)
pl.grid()
pl.show()
\end{python}
\begin{poutput}
(*\capfig{0.5\textwidth}{figures/Chapter4/parabola.png}{\label{fig:chap4:parabola}Parabolic trajectory of an object with no acceleration in the $x$ direction and a negative acceleration in the $y$ direction.}*)
\end{poutput}
As you can see, the trajectory is a parabola, and corresponds to what you would get when throwing an object with an initial velocity with upwards (positive $y$) and horizontal (positive $x$) components. If you look at only the $y$ axis, you will see that the object first goes up, then turns around and goes back down. This is exactly what happens when you throw a ball upwards, independently of whether the object is moving in the $x$ direction. In the $x$ direction, the object just moves with a constant velocity. The points on the graph are drawn for constant time intervals (the time between each point, $\Delta t$ is constant). If you look at the distance between points projected onto the $x$ axis, you will see that they are all equidistant and that along $x$, the motion corresponds to that of an object with constant velocity. 
\end{example}

\begin{checkpointMC}{In example \ref{ex:chap4:parabola}, what is the velocity vector exactly at the top of the parabola in Figure \ref{fig:chap4:parabola}?}
\item $\vec v=(\SI{10}{m/s})\hat x+(\SI{15}{m/s})\hat y$
\item $\vec v=(\SI{15}{m/s})\hat y$
\item $\vec v=(\SI{10}{m/s})\hat x$ %correct
\item none of the above
\end{checkpointMC}

\subsection{Accelerated motion when the velocity vector changes direction}
\label{sec:chap4:accvconst}
One key difference with one dimensional motion is that, in two dimensions, it is possible to have a non-zero acceleration even when the speed is constant. Recall, the acceleration \textbf{vector} is defined as the time derivative of the velocity \textbf{vector} (equation \ref{eqn:chap4:avecdef}). This means that if the velocity vector changes with time, then the acceleration vector is non-zero. The length of the velocity vector is called the speed. If the length of the velocity vector (speed) is constant, it is still possible that the \textbf{direction} of the velocity vector changes with time, and thus, that the acceleration vector is non-zero. In this case, the acceleration would not result in a change of speed, but rather in a change of the direction of motion. This is exactly what happens when an object goes around in a circle with a constant speed (the direction of the velocity vector changes). 
\rwcapfig[14]{0.35\textwidth}{figures/Chapter4/deltav.png}{\label{fig:chap4:deltav} Illustration of how the direction of the velocity vector can change when speed is constant.}

Figure \ref{fig:chap4:deltav} shows an illustration of a velocity vector, $\vec v(t)$, at two different times, $\vec v_1$ and $\vec v_2$, as well as the vector difference, $\Delta \vec v=\vec v_2 - \vec v_1$, between the two. In this case, the length of the velocity vector did not change with time ($||\vec v_1||=||\vec v_2||$). The acceleration vector is given by:
\begin{align*}
\vec a = \lim_{\Delta t\to 0}\frac{\Delta \vec v}{\Delta t}
\end{align*}
and will thus have a direction parallel to $\Delta \vec v$, and a magnitude that is proportional to $\Delta v$. Thus, even if the velocity vector does not change amplitude (speed is constant), the acceleration vector can be non-zero if the velocity vector changes \textit{direction}.

Let us write the velocity vector, $\vec v$, in terms of its magnitude, $v$, and a unit vector, $\hat v$, in the direction of $\vec v$:
\begin{align*}
\vec v &=v_x\hat x+v_y\hat y= v \hat v\\
v&=||\vec v||=\sqrt{v_x^2+v_y^2}\\
\hat v &= \frac{v_x}{v}\hat x+\frac{v_y}{v}\hat y\\
\end{align*}
In the most general case, both the magnitude of the velocity and its direction can change with time. That is, both the direction and the magnitude of the velocity vector are functions of time:
\begin{align*}
\vec v(t)&=v(t)\hat v(t)
\end{align*}
When we take the time derivative of $\vec v(t)$ to obtain the acceleration vector, we need to take the derivative of a product of two functions of time, $v(t)$ and $\hat v(t)$. Using the rules for taking the derivative of a product, the acceleration vector is given by:
\begin{align}
\label{eqn:chap4:avecdef2}
\vec a &= \frac{d}{dt}\vec v(t)= \frac{d}{dt}v(t)\hat v(t)\nonumber\\
\Aboxed{\vec a&=\frac{dv}{dt}\hat v(t)+v(t)\frac{d\hat v}{dt}}
\end{align}
and has two terms. The first term, $\frac{dv}{dt}\hat v(t)$, is zero if the speed is constant ($\frac{dv}{dt}=0$). The second term, $v(t)\frac{d\hat v}{dt}$, is zero if the direction of the velocity vector is constant ($\frac{d\hat v}{dt}=0$). In general though, the acceleration vector has two terms corresponding to the change in speed, and to the change in the direction of the velocity, respectively.

The specific functional form of the acceleration vector will depend on the path being taken by the object. If we consider the case where speed is constant, then we have:
\begin{align*}
v(t) &= v \\
\frac{dv}{dt}&=0\\
v_x^2(t)+v_y^2(t) &=v^2 \\
\therefore v_y(t)&=\sqrt{v^2-v_x(t)^2}
\end{align*}
\capfig{0.35\textwidth}{figures/Chapter4/aperpv.png}{\label{fig:chap4:aperpv} Illustration that the acceleration vector is perpendicular to the velocity vector if speed is constant.}
In other words, if the magnitude of the velocity is constant, then the $x$ and $y$ components are no longer independent (if the $x$ component gets larger, then the $y$ component must get smaller so that the total magnitude remains unchanged). If the speed is constant, then the acceleration vector is given by:
\begin{align}
\label{eqn:chap4:vecaconstv}
\vec a&=\frac{dv}{dt}\hat v(t)+v\frac{d\hat v}{dt}\nonumber\\
&=0 + v\frac{d}{dt}\hat v(t)\nonumber\\
&=v\frac{d}{dt}\left(\frac{v_x(t)}{v}\hat x+\frac{v_y(t)}{v}\hat y   )\right)\nonumber\\
&=\frac{dv_x}{dt}\hat x + \frac{d}{dt}\sqrt{v^2-v_x(t)^2}\hat y\nonumber\\
&=\frac{dv_x}{dt}\hat x + \frac{1}{2\sqrt{v^2-v_x(t)^2}}(-2v_x(t))\frac{dv_x}{dt}\hat y\nonumber\\
&=\frac{dv_x}{dt}\hat x - \frac{v_x(t)}{\sqrt{v^2-v_x(t)^2}}\frac{dv_x}{dt}\hat y\nonumber\\
&=\frac{dv_x}{dt}\hat x - \frac{v_x(t)}{v_y(t)}\frac{dv_x}{dt}\hat y\nonumber\\
\therefore\quad\Aboxed{\vec a&=\frac{dv_x}{dt} \left(\hat x - \frac{v_x(t)}{v_y(t)}\hat y\right)}
\end{align}
where most of the algebra that we did was to separate out the $x$ and $y$ components of the acceleration vector. The resulting acceleration vector is illustrated in Figure \ref{fig:chap4:aperpv} along with the velocity vector. Rather, a vector parallel to the acceleration vector is illustrated, as the factor of $\frac{dv_x}{dt}$ was omitted (as you recall, multiplying by a scalar only changes the length, not the direction). The velocity vector has components $v_x$ and $v_y$, which allows us to calculate the angle, $\theta$ that it makes with the $x$ axis:
\begin{align*}
\tan(\theta)=\frac{v_y}{v_x}
\end{align*}
Similarly, the vector that is parallel to the acceleration has components of $1$ and $-\frac{v_x}{v_y}$, allowing us to determine the angle, $\phi$, that it makes with the $x$ axis:
\begin{align*}
\tan(\phi)=\frac{v_x}{v_y}
\end{align*}
Note that $\tan(\theta)$ is the inverse of $\tan(\phi)$, or in other words, $\tan(\theta)=\cot(\phi)$, meaning that $\theta$ and $\phi$ are complementary and thus must sum to $\frac{\pi}{2}$ (\SI{90}{\degree}). This means that \textbf{the acceleration vector is perpendicular to the velocity vector if the speed is constant and the direction of the velocity changes}. 

In other words, when we write the acceleration vector, we can identify two components, $\vec a_{\parallel}(t)$ and $\vec a_{\perp}(t)$:
\begin{align*}
\vec a&=\frac{dv}{dt}\hat v(t)+v(t)\frac{d\hat v}{dt}\\
&=\vec a_{\parallel}(t) + \vec a_{\perp}(t)\\
\therefore \vec a_{\parallel}(t)&=\frac{dv}{dt}\hat v(t)\\
\therefore \vec a_{\perp}(t)&=v\frac{d\hat v}{dt}=\frac{dv_x}{dt} \left(\hat x - \frac{v_x(t)}{v_y(t)}\hat y\right)
\end{align*}
where $\vec a_{\parallel}(t)$ is the component of the acceleration that is parallel to the velocity vector, and is responsible for changing its magnitude, and $\vec a_{\perp}(t)$, is the component that is perpendicular to the velocity vector and is responsible for changing the direction of the motion.

\begin{checkpointMC}{A satellite moves in a circular orbit around the Earth with a constant speed. What can you say about its acceleration vector?}
\item it has a magnitude of zero.
\item it is perpendicular to the velocity vector.
\item it is parallel to the velocity vector.
\item it is in a direction other than parallel or perpendicular to the velocity vector.
\end{checkpointMC}

\subsection{Relative motion}
In the previous chapter, we examined how to convert the description of motion from one reference frame to another. Recall the one dimensional situation where we described the position of an object, $A$, using an axis $x$ as $x^A(t)$. Suppose that the reference frame, $x$, is moving with a constant speed, $v'^B$, relative to a second reference frame, $x'$. We found that the position of the object is described in the $x'$ reference frame as:
\begin{align*}
x'^A(t)=v'^Bt+x^A(t)
\end{align*}
if the origins of the two systems coincided at $t=0$. The equation above simply states that the distance of the object to the $x'$ origin is the sum of the distance from the $x'$ origin to the $x$ origin \textbf{and} the distance from the $x$ origin to the object.

In two dimensions, we proceed in exactly the same way, but use vectors instead:
\begin{align*}
\pvec r'^A(t) = \pvec v'^Bt+\vec r^A(t)
\end{align*}
where $r^A(t)$ is the position of the object as described in the $xy$ reference frame, $\pvec v'^B$, is the velocity vector describing the motion of the origin of the $xy$ coordinate system relative to an $x'y'$ coordinate system. $\pvec r'^A(t)$ is the position of the object in the $x'y'$ coordinate system. We have assumed that the origins of the two coordinate systems coincided at $t=0$ and that the axes of the coordinate systems are parallel ($x$ parallel to $x'$ and $y$ parallel to $y'$).

Note that the velocity of the object in the $x'y'$ system is found by adding the velocity of $xy$ relative to $x'y'$ and the velocity of the object in the $xy$ frame ($\vec v^A(t)$):
\begin{align*}
\frac{d}{dt}\pvec r'^A(t) &=\frac{d}{dt}(\pvec v'^Bt+\vec r^A(t))\\
&=\pvec v'^B+\vec v^A(t)
\end{align*}

As an example, consider the situation depicted in Figure \ref{fig:chap4:2drel}. Brice is on a boat off the shore of Nice, with a coordinate system $xy$, and is describing the position of a boat carrying Alice. He describes Alice's position as $\vec r^A(t)$ in the $xy$ coordinate system. Igor is on the shore and also wishes to describe Alice's position using the work done by Brice. Igor sees Brice's boat move with a velocity $\vec v'^B$ as measured in his $x'y'$ coordinate system. In order to find the vector pointing to Alice's position $\pvec r'^A(t)$, he adds the vector from his origin to Brice's origin ($\pvec v'^B t$) and the vector from Brice's origin to Alice $\vec r^A(t)$.

\capfig{0.7\textwidth}{figures/Chapter4/2drel.png}{\label{fig:chap4:2drel} Example of converting from one reference frame to another in two dimensions using vector addition.}

Writing this out by coordinate, we have:
\begin{align*}
x'^A(t)&=v'^B_xt+x^A(t)\\
y'^A(t)&=v'^B_yt+y^A(t)
\end{align*}
and for the velocities:
\begin{align*}
v_x'^A(t)&=v'^B_x+v_x^A(t)\\
v_y'^A(t)&=v'^B_y+v_y^A(t)
\end{align*}


\begin{checkpointMC}{You are on a boat and crossing a North-flowing river, from the East bank to the West bank. You point your boat in the West direction and cross the river. \chloe is watching your boat cross the river from the shore, in which direction does she measure your velocity vector to be?}
\item in the North direction
\item in the West direction
\item a combination of North and West directions
\end{checkpointMC}


\section{Motion in three dimensions}
The big challenge was to expand our description of motion from one dimension to two. Adding a third dimension ends up being trivial now that we know how to use vectors. In three dimensions, we describe the position of a point using three coordinates, so all of the vectors simply have three independent components, but are treated in exactly the same way as in the two dimensional case. The position of an object is now described by three independent functions, $x(t)$, $y(t)$, $z(t)$, that make up the three components of a position vector $\vec r(t)$:
\begin{align*}
\vec r(t) &= \begin{pmatrix}
           x(t) \\
           y(t) \\
           z(t)  \\
         \end{pmatrix}\\
\therefore \vec r(t)  &= x(t) \hat x + y(t) \hat y + z(t) \hat z
\end{align*}
The velocity vector now has three components and is defined analogously to the 2D case:
\begin{align*}
\vec v(t) &=\frac{d\vec r}{dt}
 =\begin{pmatrix}
           \frac{dx}{dt}  \\
          \frac{dy}{dt}  \\
          \frac{dz}{dt}  \\
         \end{pmatrix}
 =\begin{pmatrix}
           v_x(t) \\
           v_y(t) \\
           v_z(t) \\
         \end{pmatrix}\\   
\therefore \vec v(t) &= v_x(t)\hat x+v_y(t)\hat y+v_z(t)\hat z  \nonumber 
\end{align*}
and the acceleration is defined in a similar way:
\begin{align*}
\vec a(t)  &=\frac{d\vec v}{dt}
 =\begin{pmatrix}
           \frac{dv_x}{dt}  \\
          \frac{dv_y}{dt}  \\
          \frac{dv_z}{dt}  \\
         \end{pmatrix}
 =\begin{pmatrix}
           a_x(t) \\
           a_y(t) \\
           a_z(t) \\
         \end{pmatrix}\\   
\therefore \vec a(t) &= a_x(t)\hat x+a_y(t)\hat y+a_z(t)\hat z  \nonumber 
\end{align*}

In particular, if an object has a constant acceleration, $\vec a=a_x\hat x+a_y\hat y+a_z\hat z$, and started at $t=0$ with a position $\vec r_0$ and velocity $\vec v_0$, then its velocity vector is given by:
\begin{align*}
\vec v(t)  &= \vec v_0+\vec at=\begin{pmatrix}
           v_{0x}+ a_xt \\
           v_{0y}+ a_yt \\
           v_{0z}+ a_zt \\
         \end{pmatrix}\\
\end{align*}
and the position vector is given by:
\begin{align*}
\vec r(t)= \vec r_0+\vec v_0 t+\frac{1}{2}\vec a t^2=\begin{pmatrix}
           x_0+v_{0x}t+\frac{1}{2} a_xt^2 \\
           y_0+v_{0y}t+\frac{1}{2} a_yt^2 \\
           z_0+v_{0z}t+\frac{1}{2} a_zt^2 \\
         \end{pmatrix}\\
\end{align*}
where again, we see how writing a single vector equation (e.g. $\vec v(t) = \vec v_0+\vec at$) is really just a way to write the three independent equations that are true for each component.
\section{Circular motion}
We often consider the motion of an object around a circle of fixed radius, $R$. In principle, this is motion in two dimensions, as a circle is necessarily in a two dimensional plane. However, since the object is constrained to move along the circumference of the circle, it can be thought of (and treated as) motion along a one dimensional axis that is curved. 
\capfig{0.35\textwidth}{figures/Chapter4/circle.png}{\label{fig:chap4:circle} Describing the motion of an object around a circle of radius $R$.}

Figure \ref{fig:chap4:circle} shows how we can describe motion on a circle. We could use $x(t)$ and $y(t)$ to describe the position on the circle, however, $x(t)$ and $y(t)$ are no longer independent since they have to correspond to the coordinates of points on a circle:
\begin{align*}
x^2(t)+y^2(t)=R^2
\end{align*}
Instead of using $x$ and $y$, we could think of an axis that is bent around the circle (as shown by the curved arrow in Figure \ref{fig:chap4:circle}, the $s$ axis). The $s$ axis is such that $s=0$ where the circle intersects the $x$ axis, and the value of $s$ increases as we move counter-clockwise along the circle. Distance along the $s$ axis thus corresponds to the distance along the circumference of the circle.

Another variable that could be used for position instead of $s$ is the angle, $\theta$, between the position vector of the object and the $x$ axis, as illustrated in Figure \ref{fig:chap4:circle}. If we express the angle $\theta$ in radians, then it easy to convert between $s$ and $\theta$. Recall, an angle in radians is defined as the length of an arc subtended by that angle divided by the radius of the circle. We thus have:
\begin{align}
\label{eqn:chap4:raddef}
\Aboxed{\theta(t)=\frac{s(t)}{R}}
\end{align}
In particular, if the object has gone around the whole circle, then $s=2\pi R$ (the circumference of a circle), and the corresponding angle is, $\theta=\frac{2\pi R}{R}=2\pi$, namely \SI{360}{\degree}. 

By using the angle, $\theta$, instead of $x$ and $y$, we are effectively using polar coordinates, with a fixed radius. As we already saw, the $x$ and $y$ positions are related to $\theta$ by:
\begin{align*}
x(t) &= R\cos(\theta(t))\\
y(t) &= R\sin(\theta(t))\\
\end{align*}
where $R$ is a constant. For an object moving along the circle, we can write its position vector, $\vec r(t)$, as:
\begin{align*}
\vec r(t)&= \begin{pmatrix}
           x(t) \\
           y(t) \\
         \end{pmatrix}
         =R \begin{pmatrix}
           \cos(\theta(t)) \\
           \sin(\theta(t)) \\
         \end{pmatrix}
\end{align*}
\capfig{0.35\textwidth}{figures/Chapter4/vcircle.png}{\label{fig:chap4:vcircle} The position vector, $\vec r(t)$ is always perpendicular to the velocity vector, $\vec v(t)$, for motion on a circle.}
and the velocity vector is thus given by:
\begin{align*}
\vec v(t) &=\frac{d}{dt}\vec r(t) 
=\frac{d}{dt} R \begin{pmatrix}
           \cos(\theta(t)) \\
           \sin(\theta(t)) \\
         \end{pmatrix} \\
&= R \begin{pmatrix}
           \frac{d}{dt}\cos(\theta(t)) \\
           \frac{d}{dt}\sin(\theta(t)) \\
         \end{pmatrix} \\
 &= R \begin{pmatrix}
           -\sin(\theta(t))\frac{d\theta}{dt} \\
           \cos(\theta(t))\frac{d\theta}{dt} \\
         \end{pmatrix}     
\end{align*}         
where we used the Chain Rule to calculate the time derivatives of the trigonometric functions (since $\theta(t)$ is function of time). The magnitude of the velocity vector is given by:
\begin{align*}
||\vec v|| &=\sqrt{ v_x^2+v_y^2}\\
&=\sqrt{ \left(-R\sin(\theta(t))\frac{d\theta}{dt}\right)^2+\left(R\cos(\theta(t))\frac{d\theta}{dt}\right)^2}\\
&=\sqrt{ R^2\left( \frac{d\theta}{dt}\right)^2[\sin^2(\theta(t))+\cos^2(\theta(t)]}\\
&=R\left |\frac{d\theta}{dt}\right|
\end{align*}

The position and velocity vectors are illustrated in Figure \ref{fig:chap4:vcircle} for an angle $\theta$ in the first quadrant ($0<\theta<\frac{\pi}{2}$). In this case, you can note that the $x$ component of the velocity is negative (in the equation above, and in the Figure). From the equation above, you can also see that $\frac{|v_x|}{|v_y|}=\tan(\theta)$, which is illustrated in Figure \ref{fig:chap4:vcircle}, showing that \textbf{the velocity vector is tangent to the circle} and perpendicular to the position vector. This is always the case for motion along a circle.

We can simplify our description of motion along the circle by using either $s(t)$ or $\theta(t)$ instead of the vectors for position and velocity. If we use $s(t)$ to represent position along the circumference ($s=0$ where the circle intersects the $x$ axis), then the velocity along the $s$ axis is:
\begin{align*}
v_s(t)&=\frac{d}{dt}s(t)\\
&=\frac{d}{dt}R\theta(t)\\
&=R\frac{d\theta}{dt}
\end{align*}
where we used the fact that $\theta=\frac{s}{R}$ to convert from $s$ to $\theta$. The velocity along the $s$ axis is thus precisely equal to the magnitude of the two-dimensional velocity vector (derived above), which makes sense since the velocity vector is tangent to the circle (and thus in the $s$ ``direction'').

If the object has a \textbf{constant speed}, $v_s$, along the circle and started at a position along the circumference $s=s_0$, then its position along the $s$ axis can be described as:
\begin{align*}
s(t)=s_0+v_st
\end{align*}
or, in terms of $\theta$:
\begin{align*}
\theta(t)&=\frac{s(t)}{R}=\frac{s_0}{R}+\frac{v_s}{R}t\\
&=\theta_0 + \frac{d\theta}{dt}t\\
&=\theta_0 + \omega t\\
\Aboxed{\therefore \omega &= \frac{d\theta}{dt}}
\end{align*}
where we introduced $\theta_0$ as the angle corresponding to the position $s_0$, and we introduced $\omega=\frac{d\theta}{dt}$, which is analogous to velocity, but for an angle. $\omega$ is called the \textbf{angular velocity} and is a measure of the rate of change of the angle $\theta$ (as it is the time derivative of the angle). The relation between the ``linear'' velocity $v_s$ (the magnitude of the velocity vector, which corresponds to the velocity in the direction tangent to the circle) and $\omega$ is:
\begin{align*}
\Aboxed{v_s=R\frac{d\theta}{dt}=R\omega }
\end{align*}

Similarly, if the object is accelerating, we can define an \textbf{angular acceleration}, $\alpha(t)$, as the rate of change of the angular acceleration:
\begin{align*}
\alpha(t)=\frac{d\omega}{dt}
\end{align*}
which can directly be related to the acceleration in the $s$ direction, $a_s(t)$:
\begin{align*}
a_d(t) &= \frac{d}{dt}v_s\\
&=\frac{d}{dt}\omega R=R\frac{d\omega}{dt}\\
\Aboxed{a_d(t)&=R\alpha }
\end{align*}
Thus, the linear quantities (those along the $s$ axis) can be related to the angular quantities by multiplying the angular quantities by $R$:
\begin{align}
s&=R\theta\\
v_s&=R\omega\\
a_s&=R\alpha
\end{align}
If the object started at $t=0$ with a position $s=s_0$ ($\theta=\theta_0$), and an initial linear velocity $v_{0s}$ (angular velocity $\omega_0$), and has a \textbf{constant linear acceleration} around the circle, $a_s$ (angular acceleration, $\alpha$), then the position of the object can be described as:
\begin{align*}
s(t) &= s_0+v_{s0}t+\frac{1}{2}a_s t^2\\
\theta(t) &= \theta_0+\omega_0t+\frac{1}{2}\alpha t^2
\end{align*}
which corresponds to an object that is going around the circle faster and faster.

As you recall from section \ref{sec:chap4:accvconst}, we can compute the acceleration \textbf{vector} and identify components that are parallel and perpendicular to the velocity vector:
\begin{align*}
\vec a&=\vec a_{\parallel}(t) + \vec a_{\bot}(t)\\
&=\frac{dv}{dt}\hat v(t)+v\frac{d\hat v}{dt}\\
\end{align*}
The first term, $\vec a_{\parallel}(t)=\frac{dv}{dt}\hat v(t)$, is parallel to the velocity vector $\hat v$, and has a magnitude given by:
\begin{align*}
||\vec a_{\parallel}(t)||&=\frac{dv}{dt}=\ddt v(t)=\ddt R\omega=R\alpha
\end{align*}
That is, the component of the acceleration vector that is parallel to the velocity is precisely the acceleration in the $s$ direction (the linear acceleration). This component of the acceleration is responsible for increasing (or decreasing) the speed of the object and is zero if the object goes around the circle with a constant speed (linear or angular). 

As we saw earlier, the perpendicular component of the acceleration, $\vec a_{\bot}(t)$, is responsible for changing the direction of the velocity vector (as the object continuously changes direction when going in a circle). When the motion is around a circle, this component of the acceleration vector is called ``centripetal'' acceleration (i.e. acceleration pointing towards the centre of the circle, as we will see). We can calculate the centripetal acceleration in terms of our angular variables, noting that the unit vector in the direction of the velocity is $\hat v=-\sin(\theta)\hat x+\cos(\theta)\hat y$:
\begin{align}
\vec a_{\bot}(t)&=v\frac{d\hat v}{dt}\nonumber\\
&=(\omega R)\ddt \left[-\sin(\theta)\hat x+\cos(\theta)\hat y\right]\nonumber\\
&=\omega R \left[-\ddt\sin(\theta)\hat x+\ddt\cos(\theta)\hat y\right]\nonumber\\
&=\omega R \left[-\cos(\theta)\frac{d\theta}{dt}\hat x-\sin(\theta)\frac{d\theta}{dt}\hat y\right]\nonumber\\
&=\omega R [-\cos(\theta)\omega\hat x-\sin(\theta)\omega\hat y]\nonumber\\
\Aboxed{\vec a_{\bot}(t)&=\omega^2 R[-\cos(\theta)\hat x-\sin(\theta)\hat y]}
\end{align}
where you can easily verify that the vector $[-\cos(\theta)\hat x-\sin(\theta)\hat y]$ has unit length and points towards the centre of the circle (when the tail is placed on a point on the circle at angle $\theta$). The centripetal acceleration thus points towards the centre of the circle and has magnitude:
\begin{align}
a_c(t) = ||\vec a_{\bot}(t)||=\omega^2(t) R = \frac{v^2(t)}{R}
\end{align}
where in the last equal sign, we wrote the centripetal acceleration in terms of the speed around the circle ($v=||\vec v||=v_s$).

If an object goes around a circle, it will always have a centripetal acceleration (since its velocity vector must change direction). In addition, if the object's speed is changing, it will also have a linear acceleration, which points in the same direction as the velocity vector (it changes the velocity vector's length but not its direction).

\begin{checkpointMC}{A vicu\~na is going clockwise around a circle that is centred at the origin of an $xy$ coordinate system that is in the plane of the circle. The vicu\~na runs faster and faster around the circle. In which direction does its acceleration vector point just as the vicu\~na is at the point where the circle intersects the positive $y$ axis?}
\item In the negative $y$ direction
\item In the positive $y$ direction
\item A combination of the positive $y$ and positive $x$ directions
\item A combination of the negative $y$ and positive $x$ directions %correct
\item A combination of the negative $y$ and negative $x$ directions
\end{checkpointMC}

\subsection{Period and frequency}
When an object is moving around in a circle, it will typically complete more than one revolution. If the object is going around the circle with a constant speed, we call the motion ``uniform circular motion'', and we can define the \textbf{period and frequency} of the motion. 

The period, $T$, is defined to be the time that it takes to complete one revolution around the circle. If the object has constant angular speed $\omega$, we can find the time, $T$, that it takes to complete one full revolution, from $\theta=0$ to $\theta=2\pi$:
\begin{align}
\omega&=\frac{\Delta \theta}{T}=\frac{2\pi}{T}\nonumber\\
\Aboxed{\therefore T&=\frac{2\pi}{\omega}}
\end{align}
We would obtain the same result using the linear quantities; in one revolution, the object covers a distance of $2\pi R$ at a speed of $v$:
\begin{align*}
v&=\frac{2\pi R}{T}\\
T&=\frac{2\pi R}{v}=\frac{2\pi R}{\omega R}=\frac{2\pi}{\omega}
\end{align*}

The frequency, $f$, is defined to be the inverse of the period:
\begin{align*}
f&=\frac{1}{T}=\frac{\omega}{2\pi}
\end{align*}
and has SI units of $\si{Hz}=\si{s^{-1}}$. Think of frequency as the number of revolutions completed per second. Thus, if the frequency is $f=\SI{1}{Hz}$, the object goes around the circle once per second. 
\capfig{0.35\textwidth}{figures/Chapter4/twocircles.png}{\label{fig:chap4:twocircles} For a given angular velocity, the linear velocity will be larger on a larger circle ($v=\omega R$).} Given the frequency, we can of course obtain the angular velocity:
\begin{align*}
\omega = 2\pi f
\end{align*}
which is sometimes called the ``angular frequency'' instead of the angular velocity. The angular velocity can really be thought of as a frequency, as it represents the ``amount of angle'' per second that an object covers when going around a circle. The angular velocity does not tell us anything about the actual speed of the object, which depends on the radius $v=\omega R$. This is illustrated in Figure \ref{fig:chap4:twocircles}, where two objects can be travelling around two circles of radius $R_1$ and $R_2$ with the same angular velocity $\omega$. If they have the same angular velocity, then it will take them the same amount of time to complete a revolution. However, the outer object has to cover a much larger distance (the circumference is larger), and thus has to move with a larger linear speed.

\begin{checkpointMC}{A motor is rotating at \SI{3000}{rpm}, what is the corresponding frequency in \si{Hz}?}
\item \SI{5}{Hz}
\item \SI{50}{Hz}%correct
\item \SI{500}{Hz}
\end{checkpointMC}


\newpage
\section{Summary}
\vspace{2cm}
\begin{chapterSummary}
\item Something interesting
\end{chapterSummary}
\appendix
\renewcommand\chaptername{Appendix}
\include{AppendixA_Calculus}


\end{document}
