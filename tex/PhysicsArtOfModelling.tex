%Copyright 2017 R.D. Martin
%This book is free software: you can redistribute it and/or modify it under the terms of the GNU General Public License as published by the Free Software Foundation, either version 3 of the License, or (at your option) any later version.
%
%This book is distributed in the hope that it will be useful, but WITHOUT ANY WARRANTY; without even the implied warranty of MERCHANTABILITY or FITNESS FOR A PARTICULAR PURPOSE.  See the GNU General Public License for more details, http://www.gnu.org/licenses/.


\documentclass[11pt]{report}
\usepackage{paralist}
\usepackage{calc}
\usepackage{subfig}
\usepackage{setspace}
\usepackage{amssymb}
\usepackage{amsmath}
\usepackage{amstext}
\usepackage[font={small,it}]{caption}
\usepackage[pdftex]{graphicx} 
\usepackage{fancyhdr,lastpage}
\usepackage{url}
\usepackage{longtable}
\usepackage{comment}
\usepackage{ifthen}
\usepackage{color}
\usepackage[colorlinks=true,linkcolor=blue]{hyperref}
\usepackage[explicit]{titlesec}
\usepackage{lmodern}
\usepackage{listings}
\usepackage{parskip}
\usepackage[table]{xcolor}
\usepackage{enumitem}
%\usepackage{mdframed}

%\lstset{language=Python,showstringspaces=false,commentstyle=} 

\definecolor{mygreen}{rgb}{0,0.6,0}
\lstset{ %
  belowskip=-3em,
  backgroundcolor=\color{white},   % choose the background color; you must add \usepackage{color} or \usepackage{xcolor}
  basicstyle=\footnotesize,        % the size of the fonts that are used for the code
  breakatwhitespace=false,         % sets if automatic breaks should only happen at whitespace
  breaklines=true,                 % sets automatic line breaking
  captionpos=b,                    % sets the caption-position to bottom
  commentstyle=\color{mygreen},    % comment style
  deletekeywords={...},            % if you want to delete keywords from the given language
  escapeinside={\%*}{*)},          % if you want to add LaTeX within your code
  extendedchars=true,              % lets you use non-ASCII characters; for 8-bits encodings only, does not work with UTF-8
  frame=single,	                   % adds a frame around the code
  keepspaces=true,                 % keeps spaces in text, useful for keeping indentation of code (possibly needs columns=flexible)
  keywordstyle=\color{blue},       % keyword style
  language=Python,                 % the language of the code
  otherkeywords={*,...},           % if you want to add more keywords to the set
  numbers=none,                    % where to put the line-numbers; possible values are (none, left, right)
  numbersep=5pt,                   % how far the line-numbers are from the code
  numberstyle=\tiny\color{black}, % the style that is used for the line-numbers
  rulecolor=\color{black},         % if not set, the frame-color may be changed on line-breaks within not-black text (e.g. comments (green here))
  showspaces=false,                % show spaces everywhere adding particular underscores; it overrides 'showstringspaces'
  showstringspaces=false,          % underline spaces within strings only
  showtabs=false,                  % show tabs within strings adding particular underscores
  stepnumber=1,                    % the step between two line-numbers. If it's 1, each line will be numbered
  stringstyle=\color{red},     % string literal style
  tabsize=2,	                   % sets default tabsize to 2 spaces
  title=\lstname                   % show the filename of files included with \lstinputlisting; also try caption instead of title
}



\newcommand{\code}[1]{\texttt{#1}}

%%%%%%%%%%%%%%%%%%%%%%%%%%%%%%%%%%%%
%%%%% Choose what to show%%%%%%%%%%%
%%%%%%%%%%%%%%%%%%%%%%%%%%%%%%%%%%%%
% "example" environment, choose to show in notes or not
\newboolean{showexamples}
\setboolean{showexamples}{true}

% "detailedderivation" environment
\newboolean{showdetailedderivations}
\setboolean{showdetailedderivations}{true}

% "hideable section" environment (e.g. problems section)
\newboolean{showhiddensections}
\setboolean{showhiddensections}{false}

% "solution" environment (for solutions to problems at end of chapter)
\newboolean{showproblemsolutions}
\setboolean{showproblemsolutions}{false}


%%%spacing around titles
%\titlespacing*{\chapter}
%{0pt}{0ex}{0ex}
\titlespacing{\section}
{0pt}{0ex}{0ex}
\titlespacing{\subsection}
{0pt}{0ex}{0ex}
\titlespacing{\subsubsection}
{0pt}{0ex}{0ex}


\newlength\chapnumb
\setlength\chapnumb{4cm}

\titleformat{\chapter}[block]
{\normalfont\sffamily}{}{0pt}
{\parbox[b]{\chapnumb}{%
   \fontsize{120}{110}\selectfont\thechapter}%
  \parbox[b]{\dimexpr\textwidth-\chapnumb\relax}{%
    \raggedleft%
    \hfill{\LARGE#1}\\
    \rule{\dimexpr\textwidth-\chapnumb\relax}{0.4pt}}}
\titleformat{name=\chapter,numberless}[block]
{\normalfont\sffamily}{}{0pt}
{\parbox[b]{\chapnumb}{%
   \mbox{}}%
  \parbox[b]{\dimexpr\textwidth-\chapnumb\relax}{%
    \raggedleft%
    \hfill{\LARGE#1}\\
    \rule{\dimexpr\textwidth-\chapnumb\relax}{0.4pt}}}



\newcommand{\die}[2]{\frac{\partial #1}{\partial #2}}
\newcommand{\lagd}{\mathcal{L}}

\setlength{\parindent}{0pt}
\parskip = \baselineskip

\newenvironment{capfig}[3]{\begin{figure}[h!]\center\includegraphics[width=#1]{#2}\caption{#3}\end{figure}}{}


\newenvironment{learningObjectives}{\textbf{Learning Objectives:}\begin{itemize}}{\end{itemize}}
\newenvironment{chapterSummary}{\begin{itemize}}{\end{itemize}}

%%% Environments for hiding
\newcounter{example}[chapter]
\def\theexample{\thechapter-\arabic{example}}

\ifthenelse{\boolean{showexamples}}{%
  \newenvironment{example}[3]
  {\refstepcounter{example} \vspace{2ex}\hrule\vspace*{1ex}\noindent EXAMPLE \theexample: #2\\ #3 \\ \small\itshape}
  {\vspace{1ex}\hrule\vspace{2ex}}
}
{%
  \newenvironment{example}[3]
  {\refstepcounter{example} \vspace{2ex}\hrule\vspace*{1ex}\noindent EXAMPLE \theexample: #2\\ #3 \vspace*{\dimexpr#1} \small\itshape\begingroup\color{white}}
  {\endgroup\vspace{1ex}\hrule\vspace{2ex}}
  %\excludecomment{example}
}


\ifthenelse{\boolean{showhiddensections}}{%
  \newenvironment{hideablesection}
  {}
  {}
}
{%
  \newenvironment{hideablesection}
  {\begingroup\color{white}}
  {\endgroup}
}


\ifthenelse{\boolean{showproblemsolutions}}{%
  \newenvironment{solution}
  {}
  {}
}
{%
  \newenvironment{solution}
  {\begingroup\color{white}}
  {\endgroup}
}


\ifthenelse{\boolean{showdetailedderivations}}{%
  \newenvironment{detailedderivation}
  {}
  {}
}
{%
  \newenvironment{detailedderivation}
  {\begingroup\color{white}}
  {\endgroup}
}


\newcounter{problem}[chapter]
\def\theproblem{\thechapter-\arabic{problem}}

\newenvironment{problem}[1]
  {\refstepcounter{problem}\textbf{Problem \theproblem: #1}\\}
  {\vspace{2ex}\\}


\def\secondpage{\clearpage\null\vfill
%\pagestyle{empty}
\begin{minipage}[b]{0.9\textwidth}
\footnotesize\raggedright
\setlength{\parskip}{0.5\baselineskip}
Copyright \copyright \the\year\ R.D. Martin\par
This book is free software: you can redistribute it and/or modify it under the terms of the GNU General Public License as published by the Free Software Foundation, either version 3 of the License, or (at your option) any later version.

This book is distributed in the hope that it will be useful, but WITHOUT ANY WARRANTY; without even the implied warranty of MERCHANTABILITY or FITNESS FOR A PARTICULAR PURPOSE.  See the GNU General Public License for more details, http://www.gnu.org/licenses/.
\end{minipage}
\vspace*{2\baselineskip}
\cleardoublepage
\rfoot{\thepage}}

\makeatletter
\g@addto@macro{\maketitle}{\secondpage}
\makeatother
          
\usepackage[paper=letterpaper,
            %includefoot, % Uncomment to put page number above margin
            marginparwidth=.0in,     % Length of section titles
            marginparsep=.05in,       % Space between titles and text
            margin=1in,               % 1 inch margins
            includemp]{geometry}

\setcounter{secnumdepth}{3}
\setcounter{tocdepth}{3}

\begin{document}
\title{Physics: The Art of Modelling}
\author{Ryan D. Martin}
\pagenumbering{roman}
\maketitle
\tableofcontents
\pagenumbering{arabic}

%Copyright 2016 R.D. Martin
%This book is free software: you can redistribute it and/or modify it under the terms of the GNU General Public License as published by the Free Software Foundation, either version 3 of the License, or (at your option) any later version.
%
%This book is distributed in the hope that it will be useful, but WITHOUT ANY WARRANTY; without even the implied warranty of MERCHANTABILITY or FITNESS FOR A PARTICULAR PURPOSE.  See the GNU General Public License for more details, http://www.gnu.org/licenses/.
\chapter{Introduction to Physics}
\label{chap:Intro}

\begin{learningObjectives}
\item Understand the Scientific Method
\item Define the scope of Physics
\item Understand the difference between theory and model
\end{learningObjectives}

\section{Science and the Scientific Method}
Science is, very generally, an attempt to \textit{describe} the world around us. It is important to note that describing the world around us is not the same as \textit{explaining} the world around us\footnote{Some might say this is the role of religion}. Science is an attempt to answer the question ``How?'' and not ``Why?''. As we develop our description of the physical world, you should remember this important distinction. The Scientific Method is a prescription for coming up with a description of the physical world that anyone can challenge through performing experiments. We can get some insight into the Scientific Method through a simple example. 

Imagine that we wish to describe how long it takes a tennis ball to reach the ground after being released from a certain height. One way to proceed is to describe how long it takes for a tennis ball to drop 1\,m, and then to describe how long it takes a tennis ball to drop 2\,m, etc. One could generate a giant table showing how long it takes a tennis ball to drop for any given height. Someone would then be able to perform an experiment to measure how long a tennis ball takes to drop 1\,m or 2\,m and see if they agree with those descriptions.

If we collected the descriptions for all possible heights, then we would effectively have a valid ``scientific theory'' that describes how long it takes tennis balls to drop from any height. Suppose that a budding scientist, let's call her Chlo\"e, then came along and noticed that there is a pattern in the theory that can be described much more succinctly and generally than the gian table. In particular, suppose that she notices that, mathematically, the time, $t$, that it takes for a tennis ball to drop a height, $h$, is proportional to the square root of the height:
\begin{equation*}
t \propto \sqrt{h}
\end{equation*}
Chlo\"e's ``Theory of Tennis Ball Drop Times'' is appealing because it is succinct, and it also allows us to make \textbf{verifiable predictions}. That is, using this theory, we can predict that it will take a tennis ball $\sqrt 2$ times longer to drop from 2\,m than it will from 1\,m, and then perform an experiment to verify that prediction. If the experiment agrees with the prediction, then we conclude that Chlo\"e's theory adequately describes the result of that particular experiment. If the experiment does not agree with the prediction, then we conclude that the theory is not an adequate description of that experiment, and we try to find a new theory.

Chlo\"e's theory is also appealing because it can describe not only tennis balls, but the time it takes for other objects to fall as well. Scientists can then set out to continue testing her theory with a wide range of objects and drop heights to see if it describes those experiments as well. Inevitably, they will discover situations where Chlo\"e's theory fails to adequately describe the time that it takes for objects to fall (can you think of an example?).

We would then develop a new ``Theory of Falling Objects'' that would include Chlo\"e's theory that describes most objects falling, and additionally, a set of descriptions for the fall times for cases that are not described by Chlo\"e's theory. Ideally, we would seek a new theory that would also describe the new phenomena not described by Chlo\"e's theory in a succinct manner. There is of course no guarantee, ever, that such a theory would exist; it is just an optimistic hope of scientists to find the most general and succinct description of the physical world. 



 



\section{The scope of Physics}

\section{Theories and models}

\section{Fighting intuition}

\section{Summary}
\begin{chapterSummary}
\item Science attempts to \textit{describe} the physical world (answers the question ``How?'', not ``Why?'').
\item The Scientific Method provides a prescription for arriving at theories that describe the physical world and that can be experimentally verified.
\item An experiment can only disprove a theory, not confirm it in any general sense.
\item Theories are typically valid only in well-defined situations.
\end{chapterSummary}



\end{document}
