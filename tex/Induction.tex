\section{Electromagnetic induction}

%%%%%%%%%%%%%%%%%%%%%%%%%%%%%%%%%%%
%%
%% Multiple Choice
%%
%%%%%%%%%%%%%%%%%%%%%%%%%%%%%%%%%%%
\subsection{Multiple Choice}

%Kate
\question A loop of wire lies in the $x-y$ plane in a region with a constant magnetic field, with magnitude $B$, in the positive $z$ direction. Which of the following situations will cause the magnetic flux through the loop to decrease?
\begin{checkboxes}
\choice The loop of wire is moved upwards (positive $z$ direction) at a constant velocity.
\choice The loop of wire is moved to the right (positive $x$ direction) at a constant velocity
\choice More wire is added to the loop, causing the circumference of the loop to double. 
\CorrectChoice The loop of wire is rotated such that it becomes oriented in the $y-z$ plane. \correct
\end{checkboxes}

%Question submitted by Patrick Singal
\question Which statement best describes Faraday's law?
\begin{checkboxes}
\CorrectChoice A change in the magnetic flux across a surface induces a voltage along the loop enclosing that surface \correct
\choice A magnetically-induced electric current always creates an opposite force to the one inducing it
\choice The total electric flux out of a surface is proportional to the charge enclosed by that surface. 
\end{checkboxes}

\question A circular loop of radius $r$ and resistance $R$ lies in the $xy$-plane in a region of uniform time-varying magnetic field given by $\vec B(t)=(\SI{-2.5}{T/s})t\hat z$. In which direction does the magnetic moment of the current induced in the loop point at some time $t>0$?
\begin{checkboxes}
\CorrectChoice Positive $z$ \correct
\choice Negative $z$
\choice Other
\end{checkboxes}

\question A circular loop of radius $r$ and resistance $R$ lies in the $xy$-plane in a region of uniform time-varying magnetic field given by $\vec B(t)=(\SI{2.5}{T/s})t\hat z$. In which direction does the magnetic moment of the current induced in the loop point at some time $t>0$?
\begin{checkboxes}
\choice Positive $z$ 
\CorrectChoice Negative $z$ \correct
\choice Other
\end{checkboxes}

\question A circular loop of radius $r$ and resistance $R$ lies in the $xy$-plane in a region of uniform time-varying magnetic field given by $\vec B(t)=(\SI{20}{T})+(\SI{2.5}{T/s})t\hat z$. In which direction does the magnetic moment of the current induced in the loop point at some time $t>0$?
\begin{checkboxes}
\choice Positive $z$ 
\CorrectChoice Negative $z$ \correct
\choice Other
\end{checkboxes}

\question Chlo\"e bought herself a pair of rocket shoes (TODO:reference other example). We want to see if we can use Chlo\"e to charge a cell phone while she rides her rocket shoes. We model Chlo\"e as a thin rigid bar of length $L=\SI{85}{cm}$ that is moving at constant speed $v=\SI{20}{km/h}$ due East (perpendicular to the Earth's magnetic field, assumed to be horizontal in the North direction, $B=\SI{5e-5}{T}$), as in Figure \ref{fig:induction:RocketEMF}. Since she is moving at constant speed, she does not need to lean forward, so we assume the ``Chlo\"e-bar'' is vertical. I've attached a wire that runs from her head to her feet from which I will use the generated emf for my cell phone charger. What is the electric potential difference between the ends of the wire?
\capfig{0.3\textwidth}{figures/Induction/RocketEMF.png}{\label{fig:induction:RocketEMF}Chlo\"e with a vertical wire.}
\begin{choices} 
\CorrectChoice \SI{2.4e-4}{V} \correct
\choice \SI{4.3e-4}{V}
\choice \SI{8.5e-4}{V}
\choice \SI{12.3}{V}
\end{choices}

\question \label{q:induction:galvanometer} A galvanometer is constructed using a rectangular coil with dimension $2$\,cm$\times3$\,cm with $N=100$ turns. The coil can rotate about an axis that goes through its centre (as in Figure \ref{fig:induction:galvanometer}). The coil is immersed in a magnetic field that is shaped in such a way that the flux of the field through the coil is constant and has a density per unit area $\sigma_\Phi=0.02$\,Wb/m$^2$ (recall, 1\,Wb = 1 T$\cdot$m$^2$, the unit of magnetic flux).  If the coil is rotated by an angle $\theta$, a coil spring provides a restoring torque of $\tau=-\kappa\theta$, where $\kappa$ is the torsional spring constant of the spring. What should the value of $\kappa$ be so that a current $I=1.0\times 10^{-6}$\,A leads to a deflection of $\theta = 45^{\circ}$?
\capfig{0.9\textwidth}{figures/Induction/galvanometer.png}{\label{fig:induction:galvanometer} Galvanometer in question \ref{q:induction:galvanometer}.}
\begin{checkboxes}
\CorrectChoice $1.53\times 10^{-9}$\,Nm/rad \correct
\choice $1.53\times 10^{-11}$\,Nm/rad
\choice $2.67\times 10^{-11}$\,Nm/rad
\choice $2.67\times 10^{-13}$\,Nm/rad
\end{checkboxes}

\question A magnetic braking system for a car is constructed by attaching a conducting disk (radius $R$ and thickness $h$) to each car wheel. A portion of each disk is near an electromagnet that can be turned on when one needs to brake. The magnetic field from each electromagnet is perpendicular to the surface of each corresponding disk. For a given car speed and a given set of electromagnets, what happens if we make the disks thicker ($h$ bigger)?
\begin{checkboxes}
\choice Nothing, the braking torque from the magnetic brakes will be the same.
\choice The braking torque from the magnetic brakes will be reduced.
\CorrectChoice The braking torque from the magnetic brakes will be increased. \correct
\end{checkboxes}

\question \label{q:induction:CoilsRotating} An electric generator is made of $N=100$ coils wrapped around a circle of radius $r=0.2$\,m, and immersed in a magnetic field $B=0.2$\,T. The coils are rotated along an axis that passes through their centre and is perpendicular to the magnetic field (Figure \ref{fig:induction:CoilsRotating}). The coils have negligible resistance and are connected to a heater with total resistance $R=500\,\Omega$ (the connection to the heater is non-restoring; that is, the current in the heater from the generator is alternating). At what frequency must the coils rotate so that the average power dissipated in the heater is 100\,W?
\capfig{0.4\textwidth}{figures/Induction/CoilsRotating.png}{\label{fig:induction:CoilsRotating} Coils rotating in a magnetic field, question \ref{q:induction:CoilsRotating}.}
\begin{checkboxes}
\CorrectChoice 20\,Hz \correct
\choice 126 \,Hz
\choice 2002\,Hz
\choice 12,582\,Hz
\end{checkboxes}
\begin{solution}
\begin{align*}
\Phi_B(t)&= B\pi r^2\cos(\omega t)\\
\Delta V(t) &= -N\frac{d\Phi_B(t)}{dt}=NB\pi r^2 \omega \sin(\omega t)\\
V_0 &= NB\pi r^2 \omega \\
\bar P&=\frac{1}{2}\frac{V_0^2}{R}=\frac{1}{2}\frac{N^2B^2\pi^2 r^4 \omega^2}{R}\\
\omega &= \frac{\sqrt{2R\bar P}}{NB\pi r^2} \\
f&=\frac{\omega}{2\pi}=\frac{\sqrt{R\bar P}}{\sqrt 2 NB\pi^2 r^2}
\end{align*}
\end{solution}
%%%%%%%%%%%%%%%%%%%%%%%%%%%%%%%%%%%
%
% long answer
%
%%%%%%%%%%%%%%%%%%%%%%%%%%%%%%%%%%%
\subsection{Long answers}
%original
\question A coil is made of $N$ circular loops of radius $r=\SI{5.0}{cm}$ and has a total resistance $R=\SI{5}{\ohm}$. The axis of the coil is in the $z$ direction such that each loop lies in the $xy$-plane. The coil is in a region of uniform time-varying magnetic field given by:
\begin{align*}
\vec B(t)=B_0+at\hat z
\end{align*}
where $B_0=\SI{20}{T}$ and $a=\SI{2.5}{T/s}$.
\begin{parts}
\part How many loops, $N$, should the coil have such that the induced voltage across the coil is \SI{120}{V} (round $N$ to the nearest integer, e.g. $1.5\to 2$)?
\part What is the magnitude of the current induced in the coil?
\end{parts}
\begin{finalanswer}
\begin{enumerate}[(a)]
\item 6112
\item \SI{24}{A}
\end{enumerate}
\end{finalanswer}
\begin{solution}
\begin{parts}
\part We can choose the normal vector, $\vec A$, for the surface to be in the positive $z$ direction, $\vec A=\pi r^2 \hat z$. The flux through one loop of the coil is thus given by:
\begin{align*}
\Phi_B(t)=\vec A\cdot \vec B=\pi r^2(B_0+at)
\end{align*}
The emf induced in $N$ coils is given by:
\begin{align*}
\Delta V=-N\frac{d\Phi_B}{dt}=-N\pi r^2a
\end{align*}
If we want $\Delta V=\SI{120}{V}$, then, $N$ is given by:
\begin{align*}
N=\left|\frac{\Delta V}{\pi r^2a}\right|=\frac{(\SI{120}{V})}{\pi (\SI{0.05}{m})^2(\SI{2.5}{T/s})}=\num{6112}
\end{align*}
\part We just use Ohm's law to find the current, since we know the voltage across the coil and its resistance:
\begin{align*}
I=\frac{\Delta V}{R}=\frac{(\SI{120}{V})}{(\SI{5}{\ohm})}=\SI{24}{A}
\end{align*}
\end{parts}
\end{solution}
%Based on Giancolli 29-25
\question A square loop of side $a$ is placed a distance $b$ away from an infinitely long wire carrying current $I$, as shown in Figure \ref{fig:induction:LoopWire}
\capfig{0.2\textwidth}{figures/Induction/LoopWire.png}{\label{fig:induction:LoopWire}A loop of wire next to a current carrying wire.}
\begin{parts}
\part Find the magnetic flux through the loop.
\part If the loop is pulled away from the wire at speed $v$, what voltage difference is induced across the loop? Give a formula for the voltage as a function of time, $V(t)$, given that $b(t) = b_0+vt$.
\part If the loop has total resistance, $R$, what force, $F(t)$, is required to keep the loop moving at constant speed? Note that the force must do work to pull the loop; that work gets converted into the electric power dissipated in the loop.
\end{parts}
\begin{finalanswer}
\begin{enumerate}[(a)]
\item \begin{align*}
\Phi_B=\frac{\mu_0 Ia}{2\pi}\ln\left( \frac{b+a}{b} \right)
\end{align*}
\item \begin{align*}
\Delta V(t)=-\frac{\mu_0 Ia^2}{2\pi}\left( \frac{1}{(b_0+vt)(b_0+vt+a)} \right)v
\end{align*}
\item \begin{align*}
F(t)=\frac{\mu_0^2 I^2a^4}{4\pi^2R}\left( \frac{v}{(b_0+vt)^2(b_0+vt+a)^2} \right)
\end{align*}
\end{enumerate}
\end{finalanswer}
\begin{solution}
\begin{parts}
\part To determine the flux through the loop, we consider the flux through a small area of the loop, $dA$, a distance $r$ away from the wire, as shown in Figure \ref{fig:LoopWire_sol}.   

\capfig{0.2\textwidth}{figures/Induction/LoopWire_sol.png}{\label{fig:LoopWire_sol}A small area element in the loop.}

The magnetic field a distance $r$ away from an infinite wire carrying current $I$ is easily found from Amp\`ere's Law:
\begin{align*}
\oint \vec B(r) \cdot d\vec l&=\mu_0 I\\
\therefore B(r) &=\frac{\mu_0I}{2\pi r}
\end{align*}
The magnetic field is in the negative $z$ direction using the coordinate system that is shown. 

The flux through the area element $dA$ is then (choosing $d\vec A$ into the page):
\begin{align*}
d\Phi_B=\vec B(r)\cdot d\vec A=B(r)adr=\frac{\mu_0 I}{2\pi r}adr
\end{align*}
where the area element vector is parallel to the magnetic field (since the magnetic field is perpendicular to the plane of the loop). The total flux through the loop is given by summing over the entire loop:
\begin{align*}
\Phi_B&=\int d\Phi_B=\int_b^{b+a}\frac{\mu_0 Ia}{2\pi r}dr\\
&=\frac{\mu_0 Ia}{2\pi}\int_b^{b+a}\frac{1}{r}dr=\frac{\mu_0 Ia}{2\pi}\ln\left( \frac{b+a}{b} \right)\\
\end{align*}
\part We can find the voltage induced across the loop by taking the time derivative of the flux (Faraday's Law):
\begin{align*}
\Delta V(t)&=-\frac{d\Phi_B}{dt}=-\frac{d}{dt}\frac{\mu_0 Ia}{2\pi}\ln\left( \frac{b(t)+a}{b(t)} \right)\\
&=-\frac{\mu_0 Ia}{2\pi}\frac{d}{dt}\ln\left( \frac{b(t)+a}{b(t)} \right)\\
&=-\frac{\mu_0 Ia}{2\pi}\frac{d}{db}\ln\left( \frac{b(t)+a}{b(t)} \right)\frac{db}{dt}\\
&=-\frac{\mu_0 Ia}{2\pi}\left( \frac{b(t)}{b(t)+a} \right)\left(\frac{-a}{b(t)^2}  \right)\frac{db}{dt}\\
&=\frac{\mu_0 Ia^2}{2\pi}\left( \frac{1}{b(t)(b(t)+a)} \right)\frac{db}{dt}\\
&=\frac{\mu_0 Ia^2}{2\pi}\left( \frac{1}{(b_0+vt)(b_0+vt+a)} \right)v
\end{align*}
which is positive, as expected, since the magnetic moment associated with the induced current is into the page (same direction as $d\vec A$).
\part If a force $\vec F(t)$ is exerted on an object moving with velocity $\vec v$ we can easily find the rate at which it does work on the object. Over a small amount of time, $dt$, the force will be approximately constant and do an amount of work $dW$ over a distance $d\vec x$. The power is then
\begin{align*}
P_F=\frac{dW}{dt}=\frac{d}{dt}\vec F\cdot d\vec x=\vec F\cdot \frac{d\vec x}{dt}=\vec F\cdot \vec v
\end{align*} 
The force that is pulling the loop at constant speed is the only source of energy in the system, which is dissipated as electrical power in the loop:
\begin{align*}
P_E&=\frac{\Delta V(t)^2}{R}=P_F=\vec F\cdot \vec v=F(t)v\\
\therefore F(t)&=\frac{\Delta V(t)^2}{Rv}\\
&=\frac{\mu_0^2 I^2a^4}{4\pi^2R}\left( \frac{v}{(b_0+vt)^2(b_0+vt+a)^2} \right)
\end{align*}
\end{parts}

\end{solution}

%McLean? Smart Physics?? Modified a lot, pretty much consider original!
\question A right-angle triangular loop of wire is constructed with a base of $b$ and a height $h$, as shown in Figure \ref{fig:induction:TriangleLoop}. The loop is immersed in a uniform magnetic field $B$ which points in the positive z-direction. At time $t=0$, the loop is in the $x-y$ plane, as shown. The loop is rotated with a constant angular speed $\vec\omega=-\omega\hat y$ about an axis that is co-linear with the side that lies in the $y$ direction (its height). The loop of wire has a total resistance $R$.
\begin{parts}
\part Write an expression for the current in the loop as a function of time (indicate if the equation is for clockwise or counter-clockwise current, when the loop is viewed at $t=0$, as in the figure)
\part What is the maximum value of the current in the loop?
\part What is the average power dissipated by current in the loop?
\part What (time-dependent) torque must one exert on the loop in order to keep the loop rotating at constant angular speed? Specify the torque vector that one must exert on the loop and show that it is always in the direction of the angular velocity.
\end{parts}
\capfig{0.3\textwidth}{figures/Induction/TriangleLoop.png}{\label{fig:induction:TriangleLoop}A triangular loop rotating in a magnetic field.}
\begin{finalanswer}
\begin{enumerate}[(a)]
\item 
\begin{align*}
I=\frac{1}{2R}hbB\omega\sin(\omega t)
\end{align*}
\item The current can be found from Ohm's law:
\begin{align*}
I=\frac{1}{2R}hbB\omega\sin(\omega t)
\end{align*}
\item \begin{align*}
\bar P=\frac{1}{8R}h^2b^2B^2\omega^2
\end{align*}
\item The torque exerted by the magnetic field at some time $t$ is:
\begin{align*}
\vec \tau_B(t)-IAB\sin(\omega t)(-\hat y)=IAB\sin(\omega t)\hat y
\end{align*} 
We must then exert a torque in the opposite direction:
\begin{align*}
\vec\tau(t)=-\frac{1}{4R}h^2b^2B^2\omega\sin^2(\omega t)\hat y
\end{align*}
which is always in the same direction as the angular velocity, since the $\sin^2(\omega t)$ term will never cause the torque to change direction. 
\end{enumerate}
\end{finalanswer}
\begin{solution}
\begin{parts}
\part The area of the loop is $A=\frac{1}{2}bh$. At $t=0$, we define the area vector to be in the positive $z$ direction. As the loop rotates, the area vector rotates in the $x-z$ plane, as shown in Figure \ref{fig:induction:TriangleLoop_A}:
\begin{align*}
\vec A=A[-\sin(\omega t)\hat x+\cos(\omega t)\hat z]
\end{align*}
\capfig{0.3\textwidth}{figures/Induction/TriangleLoop_A.png}{\label{fig:induction:TriangleLoop_A}View of the loop from above, showing the area vector at time $t$.}

The flux through the loop is given by:
\begin{align*}
\Phi_B&=\vec B\cdot \vec A=B(\hat z)\cdot A[-\sin(\omega t)\hat x+\cos(\omega t)\hat z]\\
&=AB\cos(\omega t)
\end{align*}
The potential difference across the loop is given by:
\begin{align*}
\Delta V&=-\frac{d\Phi_B}{dt}=-\frac{d}{dt}AB\cos(\omega t)\\
&=AB\omega\sin(\omega t)
\end{align*}
The current can be found from Ohm's law:
\begin{align*}
I=\frac{\Delta V}{R}=\frac{1}{R}AB\omega\sin(\omega t)=\frac{1}{2R}hbB\omega\sin(\omega t)
\end{align*}
At $t=0$, the flux in the loop starts to decrease, so the current will try to restore a magnetic field in the positive $z$ direction, and will thus be in the counter-clockwise direction (as viewed in Figure \ref{fig:induction:TriangleLoop}).
\part The maximum current is:
\begin{align*}
I_0=\frac{1}{2R}hbB\omega
\end{align*}
\part The instantaneous power dissipated in the loop at time $t$ is:
\begin{align*}
P(t)=I^2(t)R=\frac{1}{4R}h^2b^2B^2\omega^2\sin^2(\omega t)
\end{align*}
The only part of the power that depends on time is $\sin^2(\omega t)$, which over one rotational period, $T$, averages to a value of $\frac{1}{2}$:
\begin{align*}
\overline{\sin^2(\omega t)}&=\frac{1}{T}\int_0^T\sin^2(\omega t)dt=\frac{\omega}{2\pi}\int_0^{\frac{2\pi}{\omega}}\sin^2(\omega t)dt=\frac{\omega}{2\pi}\int_0^{\frac{2\pi}{\omega}}\sin^2(\omega t)dt\\
&=\frac{\omega}{2\pi}\int_0^{\frac{2\pi}{\omega}}\frac{1}{2}(1-\cos(2\omega t))dt=\frac{\omega}{2\pi}\frac{1}{2}\left[t-\frac{1}{2\omega}\sin(2\omega t)  \right]_0^{\frac{2\pi}{\omega}}=\frac{1}{2}
\end{align*}
where we used the trigonometric identity $\cos2\theta = 1-2\sin^2\theta$. The average power dissipated in the loop is thus:
\begin{align*}
\bar P=\frac{1}{8R}h^2b^2B^2\omega^2
\end{align*}
\part We can find the torque exerted on the loop by the magnetic field using the magnetic moment created by the induced current in the loop. Note that one must then exert a torque on the loop that is opposite of that exerted by the magnetic field to keep the loop rotating at constant angular velocity.

The magnetic moment of the loop is:
\begin{align*}
\vec \mu(t) = I\vec A=IA(-\sin(\omega t)\hat x+\cos(\omega t)\hat z)
\end{align*}
where the direction of the vector $\vec A$ is chosen as before so that at $t=0$, the magnetic moment is in the positive $z$ direction, (consistent with what we determined from Lenz's Law in part (a)). The torque exerted by the magnetic field at some time $t$ is:
\begin{align*}
\vec \tau_B(t)&=\vec \mu(t)\times\vec B=[IA(-\sin(\omega t)\hat x+\cos(\omega t)\hat z)]\times (B\hat z)\\
&=-IAB\sin(\omega t)(\hat x\times\hat z)+IAB\cos(\omega t)(\hat z\times\hat z)\\
&=-IAB\sin(\omega t)(-\hat y)=IAB\sin(\omega t)\hat y
\end{align*} 
We must then exert a torque in the opposite direction:
\begin{align*}
\vec\tau(t)&=-\vec\tau_B(t)=-IAB\sin(\omega t)\hat y\\
&=-\left(\frac{1}{2R}hbB\omega\sin(\omega t)\right)AB\sin(\omega t)\hat y\\
&=-\frac{1}{4R}h^2b^2B^2\omega\sin^2(\omega t)\hat y
\end{align*}
which is always in the same direction as the angular velocity, since the $\sin^2(\omega t)$ term will never cause the torque to change direction. 
\end{parts}
\end{solution}

%Giancolli 29-12 -fixed.
\question A rectangular loop of length $l=\SI{0.60}{m}$, width $w = \SI{0.40}{m}$, and resistance $R = \SI{0.49}{\Omega}$ is pulled by a force $\vec{F}$ from a uniform magnetic field $B = \SI{0.80}{T}$, as shown in Figure \ref{fig:induction:LoopForce}. What force must be applied to the rectangular loop in order for it to maintain a constant speed of $v = \SI{2.15}{m/s}$ while it exits the magnetic field?
\capfig{0.4\textwidth}{figures/Induction/LoopForce.png}{\label{fig:induction:LoopForce}A rectangular loop being pulled out of a magnetic field.}
\begin{finalanswer}
\SI{0.493}{N}
\end{finalanswer}
\begin{solution}
As the loop moves out of the magnetic field, the flux through the loop decreases. This induces a counter-clockwise current $I$ that results in a force being exerted on the wire by the magnetic field. The applied force, $\vec F$, will have to be equal in magnitude to that exerted on current carrying wire by the magnetic field.

If we let $l'$ be the length of the loop that is still in the region of magnetic field, then the flux of the magnetic field through the loop is:
\begin{align*}
\Phi_B=\vec B\cdot \vec A=Bwl'
\end{align*}
since the magnetic field is perpendicular to the plane of the loop. The induced voltage across the loop is given by:
\begin{align*}
\Delta V=-\frac{d}{dt}\Phi_B=-\frac{d}{dt}Bwl'=-Bw\frac{dl'}{dt}=Bwv
\end{align*}
where we noted that since $l'$ decreases with time, its time derivative is negative and cancels the minus sign from Lenz' Law (so that $v$ is the positive magnitude of the velocity). The current in the loop will be:
\begin{align*}
I=\frac{\Delta V}{R}=\frac{Bwv}{R}
\end{align*}
and in the counter-clockwise direction, as argued above. Of the 4 section of the loop that could have a magnetic force, the vertical section on the right has no force because it is not in a magnetic field, and the forces on the two horizontal sections cancel each other because they carry currents in opposite directions. This leaves the vertical section of wire on the left, which experiences a magnetic force:
\begin{align*}
F&=IwB=\frac{Bwv}{R}wB=\frac{B^2w^2v}{R}\\
&=\frac{(\SI{0.80}{T})^2(\SI{0.40}{m})^2(\SI{2.15}{m/s})}{(\SI{0.49}{\ohm})}=\SI{0.493}{N}
\end{align*}
\end{solution}

%Giancolli 29-54 -fixed
\question A point charge q = \SI{15}{\mu C} is fixed on the x-axis \SI{12}{cm} from the centre of a uniform circular magnetic field $\vec{B}$, as shown in Figure \ref{fig:induction:ChargeField}. The magnetic field begins to decrease at a rate of \SI{0.08}{T/s}. What is the resulting force vector on the charge?
\capfig{0.4\textwidth}{figures/Induction/ChargeField.png}{\label{fig:induction:ChargeField}A charge in a region of changing magnetic field.}
\begin{finalanswer}
$(\SI{7.2e-8}{N})\hat y$
\end{finalanswer}
\begin{solution}
We need to find the electric field that results from the changing magnetic field:
\begin{align*}
\int \vec E\cdot d\vec l=-\frac{d}{dt}\Phi_B
\end{align*}
We choose a circular path of radius $r$ to calculate the circulation of the electric field at the position of the charge as well as the flux of the magnetic field through that circle:
\begin{align*}
\int \vec E\cdot d\vec l&=E(2\pi r)\\
\frac{d}{dt}\Phi_B&=\frac{d}{dt}\pi r ^2 B=\pi r^2 \frac{dB}{dt}\\
\therefore E&=\frac{1}{2\pi r}\pi r^2 \frac{dB}{dt}=\frac{1}{2}r\frac{dB}{dt}
\end{align*}
The electric field will be directed in the positive $y$ direction (inducing a current that would restore the magnetic field). The force on the charge is thus:
\begin{align*}
\vec F&=q\vec F=q\frac{1}{2}r\frac{dB}{dt}\hat y\\
&=\frac{1}{2}(\SI{15}{\micro C})(\SI{0.12}{m})(\SI{0.08}{T/s})\hat y\\
&=(\SI{7.2e-8}{N}) \hat y
\end{align*}
\end{solution}

%Modified from a question submitted by Jesse
\question Two superconducting and frictionless rails a distance $l$ apart are connected by a superconducting wire, as shown in Figure \ref{fig:induction:BarRails}. A platform of mass $m$ connects the two rails and can slide without friction along the rails. The platform has a resistance $R$ between the points where it makes contact with the rails. The rails are infinitely long and a uniform magnetic field is perpendicular to the plane of the rails and the platform at all points in space. If the platform is put in motion with a speed $v$ in the direction opposite of the superconducting wire that connects the rails, show that it will take a time:
\begin{align*}
t&=\ln(2)\frac{mR}{l^2B^2}
\end{align*}
for the platform to slow down to half of its initial speed.
\capfig{0.4\textwidth}{figures/Induction/BarRails.png}{\label{fig:induction:BarRails}A platform moving along frictionless rails.}
\begin{solution}
As the platform moves on the rails, the flux through a loop that comprises the platform, the two rails and the superconducting wire will change and a current will be induced. Since the flux through the loop increases, a clockwise current will be generated in the loop (as viewed in Figure \ref{fig:induction:BarRails}). The current will be upwards in the platform (as viewed in Figure \ref{fig:induction:BarRails}), resulting in a force towards the right that will slow the platform down. 

Suppose that at $t=0$ the platform was where the superconducting wire is located, then at some time $t$ it will be a distance $vt$ away from the wire. The flux through the loop will be:
\begin{align*}
\Phi_B(t)=lvtB
\end{align*}
The voltage induced across the loop will be:
\begin{align*}
\Delta V=-\frac{d}{dt}\Phi_B(t)=lvB
\end{align*}
The current will be given by:
\begin{align*}
I=\frac{\Delta V}{R}=\frac{1}{R}lvB
\end{align*}
And the force:
\begin{align*}
F=IlB=\frac{1}{R}l^2B^2v
\end{align*}
We can use Newton's 2nd Law to determine how the speed changes in response to this speed. The tricky part is that the force itself depends on the speed. Writing Newton's 2nd Law, and noting that the force is in the opposite direction of $v$:
\begin{align*}
F&=ma=m\frac{dv}{dt}=-\frac{l^2B^2}{R}v\\
\therefore \frac{dv}{dt}&=-\left(\frac{l^2B^2}{mR}\right)v
\end{align*}
which is a differential equation for $v(t)$ (an equation that contains both the function and its derivative). This is a separable differential equation (we can put all of the quantities that depend on $v$ and $t$ on different sides of the equation:
\begin{align*}
\frac{dv}{v}=-\left(\frac{l^2B^2}{mR}\right)dt
\end{align*}
If we let $v(t=0)=v_0$, we can integrate both sides to find $v(t)$:
\begin{align*}
\int_{v_0}^{v(t)}\frac{dv}{v}&=-\left(\frac{l^2B^2}{mR}\right)\int_0^tdt'\\
\ln\left( \frac{v(t)}{v_0} \right)&=-\left(\frac{l^2B^2}{mR}\right)t\\
\therefore v(t)&=v_0e^{-\frac{l^2B^2}{mR}t}
\end{align*}
So we find that $v(t)$ decreases exponentially with time. We can find the time, $t_1$, at which $v(t)$ has decreased to $\frac{1}{2}v_0$:
\begin{align*}
v(t_1)&=v_0e^{-\frac{l^2B^2}{mR}t_1}=\frac{1}{2}v_0\\
e^{-\frac{l^2B^2}{mR}t_1}&=\frac{1}{2}\\
\frac{-l^2B^2}{mR}t_1&=\ln\left(\frac{1}{2}  \right)\\
t_1&=\ln(2)\frac{mR}{l^2B^2}
\end{align*}
\end{solution}

%Giancolli 29-38 -fixed
\question A generator is made up of a square coil with a \SI{35}{cm} side and 600 loops. The coil is set to generate a \SI{100}{V} peak output. If the coil is in a \SI{0.73}{T} magnetic field, at what frequency must the coil rotate to generate the peak output?
\begin{finalanswer}
\SI{0.297}{Hz}
\end{finalanswer}
\begin{solution}
We can assume that the coils are set up such that at some point in time, their plane is perpendicular to the magnetic field, which we assume to be constant and in the $x$ direction. We can write their area vector as:
\begin{align*}
\vec A=A[\cos(\omega t)\hat x +\sin(\omega t) \hat y]
\end{align*}
where $A$ is the area of the square coil, and we assume that the coils rotate about the $z$ axis. We can thus write the flux through the coils as:
\begin{align*}
\Phi_B=N\vec A\cdot \vec B=NAB\cos(\omega t)
\end{align*}
The induced voltage is:
\begin{align*}
\Delta V=-\frac{d\Phi_B}{dt}=-\frac{d}{d}NAB\cos(\omega t)=NAB\omega\sin(\omega t)
\end{align*}
The maximum output voltage is:
\begin{align*}
V_{max}&=NAB\omega\\
\therefore \omega &=\frac{V_{max}}{NAB}=\frac{(\SI{100}{V})}{600(\SI{0.35}{m})^2(\SI{0.73}{T})}=\SI{1.86}{rad/s}
\end{align*}
The frequency is thus:
\begin{align*}
f=\frac{\omega}{2\pi}=\frac{\SI{9.39}{rad/s}}{2\pi}=\SI{0.297}{Hz}
\end{align*}
\end{solution}

\question A llama farmer has decided to attempt to steal electrical power from the high voltage line that passes above his property by building an induction coil underneath the power line, as in Figure \ref{fig:induction:EvilFarmer}. The horizontal power line is a height $d=\SI{20}{m}$ above the ground and carries a sinusoidally time-varying current:
\begin{align*}
I(t)=I_0\sin(\omega t)
\end{align*}
with a frequency of $\SI{60}{Hz}$ and a peak current of $I_0=\SI{10}{A}$. The farmer builds a vertical induction coil that has a length $L=\SI{15}{m}$ and a height $h=\SI{2}{m}$ with $N$ loops, and that is located directly below the power line, as shown.
\begin{parts}
\part Assuming that the power line is very long, show that the magnetic field a distance $r$ from the line is given by:
\begin{align*}
B(r,t)=(\SI{2e-6}{Tm})\frac{1}{r}\sin(\omega t)
\end{align*}
\part Show that the magnetic flux from the power line through one loop of the induction coil is given by:
\begin{align*}
\Phi(t)=(\SI{3.161e-6}{Wb})\sin(\omega t)
\end{align*}
\part How many loops, $N$, would the farmer need to put in his coil if he is to generate a maximal potential difference of \SI{170}{V} in his induction coil (the voltage in the coil will vary sinusoidally and we want the maximum value to be \SI{170}{V})? Comment on whether this would be practical. 
\end{parts}
\capfig{0.6\textwidth}{figures/Induction/EvilFarmer.png}{\label{fig:induction:EvilFarmer}A farmer stealing power.}
\begin{finalanswer}
\begin{enumerate}[(a)]
\item N/A
\item N/A
\item 140000 loops. This would not be practical. If the wire had a diameter of \SI{1}{mm}, the coul would be just \SI{140}{m} deep.
\end{enumerate}
\end{finalanswer}
\begin{solution}
\begin{parts}
\part We use Amp\`ere's law to find the magnetic field a distance $r$ away from a current carrying wire. Using a circular amperian loop, we have:
\begin{align*}
\oint \vec B(r,t)\cdot d\vec l&=\mu_0 I(t)\\
B(r,t)(2\pi r)&=\mu_0I_0\sin(\omega t)\\
\therefore B(r,t)&=\frac{\mu_0I_0}{2\pi r}\sin(\omega t)\\
&=\frac{4\pi(\SI{1e-7}{TmA^{-1}})(\SI{10}{A})}{2\pi}\frac{1}{r}\sin(\omega t)\\
&=(\SI{2e-6}{Tm})\frac{1}{r}\sin(\omega t)
\end{align*}

\part Since the magnetic field changes with distance from the wire, we need to integrate the flux over the area of the coil to get the total flux. Although the magnetic field changes direction with the oscillating current, it is always perpendicular to the plane of the coil. 

We define a small surface element $dA=Ldr$ with a length $L$ and a height $dr$ that is located a distance $r$ from the power line. The flux through that area element is:
\begin{align*}
d\Phi(r,t)=B(r,t)dA=B(r,t)Ldr
\end{align*}
The total flux through the coil is thus:
\begin{align*}
\Phi(r,t)&=\int d\Phi(r,t) =\int_{d-h}^{d}B(r,t)Ldr=\int_{d-h}^{d}\frac{\mu_0I_0L}{2\pi r}\sin(\omega t)dr\\
&=\frac{\mu_0I_0L}{2\pi}\sin(\omega t)\int_{d-h}^{d}\frac{1}{r}dr\\
&=\frac{\mu_0I_0L}{2\pi}\ln\left(\frac{d}{d-h} \right)\sin(\omega t)\\
&=\frac{4\pi(\SI{1e-7}{TmA^{-1}})(\SI{10}{A})(\SI{15}{m})}{2\pi}\ln\left(\frac{(\SI{20}{m})}{(\SI{18}{m})} \right)\sin(\omega t)\\
&=(\SI{3.161e-6}{Wb})\sin(\omega t)
\end{align*}
\part The induced emf is given by (taking the absolute value, since it is AC current):
\begin{align*}
\Delta V&=N\frac{d\Phi}{dt}=N\frac{d}{dt}\frac{\mu_0I_0L}{2\pi}\ln\left(\frac{d-h}{h} \right)\sin(\omega t)\\
&=N\omega\frac{\mu_0I_0L}{2\pi}\ln\left(\frac{d-h}{h} \right)\cos(\omega t)\\
&=N\omega(\SI{3.161e-6}{Wb})\cos(\omega t)\\
&=N 2\pi f(\SI{3.161e-6}{Wb})\cos(\omega t)\\
&=N 2\pi (\SI{60}{Hz})(\SI{3.161e-6}{Wb})\cos(\omega t)\\
&=N (\SI{1.192e-3}{V})\cos(\omega t)
\end{align*}
where we replaced the angular frequency, $\omega$, with the given frequency, $f$. If we want the maximum value of the potential difference to be \SI{170}{V}, then the number of loops in the coil is given by:
\begin{align*}
N(\SI{1.192e-3}{V})&=\SI{170}{V}\\
\therefore N&=\frac{(\SI{170}{V})}{(\SI{1.192e-3}{V})}=\num{142665}
\end{align*}
The farmer would need to have just over \num{140000} loops in his coil, which would not be practical (if the wire had a diameter of \SI{1}{mm}, the coil would be \SI{140}{m} deep).
\end{parts}
\end{solution}

\question \label{q:induction:Helmholtz} In order to create a region of uniform magnetic field, you use two identical circular coils in the Helmholtz configuration (Figure \ref{fig:induction:HelmholtzCoils}). The coils each have $N=100$ turns and a radius $R=20$\,cm. The centres of the coils are on the z-axis at positions $z=\pm\frac{R}{2}$ and both coils are parallel to the xy-plane. The coils each carry a current, $I=2$\,A, such that the magnetic dipole moment of each coil is in the positive z-direction as shown.
\capfig{0.5\textwidth}{figures/Induction/HelmholtzCoils.png}{\label{fig:induction:HelmholtzCoils} Helmholtz coils (the x-axis is into the page), Question \ref{q:induction:Helmholtz}a.}

\begin{parts}
\part Give an expression for the magnetic field along the z-axis and evaluate the strength of the magnetic field at $z=0$.

\part A small circular pickup coil with a single turn and a radius $r=1$\,cm is placed with its centre at the origin of the coordinate system parallel to the Helmoltz coils. The pickup coil has a total resistance, $R_{pickup}=1\,\Omega$, and is rotated at a frequency of 10\,Hz about the x-axis (Figure \ref{fig:induction:HelmholtzCoilsPickup}, the angular velocity is in the positive x direction). At time $t=0$, the pickup coil is in the xy-plane. Assume that the magnetic field from the Helmholtz coils is uniform near the pickup coil, with magnitude $B_0=9.0\times 10^{-4}$\,T.

Make a graph showing the torque that is required to rotate the pickup coil as a function of time. Make your graph as precise as possible (quantitative labels) and make sure it covers at least one rotation period in time.

\capfig{0.5\textwidth}{figures/Induction/HelmholtzCoilsPickup.png}{\label{fig:induction:HelmholtzCoilsPickup} Pickup coil in uniform magnetic field $B_0\hat z$ (the x-axis is into the page). The coil rotates about the x-axis, with angular velocity in the positive x direction. Question \ref{q:induction:Helmholtz}b.}
\end{parts}
\begin{finalanswer}
\begin{enumerate}[(a)]
\item \begin{align*}
B^{tot}(z)=\frac{\mu_0NIR^2}{2} \left( \frac{1}{((\frac{R}{2}+z)^2+R^2)^{\frac{3}{2}}}+\frac{1}{((\frac{R}{2}-z)^2+R^2)^{\frac{3}{2}}}\right)
\end{align*}
\item \begin{align*}
\tau=\frac{B_0^2\pi^2 r^4\omega}{R_{pickup}}\sin^2(\omega t)
\end{align*}
\capfig{0.5\textwidth}{figures/Induction/taut.png}{\label{fig:induction:taut2} Torque as a function of time.}
\end{enumerate}
\end{finalanswer}
\begin{solution}
a) We use the Biot-Savart Law to calculate the field a distance $h$ from it centre. The field will clearly be in the z-direction, and we need to sum the z-components $\cos\theta=\frac{R}{r}$:
\begin{align*}
B(h) &= \frac{\mu_0NI}{4\pi}\int_0^{2\pi R}\frac{dl}{r^2}\cos\theta=\frac{\mu_0NI}{4\pi}\int_0^{2\pi R}\frac{dl}{r^2}\frac{R}{r}=\frac{\mu_0NI}{4\pi}2\pi R\frac{R}{(h^2+R^2)^{\frac{3}{2}}}\\
&=\frac{\mu_0NIR^2}{2}\frac{1}{(h^2+R^2)^{\frac{3}{2}}}
\end{align*}
The sum of the fields from both coils, along the z-axis is:
\begin{align*}
B^{tot}(z) &= B(\frac{R}{2}+z)+B(\frac{R}{2}-z)\\
&=\frac{\mu_0NIR^2}{2} \left( \frac{1}{((\frac{R}{2}+z)^2+R^2)^{\frac{3}{2}}}+\frac{1}{((\frac{R}{2}-z)^2+R^2)^{\frac{3}{2}}}\right)
\end{align*}
Evaluating, at $B^{tot}(z=0\,m)=8.99\times 10^{-4}$\,T, and, for reference at $z=5$\,cm $B^{tot}(z=0.05\,m)=8.95\times 10^{-4}$\,T, so we see that it does not vary much.
\\
\\
b) We can easily calculate the flux in the coil, and thus the induced current:
\begin{align*}
I^{ind}&=\frac{1}{R_{pickup}}\Delta V=-\frac{1}{R_{pickup}}\frac{d\Phi_B(t)}{dt}=-\frac{1}{R_{pickup}}\frac{d}{dt}B_0\pi r^2\cos(\omega t)\\
&=\frac{B_0\pi r^2\omega}{R_{pickup}}\sin(\omega t)
\end{align*}
Note that just after $t=0$, the flux through the pickup coil is decreasing, so the magnetic moment associated with the induced current is in the positive z direction.

The torque is given by the cross product of the magnetic moment and the magnetic field vector. The torque required to keep the coil turning is in the positive x-direction.
\begin{align*}
\tau &= \mu B_0 \sin(\omega t) = I^{ind}\pi r^2 B_0  \sin(\omega t)= \frac{B_0\pi r^2\omega}{R_{pickup}}\sin(\omega t)\pi r^2 B_0  \sin(\omega t)\\
&=\frac{B_0^2\pi^2 r^4\omega}{R_{pickup}}\sin^2(\omega t)
\end{align*}
The maximum torque is $\tau_0=4.06\times 10^{-10}$\,Nm
\capfig{0.5\textwidth}{figures/Induction/taut.png}{\label{fig:induction:taut} Torque as a function of time.}
\end{solution}
\question You wish to build a little generator for your house using the little stream of water that runs behind it. You construct a coil with $N=\num{1e4}$ circular loops of radius $r=\SI{20}{cm}$, which you attach to a paddle wheel so that it can rotate. You then go on ebay to search for a suitable magnet. Assuming that you want to generate AC voltage with a frequency of $\SI{60}{Hz}$ and a maximum amplitude of $\Delta V=\SI{110}{V}$, how strong a magnetic field do you need? Assume that the magnetic field is uniform through the volume of the coil. 
\begin{finalanswer}
	$\SI{2.322e-4}{T}$
\end{finalanswer}
\begin{solution}
	We can write the flux as a function of time as:
	\begin{align*}
	\Phi_B = AB\cos(\omega t)
	\end{align*}
	where $A=\pi r^2$ is the area of the loop. The induced emf is thus:
	\begin{align*}
	\Delta V = -N\frac{d\Phi_B}{dt}=\omega NAB \sin(\omega t)
	\end{align*}
	The maximum voltage is given by:
	\begin{align*}
	\omega  NAB &= \SI{110}{V}\\
	\therefore B&=\frac{(\SI{110}{V})}{NA\omega}=\frac{(\SI{110}{V})}{NA2\pi f}\\
	&=\frac{(\SI{110}{V})}{2\pi^2(10000)(\SI{0.2}{m})^2(\SI{60}{Hz})}=\SI{2.322e-4}{T}
	\end{align*}
	
\end{solution}
